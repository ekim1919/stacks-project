\IfFileExists{stacks-project.cls}{%
\documentclass{stacks-project}
}{%
\documentclass{amsart}
}

% For dealing with references we use the comment environment
\usepackage{verbatim}
\newenvironment{reference}{\comment}{\endcomment}
%\newenvironment{reference}{}{}
\newenvironment{slogan}{\comment}{\endcomment}
\newenvironment{history}{\comment}{\endcomment}

% For commutative diagrams we use Xy-pic
\usepackage[all]{xy}

% We use 2cell for 2-commutative diagrams.
\xyoption{2cell}
\UseAllTwocells

% We use multicol for the list of chapters between chapters
\usepackage{multicol}

% This is generall recommended for better output
\usepackage{lmodern}
\usepackage[T1]{fontenc}

% For cross-file-references
\usepackage{xr-hyper}

% Package for hypertext links:
\usepackage{hyperref}

% For any local file, say "hello.tex" you want to link to please
% use \externaldocument[hello-]{hello}
\externaldocument[introduction-]{introduction}
\externaldocument[conventions-]{conventions}
\externaldocument[sets-]{sets}
\externaldocument[categories-]{categories}
\externaldocument[topology-]{topology}
\externaldocument[sheaves-]{sheaves}
\externaldocument[sites-]{sites}
\externaldocument[stacks-]{stacks}
\externaldocument[fields-]{fields}
\externaldocument[algebra-]{algebra}
\externaldocument[brauer-]{brauer}
\externaldocument[homology-]{homology}
\externaldocument[derived-]{derived}
\externaldocument[simplicial-]{simplicial}
\externaldocument[more-algebra-]{more-algebra}
\externaldocument[smoothing-]{smoothing}
\externaldocument[modules-]{modules}
\externaldocument[sites-modules-]{sites-modules}
\externaldocument[injectives-]{injectives}
\externaldocument[cohomology-]{cohomology}
\externaldocument[sites-cohomology-]{sites-cohomology}
\externaldocument[dga-]{dga}
\externaldocument[dpa-]{dpa}
\externaldocument[sdga-]{sdga}
\externaldocument[hypercovering-]{hypercovering}
\externaldocument[schemes-]{schemes}
\externaldocument[constructions-]{constructions}
\externaldocument[properties-]{properties}
\externaldocument[morphisms-]{morphisms}
\externaldocument[coherent-]{coherent}
\externaldocument[divisors-]{divisors}
\externaldocument[limits-]{limits}
\externaldocument[varieties-]{varieties}
\externaldocument[topologies-]{topologies}
\externaldocument[descent-]{descent}
\externaldocument[perfect-]{perfect}
\externaldocument[more-morphisms-]{more-morphisms}
\externaldocument[flat-]{flat}
\externaldocument[groupoids-]{groupoids}
\externaldocument[more-groupoids-]{more-groupoids}
\externaldocument[etale-]{etale}
\externaldocument[chow-]{chow}
\externaldocument[intersection-]{intersection}
\externaldocument[pic-]{pic}
\externaldocument[weil-]{weil}
\externaldocument[adequate-]{adequate}
\externaldocument[dualizing-]{dualizing}
\externaldocument[duality-]{duality}
\externaldocument[discriminant-]{discriminant}
\externaldocument[derham-]{derham}
\externaldocument[local-cohomology-]{local-cohomology}
\externaldocument[algebraization-]{algebraization}
\externaldocument[curves-]{curves}
\externaldocument[resolve-]{resolve}
\externaldocument[models-]{models}
\externaldocument[equiv-]{equiv}
\externaldocument[pione-]{pione}
\externaldocument[etale-cohomology-]{etale-cohomology}
\externaldocument[proetale-]{proetale}
\externaldocument[more-etale-]{more-etale}
\externaldocument[trace-]{trace}
\externaldocument[crystalline-]{crystalline}
\externaldocument[spaces-]{spaces}
\externaldocument[spaces-properties-]{spaces-properties}
\externaldocument[spaces-morphisms-]{spaces-morphisms}
\externaldocument[decent-spaces-]{decent-spaces}
\externaldocument[spaces-cohomology-]{spaces-cohomology}
\externaldocument[spaces-limits-]{spaces-limits}
\externaldocument[spaces-divisors-]{spaces-divisors}
\externaldocument[spaces-over-fields-]{spaces-over-fields}
\externaldocument[spaces-topologies-]{spaces-topologies}
\externaldocument[spaces-descent-]{spaces-descent}
\externaldocument[spaces-perfect-]{spaces-perfect}
\externaldocument[spaces-more-morphisms-]{spaces-more-morphisms}
\externaldocument[spaces-flat-]{spaces-flat}
\externaldocument[spaces-groupoids-]{spaces-groupoids}
\externaldocument[spaces-more-groupoids-]{spaces-more-groupoids}
\externaldocument[bootstrap-]{bootstrap}
\externaldocument[spaces-pushouts-]{spaces-pushouts}
\externaldocument[spaces-chow-]{spaces-chow}
\externaldocument[groupoids-quotients-]{groupoids-quotients}
\externaldocument[spaces-more-cohomology-]{spaces-more-cohomology}
\externaldocument[spaces-simplicial-]{spaces-simplicial}
\externaldocument[spaces-duality-]{spaces-duality}
\externaldocument[formal-spaces-]{formal-spaces}
\externaldocument[restricted-]{restricted}
\externaldocument[spaces-resolve-]{spaces-resolve}
\externaldocument[formal-defos-]{formal-defos}
\externaldocument[defos-]{defos}
\externaldocument[cotangent-]{cotangent}
\externaldocument[examples-defos-]{examples-defos}
\externaldocument[algebraic-]{algebraic}
\externaldocument[examples-stacks-]{examples-stacks}
\externaldocument[stacks-sheaves-]{stacks-sheaves}
\externaldocument[criteria-]{criteria}
\externaldocument[artin-]{artin}
\externaldocument[quot-]{quot}
\externaldocument[stacks-properties-]{stacks-properties}
\externaldocument[stacks-morphisms-]{stacks-morphisms}
\externaldocument[stacks-limits-]{stacks-limits}
\externaldocument[stacks-cohomology-]{stacks-cohomology}
\externaldocument[stacks-perfect-]{stacks-perfect}
\externaldocument[stacks-introduction-]{stacks-introduction}
\externaldocument[stacks-more-morphisms-]{stacks-more-morphisms}
\externaldocument[stacks-geometry-]{stacks-geometry}
\externaldocument[moduli-]{moduli}
\externaldocument[moduli-curves-]{moduli-curves}
\externaldocument[examples-]{examples}
\externaldocument[exercises-]{exercises}
\externaldocument[guide-]{guide}
\externaldocument[desirables-]{desirables}
\externaldocument[coding-]{coding}
\externaldocument[obsolete-]{obsolete}
\externaldocument[fdl-]{fdl}
\externaldocument[index-]{index}

% Theorem environments.
%
\theoremstyle{plain}
\newtheorem{theorem}[subsection]{Theorem}
\newtheorem{proposition}[subsection]{Proposition}
\newtheorem{lemma}[subsection]{Lemma}

\theoremstyle{definition}
\newtheorem{definition}[subsection]{Definition}
\newtheorem{example}[subsection]{Example}
\newtheorem{exercise}[subsection]{Exercise}
\newtheorem{situation}[subsection]{Situation}

\theoremstyle{remark}
\newtheorem{remark}[subsection]{Remark}
\newtheorem{remarks}[subsection]{Remarks}

\numberwithin{equation}{subsection}

% Macros
%
\def\lim{\mathop{\mathrm{lim}}\nolimits}
\def\colim{\mathop{\mathrm{colim}}\nolimits}
\def\Spec{\mathop{\mathrm{Spec}}}
\def\Hom{\mathop{\mathrm{Hom}}\nolimits}
\def\Ext{\mathop{\mathrm{Ext}}\nolimits}
\def\SheafHom{\mathop{\mathcal{H}\!\mathit{om}}\nolimits}
\def\SheafExt{\mathop{\mathcal{E}\!\mathit{xt}}\nolimits}
\def\Sch{\mathit{Sch}}
\def\Mor{\mathop{\mathrm{Mor}}\nolimits}
\def\Ob{\mathop{\mathrm{Ob}}\nolimits}
\def\Sh{\mathop{\mathit{Sh}}\nolimits}
\def\NL{\mathop{N\!L}\nolimits}
\def\CH{\mathop{\mathrm{CH}}\nolimits}
\def\proetale{{pro\text{-}\acute{e}tale}}
\def\etale{{\acute{e}tale}}
\def\QCoh{\mathit{QCoh}}
\def\Ker{\mathop{\mathrm{Ker}}}
\def\Im{\mathop{\mathrm{Im}}}
\def\Coker{\mathop{\mathrm{Coker}}}
\def\Coim{\mathop{\mathrm{Coim}}}

% Boxtimes
%
\DeclareMathSymbol{\boxtimes}{\mathbin}{AMSa}{"02}

%
% Macros for moduli stacks/spaces
%
\def\QCohstack{\mathcal{QC}\!\mathit{oh}}
\def\Cohstack{\mathcal{C}\!\mathit{oh}}
\def\Spacesstack{\mathcal{S}\!\mathit{paces}}
\def\Quotfunctor{\mathrm{Quot}}
\def\Hilbfunctor{\mathrm{Hilb}}
\def\Curvesstack{\mathcal{C}\!\mathit{urves}}
\def\Polarizedstack{\mathcal{P}\!\mathit{olarized}}
\def\Complexesstack{\mathcal{C}\!\mathit{omplexes}}
% \Pic is the operator that assigns to X its picard group, usage \Pic(X)
% \Picardstack_{X/B} denotes the Picard stack of X over B
% \Picardfunctor_{X/B} denotes the Picard functor of X over B
\def\Pic{\mathop{\mathrm{Pic}}\nolimits}
\def\Picardstack{\mathcal{P}\!\mathit{ic}}
\def\Picardfunctor{\mathrm{Pic}}
\def\Deformationcategory{\mathcal{D}\!\mathit{ef}}


% OK, start here.
%
\begin{document}

\title{Local Cohomology}


\maketitle

\phantomsection
\label{section-phantom}

\tableofcontents

\section{Introduction}
\label{section-introduction}

\noindent
This chapter continues the study of local cohomology.
A reference is \cite{SGA2}.
The definition of local cohomology can be found in
Dualizing Complexes, Section \ref{dualizing-section-local-cohomology}.
For Noetherian rings taking local cohomology is the same
as deriving a suitable torsion functor as is shown in
Dualizing Complexes, Section
\ref{dualizing-section-local-cohomology-noetherian}.
The relationship with depth can be found in
Dualizing Complexes, Section
\ref{dualizing-section-depth}.

\medskip\noindent
We discuss finiteness properties of local cohomology leading to a proof
of a fairly general version of
Grothendieck's finiteness theorem, see Theorem \ref{theorem-finiteness}
and Lemma \ref{lemma-finiteness-Rjstar} (higher direct images
of coherent modules under open immersions).
Our methods incorporate a few very slick arguments the reader
can find in papers of Faltings, see
\cite{Faltings-annulators} and \cite{Faltings-finiteness}.

\medskip\noindent
As applications we offer a discussion of
Hartshorne-Lichtenbaum vanishing. We also discuss
the action of Frobenius and of differential operators
on local cohomology.




\section{Generalities}
\label{section-generalities}

\noindent
The following lemma tells us that the functor $R\Gamma_Z$
is related to cohomology with supports.

\begin{lemma}
\label{lemma-local-cohomology-is-local-cohomology}
Let $A$ be a ring and let $I$ be a finitely generated ideal.
Set $Z = V(I) \subset X = \Spec(A)$. For $K \in D(A)$ corresponding
to $\widetilde{K} \in D_\QCoh(\mathcal{O}_X)$ via
Derived Categories of Schemes, Lemma \ref{perfect-lemma-affine-compare-bounded}
there is a functorial isomorphism
$$
R\Gamma_Z(K) = R\Gamma_Z(X, \widetilde{K})
$$
where on the left we have
Dualizing Complexes, Equation (\ref{dualizing-equation-local-cohomology})
and on the right we have the functor of
Cohomology, Section \ref{cohomology-section-cohomology-support-bis}.
\end{lemma}

\begin{proof}
By Cohomology, Lemma \ref{cohomology-lemma-triangle-sections-with-support}
there exists a distinguished triangle
$$
R\Gamma_Z(X, \widetilde{K})
\to R\Gamma(X, \widetilde{K})
\to R\Gamma(U, \widetilde{K})
\to R\Gamma_Z(X, \widetilde{K})[1]
$$
where $U = X \setminus Z$. We know that $R\Gamma(X, \widetilde{K}) = K$
by Derived Categories of Schemes, Lemma
\ref{perfect-lemma-affine-compare-bounded}.
Say $I = (f_1, \ldots, f_r)$. Then we obtain a finite affine
open covering $\mathcal{U} : U = D(f_1) \cup \ldots \cup D(f_r)$.
By Derived Categories of Schemes, Lemma
\ref{perfect-lemma-alternating-cech-complex-complex-computes-cohomology}
the alternating {\v C}ech complex
$\text{Tot}(\check{\mathcal{C}}_{alt}^\bullet(\mathcal{U},
\widetilde{K^\bullet}))$
computes $R\Gamma(U, \widetilde{K})$ where $K^\bullet$ is any
complex of $A$-modules representing $K$. Working through the
definitions we find
$$
R\Gamma(U, \widetilde{K}) =
\text{Tot}\left(
K^\bullet \otimes_A
(\prod\nolimits_{i_0} A_{f_{i_0}} \to
\prod\nolimits_{i_0 < i_1} A_{f_{i_0}f_{i_1}} \to
\ldots \to A_{f_1\ldots f_r})\right)
$$
It is clear that
$K^\bullet = R\Gamma(X, \widetilde{K^\bullet}) \to
R\Gamma(U, \widetilde{K}^\bullet)$
is induced by the diagonal map from $A$ into $\prod A_{f_i}$.
Hence we conclude that
$$
R\Gamma_Z(X, \mathcal{F}^\bullet) =
\text{Tot}\left(
K^\bullet \otimes_A
(A \to \prod\nolimits_{i_0} A_{f_{i_0}} \to
\prod\nolimits_{i_0 < i_1} A_{f_{i_0}f_{i_1}} \to
\ldots \to A_{f_1\ldots f_r})\right)
$$
By Dualizing Complexes, Lemma \ref{dualizing-lemma-local-cohomology-adjoint}
this complex computes $R\Gamma_Z(K)$ and we see the lemma holds.
\end{proof}

\begin{lemma}
\label{lemma-local-cohomology}
Let $A$ be a ring and let $I \subset A$ be a finitely generated ideal.
Set $X = \Spec(A)$, $Z = V(I)$, $U = X \setminus Z$, and $j : U \to X$
the inclusion morphism. Let $\mathcal{F}$ be a quasi-coherent
$\mathcal{O}_U$-module. Then
\begin{enumerate}
\item there exists an $A$-module $M$ such that $\mathcal{F}$ is the
restriction of $\widetilde{M}$ to $U$,
\item given $M$ there is an exact sequence
$$
0 \to H^0_Z(M) \to M \to H^0(U, \mathcal{F}) \to H^1_Z(M) \to 0
$$
and isomorphisms $H^p(U, \mathcal{F}) = H^{p + 1}_Z(M)$ for $p \geq 1$,
\item we may take $M = H^0(U, \mathcal{F})$ in which case
we have $H^0_Z(M) = H^1_Z(M) = 0$.
\end{enumerate}
\end{lemma}

\begin{proof}
The existence of $M$ follows from
Properties, Lemma \ref{properties-lemma-extend-trivial}
and the fact that quasi-coherent sheaves on $X$ correspond
to $A$-modules (Schemes, Lemma \ref{schemes-lemma-equivalence-quasi-coherent}).
Then we look at the distinguished triangle
$$
R\Gamma_Z(X, \widetilde{M}) \to R\Gamma(X, \widetilde{M}) \to
R\Gamma(U, \widetilde{M}|_U) \to R\Gamma_Z(X, \widetilde{M})[1]
$$
of Cohomology, Lemma \ref{cohomology-lemma-triangle-sections-with-support}.
Since $X$ is affine we have $R\Gamma(X, \widetilde{M}) = M$
by Cohomology of Schemes, Lemma
\ref{coherent-lemma-quasi-coherent-affine-cohomology-zero}.
By our choice of $M$ we have $\mathcal{F} = \widetilde{M}|_U$
and hence this produces an exact sequence
$$
0 \to H^0_Z(X, \widetilde{M}) \to M \to H^0(U, \mathcal{F}) \to
H^1_Z(X, \widetilde{M}) \to 0
$$
and isomorphisms $H^p(U, \mathcal{F}) = H^{p + 1}_Z(X, \widetilde{M})$
for $p \geq 1$. By Lemma \ref{lemma-local-cohomology-is-local-cohomology}
we have $H^i_Z(M) = H^i_Z(X, \widetilde{M})$ for all $i$.
Thus (1) and (2) do hold.
Finally, setting $M' = H^0(U, \mathcal{F})$ we see that
the kernel and cokernel of $M \to M'$ are $I$-power torsion.
Therefore $\widetilde{M}|_U \to \widetilde{M'}|_U$ is an isomorphism
and we can indeed use $M'$ as predicted in (3). It goes without saying
that we obtain zero for both $H^0_Z(M')$ and $H^0_Z(M')$.
\end{proof}

\begin{lemma}
\label{lemma-already-torsion}
Let $I, J \subset A$ be finitely generated ideals of a ring $A$.
If $M$ is an $I$-power torsion module, then the
canonical map
$$
H^i_{V(I) \cap V(J)}(M) \to H^i_{V(J)}(M)
$$
is an isomorphism for all $i$.
\end{lemma}

\begin{proof}
Use the spectral sequence of
Dualizing Complexes, Lemma \ref{dualizing-lemma-local-cohomology-ss}
to reduce to the statement $R\Gamma_I(M) = M$ which is immediate
from the construction of local cohomology
in Dualizing Complexes, Section \ref{dualizing-section-local-cohomology}.
\end{proof}

\begin{lemma}
\label{lemma-multiplicative}
Let $S \subset A$ be a multiplicative set of a ring $A$.
Let $M$ be an $A$-module with $S^{-1}M = 0$. Then
$\colim_{f \in S} H^0_{V(f)}(M) = M$ and
$\colim_{f \in S} H^1_{V(f)}(M) = 0$.
\end{lemma}

\begin{proof}
The statement on $H^0$ follows directly from the definitions.
To see the statement on $H^1$ observe that $R\Gamma_{V(f)}$
and $H^1_{V(f)}$ commute with colimits. Hence we may assume
$M$ is annihilated by some $f \in S$. Then
$H^1_{V(ff')}(M) = 0$ for all $f' \in S$ (for example by
Lemma \ref{lemma-already-torsion}).
\end{proof}

\begin{lemma}
\label{lemma-elements-come-from-bigger}
Let $I \subset A$ be a finitely generated ideal of a ring $A$.
Let $\mathfrak p$ be a prime ideal. Let $M$ be an $A$-module.
Let $i \geq 0$ be an integer and consider the map
$$
\Psi :
\colim_{f \in A, f \not \in \mathfrak p} H^i_{V((I, f))}(M)
\longrightarrow
H^i_{V(I)}(M)
$$
Then
\begin{enumerate}
\item $\Im(\Psi)$ is the set of elements which map to zero in
$H^i_{V(I)}(M)_\mathfrak p$,
\item if $H^{i - 1}_{V(I)}(M)_\mathfrak p = 0$, then $\Psi$ is injective,
\item if $H^{i - 1}_{V(I)}(M)_\mathfrak p = H^i_{V(I)}(M)_\mathfrak p = 0$,
then $\Psi$ is an isomorphism.
\end{enumerate}
\end{lemma}

\begin{proof}
For $f \in A$, $f \not \in \mathfrak p$ the spectral sequence of
Dualizing Complexes, Lemma \ref{dualizing-lemma-local-cohomology-ss}
degenerates to give short exact sequences
$$
0 \to H^1_{V(f)}(H^{i - 1}_{V(I)}(M)) \to
H^i_{V((I, f))}(M) \to H^0_{V(f)}(H^i_{V(I)}(M)) \to 0
$$
This proves (1) and part (2) follows from this and
Lemma \ref{lemma-multiplicative}.
Part (3) is a formal consequence.
\end{proof}

\begin{lemma}
\label{lemma-isomorphism}
Let $I \subset I' \subset A$ be finitely generated ideals of a
Noetherian ring $A$. Let $M$ be an $A$-module. Let $i \geq 0$ be an integer.
Consider the map
$$
\Psi : H^i_{V(I')}(M) \to H^i_{V(I)}(M)
$$
The following are true:
\begin{enumerate}
\item if $H^i_{\mathfrak pA_\mathfrak p}(M_\mathfrak p) = 0$
for all $\mathfrak p \in V(I) \setminus V(I')$, then
$\Psi$ is surjective,
\item if $H^{i - 1}_{\mathfrak pA_\mathfrak p}(M_\mathfrak p) = 0$
for all $\mathfrak p \in V(I) \setminus V(I')$, then
$\Psi$ is injective,
\item if $H^i_{\mathfrak pA_\mathfrak p}(M_\mathfrak p) =
H^{i - 1}_{\mathfrak pA_\mathfrak p}(M_\mathfrak p) = 0$
for all $\mathfrak p \in V(I) \setminus V(I')$, then
$\Psi$ is an isomorphism.
\end{enumerate}
\end{lemma}

\begin{proof}
Proof of (1).
Let $\xi \in H^i_{V(I)}(M)$. Since $A$ is Noetherian, there exists a
largest ideal $I \subset I'' \subset I'$ such that $\xi$ is the image
of some $\xi'' \in H^i_{V(I'')}(M)$. If $V(I'') = V(I')$, then we are
done. If not, choose a generic point $\mathfrak p \in V(I'')$ not in $V(I')$.
Then we have $H^i_{V(I'')}(M)_\mathfrak p =
H^i_{\mathfrak pA_\mathfrak p}(M_\mathfrak p) = 0$ by assumption.
By Lemma \ref{lemma-elements-come-from-bigger} we can increase $I''$
which contradicts maximality.

\medskip\noindent
Proof of (2). Let $\xi' \in H^i_{V(I')}(M)$ be in the kernel of $\Psi$.
Since $A$ is Noetherian, there exists a
largest ideal $I \subset I'' \subset I'$ such that $\xi'$
maps to zero in $H^i_{V(I'')}(M)$. If $V(I'') = V(I')$, then we are
done. If not, then choose a generic point $\mathfrak p  \in V(I'')$
not in $V(I')$. Then we have $H^{i - 1}_{V(I'')}(M)_\mathfrak p =
H^{i - 1}_{\mathfrak pA_\mathfrak p}(M_\mathfrak p) = 0$ by assumption.
By Lemma \ref{lemma-elements-come-from-bigger} we can increase $I''$
which contradicts maximality.

\medskip\noindent
Part (3) is formal from parts (1) and (2).
\end{proof}





\section{Hartshorne's connectedness lemma}
\label{section-hartshorne-connectedness}

\noindent
The title of this section refers to the following result.

\begin{lemma}
\label{lemma-depth-2-connected-punctured-spectrum}
\begin{reference}
\cite[Proposition 2.1]{Hartshorne-connectedness}
\end{reference}
\begin{slogan}
Hartshorne's connectedness
\end{slogan}
Let $A$ be a Noetherian local ring of depth $\geq 2$.
Then the punctured spectra of $A$, $A^h$, and $A^{sh}$ are connected.
\end{lemma}

\begin{proof}
Let $U$ be the punctured spectrum of $A$.
If $U$ is disconnected then we see that
$\Gamma(U, \mathcal{O}_U)$ has a nontrivial idempotent.
But $A$, being local, does not have a nontrivial idempotent.
Hence $A \to \Gamma(U, \mathcal{O}_U)$ is not an isomorphism.
By Lemma \ref{lemma-local-cohomology}
we conclude that either $H^0_\mathfrak m(A)$ or $H^1_\mathfrak m(A)$
is nonzero. Thus $\text{depth}(A) \leq 1$ by
Dualizing Complexes, Lemma \ref{dualizing-lemma-depth}.
To see the result for $A^h$ and $A^{sh}$ use
More on Algebra, Lemma \ref{more-algebra-lemma-henselization-depth}.
\end{proof}

\begin{lemma}
\label{lemma-catenary-S2-equidimensional}
\begin{reference}
\cite[Corollary 5.10.9]{EGA}
\end{reference}
Let $A$ be a Noetherian local ring which is catenary and $(S_2)$.
Then $\Spec(A)$ is equidimensional.
\end{lemma}

\begin{proof}
Set $X = \Spec(A)$. Say $d = \dim(A) = \dim(X)$. Inside $X$ consider the
union $X_1$ of the irreducible components of dimension $d$ and the union
$X_2$ of the irreducible components of dimension $< d$. Of course
$X = X_1 \cup X_2$. If $X_2 = \emptyset$,
then the lemma holds. If not, then $Z = X_1 \cap X_2$ is a nonempty closed
subset of $X$ because it contains at least the closed point of $X$.
Hence we can choose a generic point $z \in Z$ of an irreducible component
of $Z$. Recall that the spectrum of $\mathcal{O}_{Z, z}$ is the set of points
of $X$ specializing to $z$. Since $z$ is both contained in an
irreducible component of dimension $d$ and in an irreducible component
of dimension $< d$ we obtain nontrivial specializations $x_1 \leadsto z$ and
$x_2 \leadsto z$ such that the closures of $x_1$ and $x_2$ have different
dimensions. Since $X$ is catenary, this can only happen if at least
one of the specializations $x_1 \leadsto z$ and $x_2 \leadsto z$ is not
immediate! Thus $\dim(\mathcal{O}_{Z, z}) \geq 2$. Therefore
$\text{depth}(\mathcal{O}_{Z, z}) \geq 2$ because $A$ is $(S_2)$.
However, the punctured spectrum $U$ of $\mathcal{O}_{Z, z}$ is disconnected
because the closed subsets $U \cap X_1$ and $U \cap X_2$ are disjoint
(by our choice of $z$) and cover $U$. This is a contradiction with
Lemma \ref{lemma-depth-2-connected-punctured-spectrum}
and the proof is complete.
\end{proof}



\section{Cohomological dimension}
\label{section-cd}

\noindent
A quick section about cohomological dimension.

\begin{lemma}
\label{lemma-cd}
Let $I \subset A$ be a finitely generated ideal of a ring $A$.
Set $Y = V(I) \subset X = \Spec(A)$. Let $d \geq -1$ be an integer.
The following are equivalent
\begin{enumerate}
\item $H^i_Y(A) = 0$ for $i > d$,
\item $H^i_Y(M) = 0$ for $i > d$ for every $A$-module $M$, and
\item if $d = -1$, then $Y = \emptyset$, if $d = 0$, then
$Y$ is open and closed in $X$, and if $d > 0$ then
$H^i(X \setminus Y, \mathcal{F}) = 0$ for $i \geq d$
for every quasi-coherent $\mathcal{O}_{X \setminus Y}$-module $\mathcal{F}$.
\end{enumerate}
\end{lemma}

\begin{proof}
Observe that $R\Gamma_Y(-)$ has finite cohomological dimension by
Dualizing Complexes, Lemma \ref{dualizing-lemma-local-cohomology-adjoint}
for example. Hence there exists an integer $i_0$ such that
$H^i_Y(M) = 0$ for all $A$-modules $M$ and $i \geq i_0$.

\medskip\noindent
Let us prove that (1) and (2) are equivalent. It is immediate that
(2) implies (1). Assume (1). By descending induction on $i > d$
we will show that $H^i_Y(M) = 0$ for all $A$-modules $M$.
For $i \geq i_0$ we have seen this above. To do the induction step,
let $i_0 > i > d$. Choose any $A$-module $M$ and fit it into
a short exact sequence $0 \to N \to F \to M \to 0$ where $F$ is a
free $A$-module. Since $R\Gamma_Y$ is a right adjoint, we see that
$H^i_Y(-)$ commutes with direct sums. Hence $H^i_Y(F) = 0$
as $i > d$ by assumption (1). Then we see that
$H^i_Y(M) = H^{i + 1}_Y(N) = 0$ as desired.

\medskip\noindent
Assume $d = -1$ and (2) holds. Then $0 = H^0_Y(A/I) = A/I \Rightarrow A = I
\Rightarrow Y = \emptyset$. Thus (3) holds. We omit the proof of the converse.

\medskip\noindent
Assume $d = 0$ and (2) holds. Set
$J = H^0_I(A) = \{x \in A \mid I^nx = 0 \text{ for some }n > 0\}$.
Then
$$
H^1_Y(A) = \Coker(A \to \Gamma(X \setminus Y, \mathcal{O}_{X \setminus Y}))
\quad\text{and}\quad
H^1_Y(I) = \Coker(I \to \Gamma(X \setminus Y, \mathcal{O}_{X \setminus Y}))
$$
and the kernel of the first map is equal to $J$. See
Lemma \ref{lemma-local-cohomology}.
We conclude from (2) that $I(A/J) = A/J$.
Thus we may pick $f \in I$
mapping to $1$ in $A/J$. Then $1 - f \in J$ so $I^n(1 - f) = 0$ for some
$n > 0$. Hence $f^n = f^{n + 1}$. Then $e = f^n \in I$ is an idempotent.
Consider the complementary idempotent $e' = 1 - f^n \in J$.
For any element $g \in I$ we have $g^m e' = 0$ for some $m > 0$.
Thus $I$ is contained in the radical of ideal $(e) \subset I$.
This means $Y = V(I) = V(e)$ is open and closed in $X$ as predicted in (3).
Conversely, if $Y = V(I)$ is open and closed, then the functor
$H^0_Y(-)$ is exact and has vanshing higher derived functors.

\medskip\noindent
If $d > 0$, then we see immediately from
Lemma \ref{lemma-local-cohomology} that (2) is equivalent to (3).
\end{proof}

\begin{definition}
\label{definition-cd}
Let $I \subset A$ be a finitely generated ideal of a ring $A$.
The smallest integer $d \geq -1$ satisfying the equivalent conditions
of Lemma \ref{lemma-cd} is called the
{\it cohomological dimension of $I$ in $A$} and is
denoted $\text{cd}(A, I)$.
\end{definition}

\noindent
Thus we have $\text{cd}(A, I) = -1$ if
$I = A$ and $\text{cd}(A, I) = 0$ if $I$ is locally nilpotent
or generated by an idempotent.
Observe that $\text{cd}(A, I)$ exists by the following lemma.

\begin{lemma}
\label{lemma-bound-cd}
Let $I \subset A$ be a finitely generated ideal of a ring $A$.
Then
\begin{enumerate}
\item $\text{cd}(A, I)$ is at most equal to the number of
generators of $I$,
\item $\text{cd}(A, I) \leq r$ if there exist $f_1, \ldots, f_r \in A$
such that $V(f_1, \ldots, f_r) = V(I)$,
\item $\text{cd}(A, I) \leq c$ if $\Spec(A) \setminus V(I)$
can be covered by $c$ affine opens.
\end{enumerate}
\end{lemma}

\begin{proof}
The explicit description for $R\Gamma_Y(-)$ given in
Dualizing Complexes, Lemma \ref{dualizing-lemma-local-cohomology-adjoint}
shows that (1) is true. We can deduce (2) from (1) using the
fact that $R\Gamma_Z$ depends only on the closed subset
$Z$ and not on the choice of the finitely generated ideal
$I \subset A$ with $V(I) = Z$. This follows either from the
construction of local cohomology in
Dualizing Complexes, Section \ref{dualizing-section-local-cohomology}
combined with
More on Algebra, Lemma \ref{more-algebra-lemma-local-cohomology-closed}
or it follows from Lemma \ref{lemma-local-cohomology-is-local-cohomology}.
To see (3) we use Lemma \ref{lemma-cd}
and the vanishing result of Cohomology of Schemes, Lemma
\ref{coherent-lemma-vanishing-nr-affines}.
\end{proof}

\begin{lemma}
\label{lemma-cd-sum}
Let $I, J \subset A$ be finitely generated ideals of a ring $A$.
Then $\text{cd}(A, I + J) \leq \text{cd}(A, I) + \text{cd}(A, J)$.
\end{lemma}

\begin{proof}
Use the definition and Dualizing Complexes, Lemma
\ref{dualizing-lemma-local-cohomology-ss}.
\end{proof}

\begin{lemma}
\label{lemma-cd-change-rings}
Let $A \to B$ be a ring map. Let $I \subset A$ be a finitely generated ideal.
Then $\text{cd}(B, IB) \leq \text{cd}(A, I)$. If $A \to B$ is faithfully
flat, then equality holds.
\end{lemma}

\begin{proof}
Use the definition and
Dualizing Complexes, Lemma \ref{dualizing-lemma-torsion-change-rings}.
\end{proof}

\begin{lemma}
\label{lemma-cd-local}
Let $I \subset A$ be a finitely generated ideal of a ring $A$.
Then $\text{cd}(A, I) = \max \text{cd}(A_\mathfrak p, I_\mathfrak p)$.
\end{lemma}

\begin{proof}
Let $Y = V(I)$ and $Y' = V(I_\mathfrak p) \subset \Spec(A_\mathfrak p)$.
Recall that
$R\Gamma_Y(A) \otimes_A A_\mathfrak p = R\Gamma_{Y'}(A_\mathfrak p)$
by Dualizing Complexes, Lemma \ref{dualizing-lemma-torsion-change-rings}.
Thus we conclude by Algebra, Lemma \ref{algebra-lemma-characterize-zero-local}.
\end{proof}

\begin{lemma}
\label{lemma-cd-dimension}
Let $I \subset A$ be a finitely generated ideal of a ring $A$.
If $M$ is a finite $A$-module, then
$H^i_{V(I)}(M) = 0$ for $i > \dim(\text{Supp}(M))$.
In particular, we have $\text{cd}(A, I) \leq \dim(A)$.
\end{lemma}

\begin{proof}
We first prove the second statement.
Recall that $\dim(A)$ denotes the Krull dimension. By
Lemma \ref{lemma-cd-local} we may assume $A$ is local.
If $V(I) = \emptyset$, then the result is true.
If $V(I) \not = \emptyset$, then
$\dim(\Spec(A) \setminus V(I)) < \dim(A)$ because
the closed point is missing. Observe that
$U = \Spec(A) \setminus V(I)$ is a quasi-compact
open of the spectral space $\Spec(A)$, hence a spectral space itself.
See Algebra, Lemma \ref{algebra-lemma-spec-spectral} and
Topology, Lemma \ref{topology-lemma-spectral-sub}.
Thus Cohomology, Proposition
\ref{cohomology-proposition-cohomological-dimension-spectral}
implies $H^i(U, \mathcal{F}) = 0$ for $i \geq \dim(A)$
which implies what we want by Lemma \ref{lemma-cd}.
In the Noetherian case the reader may use
Grothendieck's Cohomology, Proposition
\ref{cohomology-proposition-vanishing-Noetherian}.

\medskip\noindent
We will deduce the first statement from the second.
Let $\mathfrak a$ be the annihilator of the finite $A$-module $M$.
Set $B = A/\mathfrak a$. Recall that $\Spec(B) = \text{Supp}(M)$, see
Algebra, Lemma \ref{algebra-lemma-support-closed}.
Set $J = IB$. Then $M$ is a $B$-module
and $H^i_{V(I)}(M) = H^i_{V(J)}(M)$, see
Dualizing Complexes, Lemma
\ref{dualizing-lemma-local-cohomology-and-restriction}.
Since $\text{cd}(B, J) \leq \dim(B) = \dim(\text{Supp}(M))$
by the first part we conclude.
\end{proof}

\begin{lemma}
\label{lemma-cd-is-one}
Let $I \subset A$ be a finitely generated ideal of a ring $A$. If
$\text{cd}(A, I) = 1$ then $\Spec(A) \setminus V(I)$ is nonempty affine.
\end{lemma}

\begin{proof}
This follows from Lemma \ref{lemma-cd} and
Cohomology of Schemes, Lemma
\ref{coherent-lemma-quasi-compact-h1-zero-covering}.
\end{proof}

\begin{lemma}
\label{lemma-cd-maximal}
Let $(A, \mathfrak m)$ be a Noetherian local ring of dimension $d$.
Then $H^d_\mathfrak m(A)$ is nonzero and $\text{cd}(A, \mathfrak m) = d$.
\end{lemma}

\begin{proof}
By one of the characterizations of dimension, there exists
an ideal of definition for $A$ generated by $d$ elements, see
Algebra, Proposition \ref{algebra-proposition-dimension}.
Hence $\text{cd}(A, \mathfrak m) \leq d$ by
Lemma \ref{lemma-bound-cd}. Thus $H^d_\mathfrak m(A)$ is
nonzero if and only if $\text{cd}(A, \mathfrak m) = d$ if and only if
$\text{cd}(A, \mathfrak m) \geq d$.

\medskip\noindent
Let $A \to A^\wedge$ be the map from $A$ to its completion.
Observe that $A^\wedge$ is a Noetherian local ring of the
same dimension as $A$ with maximal ideal $\mathfrak m A^\wedge$.
See Algebra, Lemmas
\ref{algebra-lemma-completion-Noetherian-Noetherian},
\ref{algebra-lemma-completion-complete}, and
\ref{algebra-lemma-completion-faithfully-flat} and
More on Algebra, Lemma \ref{more-algebra-lemma-completion-dimension}.
By Lemma \ref{lemma-cd-change-rings}
it suffices to prove the lemma for $A^\wedge$.

\medskip\noindent
By the previous paragraph we may assume that $A$ is
a complete local ring. Then $A$ has a normalized dualizing complex
$\omega_A^\bullet$ (Dualizing Complexes, Lemma
\ref{dualizing-lemma-ubiquity-dualizing}).
The local duality theorem (in the form of
Dualizing Complexes, Lemma \ref{dualizing-lemma-special-case-local-duality})
tells us $H^d_\mathfrak m(A)$ is Matlis dual to
$\text{Ext}^{-d}(A, \omega_A^\bullet) = H^{-d}(\omega_A^\bullet)$
which is nonzero for example by
Dualizing Complexes, Lemma
\ref{dualizing-lemma-nonvanishing-generically-local}.
\end{proof}

\begin{lemma}
\label{lemma-cd-bound-dim-local}
Let $(A, \mathfrak m)$ be a Noetherian local ring.
Let $I \subset A$ be a proper ideal.
Let $\mathfrak p \subset A$ be a prime ideal
such that $V(\mathfrak p) \cap V(I) = \{\mathfrak m\}$.
Then $\dim(A/\mathfrak p) \leq \text{cd}(A, I)$.
\end{lemma}

\begin{proof}
By Lemma \ref{lemma-cd-change-rings} we have
$\text{cd}(A, I) \geq \text{cd}(A/\mathfrak p, I(A/\mathfrak p))$.
Since $V(I) \cap V(\mathfrak p) = \{\mathfrak m\}$ we have
$\text{cd}(A/\mathfrak p, I(A/\mathfrak p)) =
\text{cd}(A/\mathfrak p, \mathfrak m/\mathfrak p)$.
By Lemma \ref{lemma-cd-maximal} this is equal to $\dim(A/\mathfrak p)$.
\end{proof}

\begin{lemma}
\label{lemma-cd-blowup}
Let $A$ be a Noetherian ring. Let $I \subset A$ be an ideal.
Let $b : X' \to X = \Spec(A)$ be the blowing up of $I$.
If the fibres of $b$ have dimension $\leq d - 1$, then
$\text{cd}(A, I) \leq d$.
\end{lemma}

\begin{proof}
Set $U = X \setminus V(I)$. Denote $j : U \to X'$ the canonical open
immersion, see Divisors, Section \ref{divisors-section-blowing-up}.
Since the exceptional divisor is an effective Cartier divisor
(Divisors, Lemma
\ref{divisors-lemma-blowing-up-gives-effective-Cartier-divisor})
we see that $j$ is affine, see
Divisors, Lemma
\ref{divisors-lemma-complement-locally-principal-closed-subscheme}.
Let $\mathcal{F}$ be a quasi-coherent $\mathcal{O}_U$-module.
Then $R^pj_*\mathcal{F} = 0$ for $p > 0$, see
Cohomology of Schemes, Lemma
\ref{coherent-lemma-relative-affine-vanishing}.
On the other hand, we have $R^qb_*(j_*\mathcal{F}) = 0$ for
$q \geq d$ by Limits, Lemma
\ref{limits-lemma-higher-direct-images-zero-above-dimension-fibre}.
Thus by the Leray spectral sequence
(Cohomology, Lemma \ref{cohomology-lemma-relative-Leray})
we conclude that $R^n(b \circ j)_*\mathcal{F} = 0$ for
$n \geq d$. Thus $H^n(U, \mathcal{F}) = 0$ for $n \geq d$
(by Cohomology, Lemma \ref{cohomology-lemma-apply-Leray}).
This means that $\text{cd}(A, I) \leq d$ by definition.
\end{proof}







\section{More general supports}
\label{section-supports}

\noindent
Let $A$ be a Noetherian ring. Let $M$ be an $A$-module.
Let $T \subset \Spec(A)$ be a subset stable under specialization
(Topology, Definition \ref{topology-definition-specialization}).
Let us define
$$
H^0_T(M) = \colim_{Z \subset T} H^0_Z(M)
$$
where the colimit is over the directed partially ordered set of
closed subsets $Z$ of $\Spec(A)$ contained in
$T$\footnote{Since $T$ is stable under specialization
we have $T = \bigcup_{Z \subset T} Z$, see
Topology, Lemma \ref{topology-lemma-stable-specialization}.}.
In other words, an element $m$ of $M$ is in $H^0_T(M) \subset M$
if and only if the support $V(\text{Ann}_R(m))$ of $m$
is contained in $T$.

\begin{lemma}
\label{lemma-support}
Let $A$ be a Noetherian ring. Let $T \subset \Spec(A)$ be a subset stable
under specialization. For an $A$-module $M$ the following are equivalent
\begin{enumerate}
\item $H^0_T(M) = M$, and
\item $\text{Supp}(M) \subset T$.
\end{enumerate}
The category of such $A$-modules is a Serre subcategory
of the category $A$-modules closed under direct sums.
\end{lemma}

\begin{proof}
The equivalence holds because the support of an element of $M$
is contained in the support of $M$ and conversely the support of
$M$ is the union of the supports of its elements.
The category of these modules is a Serre subcategory
(Homology, Definition \ref{homology-definition-serre-subcategory})
of $\text{Mod}_A$ by
Algebra, Lemma \ref{algebra-lemma-support-quotient}.
We omit the proof of the statement on direct sums.
\end{proof}

\noindent
Let $A$ be a Noetherian ring. Let $T \subset \Spec(A)$ be a subset stable
under specialization. Let us denote $\text{Mod}_{A, T} \subset \text{Mod}_A$
the Serre subcategory described in Lemma \ref{lemma-support}.
Let us denote $D_T(A) \subset D(A)$ the
strictly full saturated triangulated subcategory of $D(A)$
(Derived Categories, Lemma \ref{derived-lemma-cohomology-in-serre-subcategory})
consisting of complexes of $A$-modules whose cohomology modules
are in $\text{Mod}_{A, T}$. We obtain functors
$$
D(\text{Mod}_{A, T}) \to D_T(A) \to D(A)
$$
See discussion in
Derived Categories, Section \ref{derived-section-triangulated-sub}.
Denote $RH^0_T : D(A) \to D(\text{Mod}_{A, T})$ the right
derived extension of $H^0_T$. We will denote
$$
R\Gamma_T : D^+(A) \to D^+_T(A),
$$
the composition of $RH^0_T : D^+(A) \to D^+(\text{Mod}_{A, T})$ with
$D^+(\text{Mod}_{A, T}) \to D^+_T(A)$. If the dimension of $A$ is
finite\footnote{If $\dim(A) = \infty$ the construction
may have unexpected properties on unbounded complexes.},
then we will denote
$$
R\Gamma_T : D(A) \to D_T(A)
$$
the composition of $RH^0_T$ with
$D(\text{Mod}_{A, T}) \to D_T(A)$.

\begin{lemma}
\label{lemma-adjoint}
Let $A$ be a Noetherian ring. Let $T \subset \Spec(A)$
be a subset stable under specialization. The functor
$RH^0_T$ is the right adjoint to the functor
$D(\text{Mod}_{A, T}) \to D(A)$.
\end{lemma}

\begin{proof}
This follows from the fact that the functor $H^0_T(-)$ is
the right adjoint to the inclusion functor
$\text{Mod}_{A, T} \to \text{Mod}_A$, see
Derived Categories, Lemma \ref{derived-lemma-derived-adjoint-functors}.
\end{proof}

\begin{lemma}
\label{lemma-adjoint-ext}
Let $A$ be a Noetherian ring. Let $T \subset \Spec(A)$
be a subset stable under specialization.
For any object $K$ of $D(A)$ we have
$$
H^i(RH^0_T(K)) = \colim_{Z \subset T\text{ closed}} H^i_Z(K)
$$
\end{lemma}

\begin{proof}
Let $J^\bullet$ be a K-injective complex representing $K$.
By definition $RH^0_T$ is represented by the complex
$$
H^0_T(J^\bullet) = \colim H^0_Z(J^\bullet)
$$
where the equality follows from our definition of $H^0_T$.
Since filtered colimits are exact the cohomology of this
complex in degree $i$ is
$\colim H^i(H^0_Z(J^\bullet)) = \colim H^i_Z(K)$
as desired.
\end{proof}

\begin{lemma}
\label{lemma-equal-plus}
Let $A$ be a Noetherian ring. Let $T \subset \Spec(A)$ be a subset stable
under specialization. The functor $D^+(\text{Mod}_{A, T}) \to D^+_T(A)$
is an equivalence.
\end{lemma}

\begin{proof}
Let $M$ be an object of $\text{Mod}_{A, T}$. Choose an embedding
$M \to J$ into an injective $A$-module. By
Dualizing Complexes, Proposition
\ref{dualizing-proposition-structure-injectives-noetherian}
the module $J$ is a direct sum of injective hulls of residue fields.
Let $E$ be an injective hull of the residue field of $\mathfrak p$.
Since $E$ is $\mathfrak p$-power torsion we see that
$H^0_T(E) = 0$ if $\mathfrak p \not \in T$ and
$H^0_T(E) = E$ if $\mathfrak p \in T$.
Thus $H^0_T(J)$ is injective as a direct sum of injective hulls
(by the proposition) and we have an embedding $M \to H^0_T(J)$.
Thus every object $M$ of $\text{Mod}_{A, T}$ has an injective resolution
$M \to J^\bullet$ with $J^n$ also in $\text{Mod}_{A, T}$. It follows
that $RH^0_T(M) = M$.

\medskip\noindent
Next, suppose that $K \in D_T^+(A)$. Then the spectral sequence
$$
R^qH^0_T(H^p(K)) \Rightarrow R^{p + q}H^0_T(K)
$$
(Derived Categories, Lemma \ref{derived-lemma-two-ss-complex-functor})
converges and above we have seen that only the terms with $q = 0$
are nonzero. Thus we see that $RH^0_T(K) \to K$ is an isomorphism.
Thus the functor $D^+(\text{Mod}_{A, T}) \to D^+_T(A)$
is an equivalence with quasi-inverse given by $RH^0_T$.
\end{proof}

\begin{lemma}
\label{lemma-equal-full}
Let $A$ be a Noetherian ring. Let $T \subset \Spec(A)$ be a subset stable
under specialization. If $\dim(A) < \infty$, then functor
$D(\text{Mod}_{A, T}) \to D_T(A)$ is an equivalence.
\end{lemma}

\begin{proof}
Say $\dim(A) = d$. Then we see that $H^i_Z(M) = 0$ for $i > d$
for every closed subset $Z$ of $\Spec(A)$, see
Lemma \ref{lemma-cd-dimension}.
By Lemma \ref{lemma-adjoint-ext} we find that $H^0_T$ has bounded
cohomological dimension.

\medskip\noindent
Let $K \in D_T(A)$. We claim that $RH^0_T(K) \to K$ is an
isomorphism. We know this is true when $K$ is bounded below, see
Lemma \ref{lemma-equal-plus}. However, since $H^0_T$ has bounded
cohomological dimension, we see that the $i$th cohomology of
$RH_T^0(K)$ only depends on $\tau_{\geq -d + i}K$ and we conclude.
Thus $D(\text{Mod}_{A, T}) \to D_T(A)$ is an equivalence with
quasi-inverse $RH^0_T$.
\end{proof}

\begin{remark}
\label{remark-upshot}
Let $A$ be a Noetherian ring. Let $T \subset \Spec(A)$ be a
subset stable under specialization.
The upshot of the discussion above is that
$R\Gamma_T : D^+(A) \to D_T^+(A)$ is the right adjoint
to the inclusion functor $D_T^+(A) \to D^+(A)$.
If $\dim(A) < \infty$, then
$R\Gamma_T : D(A) \to D_T(A)$ is the right adjoint
to the inclusion functor $D_T(A) \to D(A)$.
In both cases we have
$$
H^i_T(K) = H^i(R\Gamma_T(K)) = R^iH^0_T(K) =
\colim_{Z \subset T\text{ closed}} H^i_Z(K)
$$
This follows by combining
Lemmas \ref{lemma-adjoint}, \ref{lemma-adjoint-ext},
\ref{lemma-equal-plus}, and \ref{lemma-equal-full}.
\end{remark}

\begin{lemma}
\label{lemma-torsion-change-rings}
Let $A \to B$ be a flat homomorphism of Noetherian rings.
Let $T \subset \Spec(A)$ be a subset stable under specialization.
Let $T' \subset \Spec(B)$ be the inverse image of $T$.
Then the canonical map
$$
R\Gamma_T(K) \otimes_A^\mathbf{L} B
\longrightarrow
R\Gamma_{T'}(K \otimes_A^\mathbf{L} B)
$$
is an isomorphism for $K \in D^+(A)$. If $A$ and $B$ have finite
dimension, then this is true for $K \in D(A)$.
\end{lemma}

\begin{proof}
From the map $R\Gamma_T(K) \to K$ we get a map
$R\Gamma_T(K) \otimes_A^\mathbf{L} B \to K \otimes_A^\mathbf{L} B$.
The cohomology modules of $R\Gamma_T(K) \otimes_A^\mathbf{L} B$
are supported on $T'$ and hence we get the arrow of the lemma.
This arrow is an isomorphism if $T$ is a closed subset of $\Spec(A)$ by
Dualizing Complexes, Lemma \ref{dualizing-lemma-torsion-change-rings}.
Recall that $H^i_T(K)$ is the colimit of $H^i_Z(K)$ where $Z$ runs over
the (directed set of) closed subsets of $T$, see
Lemma \ref{lemma-adjoint-ext}.
Correspondingly
$H^i_{T'}(K \otimes_A^\mathbf{L} B) =
\colim H^i_{Z'}(K \otimes_A^\mathbf{L} B)$ where $Z'$ is the inverse
image of $Z$. Thus the result because $\otimes_A B$ commutes
with filtered colimits and there are no higher Tors.
\end{proof}

\begin{lemma}
\label{lemma-local-cohomology-ss}
Let $A$ be a ring and let $T, T' \subset \Spec(A)$ subsets
stable under specialization. For $K \in D^+(A)$
there is a spectral sequence
$$
E_2^{p, q} = H^p_T(H^p_{T'}(K)) \Rightarrow H^{p + q}_{T \cap T'}(K)
$$
as in Derived Categories, Lemma
\ref{derived-lemma-grothendieck-spectral-sequence}.
\end{lemma}

\begin{proof}
Let $E$ be an object of $D_{T \cap T'}(A)$. Then we have
$$
\Hom(E, R\Gamma_T(R\Gamma_{T'}(K))) =
\Hom(E, R\Gamma_{T'}(K)) =
\Hom(E, K)
$$
The first equality by the adjointness property of $R\Gamma_T$
and the second by the adjointness property of $R\Gamma_{T'}$.
On the other hand, if $J^\bullet$ is a bounded below complex
of injectives representing $K$, then $H^0_{T'}(J^\bullet)$
is a complex of injective $A$-modules representing $R\Gamma_{T'}(K)$
and hence $H^0_T(H^0_{T'}(J^\bullet))$ is a complex representing
$R\Gamma_T(R\Gamma_{T'}(K))$. Thus $R\Gamma_T(R\Gamma_{T'}(K))$
is an object of $D^+_{T \cap T'}(A)$. Combining these two
facts we find that $R\Gamma_{T \cap T'} = R\Gamma_T \circ R\Gamma_{T'}$.
This produces the spectral sequence by the lemma referenced
in the statement.
\end{proof}

\begin{lemma}
\label{lemma-torsion-tensor-product}
Let $A$ be a Noetherian ring. Let $T \subset \Spec(A)$ be a subset
stable under specialization. Assume $A$ has finite dimension. Then
$$
R\Gamma_T(K) = R\Gamma_T(A) \otimes_A^\mathbf{L} K
$$
for $K \in D(A)$. For $K, L \in D(A)$ we have
$$
R\Gamma_T(K \otimes_A^\mathbf{L} L) =
K \otimes_A^\mathbf{L} R\Gamma_T(L) =
R\Gamma_T(K) \otimes_A^\mathbf{L} L =
R\Gamma_T(K) \otimes_A^\mathbf{L} R\Gamma_T(L)
$$
If $K$ or $L$ is in $D_T(A)$ then so is $K \otimes_A^\mathbf{L} L$.
\end{lemma}

\begin{proof}
By construction we may represent $R\Gamma_T(A)$ by a complex $J^\bullet$ in
$\text{Mod}_{A, T}$. Thus if we represent $K$ by a K-flat complex $K^\bullet$
then we see that $R\Gamma_T(A) \otimes_A^\mathbf{L} K$ is represented
by the complex $\text{Tot}(J^\bullet \otimes_A K^\bullet)$ in
$\text{Mod}_{A, T}$. Using the map $R\Gamma_T(A) \to A$ we obtain
a map $R\Gamma_T(A) \otimes_A^\mathbf{L} K\to K$. Thus by the adjointness
property of $R\Gamma_T$ we obtain a canonical map
$$
R\Gamma_T(A) \otimes_A^\mathbf{L} K \longrightarrow R\Gamma_T(K)
$$
factoring the just constructed map. Observe that $R\Gamma_T$ commutes
with direct sums in $D(A)$ for example by Lemma \ref{lemma-adjoint-ext},
the fact that directed colimits commute with direct sums, and the
fact that usual local cohomology commutes with direct sums
(for example by Dualizing Complexes, Lemma
\ref{dualizing-lemma-local-cohomology-adjoint}).
Thus by More on Algebra, Remark \ref{more-algebra-remark-P-resolution}
it suffices to check the map is an isomorphism for
$K = A[k]$ where $k \in \mathbf{Z}$. This is clear.

\medskip\noindent
The final statements follow from the result we've just shown
by transitivity of derived tensor products.
\end{proof}





\section{Filtrations on local cohomology}
\label{section-filter-local-cohomology}

\noindent
Some tricks related to the spectral sequence of
Lemma \ref{lemma-local-cohomology-ss}.

\begin{lemma}
\label{lemma-filter-local-cohomology}
Let $A$ be a Noetherian ring. Let $T \subset \Spec(A)$
be a subset stable under specialization. Let $T' \subset T$ be
the set of nonminimal primes in $T$. Then $T'$
is a subset of $\Spec(A)$ stable under specialization
and for every $A$-module $M$ there is an exact sequence
$$
0 \to
\colim_{Z, f} H^1_f(H^{i - 1}_Z(M)) \to
H^i_{T'}(M) \to H^i_T(M) \to
\bigoplus\nolimits_{\mathfrak p \in T \setminus T'}
H^i_{\mathfrak p A_\mathfrak p}(M_\mathfrak p)
$$
where the colimit is over closed subsets $Z \subset T$
and $f \in A$ with $V(f) \cap Z \subset T'$.
\end{lemma}

\begin{proof}
For every $Z$ and $f$ the spectral sequence of
Dualizing Complexes, Lemma \ref{dualizing-lemma-local-cohomology-ss}
degenerates to give short exact sequences
$$
0 \to H^1_f(H^{i - 1}_Z(M)) \to
H^i_{Z \cap V(f)}(M) \to H^0_f(H^i_Z(M)) \to 0
$$
We will use this without further mention below.

\medskip\noindent
Let $\xi \in H^i_T(M)$ map to zero in the direct sum.
Then we first write $\xi$ as the image of some $\xi' \in H^i_Z(M)$
for some closed subset $Z \subset T$, see Lemma \ref{lemma-adjoint-ext}.
Then $\xi'$ maps to zero in $H^i_{\mathfrak p A_\mathfrak p}(M_\mathfrak p)$
for every $\mathfrak p \in Z$, $\mathfrak p \not \in T'$.
Since there are finitely many of these primes,
we may choose $f \in A$ not contained in any of these
such that $f$ annihilates $\xi'$. Then $\xi'$
is the image of some $\xi'' \in H^i_{Z'}(M)$
where $Z' = Z \cap V(f)$. By our choice of $f$ we have
$Z' \subset T'$ and we get exactness at the penultimate spot.

\medskip\noindent
Let $\xi \in H^i_{T'}(M)$ map to zero in $H^i_T(M)$.
Choose closed subsets $Z' \subset Z$ with $Z' \subset T'$
and $Z \subset T$ such that $\xi$ comes from $\xi' \in H^i_{Z'}(M)$
and maps to zero in $H^i_Z(M)$. Then we can find $f \in A$
with $V(f) \cap Z = Z'$ and we conclude.
\end{proof}

\begin{lemma}
\label{lemma-zero}
Let $A$ be a Noetherian ring of finite dimension.
Let $T \subset \Spec(A)$ be a subset stable under specialization.
Let $\{M_n\}_{n \geq 0}$ be an inverse system of $A$-modules.
Let $i \geq 0$ be an integer. Assume that for every $m$ there
exists an integer $m'(m) \geq m$ such  that for all
$\mathfrak p \in T$ the induced map
$$
H^i_{\mathfrak p A_\mathfrak p}(M_{k, \mathfrak p})
\longrightarrow
H^i_{\mathfrak p A_\mathfrak p}(M_{m, \mathfrak p})
$$
is zero for $k \geq m'(m)$. Let $m'' : \mathbf{N} \to \mathbf{N}$
be the $2^{\dim(T)}$-fold self-composition of $m'$. Then the map
$H^i_T(M_k) \to H^i_T(M_m)$ is zero for all $k \geq m''(m)$.
\end{lemma}

\begin{proof}
We first make a general remark: suppose we have an exact
sequence
$$
(A_n) \to (B_n) \to (C_n)
$$
of inverse systems of abelian groups. Suppose that for every
$m$ there exists an integer $m'(m) \geq m$ such that
$$
A_k \to A_m
\quad\text{and}\quad
C_k \to C_m
$$
are zero for $k \geq m'(m)$. Then for $k \geq m'(m'(m))$
the map $B_k \to B_m$ is zero.

\medskip\noindent
We will prove the lemma by induction on $\dim(T)$ which is
finite because $\dim(A)$ is finite. Let $T' \subset T$ be
the set of nonminimal primes in $T$. Then $T'$
is a subset of $\Spec(A)$ stable under specialization
and the hypotheses of the lemma apply to $T'$.
Since $\dim(T') < \dim(T)$ we know the lemma holds for $T'$.
For every $A$-module $M$ there is an exact sequence
$$
H^i_{T'}(M) \to H^i_T(M) \to
\bigoplus\nolimits_{\mathfrak p \in T \setminus T'}
H^i_{\mathfrak p A_\mathfrak p}(M_\mathfrak p)
$$
by Lemma \ref{lemma-filter-local-cohomology}.
Thus we conclude by the initial remark of the proof.
\end{proof}

\begin{lemma}
\label{lemma-essential-image}
Let $A$ be a Noetherian ring. Let $T \subset \Spec(A)$ be a subset
stable under specialization. Let $\{M_n\}_{n \geq 0}$ be an inverse system
of $A$-modules. Let $i \geq 0$ be an integer. Assume the dimension of $A$
is finite and that for every $m$ there exists an integer $m'(m) \geq m$
such that for all $\mathfrak p \in T$ we have
\begin{enumerate}
\item $H^{i - 1}_{\mathfrak p A_\mathfrak p}(M_{k, \mathfrak p})
\to H^{i - 1}_{\mathfrak p A_\mathfrak p}(M_{m, \mathfrak p})$
is zero for $k \geq m'(m)$, and
\item $ H^i_{\mathfrak p A_\mathfrak p}(M_{k, \mathfrak p}) \to
H^i_{\mathfrak p A_\mathfrak p}(M_{m, \mathfrak p})$
has image $G(\mathfrak p, m)$ independent of $k \geq m'(m)$ and moreover
$G(\mathfrak p, m)$ maps injectively into
$H^i_{\mathfrak p A_\mathfrak p}(M_{0, \mathfrak p})$.
\end{enumerate}
Then there exists an integer $m_0$ such that for every $m \geq m_0$
there exists an integer $m''(m) \geq m$ such that
for $k \geq m''(m)$ the image of $H^i_T(M_k) \to H^i_T(M_m)$
maps injectively into $H^i_T(M_{m_0})$.
\end{lemma}

\begin{proof}
We first make a general remark: suppose we have an exact
sequence
$$
(A_n) \to (B_n) \to (C_n) \to (D_n)
$$
of inverse systems of abelian groups. Suppose that there exists
an integer $m_0$ such that for every $m \geq m_0$
there exists an integer $m'(m) \geq m$ such that the maps
$$
\Im(B_k \to B_m) \longrightarrow B_{m_0}
\quad\text{and}\quad
\Im(D_k \to D_m) \longrightarrow D_{m_0}
$$
are injective for $k \geq m'(m)$ and $A_k \to A_m$ is zero
for $k \geq m'(m)$. Then for $m \geq m'(m_0)$ and $k \geq m'(m'(m))$
the map
$$
\Im(C_k \to C_m) \to C_{m'(m_0)}
$$
is injective. Namely, let $c_0 \in C_m$ be the image of $c_3 \in C_k$
and say $c_0$ maps to zero in $C_{m'(m_0)}$. Picture
$$
C_k \to C_{m'(m'(m))} \to C_{m'(m)} \to C_m \to C_{m'(m_0)},\quad
c_3 \mapsto c_2 \mapsto c_1 \mapsto c_0 \mapsto 0
$$
We have to show $c_0 = 0$.
The image $d_3$ of $c_3$ maps to zero in $C_{m_0}$ and hence
we see that the image $d_1 \in D_{m'(m)}$ is zero.
Thus we can choose $b_1 \in B_{m'(m)}$ mapping to
the image $c_1$. Since $c_3$ maps to zero in
$C_{m'(m_0)}$ we find an element $a_{-1} \in A_{m'(m_0)}$
which maps to the image $b_{-1} \in B_{m'(m_0)}$ of $b_1$.
Since $a_{-1}$ maps to zero in $A_{m_0}$ we conclude that
$b_1$ maps to zero in $B_{m_0}$. Thus the image $b_0 \in B_m$
is zero which of course implies $c_0 = 0$ as desired.

\medskip\noindent
We will prove the lemma by induction on $\dim(T)$ which is
finite because $\dim(A)$ is finite. Let $T' \subset T$ be
the set of nonminimal primes in $T$. Then $T'$ is a subset
of $\Spec(A)$ stable under specialization and the hypotheses
of the lemma apply to $T'$. Since $\dim(T') < \dim(T)$ we know
the lemma holds for $T'$. For every $A$-module $M$ there is an
exact sequence
$$
0 \to \colim_{Z, f} H^1_f(H^{i - 1}_Z(M)) \to
H^i_{T'}(M) \to H^i_T(M) \to
\bigoplus\nolimits_{\mathfrak p \in T \setminus T'}
H^i_{\mathfrak p A_\mathfrak p}(M_\mathfrak p)
$$
by Lemma \ref{lemma-filter-local-cohomology}.
Thus we conclude by the initial remark of the proof
and the fact that we've seen the system of groups
$$
\left\{\colim_{Z, f} H^1_f(H^{i - 1}_Z(M_n))\right\}_{n \geq 0}
$$
is pro-zero in Lemma \ref{lemma-zero}; this uses that the function
$m''(m)$ in that lemma for $H^{i - 1}_Z(M)$ is independent of $Z$.
\end{proof}










\section{Finiteness of local cohomology, I}
\label{section-finiteness}

\noindent
We will follow Faltings approach to finiteness of local cohomology
modules, see \cite{Faltings-annulators} and \cite{Faltings-finiteness}.
Here is a lemma which shows that it suffices to prove
local cohomology modules have an annihilator in order to prove that
they are finite modules.

\begin{lemma}
\label{lemma-check-finiteness-local-cohomology-by-annihilator}
\begin{reference}
\cite[Lemma 3]{Faltings-annulators}
\end{reference}
Let $A$ be a Noetherian ring. Let $T \subset \Spec(A)$ be a subset stable
under specialization. Let $M$ be a finite $A$-module. Let $n \geq 0$.
The following are equivalent
\begin{enumerate}
\item $H^i_T(M)$ is finite for $i \leq n$,
\item there exists an ideal $J \subset A$ with $V(J) \subset T$
such that $J$ annihilates $H^i_T(M)$ for $i \leq n$.
\end{enumerate}
If $T = V(I) = Z$ for an ideal $I \subset A$, then these are also
equivalent to
\begin{enumerate}
\item[(3)] there exists an $e \geq 0$ such that $I^e$ annihilates
$H^i_Z(M)$ for $i \leq n$.
\end{enumerate}
\end{lemma}

\begin{proof}
We prove the equivalence of (1) and (2) by induction on $n$.
For $n = 0$ we have $H^0_T(M) \subset M$ is finite. Hence (1) is true.
Since $H^0_T(M) = \colim H^0_{V(J)}(M)$ with $J$ as in (2) we see
that (2) is true. Assume that $n > 0$.

\medskip\noindent
Assume (1) is true. Recall that $H^i_J(M) = H^i_{V(J)}(M)$, see
Dualizing Complexes, Lemma \ref{dualizing-lemma-local-cohomology-noetherian}.
Thus $H^i_T(M) = \colim H^i_J(M)$ where the colimit is over ideals
$J \subset A$ with $V(J) \subset T$, see
Lemma \ref{lemma-adjoint-ext}. Since $H^i_T(M)$ is finitely generated
for $i \leq n$ we can find a $J \subset A$ as in (2) such that
$H^i_J(M) \to H^i_T(M)$ is surjective for $i \leq n$.
Thus the finite list of generators are $J$-power torsion elements
and we see that (2) holds with $J$ replaced by some power.

\medskip\noindent
Assume we have $J$ as in (2). Let $N = H^0_T(M)$ and $M' = M/N$.
By construction of $R\Gamma_T$ we find that
$H^i_T(N) = 0$ for $i > 0$ and $H^0_T(N) = N$, see
Remark \ref{remark-upshot}. Thus we find that
$H^0_T(M') = 0$ and $H^i_T(M') = H^i_T(M)$ for $i > 0$.
We conclude that we may replace $M$ by $M'$.
Thus we may assume that $H^0_T(M) = 0$.
This means that the finite set of associated primes of $M$
are not in $T$. By prime avoidance (Algebra, Lemma \ref{algebra-lemma-silly})
we can find $f \in J$ not contained in any of the associated primes of $M$.
Then the long exact local cohomology sequence associated to the short
exact sequence
$$
0 \to M \to M \to M/fM \to 0
$$
turns into short exact sequences
$$
0 \to H^i_T(M) \to H^i_T(M/fM) \to H^{i + 1}_T(M) \to 0
$$
for $i < n$. We conclude that $J^2$ annihilates $H^i_T(M/fM)$
for $i < n$. By induction hypothesis we see that $H^i_T(M/fM)$
is finite for $i < n$. Using the short exact sequence once more
we see that $H^{i + 1}_T(M)$ is finite for $i < n$ as desired.

\medskip\noindent
We omit the proof of the equivalence of (2) and (3)
in case $T = V(I)$.
\end{proof}

\noindent
The following result of Faltings allows us to prove finiteness
of local cohomology at the level of local rings.

\begin{lemma}
\label{lemma-check-finiteness-local-cohomology-locally}
\begin{reference}
This is a special case of \cite[Satz 1]{Faltings-finiteness}.
\end{reference}
Let $A$ be a Noetherian ring, $I \subset A$ an ideal, $M$ a finite
$A$-module, and $n \geq 0$ an integer. Let $Z = V(I)$.
The following are equivalent
\begin{enumerate}
\item the modules $H^i_Z(M)$ are finite for $i \leq n$, and
\item for all $\mathfrak p \in \Spec(A)$ the modules
$H^i_Z(M)_\mathfrak p$, $i \leq n$ are finite $A_\mathfrak p$-modules.
\end{enumerate}
\end{lemma}

\begin{proof}
The implication (1) $\Rightarrow$ (2) is immediate. We prove the converse
by induction on $n$. The case $n = 0$ is clear because both (1) and
(2) are always true in that case.

\medskip\noindent
Assume $n > 0$ and that (2) is true. Let $N = H^0_Z(M)$ and $M' = M/N$.
By Dualizing Complexes, Lemma \ref{dualizing-lemma-divide-by-torsion}
we may replace $M$ by $M'$.
Thus we may assume that $H^0_Z(M) = 0$.
This means that $\text{depth}_I(M) > 0$
(Dualizing Complexes, Lemma \ref{dualizing-lemma-depth}).
Pick $f \in I$ a nonzerodivisor on $M$ and consider the short
exact sequence
$$
0 \to M \to M \to M/fM \to 0
$$
which produces a long exact sequence
$$
0 \to H^0_Z(M/fM) \to H^1_Z(M) \to H^1_Z(M) \to H^1_Z(M/fM) \to
H^2_Z(M) \to \ldots
$$
and similarly after localization. Thus assumption (2) implies that
the modules $H^i_Z(M/fM)_\mathfrak p$ are finite for $i < n$. Hence
by induction assumption $H^i_Z(M/fM)$ are finite for $i < n$.

\medskip\noindent
Let $\mathfrak p$ be a prime of $A$ which is associated to
$H^i_Z(M)$ for some $i \leq n$. Say $\mathfrak p$ is the annihilator
of the element $x \in H^i_Z(M)$. Then $\mathfrak p \in Z$, hence
$f \in \mathfrak p$. Thus $fx = 0$ and hence $x$ comes from an
element of $H^{i - 1}_Z(M/fM)$ by the boundary map $\delta$ in the long
exact sequence above. It follows that $\mathfrak p$ is an associated
prime of the finite module $\Im(\delta)$. We conclude that
$\text{Ass}(H^i_Z(M))$ is finite for $i \leq n$, see
Algebra, Lemma \ref{algebra-lemma-finite-ass}.

\medskip\noindent
Recall that
$$
H^i_Z(M) \subset
\prod\nolimits_{\mathfrak p \in \text{Ass}(H^i_Z(M))}
H^i_Z(M)_\mathfrak p
$$
by Algebra, Lemma \ref{algebra-lemma-zero-at-ass-zero}. Since by
assumption the modules on the right hand side are finite and $I$-power
torsion, we can find integers $e_{\mathfrak p, i} \geq 0$, $i \leq n$,
$\mathfrak p \in \text{Ass}(H^i_Z(M))$ such that
$I^{e_{\mathfrak p, i}}$ annihilates $H^i_Z(M)_\mathfrak p$. We conclude
that $I^e$ with $e = \max\{e_{\mathfrak p, i}\}$ annihilates $H^i_Z(M)$
for $i \leq n$. By
Lemma \ref{lemma-check-finiteness-local-cohomology-by-annihilator}
we see that $H^i_Z(M)$ is finite for $i \leq n$.
\end{proof}

\begin{lemma}
\label{lemma-annihilate-local-cohomology}
Let $A$ be a ring and let $J \subset I \subset A$ be finitely generated ideals.
Let $i \geq 0$ be an integer. Set $Z = V(I)$. If
$H^i_Z(A)$ is annihilated by $J^n$ for some $n$, then
$H^i_Z(M)$ annihilated by $J^m$ for some $m = m(M)$
for every finitely presented $A$-module $M$ such that
$M_f$ is a finite locally free $A_f$-module for all $f \in I$.
\end{lemma}

\begin{proof}
Consider the annihilator $\mathfrak a$ of $H^i_Z(M)$.
Let $\mathfrak p \subset A$ with $\mathfrak p \not \in Z$.
By assumption there exists an $f \in I$, $f \not \in \mathfrak p$
and an isomorphism $\varphi : A_f^{\oplus r} \to M_f$
of $A_f$-modules. Clearing denominators (and using that
$M$ is of finite presentation) we find maps
$$
a : A^{\oplus r} \longrightarrow M
\quad\text{and}\quad
b : M \longrightarrow A^{\oplus r}
$$
with $a_f = f^N \varphi$ and $b_f = f^N \varphi^{-1}$ for some $N$.
Moreover we may assume that $a \circ b$ and $b \circ a$ are equal to
multiplication by $f^{2N}$. Thus we see that $H^i_Z(M)$ is annihilated by
$f^{2N}J^n$, i.e., $f^{2N}J^n \subset \mathfrak a$.

\medskip\noindent
As $U = \Spec(A) \setminus Z$ is quasi-compact we can find finitely many
$f_1, \ldots, f_t$ and $N_1, \ldots, N_t$ such that $U = \bigcup D(f_j)$ and
$f_j^{2N_j}J^n \subset \mathfrak a$. Then $V(I) = V(f_1, \ldots, f_t)$
and since $I$ is finitely generated we conclude
$I^M \subset (f_1, \ldots, f_t)$ for some $M$.
All in all we see that $J^m \subset \mathfrak a$ for
$m \gg 0$, for example $m = M (2N_1 + \ldots + 2N_t) n$ will do.
\end{proof}

\begin{lemma}
\label{lemma-local-finiteness-for-finite-locally-free}
Let $A$ be a Noetherian ring. Let $I \subset A$ be an ideal. Set $Z = V(I)$.
Let $n \geq 0$ be an integer. If $H^i_Z(A)$ is finite for $0 \leq i \leq n$,
then the same is true for $H^i_Z(M)$, $0 \leq i \leq n$ for
any finite $A$-module $M$ such that $M_f$ is a finite locally free
$A_f$-module for all $f \in I$.
\end{lemma}

\begin{proof}
The assumption that $H^i_Z(A)$ is finite for $0 \leq i \leq n$
implies there exists an $e \geq 0$ such that $I^e$ annihilates
$H^i_Z(A)$ for $0 \leq i \leq n$, see
Lemma \ref{lemma-check-finiteness-local-cohomology-by-annihilator}.
Then Lemma \ref{lemma-annihilate-local-cohomology}
implies that $H^i_Z(M)$, $0 \leq i \leq n$ is annihilated
by $I^m$ for some $m = m(M, i)$. We may take the same $m$
for all $0 \leq i \leq n$. Then
Lemma \ref{lemma-check-finiteness-local-cohomology-by-annihilator}
implies that $H^i_Z(M)$ is finite for $0 \leq i \leq n$
as desired.
\end{proof}





\section{Finiteness of pushforwards, I}
\label{section-finiteness-pushforward}

\noindent
In this section we discuss the easiest nontrivial case of the
finiteness theorem, namely, the finiteness of the first local
cohomology or what is equivalent, finiteness of $j_*\mathcal{F}$
where $j : U \to X$ is an open immersion, $X$ is locally Noetherian, and
$\mathcal{F}$ is a coherent sheaf on $U$. Following a method of Koll\'ar
(\cite{Kollar-variants} and \cite{Kollar-local-global-hulls})
we find a necessary and sufficient condition, see
Proposition \ref{proposition-kollar}. The reader who is interested
in higher direct images or higher local cohomology groups should skip
ahead to Section \ref{section-finiteness-pushforward-II} or
Section \ref{section-finiteness-II} (which are developed
independently of the rest of this section).

\begin{lemma}
\label{lemma-check-finiteness-pushforward-on-associated-points}
Let $X$ be a locally Noetherian scheme. Let $j : U \to X$ be the inclusion
of an open subscheme with complement $Z$. For $x \in U$ let
$i_x : W_x \to U$ be the integral closed subscheme with generic point $x$.
Let $\mathcal{F}$ be a coherent $\mathcal{O}_U$-module.
The following are equivalent
\begin{enumerate}
\item for all $x \in \text{Ass}(\mathcal{F})$ the
$\mathcal{O}_X$-module $j_*i_{x, *}\mathcal{O}_{W_x}$ is coherent,
\item $j_*\mathcal{F}$ is coherent.
\end{enumerate}
\end{lemma}

\begin{proof}
We first prove that (1) implies (2). Assume (1) holds.
The statement is local on $X$, hence we may assume $X$ is affine.
Then $U$ is quasi-compact, hence $\text{Ass}(\mathcal{F})$ is finite
(Divisors, Lemma \ref{divisors-lemma-finite-ass}). Thus we may argue by
induction on the number of associated points. Let $x \in U$ be a generic
point of an irreducible component of the support of $\mathcal{F}$.
By Divisors, Lemma \ref{divisors-lemma-finite-ass} we have
$x \in \text{Ass}(\mathcal{F})$. By our choice of $x$ we have
$\dim(\mathcal{F}_x) = 0$ as $\mathcal{O}_{X, x}$-module.
Hence $\mathcal{F}_x$ has finite length as an $\mathcal{O}_{X, x}$-module
(Algebra, Lemma \ref{algebra-lemma-support-point}).
Thus we may use induction on this length.

\medskip\noindent
Set $\mathcal{G} = j_*i_{x, *}\mathcal{O}_{W_x}$. This is a coherent
$\mathcal{O}_X$-module by assumption. We have $\mathcal{G}_x = \kappa(x)$.
Choose a nonzero map
$\varphi_x : \mathcal{F}_x \to \kappa(x) = \mathcal{G}_x$.
By Cohomology of Schemes, Lemma \ref{coherent-lemma-map-stalks-local-map}
there is an open $x \in V \subset U$ and a map
$\varphi_V : \mathcal{F}|_V \to \mathcal{G}|_V$ whose stalk
at $x$ is $\varphi_x$. Choose $f \in \Gamma(X, \mathcal{O}_X)$
which does not vanish at $x$ such that $D(f) \subset V$. By
Cohomology of Schemes, Lemma \ref{coherent-lemma-homs-over-open}
(for example) we see that $\varphi_V$ extends to
$f^n\mathcal{F} \to \mathcal{G}|_U$ for some $n$.
Precomposing with multiplication by $f^n$ we obtain a map
$\mathcal{F} \to \mathcal{G}|_U$ whose stalk at $x$ is nonzero.
Let $\mathcal{F}' \subset \mathcal{F}$ be the kernel.
Note that $\text{Ass}(\mathcal{F}') \subset \text{Ass}(\mathcal{F})$, see
Divisors, Lemma \ref{divisors-lemma-ses-ass}.
Since
$\text{length}_{\mathcal{O}_{X, x}}(\mathcal{F}'_x) =
\text{length}_{\mathcal{O}_{X, x}}(\mathcal{F}_x) - 1$
we may apply the
induction hypothesis to conclude $j_*\mathcal{F}'$ is coherent.
Since $\mathcal{G} = j_*(\mathcal{G}|_U) = j_*i_{x, *}\mathcal{O}_{W_x}$
is coherent, we can consider the exact sequence
$$
0 \to j_*\mathcal{F}' \to j_*\mathcal{F} \to \mathcal{G}
$$
By Schemes, Lemma \ref{schemes-lemma-push-forward-quasi-coherent}
the sheaf $j_*\mathcal{F}$ is quasi-coherent.
Hence the image of $j_*\mathcal{F}$ in $j_*(\mathcal{G}|_U)$
is coherent by Cohomology of Schemes, Lemma
\ref{coherent-lemma-coherent-Noetherian-quasi-coherent-sub-quotient}.
Finally, $j_*\mathcal{F}$ is coherent by
Cohomology of Schemes, Lemma \ref{coherent-lemma-coherent-abelian-Noetherian}.

\medskip\noindent
Assume (2) holds. Exactly in the same manner as above we reduce
to the case $X$ affine. We pick $x \in \text{Ass}(\mathcal{F})$
and we set $\mathcal{G} = j_*i_{x, *}\mathcal{O}_{W_x}$.
Then we choose a nonzero map
$\varphi_x : \mathcal{G}_x = \kappa(x) \to \mathcal{F}_x$
which exists exactly because $x$ is an associated point of $\mathcal{F}$.
Arguing exactly as above we may assume $\varphi_x$
extends to an $\mathcal{O}_U$-module map
$\varphi : \mathcal{G}|_U \to \mathcal{F}$.
Then $\varphi$ is injective (for example by
Divisors, Lemma \ref{divisors-lemma-check-injective-on-ass})
and we find an injective map
$\mathcal{G} = j_*(\mathcal{G}|_V) \to j_*\mathcal{F}$.
Thus (1) holds.
\end{proof}

\begin{lemma}
\label{lemma-finiteness-pushforwards-and-H1-local}
Let $A$ be a Noetherian ring and let $I \subset A$ be an ideal.
Set $X = \Spec(A)$, $Z = V(I)$, $U = X \setminus Z$, and $j : U \to X$
the inclusion morphism. Let $\mathcal{F}$ be a coherent $\mathcal{O}_U$-module.
Then
\begin{enumerate}
\item there exists a finite $A$-module $M$ such that $\mathcal{F}$ is the
restriction of $\widetilde{M}$ to $U$,
\item given $M$ there is an exact sequence
$$
0 \to H^0_Z(M) \to M \to H^0(U, \mathcal{F}) \to H^1_Z(M) \to 0
$$
and isomorphisms $H^p(U, \mathcal{F}) = H^{p + 1}_Z(M)$ for $p \geq 1$,
\item given $M$ and $p \geq 0$ the following are equivalent
\begin{enumerate}
\item $R^pj_*\mathcal{F}$ is coherent,
\item $H^p(U, \mathcal{F})$ is a finite $A$-module,
\item $H^{p + 1}_Z(M)$ is a finite $A$-module,
\end{enumerate}
\item if the equivalent conditions in (3) hold for $p = 0$, we may take
$M = \Gamma(U, \mathcal{F})$ in which case we have $H^0_Z(M) = H^1_Z(M) = 0$.
\end{enumerate}
\end{lemma}

\begin{proof}
By Properties, Lemma \ref{properties-lemma-lift-finite-presentation}
there exists a coherent $\mathcal{O}_X$-module $\mathcal{F}'$
whose restriction to $U$ is isomorphic to $\mathcal{F}$.
Say $\mathcal{F}'$ corresponds to the finite $A$-module $M$
as in (1).
Note that $R^pj_*\mathcal{F}$ is quasi-coherent
(Cohomology of Schemes, Lemma
\ref{coherent-lemma-quasi-coherence-higher-direct-images})
and corresponds to the $A$-module $H^p(U, \mathcal{F})$.
By Lemma \ref{lemma-local-cohomology-is-local-cohomology}
and the discussion in
Cohomology, Sections \ref{cohomology-section-cohomology-support} and
\ref{cohomology-section-cohomology-support-bis}
we obtain an exact sequence
$$
0 \to H^0_Z(M) \to M \to H^0(U, \mathcal{F}) \to H^1_Z(M) \to 0
$$
and isomorphisms $H^p(U, \mathcal{F}) = H^{p + 1}_Z(M)$ for $p \geq 1$.
Here we use that $H^j(X, \mathcal{F}') = 0$ for $j > 0$ as $X$ is affine
and $\mathcal{F}'$ is quasi-coherent (Cohomology of Schemes,
Lemma \ref{coherent-lemma-quasi-coherent-affine-cohomology-zero}).
This proves (2).
Parts (3) and (4) are straightforward from (2); see also
Lemma \ref{lemma-local-cohomology}.
\end{proof}

\begin{lemma}
\label{lemma-finiteness-pushforward}
Let $X$ be a locally Noetherian scheme.
Let $j : U \to X$ be the inclusion of an
open subscheme with complement $Z$. Let $\mathcal{F}$ be a coherent
$\mathcal{O}_U$-module. Assume
\begin{enumerate}
\item $X$ is Nagata,
\item $X$ is universally catenary, and
\item for $x \in \text{Ass}(\mathcal{F})$ and
$z \in Z \cap \overline{\{x\}}$ we have
$\dim(\mathcal{O}_{\overline{\{x\}}, z}) \geq 2$.
\end{enumerate}
Then $j_*\mathcal{F}$ is coherent.
\end{lemma}

\begin{proof}
By Lemma \ref{lemma-check-finiteness-pushforward-on-associated-points}
it suffices to prove $j_*i_{x, *}\mathcal{O}_{W_x}$ is coherent
for $x \in \text{Ass}(\mathcal{F})$.
Let $\pi : Y \to X$ be the normalization of $X$ in $\Spec(\kappa(x))$, see
Morphisms, Section \ref{morphisms-section-normalization}. By
Morphisms, Lemma \ref{morphisms-lemma-nagata-normalization-finite-general}
the morphism $\pi$ is finite. Since $\pi$ is finite
$\mathcal{G} = \pi_*\mathcal{O}_Y$ is a coherent $\mathcal{O}_X$-module by
Cohomology of Schemes, Lemma \ref{coherent-lemma-finite-pushforward-coherent}.
Observe that $W_x = U \cap \pi(Y)$. Thus
$\pi|_{\pi^{-1}(U)} : \pi^{-1}(U) \to U$ factors through $i_x : W_x \to U$
and we obtain a canonical map
$$
i_{x, *}\mathcal{O}_{W_x}
\longrightarrow
(\pi|_{\pi^{-1}(U)})_*(\mathcal{O}_{\pi^{-1}(U)}) =
(\pi_*\mathcal{O}_Y)|_U = \mathcal{G}|_U
$$
This map is injective (for example by Divisors, Lemma
\ref{divisors-lemma-check-injective-on-ass}). Hence
$j_*i_{x, *}\mathcal{O}_{W_x} \subset j_*\mathcal{G}|_U$
and it suffices to show that $j_*\mathcal{G}|_U$ is coherent.

\medskip\noindent
It remains to prove that $j_*(\mathcal{G}|_U)$ is coherent. We claim
Divisors, Lemma \ref{divisors-lemma-depth-2-hartog}
applies to
$$
\mathcal{G} \longrightarrow j_*(\mathcal{G}|_U)
$$
which finishes the proof. It suffices to show that
$\text{depth}(\mathcal{G}_z) \geq 2$ for $z \in Z$.
Let $y_1, \ldots, y_n \in Y$ be the points mapping to $z$.
By Algebra, Lemma \ref{algebra-lemma-depth-goes-down-finite}
it suffices to show that
$\text{depth}(\mathcal{O}_{Y, y_i}) \geq 2$ for $i = 1, \ldots, n$.
If not, then by Properties, Lemma \ref{properties-lemma-criterion-normal}
we see that $\dim(\mathcal{O}_{Y, y_i}) = 1$ for some $i$.
This is impossible by the dimension formula
(Morphisms, Lemma \ref{morphisms-lemma-dimension-formula})
for $\pi : Y \to \overline{\{x\}}$ and assumption (3).
\end{proof}

\begin{lemma}
\label{lemma-sharp-finiteness-pushforward}
Let $X$ be an integral locally Noetherian scheme. Let $j : U \to X$
be the inclusion of a nonempty open subscheme with complement $Z$. Assume
that for all $z \in Z$ and any associated prime $\mathfrak p$ of
the completion $\mathcal{O}_{X, z}^\wedge$
we have $\dim(\mathcal{O}_{X, z}^\wedge/\mathfrak p) \geq 2$.
Then $j_*\mathcal{O}_U$ is coherent.
\end{lemma}

\begin{proof}
We may assume $X$ is affine.
Using Lemmas \ref{lemma-check-finiteness-local-cohomology-locally} and
\ref{lemma-finiteness-pushforwards-and-H1-local} we reduce to
$X = \Spec(A)$ where $(A, \mathfrak m)$ is a Noetherian local domain
and $\mathfrak m \in Z$.
Then we can use induction on $d = \dim(A)$.
(The base case is $d = 0, 1$ which do not happen by
our assumption on the local rings.)
Set $V = \Spec(A) \setminus \{\mathfrak m\}$.
Observe that the local rings of $V$ have dimension strictly smaller than $d$.
Repeating the arguments for $j' : U \to V$ we
and using induction we conclude that $j'_*\mathcal{O}_U$ is
a coherent $\mathcal{O}_V$-module.
Pick a nonzero $f \in A$ which vanishes on $Z$.
Since $D(f) \cap V \subset U$ we find an $n$ such that
multiplication by $f^n$ on $U$ extends to a map
$f^n : j'_*\mathcal{O}_U \to \mathcal{O}_V$ over $V$
(for example by Cohomology of Schemes, Lemma
\ref{coherent-lemma-homs-over-open}). This map is injective
hence there is an injective map
$$
j_*\mathcal{O}_U = j''_* j'_* \mathcal{O}_U \to j''_*\mathcal{O}_V
$$
on $X$ where $j'' : V \to X$ is the inclusion morphism.
Hence it suffices to show that $j''_*\mathcal{O}_V$ is coherent.
In other words, we may assume that $X$ is the spectrum
of a local Noetherian domain and that $Z$
consists of the closed point.

\medskip\noindent
Assume $X = \Spec(A)$ with $(A, \mathfrak m)$ local and $Z = \{\mathfrak m\}$.
Let $A^\wedge$ be the completion of $A$.
Set $X^\wedge = \Spec(A^\wedge)$, $Z^\wedge = \{\mathfrak m^\wedge\}$,
$U^\wedge = X^\wedge \setminus Z^\wedge$, and
$\mathcal{F}^\wedge = \mathcal{O}_{U^\wedge}$.
The ring $A^\wedge$ is universally catenary and Nagata (Algebra, Remark
\ref{algebra-remark-Noetherian-complete-local-ring-universally-catenary} and
Lemma \ref{algebra-lemma-Noetherian-complete-local-Nagata}).
Moreover, condition (3) of Lemma \ref{lemma-finiteness-pushforward}
for $X^\wedge, Z^\wedge, U^\wedge, \mathcal{F}^\wedge$
holds by assumption! Thus we see that
$(U^\wedge \to X^\wedge)_*\mathcal{O}_{U^\wedge}$
is coherent. Since the morphism $c : X^\wedge \to X$
is flat we conclude that the pullback of $j_*\mathcal{O}_U$ is
$(U^\wedge \to X^\wedge)_*\mathcal{O}_{U^\wedge}$
(Cohomology of Schemes, Lemma
\ref{coherent-lemma-flat-base-change-cohomology}).
Finally, since $c$ is faithfully flat we conclude that
$j_*\mathcal{O}_U$ is coherent by
Descent, Lemma \ref{descent-lemma-finite-type-descends}.
\end{proof}

\begin{remark}
\label{remark-closure}
Let $j : U \to X$ be an open immersion of locally Noetherian schemes.
Let $x \in U$. Let $i_x : W_x \to U$ be the integral closed subscheme
with generic point $x$ and let $\overline{\{x\}}$ be the closure in $X$.
Then we have a commutative diagram
$$
\xymatrix{
W_x \ar[d]_{i_x} \ar[r]_{j'} & \overline{\{x\}} \ar[d]^i \\
U \ar[r]^j & X
}
$$
We have $j_*i_{x, *}\mathcal{O}_{W_x} = i_*j'_*\mathcal{O}_{W_x}$.
As the left vertical arrow is a closed immersion we see that
$j_*i_{x, *}\mathcal{O}_{W_x}$ is coherent if and only if
$j'_*\mathcal{O}_{W_x}$ is coherent.
\end{remark}

\begin{remark}
\label{remark-no-finiteness-pushforward}
Let $X$ be a locally Noetherian scheme. Let $j : U \to X$ be the inclusion of
an open subscheme with complement $Z$. Let $\mathcal{F}$ be a coherent
$\mathcal{O}_U$-module. If there exists an $x \in \text{Ass}(\mathcal{F})$ and
$z \in Z \cap \overline{\{x\}}$ such that
$\dim(\mathcal{O}_{\overline{\{x\}}, z}) \leq 1$, then $j_*\mathcal{F}$ is not
coherent. To prove this we can do a flat base change to the spectrum
of $\mathcal{O}_{X, z}$. Let $X' = \overline{\{x\}}$.
The assumption implies $\mathcal{O}_{X' \cap U} \subset \mathcal{F}$.
Thus it suffices to see that $j_*\mathcal{O}_{X' \cap U}$ is not
coherent. This is clear because $X' = \{x, z\}$, hence
$j_*\mathcal{O}_{X' \cap U}$ corresponds to $\kappa(x)$ as an
$\mathcal{O}_{X, z}$-module which cannot be finite as $x$ is not
a closed point.

\medskip\noindent
In fact, the converse of Lemma \ref{lemma-sharp-finiteness-pushforward}
holds true: given an open immersion $j : U \to X$ of integral Noetherian
schemes and there exists a $z \in X \setminus U$ and an associated prime
$\mathfrak p$ of the completion $\mathcal{O}_{X, z}^\wedge$
with $\dim(\mathcal{O}_{X, z}^\wedge/\mathfrak p) = 1$,
then $j_*\mathcal{O}_U$ is not coherent. Namely, you can pass to
the local ring, you can enlarge $U$ to the punctured spectrum,
you can pass to the completion, and then the argument above gives
the nonfiniteness.
\end{remark}

\begin{proposition}[Koll\'ar]
\label{proposition-kollar}
\begin{reference}
See \cite{k-coherent} and see \cite[IV, Proposition 7.2.2]{EGA}
for a special case.
\end{reference}
\begin{slogan}
Weak analogue of Hartogs' Theorem: On Noetherian schemes, the
restriction of a coherent sheaf to an open set with complement
of codimension 2 in the sheaf's support, is coherent.
\end{slogan}
Let $j : U \to X$ be an open immersion of locally Noetherian schemes
with complement $Z$. Let $\mathcal{F}$ be a coherent $\mathcal{O}_U$-module.
The following are equivalent
\begin{enumerate}
\item $j_*\mathcal{F}$ is coherent,
\item for $x \in \text{Ass}(\mathcal{F})$ and
$z \in Z \cap \overline{\{x\}}$ and any associated prime
$\mathfrak p$ of the completion $\mathcal{O}_{\overline{\{x\}}, z}^\wedge$
we have $\dim(\mathcal{O}_{\overline{\{x\}}, z}^\wedge/\mathfrak p) \geq 2$.
\end{enumerate}
\end{proposition}

\begin{proof}
If (2) holds we get (1) by a combination of
Lemmas \ref{lemma-check-finiteness-pushforward-on-associated-points},
Remark \ref{remark-closure}, and
Lemma \ref{lemma-sharp-finiteness-pushforward}.
If (2) does not hold, then $j_*i_{x, *}\mathcal{O}_{W_x}$ is not finite
for some $x \in \text{Ass}(\mathcal{F})$ by the discussion in
Remark \ref{remark-no-finiteness-pushforward}
(and Remark \ref{remark-closure}).
Thus $j_*\mathcal{F}$ is not coherent by
Lemma \ref{lemma-check-finiteness-pushforward-on-associated-points}.
\end{proof}

\begin{lemma}
\label{lemma-kollar-finiteness-H1-local}
Let $A$ be a Noetherian ring and let $I \subset A$ be an ideal.
Set $Z = V(I)$. Let $M$ be a finite $A$-module. The following
are equivalent
\begin{enumerate}
\item $H^1_Z(M)$ is a finite $A$-module, and
\item for all $\mathfrak p \in \text{Ass}(M)$, $\mathfrak p \not \in Z$
and all $\mathfrak q \in V(\mathfrak p + I)$ the completion of
$(A/\mathfrak p)_\mathfrak q$ does not have associated primes
of dimension $1$.
\end{enumerate}
\end{lemma}

\begin{proof}
Follows immediately from Proposition \ref{proposition-kollar}
via Lemma \ref{lemma-finiteness-pushforwards-and-H1-local}.
\end{proof}

\noindent
The formulation in the following lemma has the advantage that conditions
(1) and (2) are inherited by schemes of finite type over $X$.
Moreover, this is the form of finiteness which we will generalize
to higher direct images in Section \ref{section-finiteness-pushforward-II}.

\begin{lemma}
\label{lemma-finiteness-pushforward-general}
Let $X$ be a locally Noetherian scheme.
Let $j : U \to X$ be the inclusion of an
open subscheme with complement $Z$. Let $\mathcal{F}$ be a coherent
$\mathcal{O}_U$-module. Assume
\begin{enumerate}
\item $X$ is universally catenary,
\item for every $z \in Z$ the formal fibres of $\mathcal{O}_{X, z}$
are $(S_1)$.
\end{enumerate}
In this situation the following are equivalent
\begin{enumerate}
\item[(a)] for $x \in \text{Ass}(\mathcal{F})$ and
$z \in Z \cap \overline{\{x\}}$ we have
$\dim(\mathcal{O}_{\overline{\{x\}}, z}) \geq 2$, and
\item[(b)] $j_*\mathcal{F}$ is coherent.
\end{enumerate}
\end{lemma}

\begin{proof}
Let $x \in \text{Ass}(\mathcal{F})$. By Proposition \ref{proposition-kollar}
it suffices to check that $A = \mathcal{O}_{\overline{\{x\}}, z}$ satisfies
the condition of the proposition on associated primes of its completion
if and only if $\dim(A) \geq 2$.
Observe that $A$ is universally catenary (this is clear)
and that its formal fibres are $(S_1)$ as follows from
More on Algebra, Lemma \ref{more-algebra-lemma-formal-fibres-normal} and
Proposition \ref{more-algebra-proposition-finite-type-over-P-ring}.
Let $\mathfrak p' \subset A^\wedge$ be an associated prime.
As $A \to A^\wedge$ is flat,
by Algebra, Lemma \ref{algebra-lemma-bourbaki},
we find that $\mathfrak p'$ lies over $(0) \subset A$.
The formal fibre $A^\wedge \otimes_A F$ is $(S_1)$ where $F$ is
the fraction field of $A$. We conclude that $\mathfrak p'$ is a
minimal prime, see
Algebra, Lemma \ref{algebra-lemma-criterion-no-embedded-primes}.
Since $A$ is universally catenary it is formally catenary
by More on Algebra, Proposition \ref{more-algebra-proposition-ratliff}.
Hence $\dim(A^\wedge/\mathfrak p') = \dim(A)$ which
proves the equivalence.
\end{proof}






\section{Depth and dimension}
\label{section-dept-dimension}

\noindent
Some helper lemmas.

\begin{lemma}
\label{lemma-ideal-depth-function}
Let $A$ be a Noetherian ring. Let $I \subset A$ be an ideal.
Let $M$ be a finite $A$-module. Let $\mathfrak p \in V(I)$
be a prime ideal. Assume
$e = \text{depth}_{IA_\mathfrak p}(M_\mathfrak p) < \infty$.
Then there exists a nonempty open $U \subset V(\mathfrak p)$
such that $\text{depth}_{IA_\mathfrak q}(M_\mathfrak q) \geq e$
for all $\mathfrak q \in U$.
\end{lemma}

\begin{proof}
By definition of depth we have $IM_\mathfrak p \not = M_\mathfrak p$
and there exists an $M_\mathfrak p$-regular sequence
$f_1, \ldots, f_e \in IA_\mathfrak p$. After replacing $A$ by
a principal localization we may assume $f_1, \ldots, f_e \in I$
form an $M$-regular sequence, see
Algebra, Lemma \ref{algebra-lemma-regular-sequence-in-neighbourhood}.
Consider the module $M' = M/IM$. Since $\mathfrak p \in \text{Supp}(M')$
and since the support of a finite module is closed, we find
$V(\mathfrak p) \subset \text{Supp}(M')$. Thus
for $\mathfrak q \in V(\mathfrak p)$ we get
$IM_\mathfrak q \not = M_\mathfrak q$. Hence, using that
localization is exact, we see that
$\text{depth}_{IA_\mathfrak q}(M_\mathfrak q) \geq e$
for any $\mathfrak q \in V(I)$ by definition of depth.
\end{proof}

\begin{lemma}
\label{lemma-depth-function}
Let $A$ be a Noetherian ring. Let $M$ be a finite $A$-module.
Let $\mathfrak p$ be a prime ideal. Assume
$e = \text{depth}_{A_\mathfrak p}(M_\mathfrak p) < \infty$.
Then there exists a nonempty open $U \subset V(\mathfrak p)$
such that $\text{depth}_{A_\mathfrak q}(M_\mathfrak q) \geq e$
for all $\mathfrak q \in U$ and
for all but finitely many $\mathfrak q \in U$ we have
$\text{depth}_{A_\mathfrak q}(M_\mathfrak q) > e$.
\end{lemma}

\begin{proof}
By definition of depth we have $\mathfrak p M_\mathfrak p \not = M_\mathfrak p$
and there exists an $M_\mathfrak p$-regular sequence
$f_1, \ldots, f_e \in \mathfrak p A_\mathfrak p$. After replacing $A$ by
a principal localization we may assume $f_1, \ldots, f_e \in \mathfrak p$
form an $M$-regular sequence, see
Algebra, Lemma \ref{algebra-lemma-regular-sequence-in-neighbourhood}.
Consider the module $M' = M/(f_1, \ldots, f_e)M$.
Since $\mathfrak p \in \text{Supp}(M')$
and since the support of a finite module is closed, we find
$V(\mathfrak p) \subset \text{Supp}(M')$. Thus
for $\mathfrak q \in V(\mathfrak p)$ we get
$\mathfrak q M_\mathfrak q \not = M_\mathfrak q$. Hence, using that
localization is exact, we see that
$\text{depth}_{A_\mathfrak q}(M_\mathfrak q) \geq e$
for any $\mathfrak q \in V(I)$ by definition of depth.
Moreover, as soon as $\mathfrak q$ is not an associated
prime of the module $M'$, then the depth goes up.
Thus we see that the final statement holds by
Algebra, Lemma \ref{algebra-lemma-finite-ass}.
\end{proof}

\begin{lemma}
\label{lemma-finite-nr-points-next-S}
Let $X$ be a Noetherian scheme with dualizing complex $\omega_X^\bullet$.
Let $\mathcal{F}$ be a coherent $\mathcal{O}_X$-module. Let $k \geq 0$
be an integer. Assume $\mathcal{F}$ is $(S_k)$.
Then there is a finite number of points $x \in X$ such that
$$
\text{depth}(\mathcal{F}_x) = k
\quad\text{and}\quad
\dim(\text{Supp}(\mathcal{F}_x)) > k
$$
\end{lemma}

\begin{proof}
We will prove this lemma by induction on $k$. The base case $k = 0$
says that $\mathcal{F}$ has a finite number of embedded associated points,
which follows from Divisors, Lemma \ref{divisors-lemma-finite-ass}.

\medskip\noindent
Assume $k > 0$ and the result holds for all smaller $k$.
We can cover $X$ by finitely many affine opens, hence we may
assume $X = \Spec(A)$ is affine. Then $\mathcal{F}$ is the
coherent $\mathcal{O}_X$-module associated to a finite $A$-module $M$
which satisfies $(S_k)$. We will use
Algebra, Lemmas \ref{algebra-lemma-one-equation-module} and
\ref{algebra-lemma-depth-drops-by-one}
without further mention.

\medskip\noindent
Let $f \in A$ be a nonzerodivisor on $M$. Then $M/fM$ has $(S_{k - 1})$.
By induction we see that there are finitely many
primes $\mathfrak p \in V(f)$ with
$\text{depth}((M/fM)_\mathfrak p) = k - 1$ and
$\dim(\text{Supp}((M/fM)_\mathfrak p)) > k - 1$.
These are exactly the primes $\mathfrak p \in V(f)$ with
$\text{depth}(M_\mathfrak p) = k$ and
$\dim(\text{Supp}(M_\mathfrak p)) > k$.
Thus we may replace $A$ by $A_f$ and $M$ by $M_f$
in trying to prove the finiteness statement.

\medskip\noindent
Since $M$ satisfies $(S_k)$ and $k > 0$ we see that $M$ has no
embedded associated primes
(Algebra, Lemma \ref{algebra-lemma-criterion-no-embedded-primes}).
Thus $\text{Ass}(M)$ is the set of generic points of the support
of $M$. Thus Dualizing Complexes, Lemma \ref{dualizing-lemma-CM-open}
shows the set
$U = \{\mathfrak q \mid M_\mathfrak q\text{ is Cohen-Macaulay}\}$
is an open containing $\text{Ass}(M)$.
By prime avoidance (Algebra, Lemma \ref{algebra-lemma-silly})
we can pick $f \in A$ with
$f \not \in \mathfrak p$ for $\mathfrak p \in \text{Ass}(M)$
such that $D(f) \subset U$.
Then $f$ is a nonzerodivisor on $M$
(Algebra, Lemma \ref{algebra-lemma-ass-zero-divisors}).
After replacing $A$ by $A_f$ and $M$ by $M_f$ (see above) we
find that $M$ is Cohen-Macaulay.
Thus for all $\mathfrak q \subset A$ we have
$\dim(M_\mathfrak q) = \text{depth}(M_\mathfrak q)$
and hence the set described in the lemma is empty
and a fortiori finite.
\end{proof}

\begin{lemma}
\label{lemma-sitting-in-degrees}
Let $(A, \mathfrak m)$ be a Noetherian local ring with
normalized dualizing complex $\omega_A^\bullet$.
Let $M$ be a finite $A$-module.
Set $E^i = \text{Ext}_A^{-i}(M, \omega_A^\bullet)$.
Then
\begin{enumerate}
\item $E^i$ is a finite $A$-module nonzero only for
$0 \leq i \leq \dim(\text{Supp}(M))$,
\item $\dim(\text{Supp}(E^i)) \leq i$,
\item $\text{depth}(M)$ is the smallest integer $\delta \geq 0$ such that
$E^\delta \not = 0$,
\item $\mathfrak p \in \text{Supp}(E^0 \oplus \ldots \oplus E^i)
\Leftrightarrow
\text{depth}_{A_\mathfrak p}(M_\mathfrak p) + \dim(A/\mathfrak p) \leq i$,
\item the annihilator of $E^i$ is equal to the annihilator
of $H^i_\mathfrak m(M)$.
\end{enumerate}
\end{lemma}

\begin{proof}
Parts (1), (2), and (3) are copies of the statements in
Dualizing Complexes, Lemma \ref{dualizing-lemma-sitting-in-degrees}.
For a prime $\mathfrak p$ of $A$ we have that
$(\omega_A^\bullet)_\mathfrak p[-\dim(A/\mathfrak p)]$
is a normalized dualzing complex for $A_\mathfrak p$.
See Dualizing Complexes, Lemma \ref{dualizing-lemma-dimension-function}.
Thus
$$
E^i_\mathfrak p =
\text{Ext}^{-i}_A(M, \omega_A^\bullet)_\mathfrak p =
\text{Ext}^{-i + \dim(A/\mathfrak p)}_{A_\mathfrak p}
(M_\mathfrak p, (\omega_A^\bullet)_\mathfrak p[-\dim(A/\mathfrak p)])
$$
is zero for
$i - \dim(A/\mathfrak p) < \text{depth}_{A_\mathfrak p}(M_\mathfrak p)$
and nonzero for
$i = \dim(A/\mathfrak p) + \text{depth}_{A_\mathfrak p}(M_\mathfrak p)$
by part (3) over $A_\mathfrak p$.
This proves part (4).
If $E$ is an injective hull of the residue field of $A$, then we have
$$
\Hom_A(H^i_\mathfrak m(M), E) =
\text{Ext}^{-i}_A(M, \omega_A^\bullet)^\wedge =
(E^i)^\wedge =
E^i \otimes_A A^\wedge
$$
by the local duality theorem (in the form of
Dualizing Complexes, Lemma \ref{dualizing-lemma-special-case-local-duality}).
Since $A \to A^\wedge$ is faithfully flat, we find (5) is true by
Matlis duality
(Dualizing Complexes, Proposition \ref{dualizing-proposition-matlis}).
\end{proof}






\section{Annihilators of local cohomology, I}
\label{section-annihilators}

\noindent
This section discusses a result due to Faltings, see
\cite{Faltings-annulators}.

\begin{proposition}
\label{proposition-annihilator}
\begin{reference}
\cite{Faltings-annulators}.
\end{reference}
Let $A$ be a Noetherian ring which has a dualizing complex.
Let $T \subset T' \subset \Spec(A)$ be subsets stable under
specialization. Let $s \geq 0$ an integer. Let $M$ be a finite $A$-module.
The following are equivalent
\begin{enumerate}
\item there exists an ideal $J \subset A$ with $V(J) \subset T'$
such that $J$ annihilates $H^i_T(M)$ for $i \leq s$, and
\item for all $\mathfrak p \not \in T'$,
$\mathfrak q \in T$ with $\mathfrak p \subset \mathfrak q$
we have
$$
\text{depth}_{A_\mathfrak p}(M_\mathfrak p) +
\dim((A/\mathfrak p)_\mathfrak q) > s
$$
\end{enumerate}
\end{proposition}

\begin{proof}
Let $\omega_A^\bullet$ be a dualizing complex. Let $\delta$ be its
dimension function, see Dualizing Complexes, Section
\ref{dualizing-section-dimension-function}.
An important role will be played by the finite $A$-modules
$$
E^i = \Ext_A^i(M, \omega_A^\bullet)
$$
For $\mathfrak p \subset A$ we will write $H^i_\mathfrak p$ to denote the
local cohomology of an $A_\mathfrak p$-module with respect to
$\mathfrak pA_\mathfrak p$. Then we see that
the $\mathfrak pA_\mathfrak p$-adic completion of
$$
(E^i)_\mathfrak p =
\Ext^{\delta(\mathfrak p) + i}_{A_\mathfrak p}(M_\mathfrak p,
(\omega_A^\bullet)_\mathfrak p[-\delta(\mathfrak p)])
$$
is Matlis dual to
$$
H^{-\delta(\mathfrak p) - i}_{\mathfrak p}(M_\mathfrak p)
$$
by
Dualizing Complexes, Lemma \ref{dualizing-lemma-special-case-local-duality}.
In particular we deduce from this the
following fact: an ideal $J \subset A$ annihilates
$(E^i)_\mathfrak p$ if and only if $J$ annihilates
$H^{-\delta(\mathfrak p) - i}_{\mathfrak p}(M_\mathfrak p)$.

\medskip\noindent
Set $T_n = \{\mathfrak p \in T \mid \delta(\mathfrak p) \leq n\}$.
As $\delta$ is a bounded function, we see that
$T_a = \emptyset$ for $a \ll 0$ and $T_b = T$ for $b \gg 0$.

\medskip\noindent
Assume (2). Let us prove the existence of $J$ as in (1).
We will use a double induction to do this. For $i \leq s$
consider the induction hypothesis $IH_i$:
$H^a_T(M)$ is annihilated by some $J \subset A$ with $V(J) \subset T'$
for $0 \leq a \leq i$. The case $IH_0$ is trivial
because $H^0_T(M)$ is a submodule of $M$ and hence finite
and hence is annihilated by some ideal $J$ with $V(J) \subset T$.

\medskip\noindent
Induction step. Assume $IH_{i - 1}$ holds for some $0 < i \leq s$.
Pick $J'$ with $V(J') \subset T'$ annihilating $H^a_T(M)$ for
$0 \leq a \leq i - 1$ (the induction hypothesis guarantees we can
do this). We will show by descending induction on $n$
that there exists an ideal $J$ with $V(J) \subset T'$ such that the
associated primes of $J H^i_T(M)$ are in $T_n$.
For $n \ll 0$ this implies $JH^i_T(M) = 0$ 
(Algebra, Lemma \ref{algebra-lemma-ass-zero})
and hence $IH_i$ will hold.
The base case $n \gg 0$ is trivial because $T = T_n$ in this case
and all associated primes of $H^i_T(M)$ are in $T$.

\medskip\noindent
Thus we assume given $J$ with the property for $n$.
Let $\mathfrak q \in T_n$. Let $T_\mathfrak q \subset \Spec(A_\mathfrak q)$
be the inverse image of $T$. We have
$H^j_T(M)_\mathfrak q = H^j_{T_\mathfrak q}(M_\mathfrak q)$
by Lemma \ref{lemma-torsion-change-rings}.
Consider the spectral sequence
$$
H_\mathfrak q^p(H^q_{T_\mathfrak q}(M_\mathfrak q))
\Rightarrow
H^{p + q}_\mathfrak q(M_\mathfrak q)
$$
of Lemma \ref{lemma-local-cohomology-ss}.
Below we will find an ideal $J'' \subset A$ with $V(J'') \subset T'$
such that $H^i_\mathfrak q(M_\mathfrak q)$ is annihilated by $J''$ for all
$\mathfrak q \in T_n \setminus T_{n - 1}$.
Claim: $J (J')^i J''$ will work for $n - 1$.
Namely, let $\mathfrak q \in T_n \setminus T_{n - 1}$.
The spectral sequence above defines a filtration
$$
E_\infty^{0, i} = E_{i + 2}^{0, i} \subset \ldots \subset E_3^{0, i} \subset
E_2^{0, i} = H^0_\mathfrak q(H^i_{T_\mathfrak q}(M_\mathfrak q))
$$
The module $E_\infty^{0, i}$ is annihilated by $J''$.
The subquotients $E_j^{0, i}/E_{j + 1}^{0, i}$ for $i + 1 \geq j \geq 2$
are annihilated by $J'$ because the target of $d_j^{0, i}$
is a subquotient of
$$
H^j_\mathfrak q(H^{i - j + 1}_{T_\mathfrak q}(M_\mathfrak q)) =
H^j_\mathfrak q(H^{i - j + 1}_T(M)_\mathfrak q)
$$
and $H^{i - j + 1}_T(M)_\mathfrak q$ is annihilated by $J'$ by choice of $J'$.
Finally, by our choice of $J$ we have
$J H^i_T(M)_\mathfrak q \subset H^0_\mathfrak q(H^i_T(M)_\mathfrak q)$
since the non-closed points of $\Spec(A_\mathfrak q)$ have higher
$\delta$ values. Thus $\mathfrak q$ cannot be an associated prime of
$J(J')^iJ'' H^i_T(M)$ as desired.

\medskip\noindent
By our initial remarks we see that $J''$ should annihilate
$$
(E^{-\delta(\mathfrak q) - i})_\mathfrak q =
(E^{-n - i})_\mathfrak q
$$
for all $\mathfrak q \in T_n \setminus T_{n - 1}$.
But if $J''$ works for one $\mathfrak q$, then it works for all
$\mathfrak q$ in an open neighbourhood of $\mathfrak q$
as the modules $E^{-n - i}$ are finite.
Since every subset of $\Spec(A)$ is Noetherian with the induced
topology (Topology, Lemma \ref{topology-lemma-Noetherian}),
we conclude that it suffices
to prove the existence of $J''$ for one $\mathfrak q$.

\medskip\noindent
Since the ext modules are finite the existence of $J''$ is
equivalent to
$$
\text{Supp}(E^{-n - i}) \cap \Spec(A_\mathfrak q) \subset T'.
$$
This is equivalent to showing the localization of $E^{-n - i}$ at every
$\mathfrak p \subset \mathfrak q$, $\mathfrak p \not \in T'$
is zero. Using local duality over $A_\mathfrak p$ we find that we need
to prove that
$$
H^{i + n - \delta(\mathfrak p)}_\mathfrak p(M_\mathfrak p) =
H^{i - \dim((A/\mathfrak p)_\mathfrak q)}_\mathfrak p(M_\mathfrak p)
$$
is zero (this uses that $\delta$ is a dimension function).
This vanishes by the assumption in the lemma and $i \leq s$ and
Dualizing Complexes, Lemma \ref{dualizing-lemma-depth}.

\medskip\noindent
To prove the converse implication we assume (2) does not hold
and we work backwards through the arguments above. First, we pick a
$\mathfrak q \in T$, $\mathfrak p \subset \mathfrak q$
with $\mathfrak p \not \in T'$ such that
$$
i = \text{depth}_{A_\mathfrak p}(M_\mathfrak p) +
\dim((A/\mathfrak p)_\mathfrak q) \leq s
$$
is minimal. Then
$H^{i - \dim((A/\mathfrak p)_\mathfrak q)}_\mathfrak p(M_\mathfrak p)$
is nonzero by the nonvanishing in
Dualizing Complexes, Lemma \ref{dualizing-lemma-depth}.
Set $n = \delta(\mathfrak q)$. Then
there does not exist an ideal $J \subset A$ with $V(J) \subset T'$
such that $J(E^{-n - i})_\mathfrak q = 0$.
Thus $H^i_\mathfrak q(M_\mathfrak q)$ is not
annihilated by an ideal $J \subset A$ with $V(J) \subset T'$.
By minimality of $i$ it follows from the spectral sequence displayed above
that the module $H^i_T(M)_\mathfrak q$
is not annihilated by an ideal $J \subset A$
with $V(J) \subset T'$. Thus $H^i_T(M)$
is not annihilated by an ideal $J \subset A$
with $V(J) \subset T'$. This finishes the proof of the proposition.
\end{proof}

\begin{lemma}
\label{lemma-kill-local-cohomology-at-prime}
Let $I$ be an ideal of a Noetherian ring $A$.
Let $M$ be a finite $A$-module, let $\mathfrak p \subset A$ be a prime
ideal, and let $s \geq 0$ be an integer. Assume
\begin{enumerate}
\item $A$ has a dualizing complex,
\item $\mathfrak p \not \in V(I)$, and
\item for all primes $\mathfrak p' \subset \mathfrak p$
and $\mathfrak q \in V(I)$ with $\mathfrak p' \subset \mathfrak q$ we have
$$
\text{depth}_{A_{\mathfrak p'}}(M_{\mathfrak p'}) +
\dim((A/\mathfrak p')_\mathfrak q) > s
$$
\end{enumerate}
Then there exists an $f \in A$, $f \not \in \mathfrak p$ which annihilates
$H^i_{V(I)}(M)$ for $i \leq s$.
\end{lemma}

\begin{proof}
Consider the sets
$$
T = V(I)
\quad\text{and}\quad
T' = \bigcup\nolimits_{f \in A, f \not \in \mathfrak p} V(f)
$$
These are subsets of $\Spec(A)$ stable under specialization.
Observe that $T \subset T'$ and $\mathfrak p \not \in T'$.
Assumption (3) says that hypothesis (2) of
Proposition \ref{proposition-annihilator} holds.
Hence we can find $J \subset A$ with $V(J) \subset T'$
such that $J H^i_{V(I)}(M) = 0$ for $i \leq s$.
Choose $f \in A$, $f \not \in \mathfrak p$ with $V(J) \subset V(f)$.
A power of $f$ annihilates $H^i_{V(I)}(M)$ for $i \leq s$.
\end{proof}





\section{Finiteness of local cohomology, II}
\label{section-finiteness-II}

\noindent
We continue the discussion of finiteness of local cohomology started in
Section \ref{section-finiteness}. Using
Faltings Annihilator Theorem
we easily prove the following fundamental result.

\begin{proposition}
\label{proposition-finiteness}
\begin{reference}
\cite{Faltings-annulators}.
\end{reference}
Let $A$ be a Noetherian ring which has a dualizing complex.
Let $T \subset \Spec(A)$ be a subset stable under specialization.
Let $s \geq 0$ an integer. Let $M$ be a finite $A$-module.
The following are equivalent
\begin{enumerate}
\item $H^i_T(M)$ is a finite $A$-module for $i \leq s$, and
\item for all $\mathfrak p \not \in T$, $\mathfrak q \in T$ with
$\mathfrak p \subset \mathfrak q$ we have
$$
\text{depth}_{A_\mathfrak p}(M_\mathfrak p) +
\dim((A/\mathfrak p)_\mathfrak q) > s
$$
\end{enumerate}
\end{proposition}

\begin{proof}
Formal consequence of Proposition \ref{proposition-annihilator} and
Lemma \ref{lemma-check-finiteness-local-cohomology-by-annihilator}.
\end{proof}

\noindent
Besides some lemmas for later use, the rest of this section is
concerned with the question to what extend the condition in
Proposition \ref{proposition-finiteness}
that $A$ has a dualizing complex can be weakened. The answer is roughly
that one has to assume the formal fibres of $A$ are $(S_n)$
for sufficiently large $n$.

\medskip\noindent
Let $A$ be a Noetherian ring and let $I \subset A$ be an ideal.
Set $X = \Spec(A)$ and $Z = V(I) \subset X$. Let $M$ be a finite $A$-module.
We define
\begin{equation}
\label{equation-cutoff}
s_{A, I}(M) =
\min \{
\text{depth}_{A_\mathfrak p}(M_\mathfrak p) + \dim((A/\mathfrak p)_\mathfrak q)
\mid
\mathfrak p \in X \setminus Z, \mathfrak q \in Z,
\mathfrak p \subset \mathfrak q
\}
\end{equation}
Our conventions on depth are that the depth of $0$ is $\infty$
thus we only need to consider primes $\mathfrak p$ in the support
of $M$. It will turn out that $s_{A, I}(M)$ is an important invariant of
the situation.

\begin{lemma}
\label{lemma-cutoff}
Let $A \to B$ be a finite homomorphism of Noetherian rings.
Let $I \subset A$ be an ideal and set $J = IB$. Let $M$ be
a finite $B$-module. If $A$ is universally catenary, then
$s_{B, J}(M) = s_{A, I}(M)$.
\end{lemma}

\begin{proof}
Let $\mathfrak p \subset \mathfrak q \subset A$ be primes with
$I \subset \mathfrak q$ and $I \not \subset \mathfrak p$.
Since $A \to B$ is finite there are finitely many primes
$\mathfrak p_i$ lying over $\mathfrak p$. By
Algebra, Lemma \ref{algebra-lemma-depth-goes-down-finite}
we have
$$
\text{depth}(M_\mathfrak p) = \min \text{depth}(M_{\mathfrak p_i})
$$
Let $\mathfrak p_i \subset \mathfrak q_{ij}$ be primes lying
over $\mathfrak q$. By going up for $A \to B$
(Algebra, Lemma \ref{algebra-lemma-integral-going-up})
there is at least one $\mathfrak q_{ij}$ for each $i$.
Then we see that
$$
\dim((B/\mathfrak p_i)_{\mathfrak q_{ij}}) =
\dim((A/\mathfrak p)_\mathfrak q)
$$
by the dimension formula, see
Algebra, Lemma \ref{algebra-lemma-dimension-formula}.
This implies that the minimum of the quantities
used to define $s_{B, J}(M)$
for the pairs $(\mathfrak p_i, \mathfrak q_{ij})$
is equal to the quantity for the pair $(\mathfrak p, \mathfrak q)$.
This proves the lemma.
\end{proof}

\begin{lemma}
\label{lemma-change-completion}
Let $A$ be a Noetherian ring which has a dualizing complex.
Let $I \subset A$ be an ideal.
Let $M$ be a finite $A$-module. Let $A', M'$ be the $I$-adic
completions of $A, M$. Let $\mathfrak p' \subset \mathfrak q'$
be prime ideals of $A'$ with $\mathfrak q' \in V(IA')$
lying over $\mathfrak p \subset \mathfrak q$ in $A$. Then
$$
\text{depth}_{A_{\mathfrak p'}}(M'_{\mathfrak p'})
\geq
\text{depth}_{A_\mathfrak p}(M_\mathfrak p)
$$
and
$$
\text{depth}_{A_{\mathfrak p'}}(M'_{\mathfrak p'}) +
\dim((A'/\mathfrak p')_{\mathfrak q'}) =
\text{depth}_{A_\mathfrak p}(M_\mathfrak p) +
\dim((A/\mathfrak p)_\mathfrak q)
$$
\end{lemma}

\begin{proof}
We have
$$
\text{depth}(M'_{\mathfrak p'}) =
\text{depth}(M_\mathfrak p) +
\text{depth}(A'_{\mathfrak p'}/\mathfrak p A'_{\mathfrak p'})
\geq \text{depth}(M_\mathfrak p)
$$
by flatness of $A \to A'$, see
Algebra, Lemma \ref{algebra-lemma-apply-grothendieck-module}.
Since the fibres of $A \to A'$ are Cohen-Macaulay
(Dualizing Complexes, Lemma
\ref{dualizing-lemma-dualizing-gorenstein-formal-fibres} and
More on Algebra, Section
\ref{more-algebra-section-properties-formal-fibres})
we see that
$\text{depth}(A'_{\mathfrak p'}/\mathfrak p A'_{\mathfrak p'}) =
\dim(A'_{\mathfrak p'}/\mathfrak p A'_{\mathfrak p'})$.
Thus we obtain
\begin{align*}
\text{depth}(M'_{\mathfrak p'}) +
\dim((A'/\mathfrak p')_{\mathfrak q'})
& =
\text{depth}(M_\mathfrak p) +
\dim(A'_{\mathfrak p'}/\mathfrak p A'_{\mathfrak p'}) +
\dim((A'/\mathfrak p')_{\mathfrak q'}) \\
& =
\text{depth}(M_\mathfrak p) +
\dim((A'/\mathfrak p A')_{\mathfrak q'}) \\
& =
\text{depth}(M_\mathfrak p) +
\dim((A/\mathfrak p)_\mathfrak q)
\end{align*}
Second equality because $A'$ is catenary and third equality by
More on Algebra, Lemma \ref{more-algebra-lemma-completion-dimension}
as $(A/\mathfrak p)_\mathfrak q$ and $(A'/\mathfrak p A')_{\mathfrak q'}$
have the same $I$-adic completions.
\end{proof}





\begin{lemma}
\label{lemma-cutoff-completion}
Let $A$ be a universally catenary Noetherian local ring.
Let $I \subset A$ be an ideal. Let $M$ be
a finite $A$-module. Then
$$
s_{A, I}(M) \geq s_{A^\wedge, I^\wedge}(M^\wedge)
$$
If the formal fibres of $A$ are $(S_n)$, then
$\min(n + 1, s_{A, I}(M)) \leq s_{A^\wedge, I^\wedge}(M^\wedge)$.
\end{lemma}

\begin{proof}
Write $X = \Spec(A)$, $X^\wedge = \Spec(A^\wedge)$, $Z = V(I) \subset X$, and
$Z^\wedge = V(I^\wedge)$.
Let $\mathfrak p' \subset \mathfrak q' \subset A^\wedge$
be primes with $\mathfrak p' \not \in Z^\wedge$ and
$\mathfrak q' \in Z^\wedge$. Let $\mathfrak p \subset \mathfrak q$
be the corresponding primes of $A$. Then $\mathfrak p \not \in Z$
and $\mathfrak q \in Z$. Picture
$$
\xymatrix{
\mathfrak p' \ar[r] & \mathfrak q' \ar[r] & A^\wedge \\
\mathfrak p \ar[r] \ar@{-}[u] &
\mathfrak q \ar[r] \ar@{-}[u] & A \ar[u]
}
$$
Let us write
\begin{align*}
a & = \dim(A/\mathfrak p) = \dim(A^\wedge/\mathfrak pA^\wedge),\\
b & = \dim(A/\mathfrak q) = \dim(A^\wedge/\mathfrak qA^\wedge),\\
a' & = \dim(A^\wedge/\mathfrak p'),\\
b' & = \dim(A^\wedge/\mathfrak q')
\end{align*}
Equalities by
More on Algebra, Lemma \ref{more-algebra-lemma-completion-dimension}.
We also write
\begin{align*}
p & = \dim(A^\wedge_{\mathfrak p'}/\mathfrak p A^\wedge_{\mathfrak p'}) =
\dim((A^\wedge/\mathfrak p A^\wedge)_{\mathfrak p'}) \\
q & = \dim(A^\wedge_{\mathfrak q'}/\mathfrak p A^\wedge_{\mathfrak q'}) =
\dim((A^\wedge/\mathfrak q A^\wedge)_{\mathfrak q'})
\end{align*}
Since $A$ is universally catenary we see that
$A^\wedge/\mathfrak pA^\wedge = (A/\mathfrak p)^\wedge$
is equidimensional of dimension $a$
(More on Algebra, Proposition \ref{more-algebra-proposition-ratliff}).
Hence $a = a' + p$. Similarly $b = b' + q$.
By Algebra, Lemma \ref{algebra-lemma-apply-grothendieck-module}
applied to the flat local ring map
$A_\mathfrak p \to A^\wedge_{\mathfrak p'}$
we have
$$
\text{depth}(M^\wedge_{\mathfrak p'})
=
\text{depth}(M_\mathfrak p) +
\text{depth}(A^\wedge_{\mathfrak p'} / \mathfrak p A^\wedge_{\mathfrak p'})
$$
The quantity we are minimizing for $s_{A, I}(M)$ is
$$
s(\mathfrak p, \mathfrak q) =
\text{depth}(M_\mathfrak p) + \dim((A/\mathfrak p)_\mathfrak q) =
\text{depth}(M_\mathfrak p) + a - b
$$
(last equality as $A$ is catenary). The quantity we are minimizing
for $s_{A^\wedge, I^\wedge}(M^\wedge)$
is
$$
s(\mathfrak p', \mathfrak q') =
\text{depth}(M^\wedge_{\mathfrak p'}) +
\dim((A^\wedge/\mathfrak p')_{\mathfrak q'}) =
\text{depth}(M^\wedge_{\mathfrak p'}) + a' - b'
$$
(last equality as $A^\wedge$ is catenary).
Now we have enough notation in place to start the proof.

\medskip\noindent
Let $\mathfrak p \subset \mathfrak q \subset A$ be primes
with $\mathfrak p \not \in Z$ and $\mathfrak q \in Z$ such that
$s_{A, I}(M) = s(\mathfrak p, \mathfrak q)$.
Then we can pick $\mathfrak q'$ minimal over $\mathfrak q A^\wedge$
and $\mathfrak p' \subset \mathfrak q'$ minimal over
$\mathfrak p A^\wedge$ (using going down for $A \to A^\wedge$).
Then we have four primes as above with $p = 0$ and $q = 0$.
Moreover, we have
$\text{depth}(A^\wedge_{\mathfrak p'} / \mathfrak p A^\wedge_{\mathfrak p'})=0$
also because $p = 0$. This means that
$s(\mathfrak p', \mathfrak q') = s(\mathfrak p, \mathfrak q)$.
Thus we get the first inequality.

\medskip\noindent
Assume that the formal fibres of $A$ are $(S_n)$. Then
$\text{depth}(A^\wedge_{\mathfrak p'} / \mathfrak p A^\wedge_{\mathfrak p'})
\geq \min(n, p)$.
Hence
$$
s(\mathfrak p', \mathfrak q') \geq
s(\mathfrak p, \mathfrak q) + q + \min(n, p) - p \geq
s_{A, I}(M) + q + \min(n, p) - p
$$
Thus the only way we can get in trouble is if $p > n$. If this happens
then
\begin{align*}
s(\mathfrak p', \mathfrak q')
& =
\text{depth}(M^\wedge_{\mathfrak p'}) +
\dim((A^\wedge/\mathfrak p')_{\mathfrak q'}) \\
& =
\text{depth}(M_\mathfrak p) +
\text{depth}(A^\wedge_{\mathfrak p'} / \mathfrak p A^\wedge_{\mathfrak p'}) +
\dim((A^\wedge/\mathfrak p')_{\mathfrak q'}) \\
& \geq
0 + n + 1
\end{align*}
because $(A^\wedge/\mathfrak p')_{\mathfrak q'}$ has at least two primes.
This proves the second inequality.
\end{proof}

\noindent
The method of proof of the following lemma works more generally,
but the stronger results one gets will be subsumed in
Theorem \ref{theorem-finiteness} below.

\begin{lemma}
\label{lemma-local-annihilator}
\begin{reference}
This is a special case of
\cite[Satz 1]{Faltings-annulators}.
\end{reference}
Let $A$ be a Gorenstein Noetherian local ring. Let $I \subset A$
be an ideal and set $Z = V(I) \subset \Spec(A)$.
Let $M$ be a finite $A$-module. Let $s = s_{A, I}(M)$ as in
(\ref{equation-cutoff}). Then $H^i_Z(M)$ is finite for $i < s$,
but $H^s_Z(M)$ is not finite.
\end{lemma}

\begin{proof}
Since a Gorenstein local ring has a dualizing complex,
this is a special case of Proposition \ref{proposition-finiteness}.
It would be helpful to have a short proof of this special case,
which will be used in the proof of a general finiteness theorem below.
\end{proof}

\noindent
Observe that the hypotheses of the following theorem are satisfied
by excellent Noetherian rings (by definition),
by Noetherian rings which have a dualizing complex
(Dualizing Complexes, Lemma \ref{dualizing-lemma-universally-catenary} and
Dualizing Complexes, Lemma
\ref{dualizing-lemma-dualizing-gorenstein-formal-fibres}), and
by quotients of regular Noetherian rings.

\begin{theorem}
\label{theorem-finiteness}
\begin{reference}
This is a special case of \cite[Satz 2]{Faltings-finiteness}.
\end{reference}
Let $A$ be a Noetherian ring and let $I \subset A$ be an ideal.
Set $Z = V(I) \subset \Spec(A)$. Let $M$ be a finite $A$-module.
Set $s = s_{A, I}(M)$ as in (\ref{equation-cutoff}).
Assume that
\begin{enumerate}
\item $A$ is universally catenary,
\item the formal fibres of the local rings of $A$ are Cohen-Macaulay.
\end{enumerate}
Then $H^i_Z(M)$ is finite for $0 \leq i < s$ and
$H^s_Z(M)$ is not finite.
\end{theorem}

\begin{proof}
By Lemma \ref{lemma-check-finiteness-local-cohomology-locally}
we may assume that $A$ is a local ring.

\medskip\noindent
If $A$ is a Noetherian complete local ring, then we can write $A$
as the quotient of a regular complete local ring $B$ by
Cohen's structure theorem
(Algebra, Theorem \ref{algebra-theorem-cohen-structure-theorem}).
Using Lemma \ref{lemma-cutoff} and
Dualizing Complexes, Lemma
\ref{dualizing-lemma-local-cohomology-and-restriction}
we reduce to the case
of a regular local ring which is a consequence of
Lemma \ref{lemma-local-annihilator}
because a regular local ring is Gorenstein
(Dualizing Complexes, Lemma \ref{dualizing-lemma-regular-gorenstein}).

\medskip\noindent
Let $A$ be a Noetherian local ring. Let $\mathfrak m$ be the maximal ideal.
We may assume $I \subset \mathfrak m$, otherwise the lemma is trivial.
Let $A^\wedge$ be the completion of $A$, let $Z^\wedge = V(IA^\wedge)$, and
let $M^\wedge = M \otimes_A A^\wedge$ be the completion of $M$
(Algebra, Lemma \ref{algebra-lemma-completion-tensor}).
Then $H^i_Z(M) \otimes_A A^\wedge = H^i_{Z^\wedge}(M^\wedge)$ by
Dualizing Complexes, Lemma \ref{dualizing-lemma-torsion-change-rings}
and flatness of $A \to A^\wedge$
(Algebra, Lemma \ref{algebra-lemma-completion-flat}).
Hence it suffices to show that $H^i_{Z^\wedge}(M^\wedge)$ is
finite for $i < s$ and not finite for $i = s$, see
Algebra, Lemma \ref{algebra-lemma-descend-properties-modules}.
Since we know the result is true for $A^\wedge$ it suffices
to show that $s_{A, I}(M) = s_{A^\wedge, I^\wedge}(M^\wedge)$.
This follows from Lemma \ref{lemma-cutoff-completion}.
\end{proof}

\begin{remark}
\label{remark-astute-reader}
The astute reader will have realized that we can get away with a
slightly weaker condition on the formal fibres of the local rings
of $A$. Namely, in the situation of Theorem \ref{theorem-finiteness}
assume $A$ is universally catenary but make no assumptions on
the formal fibres. Suppose we have an $n$ and we want to prove that
$H^i_Z(M)$ are finite for $i \leq n$. Then the exact same proof
shows that it suffices that $s_{A, I}(M) > n$ and that
the formal fibres of local rings of $A$ are $(S_n)$.
On the other hand, if we want to show that $H^s_Z(M)$
is not finite where $s = s_{A, I}(M)$, then our arguments prove
this if the formal fibres are $(S_{s - 1})$.
\end{remark}







\section{Finiteness of pushforwards, II}
\label{section-finiteness-pushforward-II}

\noindent
This section is the continuation of
Section \ref{section-finiteness-pushforward}.
In this section we reap the fruits of the labor done in
Section \ref{section-finiteness-II}.

\begin{lemma}
\label{lemma-finiteness-Rjstar}
Let $X$ be a locally Noetherian scheme. Let $j : U \to X$ be the inclusion
of an open subscheme with complement $Z$. Let $\mathcal{F}$ be a coherent
$\mathcal{O}_U$-module. Let $n \geq 0$ be an integer. Assume
\begin{enumerate}
\item $X$ is universally catenary,
\item for every $z \in Z$ the formal fibres of
$\mathcal{O}_{X, z}$ are $(S_n)$.
\end{enumerate}
In this situation the following are equivalent
\begin{enumerate}
\item[(a)] for $x \in \text{Supp}(\mathcal{F})$ and
$z \in Z \cap \overline{\{x\}}$ we have
$\text{depth}_{\mathcal{O}_{X, x}}(\mathcal{F}_x) +
\dim(\mathcal{O}_{\overline{\{x\}}, z}) > n$,
\item[(b)] $R^pj_*\mathcal{F}$ is coherent for $0 \leq p < n$.
\end{enumerate}
\end{lemma}

\begin{proof}
The statement is local on $X$, hence we may assume $X$ is affine.
Say $X = \Spec(A)$ and $Z = V(I)$. Let $M$ be a finite $A$-module
whose associated coherent $\mathcal{O}_X$-module restricts
to $\mathcal{F}$ over $U$, see
Lemma \ref{lemma-finiteness-pushforwards-and-H1-local}.
This lemma also tells us that $R^pj_*\mathcal{F}$ is coherent
if and only if $H^{p + 1}_Z(M)$ is a finite $A$-module.
Observe that the minimum of the expressions
$\text{depth}_{\mathcal{O}_{X, x}}(\mathcal{F}_x) +
\dim(\mathcal{O}_{\overline{\{x\}}, z})$
is the number $s_{A, I}(M)$ of (\ref{equation-cutoff}).
Having said this the lemma follows from
Theorem \ref{theorem-finiteness}
as elucidated by Remark \ref{remark-astute-reader}.
\end{proof}

\begin{lemma}
\label{lemma-finiteness-for-finite-locally-free}
Let $X$ be a locally Noetherian scheme. Let $j : U \to X$ be the inclusion
of an open subscheme with complement $Z$. Let $n \geq 0$ be an integer.
If $R^pj_*\mathcal{O}_U$ is coherent for $0 \leq p < n$, then
the same is true for $R^pj_*\mathcal{F}$, $0 \leq p < n$
for any finite locally free $\mathcal{O}_U$-module $\mathcal{F}$.
\end{lemma}

\begin{proof}
The question is local on $X$, hence we may assume $X$ is affine.
Say $X = \Spec(A)$ and $Z = V(I)$. Via
Lemma \ref{lemma-finiteness-pushforwards-and-H1-local}
our lemma follows from
Lemma \ref{lemma-local-finiteness-for-finite-locally-free}.
\end{proof}

\begin{lemma}
\label{lemma-annihilate-Hp}
\begin{reference}
\cite[Lemma 1.9]{Bhatt-local}
\end{reference}
Let $A$ be a ring and let $J \subset I \subset A$ be finitely generated ideals.
Let $p \geq 0$ be an integer. Set $U = \Spec(A) \setminus V(I)$. If
$H^p(U, \mathcal{O}_U)$ is annihilated by $J^n$ for some $n$, then
$H^p(U, \mathcal{F})$ annihilated by $J^m$ for some $m = m(\mathcal{F})$
for every finite locally free $\mathcal{O}_U$-module $\mathcal{F}$.
\end{lemma}

\begin{proof}
Consider the annihilator $\mathfrak a$ of $H^p(U, \mathcal{F})$.
Let $u \in U$. There exists an open neighbourhood $u \in U' \subset U$
and an isomorphism
$\varphi : \mathcal{O}_{U'}^{\oplus r} \to \mathcal{F}|_{U'}$.
Pick $f \in A$ such that $u \in D(f) \subset U'$.
There exist maps
$$
a : \mathcal{O}_U^{\oplus r} \longrightarrow \mathcal{F}
\quad\text{and}\quad
b : \mathcal{F} \longrightarrow \mathcal{O}_U^{\oplus r}
$$
whose restriction to $D(f)$ are equal to $f^N \varphi$
and $f^N \varphi^{-1}$ for some $N$. Moreover we may assume that
$a \circ b$ and $b \circ a$ are equal to multiplication by $f^{2N}$.
This follows from Properties, Lemma
\ref{properties-lemma-section-maps-backwards}
since $U$ is quasi-compact ($I$ is finitely generated), separated, and
$\mathcal{F}$ and $\mathcal{O}_U^{\oplus r}$ are finitely presented.
Thus we see that $H^p(U, \mathcal{F})$ is annihilated by
$f^{2N}J^n$, i.e., $f^{2N}J^n \subset \mathfrak a$.

\medskip\noindent
As $U$ is quasi-compact we can find finitely many $f_1, \ldots, f_t$
and $N_1, \ldots, N_t$ such that $U = \bigcup D(f_i)$ and
$f_i^{2N_i}J^n \subset \mathfrak a$. Then $V(I) = V(f_1, \ldots, f_t)$
and since $I$ is finitely generated we conclude
$I^M \subset (f_1, \ldots, f_t)$ for some $M$.
All in all we see that $J^m \subset \mathfrak a$ for
$m \gg 0$, for example $m = M (2N_1 + \ldots + 2N_t) n$  will do.
\end{proof}











\section{Annihilators of local cohomology, II}
\label{section-annihilators-II}

\noindent
We extend the discussion of annihilators of local cohomology in
Section \ref{section-annihilators}
to bounded below complexes with finite cohomology modules.

\begin{definition}
\label{definition-depth-complex}
Let $I$ be an ideal of a Noetherian ring $A$. Let
$K \in D^+_{\textit{Coh}}(A)$. We define the {\it $I$-depth} of $K$,
denoted $\text{depth}_I(K)$, to be the maximal
$m \in \mathbf{Z} \cup \{\infty\}$ such that $H^i_I(K) = 0$ for all $i < m$.
If $A$ is local with maximal ideal $\mathfrak m$
then we call $\text{depth}_\mathfrak m(K)$ simply the {\it depth} of $K$.
\end{definition}

\noindent
This definition does not conflict with
Algebra, Definition \ref{algebra-definition-depth}
by Dualizing Complexes, Lemma \ref{dualizing-lemma-depth}.

\begin{proposition}
\label{proposition-annihilator-complex}
Let $A$ be a Noetherian ring which has a dualizing complex.
Let $T \subset T' \subset \Spec(A)$ be subsets stable under
specialization. Let $s \in \mathbf{Z}$. Let $K$ be an object of
$D_{\textit{Coh}}^+(A)$. The following are equivalent
\begin{enumerate}
\item there exists an ideal $J \subset A$ with $V(J) \subset T'$
such that $J$ annihilates $H^i_T(K)$ for $i \leq s$, and
\item for all $\mathfrak p \not \in T'$,
$\mathfrak q \in T$ with $\mathfrak p \subset \mathfrak q$
we have
$$
\text{depth}_{A_\mathfrak p}(K_\mathfrak p) +
\dim((A/\mathfrak p)_\mathfrak q) > s
$$
\end{enumerate}
\end{proposition}

\begin{proof}
This lemma is the natural generalization of
Proposition \ref{proposition-annihilator}
whose proof the reader should read first.
Let $\omega_A^\bullet$ be a dualizing complex. Let $\delta$ be its
dimension function, see Dualizing Complexes, Section
\ref{dualizing-section-dimension-function}.
An important role will be played by the finite $A$-modules
$$
E^i = \Ext_A^i(K, \omega_A^\bullet)
$$
For $\mathfrak p \subset A$ we will write $H^i_\mathfrak p$ to denote the
local cohomology of an object of $D(A_\mathfrak p)$ with respect to
$\mathfrak pA_\mathfrak p$. Then we see that
the $\mathfrak pA_\mathfrak p$-adic completion of
$$
(E^i)_\mathfrak p =
\Ext^{\delta(\mathfrak p) + i}_{A_\mathfrak p}(K_\mathfrak p,
(\omega_A^\bullet)_\mathfrak p[-\delta(\mathfrak p)])
$$
is Matlis dual to
$$
H^{-\delta(\mathfrak p) - i}_{\mathfrak p}(K_\mathfrak p)
$$
by
Dualizing Complexes, Lemma \ref{dualizing-lemma-special-case-local-duality}.
In particular we deduce from this the
following fact: an ideal $J \subset A$ annihilates
$(E^i)_\mathfrak p$ if and only if $J$ annihilates
$H^{-\delta(\mathfrak p) - i}_{\mathfrak p}(K_\mathfrak p)$.

\medskip\noindent
Set $T_n = \{\mathfrak p \in T \mid \delta(\mathfrak p) \leq n\}$.
As $\delta$ is a bounded function, we see that
$T_a = \emptyset$ for $a \ll 0$ and $T_b = T$ for $b \gg 0$.

\medskip\noindent
Assume (2). Let us prove the existence of $J$ as in (1).
We will use a double induction to do this. For $i \leq s$
consider the induction hypothesis $IH_i$:
$H^a_T(K)$ is annihilated by some $J \subset A$ with $V(J) \subset T'$
for $a \leq i$. The case $IH_i$ is trivial for $i$ small
enough because $K$ is bounded below.

\medskip\noindent
Induction step. Assume $IH_{i - 1}$ holds for some $i \leq s$.
Pick $J'$ with $V(J') \subset T'$ annihilating $H^a_T(K)$ for
$a \leq i - 1$ (the induction hypothesis guarantees we can
do this). We will show by descending induction on $n$
that there exists an ideal $J$ with $V(J) \subset T'$ such that the
associated primes of $J H^i_T(K)$ are in $T_n$.
For $n \ll 0$ this implies $JH^i_T(K) = 0$ 
(Algebra, Lemma \ref{algebra-lemma-ass-zero})
and hence $IH_i$ will hold.
The base case $n \gg 0$ is trivial because $T = T_n$ in this case
and all associated primes of $H^i_T(K)$ are in $T$.

\medskip\noindent
Thus we assume given $J$ with the property for $n$.
Let $\mathfrak q \in T_n$. Let $T_\mathfrak q \subset \Spec(A_\mathfrak q)$
be the inverse image of $T$. We have
$H^j_T(K)_\mathfrak q = H^j_{T_\mathfrak q}(K_\mathfrak q)$
by Lemma \ref{lemma-torsion-change-rings}.
Consider the spectral sequence
$$
H_\mathfrak q^p(H^q_{T_\mathfrak q}(K_\mathfrak q))
\Rightarrow
H^{p + q}_\mathfrak q(K_\mathfrak q)
$$
of Lemma \ref{lemma-local-cohomology-ss}.
Below we will find an ideal $J'' \subset A$ with $V(J'') \subset T'$
such that $H^i_\mathfrak q(K_\mathfrak q)$ is annihilated by $J''$ for all
$\mathfrak q \in T_n \setminus T_{n - 1}$.
Claim: $J (J')^i J''$ will work for $n - 1$.
Namely, let $\mathfrak q \in T_n \setminus T_{n - 1}$.
The spectral sequence above defines a filtration
$$
E_\infty^{0, i} = E_{i + 2}^{0, i} \subset \ldots \subset E_3^{0, i} \subset
E_2^{0, i} = H^0_\mathfrak q(H^i_{T_\mathfrak q}(K_\mathfrak q))
$$
The module $E_\infty^{0, i}$ is annihilated by $J''$.
The subquotients $E_j^{0, i}/E_{j + 1}^{0, i}$ for $i + 1 \geq j \geq 2$
are annihilated by $J'$ because the target of $d_j^{0, i}$
is a subquotient of
$$
H^j_\mathfrak q(H^{i - j + 1}_{T_\mathfrak q}(K_\mathfrak q)) =
H^j_\mathfrak q(H^{i - j + 1}_T(K)_\mathfrak q)
$$
and $H^{i - j + 1}_T(K)_\mathfrak q$ is annihilated by $J'$ by choice of $J'$.
Finally, by our choice of $J$ we have
$J H^i_T(K)_\mathfrak q \subset H^0_\mathfrak q(H^i_T(K)_\mathfrak q)$
since the non-closed points of $\Spec(A_\mathfrak q)$ have higher
$\delta$ values. Thus $\mathfrak q$ cannot be an associated prime of
$J(J')^iJ'' H^i_T(K)$ as desired.

\medskip\noindent
By our initial remarks we see that $J''$ should annihilate
$$
(E^{-\delta(\mathfrak q) - i})_\mathfrak q =
(E^{-n - i})_\mathfrak q
$$
for all $\mathfrak q \in T_n \setminus T_{n - 1}$.
But if $J''$ works for one $\mathfrak q$, then it works for all
$\mathfrak q$ in an open neighbourhood of $\mathfrak q$
as the modules $E^{-n - i}$ are finite.
Since every subset of $\Spec(A)$ is Noetherian with the induced
topology (Topology, Lemma \ref{topology-lemma-Noetherian}),
we conclude that it suffices
to prove the existence of $J''$ for one $\mathfrak q$.

\medskip\noindent
Since the ext modules are finite the existence of $J''$ is
equivalent to
$$
\text{Supp}(E^{-n - i}) \cap \Spec(A_\mathfrak q) \subset T'.
$$
This is equivalent to showing the localization of $E^{-n - i}$ at every
$\mathfrak p \subset \mathfrak q$, $\mathfrak p \not \in T'$
is zero. Using local duality over $A_\mathfrak p$ we find that we need
to prove that
$$
H^{i + n - \delta(\mathfrak p)}_\mathfrak p(K_\mathfrak p) =
H^{i - \dim((A/\mathfrak p)_\mathfrak q)}_\mathfrak p(K_\mathfrak p)
$$
is zero (this uses that $\delta$ is a dimension function).
This vanishes by the assumption in the lemma and $i \leq s$ and
our definition of depth in Definition \ref{definition-depth-complex}.

\medskip\noindent
To prove the converse implication we assume (2) does not hold
and we work backwards through the arguments above. First, we pick a
$\mathfrak q \in T$, $\mathfrak p \subset \mathfrak q$
with $\mathfrak p \not \in T'$ such that
$$
i = \text{depth}_{A_\mathfrak p}(K_\mathfrak p) +
\dim((A/\mathfrak p)_\mathfrak q) \leq s
$$
is minimal. Then
$H^{i - \dim((A/\mathfrak p)_\mathfrak q)}_\mathfrak p(K_\mathfrak p)$
is nonzero by the our definition of depth in
Definition \ref{definition-depth-complex}.
Set $n = \delta(\mathfrak q)$. Then
there does not exist an ideal $J \subset A$ with $V(J) \subset T'$
such that $J(E^{-n - i})_\mathfrak q = 0$.
Thus $H^i_\mathfrak q(K_\mathfrak q)$ is not
annihilated by an ideal $J \subset A$ with $V(J) \subset T'$.
By minimality of $i$ it follows from the spectral sequence displayed above
that the module $H^i_T(K)_\mathfrak q$
is not annihilated by an ideal $J \subset A$
with $V(J) \subset T'$. Thus $H^i_T(K)$
is not annihilated by an ideal $J \subset A$
with $V(J) \subset T'$. This finishes the proof of the proposition.
\end{proof}







\section{Finiteness of local cohomology, III}
\label{section-finiteness-III}

\noindent
We extend the discussion of finiteness of local cohomology
in Sections \ref{section-finiteness} and \ref{section-finiteness-II}
to bounded below complexes with finite cohomology modules.

\begin{lemma}
\label{lemma-check-finiteness-local-cohomology-by-annihilator-complex}
Let $A$ be a Noetherian ring. Let $T \subset \Spec(A)$ be a subset stable
under specialization. Let $K$ be an object of $D_{\textit{Coh}}^+(A)$.
Let $n \in \mathbf{Z}$. The following are equivalent
\begin{enumerate}
\item $H^i_T(K)$ is finite for $i \leq n$,
\item there exists an ideal $J \subset A$ with $V(J) \subset T$
such that $J$ annihilates $H^i_T(K)$ for $i \leq n$.
\end{enumerate}
If $T = V(I) = Z$ for an ideal $I \subset A$, then these are also
equivalent to
\begin{enumerate}
\item[(3)] there exists an $e \geq 0$ such that $I^e$ annihilates
$H^i_Z(K)$ for $i \leq n$.
\end{enumerate}
\end{lemma}

\begin{proof}
This lemma is the natural generalization of
Lemma \ref{lemma-check-finiteness-local-cohomology-by-annihilator}
whose proof the reader should read first.
Assume (1) is true. Recall that $H^i_J(K) = H^i_{V(J)}(K)$, see
Dualizing Complexes, Lemma \ref{dualizing-lemma-local-cohomology-noetherian}.
Thus $H^i_T(K) = \colim H^i_J(K)$ where the colimit is over ideals
$J \subset A$ with $V(J) \subset T$, see
Lemma \ref{lemma-adjoint-ext}. Since $H^i_T(K)$ is finitely generated
for $i \leq n$ we can find a $J \subset A$ as in (2) such that
$H^i_J(K) \to H^i_T(K)$ is surjective for $i \leq n$.
Thus the finite list of generators are $J$-power torsion elements
and we see that (2) holds with $J$ replaced by some power.

\medskip\noindent
Let $a \in \mathbf{Z}$ be an integer such that $H^i(K) = 0$ for $i < a$.
We prove (2) $\Rightarrow$ (1) by descending induction on $a$.
If $a > n$, then we have $H^i_T(K) = 0$ for $i \leq n$ hence both
(1) and (2) are true and there is nothing to prove.

\medskip\noindent
Assume we have $J$ as in (2). Observe that $N = H^a_T(K) = H^0_T(H^a(K))$
is finite as a submodule of the finite $A$-module $H^a(K)$.
If $n = a$ we are done; so assume $a < n$ from now on. By construction of
$R\Gamma_T$ we find that $H^i_T(N) = 0$ for $i > 0$ and $H^0_T(N) = N$, see
Remark \ref{remark-upshot}. Choose a distinguished triangle
$$
N[-a] \to K \to K' \to N[-a + 1]
$$
Then we see that $H^a_T(K') = 0$ and $H^i_T(K) = H^i_T(K')$ for $i > a$.
We conclude that we may replace $K$ by $K'$. Thus we may assume that
$H^a_T(K) = 0$. This means that the finite set of associated primes of
$H^a(K)$ are not in $T$. By prime avoidance
(Algebra, Lemma \ref{algebra-lemma-silly}) we can find $f \in J$
not contained in any of the associated primes of $H^a(K)$.
Choose a distinguished triangle
$$
L \to K \xrightarrow{f} K \to L[1]
$$
By construction we see that $H^i(L) = 0$ for $i \leq a$.
On the other hand we have a long exact cohomology sequence
$$
0 \to H^{a + 1}_T(L) \to H^{a + 1}_T(K) \xrightarrow{f}
H^{a + 1}_T(K) \to H^{a + 2}_T(L) \to H^{a + 2}_T(K) \xrightarrow{f} \ldots
$$
which breaks into the identification $H^{a + 1}_T(L) = H^{a + 1}_T(K)$
and short exact sequences
$$
0 \to H^{i - 1}_T(K) \to H^i_T(L) \to H^i_T(K) \to 0
$$
for $i \leq n$ since $f \in J$.
We conclude that $J^2$ annihilates $H^i_T(L)$ for $i \leq n$.
By induction hypothesis applied to $L$ we see that $H^i_T(L)$
is finite for $i \leq n$. Using the short exact sequence once more
we see that $H^i_T(K)$ is finite for $i \leq n$ as desired.

\medskip\noindent
We omit the proof of the equivalence of (2) and (3)
in case $T = V(I)$.
\end{proof}

\begin{proposition}
\label{proposition-finiteness-complex}
Let $A$ be a Noetherian ring which has a dualizing complex.
Let $T \subset \Spec(A)$ be a subset stable under specialization.
Let $s \in \mathbf{Z}$. Let $K \in D_{\textit{Coh}}^+(A)$.
The following are equivalent
\begin{enumerate}
\item $H^i_T(K)$ is a finite $A$-module for $i \leq s$, and
\item for all $\mathfrak p \not \in T$, $\mathfrak q \in T$ with
$\mathfrak p \subset \mathfrak q$ we have
$$
\text{depth}_{A_\mathfrak p}(K_\mathfrak p) +
\dim((A/\mathfrak p)_\mathfrak q) > s
$$
\end{enumerate}
\end{proposition}

\begin{proof}
Formal consequence of
Proposition \ref{proposition-annihilator-complex} and
Lemma \ref{lemma-check-finiteness-local-cohomology-by-annihilator-complex}.
\end{proof}






\section{Improving coherent modules}
\label{section-improve}

\noindent
Similar constructions can be found in \cite{EGA} and more recently in
\cite{Kollar-local-global-hulls} and \cite{Kollar-variants}.

\begin{lemma}
\label{lemma-get-depth-1-along-Z}
Let $X$ be a Noetherian scheme. Let $T \subset X$ be a subset
stable under specialization. Let $\mathcal{F}$ be a coherent
$\mathcal{O}_X$-module. Then there is a unique map
$\mathcal{F} \to \mathcal{F}'$ of coherent $\mathcal{O}_X$-modules
such that
\begin{enumerate}
\item $\mathcal{F} \to \mathcal{F}'$ is surjective,
\item $\mathcal{F}_x \to \mathcal{F}'_x$ is an isomorphism for $x \not \in T$,
\item $\text{depth}_{\mathcal{O}_{X, x}}(\mathcal{F}'_x) \geq 1$ for $x \in T$.
\end{enumerate}
If $f : Y \to X$ is a flat morphism with $Y$ Noetherian, then
$f^*\mathcal{F} \to f^*\mathcal{F}'$ is the corresponding
quotient for $f^{-1}(T) \subset Y$ and $f^*\mathcal{F}$.
\end{lemma}

\begin{proof}
Condition (3) just means that $\text{Ass}(\mathcal{F}') \cap T = \emptyset$.
Thus $\mathcal{F} \to \mathcal{F}'$ is the quotient of $\mathcal{F}$
by the subsheaf of sections whose support is contained in $T$.
This proves uniqueness. The statement on pullbacks follows from
Divisors, Lemma \ref{divisors-lemma-bourbaki}
and the uniqueness.

\medskip\noindent
Existence of $\mathcal{F} \to \mathcal{F}'$.
By the uniqueness it suffices to prove the
existence and uniqueness locally on $X$; small detail omitted.
Thus we may assume $X = \Spec(A)$ is affine and $\mathcal{F}$
is the coherent module associated to the finite $A$-module $M$.
Set $M' = M / H^0_T(M)$ with $H^0_T(M)$ as in Section \ref{section-supports}.
Then $M_\mathfrak p = M'_\mathfrak p$ for $\mathfrak p \not \in T$
which proves (1). On the other hand, we have
$H^0_T(M) = \colim H^0_Z(M)$ where $Z$ runs over the closed
subsets of $X$ contained in $T$. Thus by
Dualizing Complexes, Lemmas \ref{dualizing-lemma-divide-by-torsion}
we have $H^0_T(M') = 0$, i.e., no associated prime
of $M'$ is in $T$. Therefore $\text{depth}(M'_\mathfrak p) \geq 1$
for $\mathfrak p \in T$.
\end{proof}

\begin{lemma}
\label{lemma-get-depth-2-along-Z}
Let $j : U \to X$ be an open immersion of Noetherian schemes.
Let $\mathcal{F}$ be a coherent $\mathcal{O}_X$-module.
Assume $\mathcal{F}' = j_*(\mathcal{F}|_U)$ is coherent.
Then $\mathcal{F} \to \mathcal{F}'$ is the unique map
of coherent $\mathcal{O}_X$-modules such that
\begin{enumerate}
\item $\mathcal{F}|_U \to \mathcal{F}'|_U$
is an isomorphism,
\item $\text{depth}_{\mathcal{O}_{X, x}}(\mathcal{F}'_x) \geq 2$
for $x \in X$, $x \not \in U$.
\end{enumerate}
If $f : Y \to X$ is a flat morphism with $Y$ Noetherian, then
$f^*\mathcal{F} \to f^*\mathcal{F}'$ is the corresponding
map for $f^{-1}(U) \subset Y$.
\end{lemma}

\begin{proof}
We have $\text{depth}_{\mathcal{O}_{X, x}}(\mathcal{F}'_x) \geq 2$
by Divisors, Lemma \ref{divisors-lemma-depth-pushforward} part (3).
The uniqueness of $\mathcal{F} \to \mathcal{F}'$ follows from
Divisors, Lemma \ref{divisors-lemma-depth-2-hartog}.
The compatibility with flat pullbacks follows from
flat base change, see Cohomology of Schemes, Lemma
\ref{coherent-lemma-flat-base-change-cohomology}.
\end{proof}

\begin{lemma}
\label{lemma-make-S2-along-Z}
Let $X$ be a Noetherian scheme. Let $Z \subset X$ be a closed subscheme.
Let $\mathcal{F}$ be a coherent $\mathcal{O}_X$-module. Assume
$X$ is universally catenary and the formal fibres of local rings have $(S_1)$.
Then there exists a unique map $\mathcal{F} \to \mathcal{F}''$
of coherent $\mathcal{O}_X$-modules such that
\begin{enumerate}
\item $\mathcal{F}_x \to \mathcal{F}''_x$
is an isomorphism for $x \in X \setminus Z$,
\item $\mathcal{F}_x \to \mathcal{F}''_x$ is surjective and
$\text{depth}_{\mathcal{O}_{X, x}}(\mathcal{F}''_x) = 1$
for $x \in Z$ such that there exists an immediate specialization
$x' \leadsto x$ with $x' \not \in Z$ and $x' \in \text{Ass}(\mathcal{F})$,
\item $\text{depth}_{\mathcal{O}_{X, x}}(\mathcal{F}''_x) \geq 2$
for the remaining $x \in Z$.
\end{enumerate}
If $f : Y \to X$ is a Cohen-Macaulay morphism with $Y$ Noetherian,
then $f^*\mathcal{F} \to f^*\mathcal{F}''$ satisfies the same properties
with respect to $f^{-1}(Z) \subset Y$.
\end{lemma}

\begin{proof}
Let $\mathcal{F} \to \mathcal{F}'$ be the map constructed in
Lemma \ref{lemma-get-depth-1-along-Z} for the subset $Z$ of $X$.
Recall that $\mathcal{F}'$ is the quotient of $\mathcal{F}$
by the subsheaf of sections supported on $Z$.

\medskip\noindent
We first prove uniqueness. Let $\mathcal{F} \to \mathcal{F}''$
be as in the lemma. We get a factorization
$\mathcal{F} \to \mathcal{F}' \to \mathcal{F}''$
since $\text{Ass}(\mathcal{F}'') \cap Z = \emptyset$
by conditions (2) and (3). Let $U \subset X$ be a maximal open
subscheme such that $\mathcal{F}'|_U \to \mathcal{F}''|_U$
is an isomorphism. We see that $U$ contains all the points
as in (2). Then by Divisors, Lemma \ref{divisors-lemma-depth-2-hartog}
we conclude that $\mathcal{F}'' = j_*(\mathcal{F}'|_U)$.
In this way we get uniqueness (small detail: if we have two
of these $\mathcal{F}''$ then we take the intersection of the opens $U$
we get from either).

\medskip\noindent
Proof of existence. Recall that
$\text{Ass}(\mathcal{F}') = \{x_1, \ldots, x_n\}$
is finite and $x_i \not \in Z$.
Let $Y_i$ be the closure of $\{x_i\}$. Let
$Z_{i, j}$ be the irreducible components of $Z \cap Y_i$.
Observe that $\text{Supp}(\mathcal{F}') \cap Z = \bigcup Z_{i, j}$.
Let $z_{i, j} \in Z_{i, j}$ be the generic point.
Let
$$
d_{i, j} = \dim(\mathcal{O}_{\overline{\{x_i\}}, z_{i, j}})
$$
If $d_{i, j} = 1$, then $z_{i, j}$ is one of the points as in (2).
Thus we do not need to modify $\mathcal{F}'$ at these points.
Furthermore, still assuming $d_{i, j} = 1$, using
Lemma \ref{lemma-depth-function}
we can find an open neighbourhood
$z_{i, j} \in V_{i, j} \subset X$ such that
$\text{depth}_{\mathcal{O}_{X, z}}(\mathcal{F}'_z) \geq 2$
for $z \in Z_{i, j} \cap V_{i, j}$, $z \not = z_{i, j}$.
Set
$$
Z' = X \setminus
\left(
X \setminus Z \cup \bigcup\nolimits_{d_{i, j} = 1} V_{i, j})
\right)
$$
Denote $j' : X \setminus Z' \to X$. By our choice of $Z'$
the assumptions of Lemma \ref{lemma-finiteness-pushforward-general}
are satisfied.
We conclude by setting $\mathcal{F}'' = j'_*(\mathcal{F}'|_{X \setminus Z'})$
and applying Lemma \ref{lemma-get-depth-2-along-Z}.

\medskip\noindent
The final statement follows from the formula for the change in
depth along a flat local homomorphism, see
Algebra, Lemma \ref{algebra-lemma-apply-grothendieck-module}
and the assumption on the fibres of $f$ inherent in $f$ being
Cohen-Macaulay. Details omitted.
\end{proof}

\begin{lemma}
\label{lemma-make-S2-along-T-simple}
Let $X$ be a Noetherian scheme which locally has a dualizing complex.
Let $T' \subset X$ be a subset stable under specialization.
Let $\mathcal{F}$ be a coherent $\mathcal{O}_X$-module.
Assume that if $x \leadsto x'$ is an immediate specialization
of points in $X$ with $x' \in T'$ and $x \not \in T'$, then
$\text{depth}(\mathcal{F}_x) \geq 1$.
Then there exists a unique map $\mathcal{F} \to \mathcal{F}''$
of coherent $\mathcal{O}_X$-modules such that
\begin{enumerate}
\item $\mathcal{F}_x \to \mathcal{F}''_x$ is an isomorphism
for $x \not \in T'$,
\item $\text{depth}_{\mathcal{O}_{X, x}}(\mathcal{F}''_x) \geq 2$
for $x \in T'$.
\end{enumerate}
If $f : Y \to X$ is a Cohen-Macaulay morphism with $Y$ Noetherian,
then $f^*\mathcal{F} \to f^*\mathcal{F}''$ satisfies the same properties
with respect to $f^{-1}(T') \subset Y$.
\end{lemma}

\begin{proof}
Let $\mathcal{F} \to \mathcal{F}'$ be the quotient of $\mathcal{F}$
constructed in Lemma \ref{lemma-get-depth-1-along-Z} using $T'$.
Recall that $\mathcal{F}'$ is the quotient of $\mathcal{F}$
by the subsheaf of sections supported on $T'$.

\medskip\noindent
Proof of uniqueness. Let $\mathcal{F} \to \mathcal{F}''$
be as in the lemma. We get a factorization
$\mathcal{F} \to \mathcal{F}' \to \mathcal{F}''$
since $\text{Ass}(\mathcal{F}'') \cap T' = \emptyset$
by condition (2). Let $U \subset X$ be a maximal open
subscheme such that $\mathcal{F}'|_U \to \mathcal{F}''|_U$
is an isomorphism. We see that $U$ contains all the points of $T'$.
Then by Divisors, Lemma \ref{divisors-lemma-depth-2-hartog}
we conclude that $\mathcal{F}'' = j_*(\mathcal{F}'|_U)$.
In this way we get uniqueness (small detail: if we have two
of these $\mathcal{F}''$ then we take the intersection of the opens $U$
we get from either).

\medskip\noindent
Proof of existence. We will define
$$
\mathcal{F}'' = \colim j_*(\mathcal{F}'|_V)
$$
where $j : V \to X$ runs over the open subschemes such that
$X \setminus V \subset T'$. Observe that the colimit is filtered
as $T'$ is stable under specialization. Each of the
maps $\mathcal{F}' \to j_*(\mathcal{F}'|_V)$ is injective
as $\text{Ass}(\mathcal{F}')$ is disjoint from $T'$.
Thus $\mathcal{F}' \to \mathcal{F}''$ is injective.

\medskip\noindent
Suppose $X = \Spec(A)$ is affine and $\mathcal{F}$
corresponds to the finite $A$-module $M$. Then $\mathcal{F}'$
corresponds to $M' = M / H^0_{T'}(M)$, see proof of
Lemma \ref{lemma-get-depth-1-along-Z}. Applying
Lemmas \ref{lemma-local-cohomology} and \ref{lemma-adjoint-ext}
we see that $\mathcal{F}''$ corresponds to an $A$-module
$M''$ which fits into the short exact sequence
$$
0 \to M' \to M'' \to H^1_{T'}(M') \to 0
$$
By Proposition \ref{proposition-finiteness} and our condition
on immediate specializations in the statement of the lemma
we see that $M''$ is a finite $A$-module. In this way
we see that $\mathcal{F}''$ is coherent.

\medskip\noindent
The final statement follows from the formula for the change in
depth along a flat local homomorphism, see
Algebra, Lemma \ref{algebra-lemma-apply-grothendieck-module}
and the assumption on the fibres of $f$ inherent in $f$ being
Cohen-Macaulay. Details omitted.
\end{proof}

\begin{lemma}
\label{lemma-make-S2-along-T}
Let $X$ be a Noetherian scheme which locally has a dualizing complex.
Let $T' \subset T \subset X$ be subsets stable under specialization
such that if $x \leadsto x'$ is an immediate specialization
of points in $X$ and $x' \in T'$, then $x \in T$. Let $\mathcal{F}$
be a coherent $\mathcal{O}_X$-module.
Then there exists a unique map $\mathcal{F} \to \mathcal{F}''$
of coherent $\mathcal{O}_X$-modules such that
\begin{enumerate}
\item $\mathcal{F}_x \to \mathcal{F}''_x$ is an isomorphism
for $x \not \in T$,
\item $\mathcal{F}_x \to \mathcal{F}''_x$ is surjective
and $\text{depth}_{\mathcal{O}_{X, x}}(\mathcal{F}''_x) \geq 1$
for $x \in T$, $x \not \in T'$, and
\item $\text{depth}_{\mathcal{O}_{X, x}}(\mathcal{F}''_x) \geq 2$
for $x \in T'$.
\end{enumerate}
If $f : Y \to X$ is a Cohen-Macaulay morphism with $Y$ Noetherian,
then $f^*\mathcal{F} \to f^*\mathcal{F}''$ satisfies the same properties
with respect to $f^{-1}(T') \subset f^{-1}(T) \subset Y$.
\end{lemma}

\begin{proof}
First, let $\mathcal{F} \to \mathcal{F}'$ be the quotient of $\mathcal{F}$
constructed in Lemma \ref{lemma-get-depth-1-along-Z} using $T$.
Second, let $\mathcal{F}' \to \mathcal{F}''$ be the unique
map of coherent modules construction in
Lemma \ref{lemma-make-S2-along-T-simple} using $T'$.
Then $\mathcal{F} \to \mathcal{F}''$ is as desired.
\end{proof}










\section{Hartshorne-Lichtenbaum vanishing}
\label{section-Hartshorne-Lichtenbaum-vanishing}

\noindent
This vanishing result is the local analogue of
Lichtenbaum's theorem that the reader can find
in Duality for Schemes, Section \ref{duality-section-lichtenbaum}.
This and much else besides can be found in \cite{CD}.

\begin{lemma}
\label{lemma-cd-top-vanishing}
Let $A$ be a Noetherian ring of dimension $d$. Let $I \subset I' \subset A$
be ideals. If $I'$ is contained in the Jacobson radical
of $A$ and $\text{cd}(A, I') < d$, then $\text{cd}(A, I) < d$.
\end{lemma}

\begin{proof}
By Lemma \ref{lemma-cd-dimension} we know $\text{cd}(A, I) \leq d$.
We will use Lemma \ref{lemma-isomorphism} to show
$$
H^d_{V(I')}(A) \to H^d_{V(I)}(A)
$$
is surjective which will finish the proof. Pick
$\mathfrak p \in V(I) \setminus V(I')$. By our assumption
on $I'$ we see that $\mathfrak p$ is not a maximal ideal of $A$.
Hence $\dim(A_\mathfrak p) < d$. Then
$H^d_{\mathfrak pA_\mathfrak p}(A_\mathfrak p) = 0$
by Lemma \ref{lemma-cd-dimension}.
\end{proof}

\begin{lemma}
\label{lemma-cd-top-vanishing-some-module}
Let $A$ be a Noetherian ring of dimension $d$. Let $I \subset A$
be an ideal. If $H^d_{V(I)}(M) = 0$ for some finite $A$-module
whose support contains all the irreducible components of
dimension $d$, then $\text{cd}(A, I) < d$.
\end{lemma}

\begin{proof}
By Lemma \ref{lemma-cd-dimension} we know $\text{cd}(A, I) \leq d$.
Thus for any finite $A$-module $N$ we have $H^i_{V(I)}(N) = 0$
for $i > d$. Let us say property $\mathcal{P}$ holds for the
finite $A$-module $N$ if $H^d_{V(I)}(N) = 0$.
One of our assumptions is that $\mathcal{P}(M)$ holds.
Observe that $\mathcal{P}(N_1 \oplus N_2)
\Leftrightarrow (\mathcal{P}(N_1) \wedge \mathcal{P}(N_2))$.
Observe that if $N \to N'$ is surjective, then
$\mathcal{P}(N) \Rightarrow \mathcal{P}(N')$ as we
have the vanishing of $H^{d + 1}_{V(I)}$ (see above).
Let $\mathfrak p_1, \ldots, \mathfrak p_n$ be the
minimal primes of $A$ with $\dim(A/\mathfrak p_i) = d$.
Observe that $\mathcal{P}(N)$ holds if the support
of $N$ is disjoint from $\{\mathfrak p_1, \ldots, \mathfrak p_n\}$
for dimension reasons, see Lemma \ref{lemma-cd-dimension}.
For each $i$ set $M_i = M/\mathfrak p_i M$.
This is a finite $A$-module annihilated by $\mathfrak p_i$
whose support is equal to
$V(\mathfrak p_i)$ (here we use the assumption on the support of $M$).
Finally, if $J \subset A$ is an ideal, then we have $\mathcal{P}(JM_i)$
as $JM_i$ is a quotient of a direct sum of copies of $M$.
Thus it follows from Cohomology of Schemes, Lemma
\ref{coherent-lemma-property-higher-rank-cohomological}
that $\mathcal{P}$ holds for every finite $A$-module.
\end{proof}

\begin{lemma}
\label{lemma-top-coh-divisible}
Let $A$ be a Noetherian local ring of dimension $d$. Let $f \in A$
be an element which is not contained in any minimal prime of
dimension $d$. Then $f : H^d_{V(I)}(M) \to H^d_{V(I)}(M)$
is surjective for any finite $A$-module $M$ and any ideal $I \subset A$.
\end{lemma}

\begin{proof}
The support of $M/fM$ has dimension $< d$ by our assumption on $f$.
Thus $H^d_{V(I)}(M/fM) = 0$ by Lemma \ref{lemma-cd-dimension}.
Thus $H^d_{V(I)}(fM) \to H^d_{V(I)}(M)$ is surjective.
Since by Lemma \ref{lemma-cd-dimension} we know $\text{cd}(A, I) \leq d$
we also see that the surjection $M \to fM$, $x \mapsto fx$
induces a surjection $H^d_{V(I)}(M) \to H^d_{V(I)}(fM)$.
\end{proof}

\begin{lemma}
\label{lemma-cd-bound-dualizing}
Let $A$ be a Noetherian local ring with
normalized dualizing complex $\omega_A^\bullet$.
Let $I \subset A$ be an ideal.
If $H^0_{V(I)}(\omega_A^\bullet) = 0$, then $\text{cd}(A, I) < \dim(A)$.
\end{lemma}

\begin{proof}
Set $d = \dim(A)$. Let $\mathfrak p_1, \ldots, \mathfrak p_n \subset A$
be the minimal primes of dimension $d$.
Recall that the finite $A$-module
$H^{-i}(\omega_A^\bullet)$ is nonzero only for
$i \in \{0, \ldots, d\}$ and that the support
of $H^{-i}(\omega_A^\bullet)$ has dimension $\leq i$, see
Lemma \ref{lemma-sitting-in-degrees}.
Set $\omega_A = H^{-d}(\omega_A^\bullet)$.
By prime avoidence (Algebra, Lemma \ref{algebra-lemma-silly})
we can find $f \in A$, $f \not \in \mathfrak p_i$
which annihilates $H^{-i}(\omega_A^\bullet)$ for $i < d$.
Consider the distinguished triangle
$$
\omega_A[d] \to \omega_A^\bullet \to
\tau_{\geq -d + 1}\omega_A^\bullet \to \omega_A[d + 1]
$$
See Derived Categories, Remark
\ref{derived-remark-truncation-distinguished-triangle}.
By Derived Categories, Lemma \ref{derived-lemma-trick-vanishing-composition}
we see that $f^d$ induces the zero endomorphism of
$\tau_{\geq -d + 1}\omega_A^\bullet$.
Using the axioms of a triangulated category, we find a map
$$
\omega_A^\bullet \to \omega_A[d]
$$
whose composition with $\omega_A[d] \to \omega_A^\bullet$ is
multiplication by $f^d$ on $\omega_A[d]$.
Thus we conclude that $f^d$ annihilates $H^d_{V(I)}(\omega_A)$.
By Lemma \ref{lemma-top-coh-divisible} we conlude $H^d_{V(I)}(\omega_A) = 0$.
Then we conclude by Lemma \ref{lemma-cd-top-vanishing-some-module}
and the fact that $(\omega_A)_{\mathfrak p_i}$ is nonzero
(see for example
Dualizing Complexes, Lemma
\ref{dualizing-lemma-nonvanishing-generically-local}).
\end{proof}

\begin{lemma}
\label{lemma-inverse-system-symbolic-powers}
Let $(A, \mathfrak m)$ be a complete Noetherian local domain. Let
$\mathfrak p \subset A$ be a prime ideal of dimension $1$.
For every $n \geq 1$ there is an $m \geq n$ such that
$\mathfrak p^{(m)} \subset \mathfrak p^n$.
\end{lemma}

\begin{proof}
Recall that the symbolic power $\mathfrak p^{(m)}$ is defined as the
kernel of $A \to A_\mathfrak p/\mathfrak p^mA_\mathfrak p$.
Since localization is exact we conclude that in the short exact sequence
$$
0 \to \mathfrak a_n \to A/\mathfrak p^n \to A/\mathfrak p^{(n)} \to 0
$$
the support of $\mathfrak a_n$ is contained in $\{\mathfrak m\}$.
In particular, the inverse system $(\mathfrak a_n)$ is Mittag-Leffler
as each $\mathfrak a_n$ is an Artinian $A$-module.
We conclude that the lemma is equivalent to the requirement
that $\lim \mathfrak a_n = 0$. Let $f \in \lim \mathfrak a_n$.
Then $f$ is an element of $A = \lim A/\mathfrak p^n$
(here we use that $A$ is complete)
which maps to zero in the completion $A_\mathfrak p^\wedge$
of $A_\mathfrak p$. Since $A_\mathfrak p \to A_\mathfrak p^\wedge$
is faithfully flat, we see that $f$ maps to zero in $A_\mathfrak p$.
Since $A$ is a domain we see that $f$ is zero as desired.
\end{proof}

\begin{proposition}
\label{proposition-Hartshorne-Lichtenbaum-vanishing}
\begin{reference}
\cite[Theorem 3.1]{CD}
\end{reference}
Let $A$ be a Noetherian local ring with completion $A^\wedge$.
Let $I \subset A$ be an ideal such that
$$
\dim V(IA^\wedge + \mathfrak p) \geq 1
$$
for every minimal prime $\mathfrak p \subset A^\wedge$ of dimension $\dim(A)$.
Then $\text{cd}(A, I) < \dim(A)$.
\end{proposition}

\begin{proof}
Since $A \to A^\wedge$ is faithfully flat we have
$H^d_{V(I)}(A) \otimes_A A^\wedge = H^d_{V(IA^\wedge)}(A^\wedge)$
by Dualizing Complexes, Lemma \ref{dualizing-lemma-torsion-change-rings}.
Thus we may assume $A$ is complete.

\medskip\noindent
Assume $A$ is complete. Let $\mathfrak p_1, \ldots, \mathfrak p_n \subset A$
be the minimal primes of dimension $d$. Consider the complete local ring
$A_i = A/\mathfrak p_i$. We have $H^d_{V(I)}(A_i) = H^d_{V(IA_i)}(A_i)$
by Dualizing Complexes, Lemma
\ref{dualizing-lemma-local-cohomology-and-restriction}.
By Lemma \ref{lemma-cd-top-vanishing-some-module}
it suffices to prove the lemma for $(A_i, IA_i)$.
Thus we may assume $A$ is a complete local domain.

\medskip\noindent
Assume $A$ is a complete local domain. We can choose a prime ideal
$\mathfrak p \supset I$ with $\dim(A/\mathfrak p) = 1$.
By Lemma \ref{lemma-cd-top-vanishing}
it suffices to prove the lemma for $\mathfrak p$.

\medskip\noindent
By Lemma \ref{lemma-cd-bound-dualizing} it suffices to show that
$H^0_{V(\mathfrak p)}(\omega_A^\bullet) = 0$.
Recall that
$$
H^0_{V(\mathfrak p)}(\omega_A^\bullet) =
\colim \text{Ext}^0_A(A/\mathfrak p^n, \omega_A^\bullet)
$$
By Lemma \ref{lemma-inverse-system-symbolic-powers}
we see that the colimit is the same as
$$
\colim \text{Ext}^0_A(A/\mathfrak p^{(n)}, \omega_A^\bullet)
$$
Since $\text{depth}(A/\mathfrak p^{(n)}) = 1$ we see that
these ext groups are zero by Lemma \ref{lemma-sitting-in-degrees}
as desired.
\end{proof}

\begin{lemma}
\label{lemma-affine-complement}
Let $(A, \mathfrak m)$ be a Noetherian local ring.
Let $I \subset A$ be an ideal. Assume $A$ is excellent,
normal, and $\dim V(I) \geq 1$. Then $\text{cd}(A, I) < \dim(A)$.
In particular, if $\dim(A) = 2$, then $\Spec(A) \setminus V(I)$ is affine.
\end{lemma}

\begin{proof}
By More on Algebra, Lemma
\ref{more-algebra-lemma-completion-normal-local-ring}
the completion $A^\wedge$ is normal and hence a domain.
Thus the assumption of
Proposition \ref{proposition-Hartshorne-Lichtenbaum-vanishing}
holds and we conclude. The statement on affineness
follows from Lemma \ref{lemma-cd-is-one}.
\end{proof}
















\section{Frobenius action}
\label{section-frobenius}

\noindent
Let $p$ be a prime number. Let $A$ be a ring with $p = 0$ in $A$.
The {\it Frobenius endomorphism} of $A$ is the map
$$
F : A \longrightarrow A,
\quad
a \longmapsto a^p
$$
In this section we prove lemmas on modules which have
Frobenius actions.

\begin{lemma}
\label{lemma-annihilator-frobenius-module}
Let $p$ be a prime number. Let $(A, \mathfrak m, \kappa)$
be a Noetherian local ring
with $p = 0$ in $A$. Let $M$ be a finite $A$-module
such that $M \otimes_{A, F} A \cong M$. Then $M$ is finite free.
\end{lemma}

\begin{proof}
Choose a presentation $A^{\oplus m} \to A^{\oplus n} \to M$
which induces an isomorphism $\kappa^{\oplus n} \to M/\mathfrak m M$.
Let $T = (a_{ij})$ be the matrix of the map $A^{\oplus m} \to A^{\oplus n}$.
Observe that $a_{ij} \in \mathfrak m$. Applying base change by
$F$, using right exactness of base change, we get a presentation
$A^{\oplus m} \to A^{\oplus n} \to M$ where the matrix is
$T = (a_{ij}^p)$. Thus we have a presentation with
$a_{ij} \in \mathfrak m^p$. Repeating this construction we
find that for each $e \geq 1$ there exists a presentation with
$a_{ij} \in \mathfrak m^e$. This implies the fitting ideals
(More on Algebra, Definition \ref{more-algebra-definition-fitting-ideal})
$\text{Fit}_k(M)$ for $k < n$ are contained in
$\bigcap_{e \geq 1} \mathfrak m^e$. Since this is zero by
Krull's intersection theorem
(Algebra, Lemma \ref{algebra-lemma-intersect-powers-ideal-module-zero})
we conclude that
$M$ is free of rank $n$ by
More on Algebra, Lemma
\ref{more-algebra-lemma-fitting-ideal-finite-locally-free}.
\end{proof}

\noindent
In this section, we say elements $f_1, \ldots, f_r$ of a ring $A$
are {\it independent} if $\sum a_if_i = 0$ implies
$a_i \in (f_1, \ldots, f_r)$. In other words, with $I = (f_1, \ldots, f_r)$
we have $I/I^2$ is free over $A/I$ with basis $f_1, \ldots, f_r$.

\begin{lemma}
\label{lemma-1}
\begin{reference}
See \cite{Lech-inequalities} and \cite[Lemma 1 page 299]{MatCA}.
\end{reference}
Let $A$ be a ring. If $f_1, \ldots, f_{r - 1}, f_rg_r$
are independent, then $f_1, \ldots, f_r$ are independent.
\end{lemma}

\begin{proof}
Say $\sum a_if_i = 0$. Then $\sum a_ig_rf_i = 0$.
Hence $a_r \in (f_1, \ldots, f_{r - 1}, f_rg_r)$.
Write $a_r = \sum_{i < r} b_i f_i + b f_rg_r$.
Then $0 = \sum_{i < r} (a_i + b_if_r)f_i + bf_r^2g_r$.
Thus $a_i + b_i f_r \in (f_1, \ldots, f_{r - 1}, f_rg_r)$
which implies $a_i \in (f_1, \ldots, f_r)$ as desired.
\end{proof}

\begin{lemma}
\label{lemma-2}
\begin{reference}
See \cite{Lech-inequalities} and \cite[Lemma 2 page 300]{MatCA}.
\end{reference}
Let $A$ be a ring. If $f_1, \ldots, f_{r - 1}, f_rg_r$
are independent and if the $A$-module
$A/(f_1, \ldots, f_{r - 1}, f_rg_r)$ has finite length, then
\begin{align*}
& \text{length}_A(A/(f_1, \ldots, f_{r - 1}, f_rg_r)) \\
& =
\text{length}_A(A/(f_1, \ldots, f_{r - 1}, f_r)) +
\text{length}_A(A/(f_1, \ldots, f_{r - 1}, g_r))
\end{align*}
\end{lemma}

\begin{proof}
We claim there is an exact sequence
$$
0 \to
A/(f_1, \ldots, f_{r - 1}, g_r) \xrightarrow{f_r}
A/(f_1, \ldots, f_{r - 1}, f_rg_r) \to
A/(f_1, \ldots, f_{r - 1}, f_r) \to 0
$$
Namely, if $a f_r \in (f_1, \ldots, f_{r - 1}, f_rg_r)$, then
$\sum_{i < r} a_i f_i + (a + bg_r)f_r = 0$
for some $b, a_i \in A$. Hence
$\sum_{i < r} a_i g_r f_i + (a + bg_r)g_rf_r = 0$
which implies $a + bg_r \in (f_1, \ldots, f_{r - 1}, f_rg_r)$
which means that $a$ maps to zero in $A/(f_1, \ldots, f_{r - 1}, g_r)$.
This proves the claim.
To finish use additivity of lengths
(Algebra, Lemma \ref{algebra-lemma-length-additive}).
\end{proof}

\begin{lemma}
\label{lemma-3}
\begin{reference}
See \cite{Lech-inequalities} and \cite[Lemma 3 page 300]{MatCA}.
\end{reference}
Let $(A, \mathfrak m)$ be a local ring. If $\mathfrak m = (x_1, \ldots, x_r)$
and $x_1^{e_1}, \ldots, x_r^{e_r}$ are independent for some $e_i > 0$,
then $\text{length}_A(A/(x_1^{e_1}, \ldots, x_r^{e_r})) = e_1\ldots e_r$.
\end{lemma}

\begin{proof}
Use Lemmas \ref{lemma-1} and \ref{lemma-2} and induction.
\end{proof}

\begin{lemma}
\label{lemma-flat-extension-independent}
Let $\varphi : A \to B$ be a flat ring map.
If $f_1, \ldots, f_r \in A$ are independent, then
$\varphi(f_1), \ldots, \varphi(f_r) \in B$ are independent.
\end{lemma}

\begin{proof}
Let $I = (f_1, \ldots, f_r)$ and $J = F(I)B$. By flatness we have
$I/I^2 \otimes_A B = J/J^2$. Hence freeness of $I/I^2$ over $A/I$
implies freeness of $J/J^2$ over $B/J$.
\end{proof}

\begin{lemma}[Kunz]
\label{lemma-frobenius-flat-regular}
\begin{reference}
\cite{Kunz-flat}
\end{reference}
Let $p$ be a prime number.
Let $A$ be a Noetherian ring with $p = 0$.
The following are equivalent
\begin{enumerate}
\item $A$ is regular, and
\item $F : A \to A$, $a \mapsto a^p$ is flat.
\end{enumerate}
\end{lemma}

\begin{proof}
Observe that $\Spec(F) : \Spec(A) \to \Spec(A)$ is the identity map.
Being regular is defined in terms of the local rings and being flat
is something about local rings, see
Algebra, Lemma \ref{algebra-lemma-flat-localization}.
Thus we may and do assume $A$ is a Noetherian
local ring with maximal ideal $\mathfrak m$.

\medskip\noindent
Assume $A$ is regular. Let $x_1, \ldots, x_d$ be a
system of parameters for $A$. Applying $F$ we find
$F(x_1), \ldots, F(x_d) = x_1^p, \ldots, x_d^p$,
which is a system of parameters for $A$. Hence $F$ is flat, see
Algebra, Lemmas \ref{algebra-lemma-CM-over-regular-flat} and
\ref{algebra-lemma-regular-ring-CM}.

\medskip\noindent
Conversely, assume $F$ is flat. Write $\mathfrak m = (x_1, \ldots, x_r)$
with $r$ minimal. Then $x_1, \ldots, x_r$ are independent in the sense
defined above. Since $F$ is flat, we see that $x_1^p, \ldots, x_r^p$
are independent, see Lemma \ref{lemma-flat-extension-independent}.
Hence $\text{length}_A(A/(x_1^p, \ldots, x_r^p)) = p^r$ by
Lemma \ref{lemma-3}.
Let $\chi(n) = \text{length}_A(A/\mathfrak m^n)$ and recall
that this is a numerical polynomial of degree $\dim(A)$, see
Algebra, Proposition \ref{algebra-proposition-dimension}.
Choose $n \gg 0$. Observe that
$$
\mathfrak m^{pn + pr} \subset F(\mathfrak m^n)A \subset \mathfrak m^{pn}
$$
as can be seen by looking at monomials in $x_1, \ldots, x_r$. We have
$$
A/F(\mathfrak m^n)A = A/\mathfrak m^n \otimes_{A, F} A
$$
By flatness of $F$ this has length $\chi(n) \text{length}_A(A/F(\mathfrak m)A)$
(Algebra, Lemma \ref{algebra-lemma-pullback-module})
which is equal to $p^r\chi(n)$ by the above. We conclude
$$
\chi(pn + pr) \geq p^r\chi(n) \geq \chi(pn)
$$
Looking at the leading terms this implies $r = \dim(A)$, i.e., $A$ is regular.
\end{proof}




\section{Structure of certain modules}
\label{section-structure}

\noindent
Some results on the structure of certain types of
modules over regular local rings. These types of
results and much more can be found in
\cite{Huneke-Sharp}, \cite{Lyubeznik}, \cite{Lyubeznik2}.

\begin{lemma}
\label{lemma-structure-torsion-D-module-regular}
\begin{reference}
Special case of \cite[Theorem 2.4]{Lyubeznik}
\end{reference}
Let $k$ be a field of characteristic $0$. Let $d \geq 1$.
Let $A = k[[x_1, \ldots, x_d]]$ with maximal ideal $\mathfrak m$.
Let $M$ be an $\mathfrak m$-power torsion $A$-module endowed with
additive operators $D_1, \ldots, D_d$ satisfying the leibniz rule
$$
D_i(fz) = \partial_i(f) z + f D_i(z)
$$
for $f \in A$ and $z \in M$. Here $\partial_i$ is
differentiation with respect to $x_i$.
Then $M$ is isomorphic to a direct sum
of copies of the injective hull $E$ of $k$.
\end{lemma}

\begin{proof}
Choose a set $J$ and an isomorphism $M[\mathfrak m] \to \bigoplus_{j \in J} k$.
Since $\bigoplus_{j \in J} E$ is injective
(Dualizing Complexes, Lemma \ref{dualizing-lemma-sum-injective-modules})
we can extend this isomorphism to an $A$-module homomorphism
$\varphi : M \to \bigoplus_{j \in J} E$.
We claim that $\varphi$ is an isomorphism, i.e., bijective.

\medskip\noindent
Injective. Let $z \in M$ be nonzero. Since $M$ is $\mathfrak m$-power torsion
we can choose an element $f \in A$ such that $fz \in M[\mathfrak m]$ and
$fz \not = 0$. Then $\varphi(fz) = f\varphi(z)$ is nonzero, hence
$\varphi(z)$ is nonzero.

\medskip\noindent
Surjective. Let $z \in M$. Then $x_1^n z = 0$ for some $n \geq 0$.
We will prove that $z \in x_1M$ by induction on $n$.
If $n = 0$, then $z = 0$ and the result is true.
If $n > 0$, then applying $D_1$ we find $0 = n x_1^{n - 1} z + x_1^nD_1(z)$.
Hence $x_1^{n - 1}(nz + x_1D_1(z)) = 0$. By induction we get
$nz + x_1D_1(z) \in x_1M$. Since $n$ is invertible, we conclude
$z \in x_1M$. Thus we see that $M$ is $x_1$-divisible.
If $\varphi$ is not surjective, then we can choose
$e \in \bigoplus_{j \in J} E$ not in $M$.
Arguing as above we may assume $\mathfrak m e \subset M$,
in particular $x_1 e \in M$. There exists an element
$z_1 \in M$ with $x_1 z_1 = x_1 e$. Hence
$x_1(z_1 - e) = 0$. Replacing $e$ by $e - z_1$
we may assume $e$ is annihilated by $x_1$.
Thus it suffices to prove that
$$
\varphi[x_1] :
M[x_1]
\longrightarrow
\left(\bigoplus\nolimits_{j \in J} E\right)[x_1] =
\bigoplus\nolimits_{j \in J} E[x_1]
$$
is surjective. If $d = 1$, this is true by construction of $\varphi$.
If $d > 1$, then we observe that $E[x_1]$ is the injective hull
of the residue field of $k[[x_2, \ldots, x_d]]$, see
Dualizing Complexes, Lemma \ref{dualizing-lemma-quotient}.
Observe that $M[x_1]$ as a module over $k[[x_2, \ldots, x_d]]$
is $\mathfrak m/(x_1)$-power torsion and comes
equipped with operators $D_2, \ldots, D_d$ satisfying
the displayed Leibniz rule.
Thus by induction on $d$ we conclude that $\varphi[x_1]$
is surjective as desired.
\end{proof}

\begin{lemma}
\label{lemma-structure-torsion-Frobenius-regular}
\begin{reference}
Follows from \cite[Corollary 3.6]{Huneke-Sharp} with a
little bit of work. Also follows directly from
\cite[Theorem 1.4]{Lyubeznik2}.
\end{reference}
Let $p$ be a prime number. Let $(A, \mathfrak m, k)$
be a regular local ring with $p = 0$. Denote $F : A \to A$, $a \mapsto a^p$
be the Frobenius endomorphism. Let $M$ be a $\mathfrak m$-power torsion module
such that $M \otimes_{A, F} A \cong M$. Then $M$ is isomorphic to a direct sum
of copies of the injective hull $E$ of $k$.
\end{lemma}

\begin{proof}
Choose a set $J$ and an $A$-module homorphism
$\varphi : M \to \bigoplus_{j \in J} E$ which maps
$M[\mathfrak m]$ isomorphically onto
$(\bigoplus_{j \in J} E)[\mathfrak m] = \bigoplus_{j \in J} k$.
We claim that $\varphi$ is an isomorphism, i.e., bijective.

\medskip\noindent
Injective. Let $z \in M$ be nonzero. Since $M$ is $\mathfrak m$-power torsion
we can choose an element $f \in A$ such that $fz \in M[\mathfrak m]$ and
$fz \not = 0$. Then $\varphi(fz) = f\varphi(z)$ is nonzero, hence
$\varphi(z)$ is nonzero.

\medskip\noindent
Surjective. Recall that $F$ is flat, see
Lemma \ref{lemma-frobenius-flat-regular}.
Let $x_1, \ldots, x_d$ be a minimal system of generators of
$\mathfrak m$. Denote
$$
M_n = M[x_1^{p^n}, \ldots, x_d^{p^n}]
$$
the submodule of $M$ consisting of elements killed by
$x_1^{p^n}, \ldots, x_d^{p^n}$. So $M_0 = M[\mathfrak m]$
is a vector space over $k$. Also $M = \bigcup M_n$ by our
assumption that $M$ is $\mathfrak m$-power torsion. Since $F^n$ is flat and
$F^n(x_i) = x_i^{p^n}$ we have
$$
M_n \cong (M \otimes_{A, F^n} A)[x_1^{p^n}, \ldots, x_d^{p^n}] =
M[x_1, \ldots, x_d] \otimes_{A, F} A =
M_0 \otimes_k A/(x_1^{p^n}, \ldots, x_d^{p^n})
$$
Thus $M_n$ is free over $A/(x_1^{p^n}, \ldots, x_d^{p^n})$.
A computation shows that every element of $A/(x_1^{p^n}, \ldots, x_d^{p^n})$
annihilated by $x_1^{p^n - 1}$ is divisible by $x_1$; for example
you can use that $A/(x_1^{p^n}, \ldots, x_d^{p^n}) \cong
k[x_1, \ldots, x_d]/(x_1^{p^n}, \ldots, x_d^{p^n})$ by Algebra, Lemma
\ref{algebra-lemma-regular-complete-containing-coefficient-field}.
Thus the same is true for every element of $M_n$.
Since every element of $M$ is in $M_n$ for all $n \gg 0$
and since every element of $M$ is killed by some power of
$x_1$, we conclude that $M$ is $x_1$-divisible.

\medskip\noindent
Let $x = x_1$. Above we have seen that $M$ is $x$-divisible.
If $\varphi$ is not surjective, then we can choose
$e \in \bigoplus_{j \in J} E$ not in $M$.
Arguing as above we may assume $\mathfrak m e \subset M$,
in particular $x e \in M$. There exists an element
$z_1 \in M$ with $x z_1 = x e$. Hence
$x(z_1 - e) = 0$. Replacing $e$ by $e - z_1$
we may assume $e$ is annihilated by $x$.
Thus it suffices to prove that
$$
\varphi[x] :
M[x]
\longrightarrow
\left(\bigoplus\nolimits_{j \in J} E\right)[x] =
\bigoplus\nolimits_{j \in J} E[x]
$$
is surjective. If $d = 1$, this is true by construction of $\varphi$.
If $d > 1$, then we observe that $E[x]$ is the injective hull
of the residue field of the regular ring $A/xA$, see
Dualizing Complexes, Lemma \ref{dualizing-lemma-quotient}.
Observe that $M[x]$ as a module over $A/xA$
is $\mathfrak m/(x)$-power torsion and we have
\begin{align*}
M[x] \otimes_{A/xA, F} A/xA
& = M[x] \otimes_{A, F} A \otimes_A A/xA \\
& = (M \otimes_{A, F} A)[x^p] \otimes_A A/xA \\
& \cong M[x^p] \otimes_A A/xA
\end{align*}
Argue using flatness of $F$ as before.
We claim that $M[x^p] \otimes_A A/xA \to M[x]$,
$z \otimes 1 \mapsto x^{p - 1}z$ is an isomorphism.
This can be seen by proving it for
each of the modules $M_n$, $n > 0$ defined above
where it follows by the same result for
$A/(x_1^{p^n}, \ldots, x_d^{p^n})$ and $x = x_1$.
Thus by induction on $\dim(A)$ we conclude that $\varphi[x]$
is surjective as desired.
\end{proof}





\section{Additional structure on local cohomology}
\label{section-additional}

\noindent
Here is a sample result.

\begin{lemma}
\label{lemma-derivation}
Let $A$ be a ring. Let $I \subset A$ be a finitely generated ideal.
Set $Z = V(I)$.
For each derivation $\theta : A \to A$ there exists a canonical
additive operator $D$ on the local cohomology modules
$H^i_Z(A)$ satisfying the Leibniz rule with respect to $\theta$.
\end{lemma}

\begin{proof}
Let $f_1, \ldots, f_r$ be elements generating $I$.
Recall that $R\Gamma_Z(A)$ is computed by the complex
$$
A \to \prod\nolimits_{i_0} A_{f_{i_0}} \to
\prod\nolimits_{i_0 < i_1} A_{f_{i_0}f_{i_1}}
\to \ldots \to A_{f_1\ldots f_r}
$$
See Dualizing Complexes, Lemma \ref{dualizing-lemma-local-cohomology-adjoint}.
Since $\theta$ extends uniquely to an additive operator on
any localization of $A$ satisfying the Leibniz rule with
respect to $\theta$, the lemma is clear.
\end{proof}

\begin{lemma}
\label{lemma-frobenius}
Let $p$ be a prime number. Let $A$ be a ring with $p = 0$.
Denote $F : A \to A$, $a \mapsto a^p$ the Frobenius endomorphism.
Let $I \subset A$ be a finitely generated ideal. Set $Z = V(I)$.
There exists an isomorphism
$R\Gamma_Z(A) \otimes_{A, F}^\mathbf{L} A \cong R\Gamma_Z(A)$.
\end{lemma}

\begin{proof}
Follows from Dualizing Complexes, Lemma
\ref{dualizing-lemma-torsion-change-rings}
and the fact that $Z = V(f_1^p, \ldots, f_r^p)$
if $I = (f_1, \ldots, f_r)$.
\end{proof}

\begin{lemma}
\label{lemma-etale-derivation}
Let $A$ be a ring. Let $V \to \Spec(A)$ be quasi-compact, quasi-separated,
and \'etale. For each derivation $\theta : A \to A$ there exists a canonical
additive operator $D$ on $H^i(V, \mathcal{O}_V)$
satisfying the Leibniz rule with respect to $\theta$.
\end{lemma}

\begin{proof}
If $V$ is separated, then we can argue using an affine open covering
$V = \bigcup_{j = 1, \ldots m} V_j$. Namely, because $V$ is separated
we may write $V_{j_0 \ldots j_p} = \Spec(B_{j_0 \ldots j_p})$.
See Schemes, Lemma \ref{schemes-lemma-characterize-separated}. Then we
find that the $A$-module $H^i(V, \mathcal{O}_V)$
is the $i$th cohomology group of the {\v C}ech complex
$$
\prod B_{j_0} \to
\prod B_{j_0j_1} \to
\prod B_{j_0j_1j_2} \to \ldots
$$
See Cohomology of Schemes, Lemma
\ref{coherent-lemma-cech-cohomology-quasi-coherent}.
Each $B = B_{j_0 \ldots j_p}$ is an \'etale $A$-algebra.
Hence $\Omega_B = \Omega_A \otimes_A B$ and we conclude
$\theta$ extends uniquely to a derivation $\theta_B : B \to B$.
These maps define an endomorphism of the {\v C}ech complex
and define the desired operators on the cohomology groups.

\medskip\noindent
In the general case we use a hypercovering of $V$ by
affine opens, exactly as in the first part of the proof of
Cohomology of Schemes, Lemma \ref{coherent-lemma-hypercoverings}.
We omit the details.
\end{proof}

\begin{remark}
\label{remark-higher-order-operators}
We can upgrade Lemmas \ref{lemma-derivation} and \ref{lemma-etale-derivation}
to include higher order differential operators.
If we ever need this we will state and prove a
precise lemma here.
\end{remark}

\begin{lemma}
\label{lemma-etale-frobenius}
Let $p$ be a prime number. Let $A$ be a ring with $p = 0$.
Denote $F : A \to A$, $a \mapsto a^p$ the Frobenius endomorphism.
If $V \to \Spec(A)$ is quasi-compact, quasi-separated,
and \'etale, then there exists an isomorphism
$R\Gamma(V, \mathcal{O}_V) \otimes_{A, F}^\mathbf{L} A \cong
R\Gamma(V, \mathcal{O}_V)$.
\end{lemma}

\begin{proof}
Observe that the relative Frobenius morphism
$$
V \longrightarrow V \times_{\Spec(A), \Spec(F)} \Spec(A)
$$
of $V$ over $A$ is an isomorphism, see
\'Etale Morphisms, Lemma \ref{etale-lemma-relative-frobenius-etale}.
Thus the lemma follows from cohomology and base change, see
Derived Categories of Schemes, Lemma
\ref{perfect-lemma-compare-base-change}.
Observe that since $V$ is \'etale over $A$, it is flat over $A$.
\end{proof}





\section{A bit of uniformity, I}
\label{section-uniform-vanishing}

\noindent
The main task of this section is to formulate and prove
Lemma \ref{lemma-characterize-vanishing-tor-ext-above-e}.

\begin{lemma}
\label{lemma-map-tor-1-zero}
Let $R$ be a ring. Let $M \to M'$ be a map of $R$-modules with
$M$ of finite presentation
such that $\text{Tor}_1^R(M, N) \to \text{Tor}_1^R(M', N)$ is zero for all
$R$-modules $N$. Then $M \to M'$ factors through a free $R$-module.
\end{lemma}

\begin{proof}
We may choose a map of short exact sequences
$$
\xymatrix{
0 \ar[r] &
K \ar[r] \ar[d] &
R^{\oplus r} \ar[r] \ar[d] &
M \ar[r] \ar[d] &
0 \\
0 \ar[r] &
K' \ar[r] &
\bigoplus_{i \in I} R \ar[r] &
M' \ar[r] &
0
}
$$
whose right vertical arrow is the given map.
We can factor this map through the short exact sequence
\begin{equation}
\label{equation-pushout}
0 \to K' \to E \to M \to 0
\end{equation}
which is the pushout of the first short exact sequence by $K \to K'$.
By a diagram chase we see that the assumption in the lemma
implies that the boundary map $\text{Tor}_1^R(M, N) \to K' \otimes_R N$
induced by (\ref{equation-pushout}) is zero, i.e., the sequence
(\ref{equation-pushout}) is universally exact. This implies by
Algebra, Lemma \ref{algebra-lemma-universally-exact-split}
that (\ref{equation-pushout}) is split (this is where we use that
$M$ is of finite presentation). Hence the map $M \to M'$
factors through $\bigoplus_{i \in I} R$ and we win.
\end{proof}

\begin{lemma}
\label{lemma-characterize-vanishing-tor-ext-above-e}
Let $R$ be a ring. Let $\alpha : M \to M'$ be a map of $R$-modules.
Let $P_\bullet \to M$ and $P'_\bullet \to M'$ be resolutions by
projective $R$-modules. Let $e \geq 0$ be an integer.
Consider the following conditions
\begin{enumerate}
\item We can find a map of complexes $a_\bullet : P_\bullet \to P'_\bullet$
inducing $\alpha$ on cohomology with $a_i = 0$ for $i > e$.
\item We can find a map of complexes $a_\bullet : P_\bullet \to P'_\bullet$
inducing $\alpha$ on cohomology with $a_{e + 1} = 0$.
\item The map $\Ext^i_R(M', N) \to \Ext^i_R(M, N)$ is zero
for all $R$-modules $N$ and $i > e$.
\item The map $\Ext^{e + 1}_R(M', N) \to \Ext^{e + 1}_R(M, N)$ is zero
for all $R$-modules $N$.
\item Let $N = \Im(P'_{e + 1} \to P'_e)$ and denote
$\xi \in \Ext^{e + 1}_R(M', N)$ the canonical element (see proof).
Then $\xi$ maps to zero in $\Ext^{e + 1}_R(M, N)$.
\item The map $\text{Tor}_i^R(M, N) \to \text{Tor}_i^R(M', N)$
is zero for all $R$-modules $N$ and $i > e$.
\item The map $\text{Tor}_{e + 1}^R(M, N) \to \text{Tor}_{e + 1}^R(M', N)$
is zero for all $R$-modules $N$.
\end{enumerate}
Then we always have the implications
$$
(1) \Leftrightarrow (2) \Leftrightarrow (3) \Leftrightarrow (4)
\Leftrightarrow (5) \Rightarrow (6) \Leftrightarrow (7)
$$
If $M$ is $(-e - 1)$-pseudo-coherent (for example if $R$ is Noetherian
and $M$ is a finite $R$-module), then all conditions
are equivalent.
\end{lemma}

\begin{proof}
It is clear that (2) implies (1). If $a_\bullet$ is as in (1), then
we can consider the map of complexes $a'_\bullet : P_\bullet \to P'_\bullet$
with $a'_i = a_i$ for $i \leq e + 1$ and $a'_i = 0$ for $i \geq e + 1$
to get a map of complexes as in (2). Thus (1) is equivalent to (2).

\medskip\noindent
By the construction of the $\Ext$ and $\text{Tor}$ functors using
resolutions (Algebra, Sections \ref{algebra-section-ext}
and \ref{algebra-section-tor}) we see 
that (1) and (2) imply all of the other conditions.

\medskip\noindent
It is clear that (3) implies (4) implies (5). Let $N$ be as in (5).
The canonical map $\tilde \xi : P'_{e + 1} \to N$ precomposed with
$P'_{e + 2} \to P'_{e + 1}$ is zero. Hence we may consider the class
$\xi$ of $\tilde \xi$ in
$$
\Ext^{e + 1}_R(M', N) =
\frac{\Ker(\Hom(P'_{e + 1}, N \to \Hom(P'_{e + 2}, N)}{
\Im(\Hom(P'_e, N \to \Hom(P'_{e + 1}, N)}
$$
Choose a map of complexes $a_\bullet : P_\bullet \to P'_\bullet$
lifting $\alpha$, see Derived Categories, Lemma
\ref{derived-lemma-morphisms-lift-projective}.
If $\xi$ maps to zero in $\Ext^{e + 1}_R(M', N)$, then
we find a map $\varphi : P_e \to N$ such that
$\tilde \xi  \circ a_{e + 1} = \varphi \circ d$.
Thus we obtain a map of complexes
$$
\xymatrix{
\ldots \ar[r] &
P_{e + 1} \ar[r] \ar[d]^0 &
P_e \ar[r] \ar[d]^{a_e - \varphi} &
P_{e - 1} \ar[r] \ar[d]^{a_{e - 1}} &
\ldots \\
\ldots \ar[r] &
P'_{e + 1} \ar[r] &
P'_e \ar[r] &
P'_{e - 1} \ar[r] &
\ldots
}
$$
as in (2). Hence (1) -- (5) are equivalent.

\medskip\noindent
The equivalence of (6) and (7) follows from dimension shifting;
we omit the details.

\medskip\noindent
Assume $M$ is $(-e - 1)$-pseudo-coherent. (The parenthetical statement
in the lemma follows from More on Algebra, Lemma
\ref{more-algebra-lemma-Noetherian-pseudo-coherent}.)
We will show that (7) implies (4) which
finishes the proof. We will use induction on $e$.
The base case is $e = 0$. Then $M$ is of finite presentation by
More on Algebra, Lemma \ref{more-algebra-lemma-n-pseudo-module}
and we conclude from Lemma \ref{lemma-map-tor-1-zero} that
$M \to M'$ factors through a free module. Of course if $M \to M'$
factors through a free module, then $\Ext^i_R(M', N) \to \Ext^i_R(M, N)$
is zero for all $i > 0$ as desired.
Assume $e > 0$. We may choose a map of short exact sequences
$$
\xymatrix{
0 \ar[r] &
K \ar[r] \ar[d] &
R^{\oplus r} \ar[r] \ar[d] &
M \ar[r] \ar[d] &
0 \\
0 \ar[r] &
K' \ar[r] &
\bigoplus_{i \in I} R \ar[r] &
M' \ar[r] &
0
}
$$
whose right vertical arrow is the given map. We obtain
$\text{Tor}_{i + 1}^R(M, N) = \text{Tor}^R_i(K, N)$
and $\Ext^{i + 1}_R(M, N) = \Ext^i_R(K, N)$ for $i \geq 1$
and all $R$-modules $N$ and similarly for $M', K'$.
Hence we see that $\text{Tor}_e^R(K, N) \to \text{Tor}_e^R(K', N)$
is zero for all $R$-modules $N$. By More on Algebra, Lemma
\ref{more-algebra-lemma-cone-pseudo-coherent} we see that $K$
is $(-e)$-pseudo-coherent. By induction we conclude that
$\Ext^e(K', N) \to \Ext^e(K, N)$ is zero for all $R$-modules
$N$, which gives what we want.
\end{proof}

\begin{lemma}
\label{lemma-cd-sequence-Koszul}
Let $I$ be an ideal of a Noetherian ring $A$.
For all $n \geq 1$ there exists an $m > n$ such that the map
$A/I^m \to A/I^n$ satisfies the equivalent conditions of
Lemma \ref{lemma-characterize-vanishing-tor-ext-above-e} with
$e = \text{cd}(A, I)$.
\end{lemma}

\begin{proof}
Let $\xi \in \Ext^{e + 1}_A(A/I^n, N)$ be the element constructed in
Lemma \ref{lemma-characterize-vanishing-tor-ext-above-e} part (5).
Since $e = \text{cd}(A, I)$ we have
$0 = H^{e + 1}_Z(N) = H^{e + 1}_I(N) = \colim \Ext^{e + 1}(A/I^m, N)$
by Dualizing Complexes, Lemmas
\ref{dualizing-lemma-local-cohomology-noetherian} and
\ref{dualizing-lemma-local-cohomology-ext}.
Thus we may pick $m \geq n$ such that $\xi$ maps to
zero in $\Ext^{e + 1}_A(A/I^m, N)$ as desired.
\end{proof}






\section{A bit of uniformity, II}
\label{section-uniform-vanishing-bis}

\noindent
Let $I$ be an ideal of a Noetherian ring $A$. Let $M$ be a finite
$A$-module. Let $i > 0$. By More on Algebra, Lemma
\ref{more-algebra-lemma-tor-strictly-pro-zero}
there exists a $c = c(A, I, M, i)$ such that
$\text{Tor}^A_i(M, A/I^n) \to \text{Tor}^A_i(M, A/I^{n - c})$
is zero for all $n \geq c$. In this section, we discuss
some results which show that one sometimes can choose
a constant $c$ which works for all $A$-modules $M$ simultaneously
(and for a range of indices $i$).
This material is related to uniform Artin-Rees as discussed in
\cite{Huneke-uniform} and \cite{AHS}.

\medskip\noindent
In Remark \ref{remark-strict-pro-isomorphism} we will apply this to show that
various pro-systems related to derived completion are (or are not)
strictly pro-isomorphic.

\medskip\noindent
The following lemma can be significantly strengthened.

\begin{lemma}
\label{lemma-maps-zero-fixed-torsion}
Let $I$ be an ideal of a Noetherian ring $A$. For every $m \geq 0$
and $i > 0$ there exist a $c = c(A, I, m, i) \geq 0$ such that
for every $A$-module $M$ annihilated by $I^m$ the map
$$
\text{Tor}^A_i(M, A/I^n) \to \text{Tor}^A_i(M, A/I^{n - c})
$$
is zero for all $n \geq c$.
\end{lemma}

\begin{proof}
By induction on $i$. Base case $i = 1$. The short exact sequence
$0 \to I^n \to A \to A/I^n \to 0$ determines an injection
$\text{Tor}_1^A(M, A/I^n) \subset I^n \otimes_A M$, see
Algebra, Remark \ref{algebra-remark-Tor-ring-mod-ideal}.
As $M$ is annihilated by $I^m$ we see that the map
$I^n \otimes_A M \to I^{n - m} \otimes_A M$ is
zero for $n \geq m$. Hence the result holds with $c = m$.

\medskip\noindent
Induction step. Let $i > 1$ and assume $c$ works for $i - 1$.
By More on Algebra, Lemma \ref{more-algebra-lemma-tor-strictly-pro-zero}
applied to $M = A/I^m$ we can choose $c' \geq 0$ such that
$\text{Tor}_i(A/I^m, A/I^n) \to \text{Tor}_i(A/I^m, A/I^{n - c'})$
is zero for $n \geq c'$. Let $M$ be annihilated by $I^m$. Choose a short
exact sequence
$$
0 \to S \to \bigoplus\nolimits_{i \in I} A/I^m \to M \to 0
$$
The corresponding long exact sequence of tors gives an exact sequence
$$
\text{Tor}_i^A(\bigoplus\nolimits_{i \in I} A/I^m, A/I^n) \to
\text{Tor}_i^A(M, A/I^n) \to
\text{Tor}_{i - 1}^A(S, A/I^n)
$$
for all integers $n \geq 0$. If $n \geq c + c'$, then the map
$\text{Tor}_{i - 1}^A(S, A/I^n) \to \text{Tor}_{i - 1}^A(S, A/I^{n - c})$
is zero and the map $\text{Tor}_i^A(A/I^m, A/I^{n - c}) \to 
\text{Tor}_i^A(A/I^m, A/I^{n - c - c'})$ is zero. Combined with the
short exact sequences this implies the result holds for $i$ with
constant $c + c'$.
\end{proof}

\begin{lemma}
\label{lemma-annihilates-affine}
Let $I = (a_1, \ldots, a_t)$ be an ideal of a Noetherian ring $A$.
Set $a = a_1$ and denote $B = A[\frac{I}{a}]$ the affine blowup algebra.
There exists a $c > 0$ such that $\text{Tor}_i^A(B, M)$ is annihilated
by $I^c$ for all $A$-modules $M$ and $i \geq t$.
\end{lemma}

\begin{proof}
Recall that $B$ is the quotient of
$A[x_2, \ldots, x_t]/(a_1x_2 - a_2, \ldots, a_1x_t - a_t)$
by its $a_1$-torsion, see
Algebra, Lemma \ref{algebra-lemma-affine-blowup-quotient-description}. Let
$$
B_\bullet = \text{Koszul complex on }a_1x_2 - a_2, \ldots, a_1x_t - a_t
\text{ over }A[x_2, \ldots, x_t]
$$
viewed as a chain complex sitting in degrees $(t - 1), \ldots, 0$.
The complex $B_\bullet[1/a_1]$ is isomorphic to the Koszul complex
on $x_2 - a_2/a_1, \ldots, x_t - a_t/a_1$ which is a regular sequence
in $A[1/a_1][x_2, \ldots, x_t]$. Since regular sequences are
Koszul regular, we conclude that the augmentation
$$
\epsilon : B_\bullet \longrightarrow B
$$
is a quasi-isomorphism after inverting $a_1$. Since the homology modules
of the cone $C_\bullet$ on $\epsilon$ are finite $A[x_2, \ldots, x_n]$-modules
and since $C_\bullet$ is bounded,
we conclude that there exists a $c \geq 0$ such that $a_1^c$
annihilates all of these. By
Derived Categories, Lemma \ref{derived-lemma-trick-vanishing-composition}
this implies that, after possibly replacing $c$ by a larger integer,
that $a_1^c$ is zero on $C_\bullet$ in $D(A)$.
The proof is finished once the reader contemplates
the distinguished triangle
$$
B_\bullet \otimes_A^\mathbf{L} M \to
B \otimes_A^\mathbf{L} M \to
C_\bullet \otimes_A^\mathbf{L} M
$$
Namely, the first term is represented by $B_\bullet \otimes_A M$ which
is sitting in homological degrees $(t - 1), \ldots, 0$
in view of the fact that the terms in the Koszul complex $B_\bullet$
are free (and hence flat) $A$-modules. Whence
$\text{Tor}_i^A(B, M) = H_i(C_\bullet \otimes_A^\mathbf{L} M)$
for $i > t - 1$ and this is annihilated by $a_1^c$.
Since $a_1^cB = I^cB$ and since the tor module is a module
over $B$ we conclude.
\end{proof}

\noindent
For the rest of the discussion in this section we fix a Noetherian ring $A$
and an ideal $I \subset A$. We denote
$$
p : X \to \Spec(A)
$$
the blowing up of $\Spec(A)$ in the ideal $I$. In other words, $X$ is the
$\text{Proj}$ of the Rees algebra $\bigoplus_{n \geq 0} I^n$.
By Cohomology of Schemes, Lemmas \ref{coherent-lemma-coherent-on-proj} and
\ref{coherent-lemma-recover-tail-graded-module}
we can choose an integer $q(A, I) \geq 0$ such that for all $q \geq q(A, I)$
we have $H^i(X, \mathcal{O}_X(q)) = 0$ for
$i > 0$ and $H^0(X, \mathcal{O}_X(q)) = I^q$.

\begin{lemma}
\label{lemma-compute-tor-Iq}
In the situation above, for $q \geq q(A, I)$ and any $A$-module $M$ we have
$$
R\Gamma(X, Lp^*\widetilde{M}(q)) \cong M \otimes_A^\mathbf{L} I^q
$$
in $D(A)$.
\end{lemma}

\begin{proof}
Choose a free resolution $F_\bullet \to M$. Then $\widetilde{F}_\bullet$
is a flat resolution of $\widetilde{M}$. Hence $Lp^*\widetilde{M}$
is given by the complex $p^*\widetilde{F}_\bullet$. Thus
$Lp^*\widetilde{M}(q)$ is given by the complex $p^*\widetilde{F}_\bullet(q)$.
Since $p^*\widetilde{F}_i(q)$ are right acyclic for $\Gamma(X, -)$ by
our choice of $q \geq q(A, I)$ and since we have
$\Gamma(X, p^*\widetilde{F}_i(q)) = I^qF_i$ by our choice of $q \geq q(A, I)$,
we get that $R\Gamma(X, Lp^*\widetilde{M}(q))$ is given by the complex
with terms $I^qF_i$ by Derived Categories of Schemes, Lemma
\ref{perfect-lemma-acyclicity-lemma-global}.
The result follows as the complex $I^qF_\bullet$ computes
$M \otimes_A^\mathbf{L} I^q$ by definition.
\end{proof}

\begin{lemma}
\label{lemma-annihilates}
In the situation above, let $t$ be an upper bound on the number of
generators for $I$. There exists an integer $c = c(A, I) \geq 0$
such that for any $A$-module $M$ the cohomology sheaves
$H^j(Lp^*\widetilde{M})$ are annihilated by $I^c$ for $j \leq -t$.
\end{lemma}

\begin{proof}
Say $I = (a_1, \ldots, a_t)$. The question is affine local on $X$.
For $1 \leq i \leq t$ let $B_i = A[\frac{I}{a_i}]$ be the affine
blowup algebra. Then $X$ has an affine open covering by
the spectra of the rings $B_i$, see
Divisors, Lemma \ref{divisors-lemma-blowing-up-affine}.
By the description of derived pullback given in
Derived Categories of Schemes, Lemma
\ref{perfect-lemma-quasi-coherence-pullback}
we conclude it suffices to prove that for each $i$ there exists a $c \geq 0$
such that
$$
\text{Tor}_j^A(B_i, M)
$$
is annihilated by $I^c$ for $j \geq t$. This is
Lemma \ref{lemma-annihilates-affine}.
\end{proof}

\begin{lemma}
\label{lemma-annihilates-tors}
In the situation above, let $t$ be an upper bound on the number of
generators for $I$. There exists an integer $c = c(A, I) \geq 0$
such that for any $A$-module $M$ the tor modules
$\text{Tor}_i^A(M, A/I^q)$ are annihilated by $I^c$ for $i > t$
and all $q \geq 0$.
\end{lemma}

\begin{proof}
Let $q(A, I)$ be as above. For $q \geq q(A, I)$ we have
$$
R\Gamma(X, Lp^*\widetilde{M}(q)) = M \otimes_A^\mathbf{L} I^q
$$
by Lemma \ref{lemma-compute-tor-Iq}.
We have a bounded and convergent spectral sequence
$$
H^a(X, H^b(Lp^*\widetilde{M}(q))) \Rightarrow
\text{Tor}_{-a - b}^A(M, I^q)
$$
by Derived Categories of Schemes, Lemma \ref{perfect-lemma-spectral-sequence}.
Let $d$ be an integer as in Cohomology of Schemes, Lemma
\ref{coherent-lemma-vanishing-nr-affines-quasi-separated}
(actually we can take $d = t$, see
Cohomology of Schemes, Lemma \ref{coherent-lemma-vanishing-nr-affines}).
Then we see that $H^{-i}(X, Lp^*\widetilde{M}(q)) = \text{Tor}_i^A(M, I^q)$
has a finite filtration with at most $d$ steps whose graded are
subquotients of the modules
$$
H^a(X, H^{- i - a}(Lp^*\widetilde{M})(q)),\quad
a = 0, 1, \ldots, d - 1
$$
If $i \geq t$ then all of these modules are annihilated
by $I^c$ where $c = c(A, I)$ is as in Lemma \ref{lemma-annihilates}
because the cohomology sheaves $H^{- i - a}(Lp^*\widetilde{M})$
are all annihilated by $I^c$ by the lemma. Hence we see that
$\text{Tor}_i^A(M, I^q)$ is annihilated by $I^{dc}$ for
$q \geq q(A, I)$ and $i \geq t$. Using the short exact sequence
$0 \to I^q \to A \to A/I^q \to 0$ we find that
$\text{Tor}_i(M, A/I^q)$ is annihilated by $I^{dc}$ for $q \geq q(A, I)$
and $i > t$. We conclude that $I^m$ with $m = \max(dc, q(A, I) - 1)$
annihilates $\text{Tor}_i^A(M, A/I^q)$ for all $q \geq 0$
and $i > t$ as desired.
\end{proof}

\begin{lemma}
\label{lemma-tor-maps-vanish}
Let $I$ be an ideal of a Noetherian ring $A$. Let $t \geq 0$
be an upper bound on the number of generators of $I$.
There exist $N, c \geq 0$ such that the maps
$$
\text{Tor}_{t + 1}^A(M, A/I^n) \to \text{Tor}_{t + 1}^A(M, A/I^{n - c})
$$
are zero for any $A$-module $M$ and all $n \geq N$.
\end{lemma}

\begin{proof}
Let $c_1$ be the constant found in Lemma \ref{lemma-annihilates-tors}.
Please keep in mind that this constant $c_1$ works for $\text{Tor}_i$ for all
$i > t$ simultaneously.

\medskip\noindent
Say $I = (a_1, \ldots, a_t)$. For an $A$-module $M$ we set
$$
\ell(M) =
\#\{i \mid 1 \leq i \leq t,\ a_i^{c_1}\text{ is zero on }M\}
$$
This is an element of $\{0, 1, \ldots, t\}$.
We will prove by descending induction on $0 \leq s \leq t$
the following statement $H_s$: there exist $N, c \geq 0$ such that
for every module $M$ with $\ell(M) \geq s$ the maps
$$
\text{Tor}_{t + 1 + i}^A(M, A/I^n) \to \text{Tor}_{t + 1 + i}^A(M, A/I^{n - c})
$$
are zero for $i = 0, \ldots, s$ for all $n \geq N$.

\medskip\noindent
Base case: $s = t$. If $\ell(M) = t$, then $M$ is annihilated by
$(a_1^{c_1}, \ldots, a_t^{c_1}\}$ and hence by
$I^{t(c_1 - 1) + 1}$. We conclude from
Lemma \ref{lemma-maps-zero-fixed-torsion}
that $H_t$ holds by taking $c = N$ to be the maximum of the integers
$c(A, I, t(c_1 - 1) + 1, t + 1), \ldots, c(A, I, t(c_1 - 1) + 1, 2t + 1)$
found in the lemma.

\medskip\noindent
Induction step. Say $0 \leq s < t$ we have $N, c$ as in $H_{s + 1}$.
Consider a module $M$ with $\ell(M) = s$. Then we can choose an $i$
such that $a_i^{c_1}$ is nonzero on $M$. It follows that
$\ell(M[a_i^c]) \geq s + 1$ and $\ell(M/a_i^{c_1}M) \geq s + 1$
and the induction hypothesis applies to them.
Consider the exact sequence
$$
0 \to M[a_i^{c_1}] \to M \xrightarrow{a_i^{c_1}} M \to M/a_i^{c_1}M \to 0
$$
Denote $E \subset M$ the image of the middle arrow.
Consider the corresponding diagram of Tor modules
$$
\xymatrix{
&
&
\text{Tor}_{i + 1}(M/a_i^{c_1}M, A/I^q) \ar[d] \\
\text{Tor}_i(M[a_i^{c_1}], A/I^q) \ar[r] &
\text{Tor}_i(M, A/I^q) \ar[r] \ar[rd]^0 &
\text{Tor}_i(E, A/I^q) \ar[d] \\
&
&
\text{Tor}_i(M, A/I^q)
}
$$
with exact rows and columns (for every $q$). The south-east arrow
is zero by our choice of $c_1$. We conclude that the
module $\text{Tor}_i(M, A/I^q)$ is sandwiched between
a quotient module of $\text{Tor}_i(M[a_i^{c_1}], A/I^q)$
and a submodule of $\text{Tor}_{i + 1}(M/a_i^{c_1}M, A/I^q)$.
Hence we conclude $H_s$ holds with $N$ replaced by $N + c$
and $c$ replaced by $2c$. Some details omitted.
\end{proof}

\begin{proposition}
\label{proposition-uniform-artin-rees}
Let $I$ be an ideal of a Noetherian ring $A$. Let $t \geq 0$
be an upper bound on the number of generators of $I$.
There exist $N, c \geq 0$ such that for $n \geq N$ the maps
$$
A/I^n \to A/I^{n - c}
$$
satisfy the equivalent conditions of
Lemma \ref{lemma-characterize-vanishing-tor-ext-above-e} with $e = t$.
\end{proposition}

\begin{proof}
Immediate consequence of Lemmas
\ref{lemma-tor-maps-vanish} and
\ref{lemma-characterize-vanishing-tor-ext-above-e}.
\end{proof}

\begin{remark}
\label{remark-better-bound}
The paper \cite{AHS} shows, besides many other things, that if $A$ is local,
then Proposition \ref{proposition-uniform-artin-rees} also holds
with $e = t$ replaced by $e = \dim(A)$. Looking at
Lemma \ref{lemma-cd-sequence-Koszul} it is natural to ask whether
Proposition \ref{proposition-uniform-artin-rees}
holds with $e = t$ replaced with $e = \text{cd}(A, I)$. We don't know.
\end{remark}

\begin{remark}
\label{remark-strict-pro-isomorphism}
Let $I$ be an ideal of a Noetherian ring $A$. Say $I = (f_1, \ldots, f_r)$.
Denote $K_n^\bullet$ the Koszul complex on $f_1^n, \ldots, f_r^n$ as in
More on Algebra, Situation \ref{more-algebra-situation-koszul} and
denote $K_n \in D(A)$ the corresponding object.
Let $M^\bullet$ be a bounded complex of finite $A$-modules
and denote $M \in D(A)$ the corresponding object.
Consider the following inverse systems in $D(A)$:
\begin{enumerate}
\item $M^\bullet/I^nM^\bullet$, i.e., the complex whose terms are $M^i/I^nM^i$,
\item $M \otimes_A^\mathbf{L} A/I^n$,
\item $M \otimes_A^\mathbf{L} K_n$, and
\item $M \otimes_P^\mathbf{L} P/J^n$ (see below).
\end{enumerate}
All of these inverse systems are isomorphic as pro-objects:
the isomorphism between (2) and (3) follows from
More on Algebra, Lemma \ref{more-algebra-lemma-sequence-Koszul-complexes}.
The isomorphism between (1) and (2) is given in
More on Algebra, Lemma
\ref{more-algebra-lemma-derived-completion-plain-completion}.
For the last one, see below.

\medskip\noindent
However, we can ask if these isomorphisms of pro-systems are ``strict'';
this terminology and question is related to the discussion in
\cite[pages 61, 62]{quillenhomology}. Namely, given a category $\mathcal{C}$
we can define a ``strict pro-category'' whose objects are inverse systems
$(X_n)$ and whose morphisms $(X_n) \to (Y_n)$ are given by tuples
$(c, \varphi_n)$ consisting of a $c \geq 0$ and morphisms
$\varphi_n : X_n \to Y_{n - c}$ for all $n \geq c$ satisfying
an obvious compatibility condition and up to a certain equivalence
(given essentially by increasing $c$). Then we ask whether the above
inverse systems are isomorphic in this strict pro-category.

\medskip\noindent
This clearly cannot be the case for (1) and (3) even when $M = A[0]$.
Namely, the system $H^0(K_n) = A/(f_1^n, \ldots, f_r^n)$ is not strictly
pro-isomorphic in the category of modules to the system $A/I^n$ in general.
For example, if we take $A = \mathbf{Z}[x_1, \ldots, x_r]$ and $f_i = x_i$,
then $H^0(K_n)$ is not annihilated by $I^{r(n - 1)}$.\footnote{Of
course, we can ask whether these pro-systems are isomorphic in
a category whose objects are inverse systems and where maps are given
by tuples $(r, c, \varphi_n)$ consisting of $r \geq 1$, $c \geq 0$
and maps $\varphi_n : X_{rn} \to Y_{n - c}$ for $n \geq c$.}

\medskip\noindent
It turns out that the results above show that the natural map from
(2) to (1) discussed in More on Algebra, Lemma
\ref{more-algebra-lemma-derived-completion-plain-completion}
is a strict pro-isomorphism. We will sketch the proof.
Using standard arguments involving stupid truncations, we first reduce
to the case where $M^\bullet$ is given by a single finite $A$-module
$M$ placed in degree $0$. Pick $N, c \geq 0$ as in
Proposition \ref{proposition-uniform-artin-rees}.
The proposition implies that for $n \geq N$ we get factorizations
$$
M \otimes_A^\mathbf{L} A/I^n
\to
\tau_{\geq -t}(M \otimes_A^\mathbf{L} A/I^n)
\to
M \otimes_A^\mathbf{L} A/I^{n - c}
$$
of the transition maps in the system (2). On the other hand, by
More on Algebra, Lemma \ref{more-algebra-lemma-tor-strictly-pro-zero},
we can find another constant $c' = c'(M) \geq 0$ such that the maps
$\text{Tor}_i^A(M, A/I^{n'}) \to \text{Tor}_i(M, A/I^{n' - c'})$
are zero for $i = 1, 2, \ldots, t$ and $n' \geq c'$. Then it follows from
Derived Categories, Lemma \ref{derived-lemma-trick-vanishing-composition}
that the map
$$
\tau_{\geq -t}(M \otimes_A^\mathbf{L} A/I^{n + tc'})
\to
\tau_{\geq -t}(M \otimes_A^\mathbf{L} A/I^n)
$$
factors through $M \otimes_A^\mathbf{L}A/I^{n + tc'} \to M/I^{n + tc'}M$.
Combined with the previous result we obtain a factorization
$$
M \otimes_A^\mathbf{L}A/I^{n + tc'} \to M/I^{n + tc'}M
\to M \otimes_A^\mathbf{L} A/I^{n - c}
$$
which gives us what we want. If we ever need this result, we will carefully
state it and provide a detailed proof.

\medskip\noindent
For number (4) suppose we have a Noetherian ring $P$,
a ring homomorphism $P \to A$, and an ideal $J \subset P$ such that $I = JA$.
By More on Algebra, Section \ref{more-algebra-section-derived-base-change}
we get a functor $M \otimes_P^\mathbf{L} - : D(P) \to D(A)$ and we get
an inverse system $M \otimes_P^\mathbf{L} P/J^n$ in $D(A)$ as in (4).
If $P$ is Noetherian, then the system in (4) is pro-isomorphic
to the system in (1) because we can compare with Koszul complexes.
If $P \to A$ is finite, then the system (4) is strictly pro-isomorphic
to the system (2) because the inverse system $A \otimes_P^\mathbf{L} P/J^n$
is strictly pro-isomorphic to the inverse system $A/I^n$
(by the discussion above) and because we have
$$
M \otimes_P^\mathbf{L} P/J^n = M \otimes_A^\mathbf{L}
(A \otimes_P^\mathbf{L} P/J^n)
$$
by More on Algebra, Lemma \ref{more-algebra-lemma-derived-base-change}.

\medskip\noindent
A standard example in (4) is to take $P = \mathbf{Z}[x_1, \ldots, x_r]$,
the map $P \to A$ sending $x_i$ to $f_i$, and $J = (x_1, \ldots, x_r)$.
In this case one shows that
$$
M \otimes_P^\mathbf{L} P/J^n =
M \otimes_{A[x_1, \ldots, x_r]}^\mathbf{L}
A[x_1, \ldots, x_r]/(x_1, \ldots, x_r)^n
$$
and we reduce to one of the cases discussed above (although this case
is strictly easier as $A[x_1, \ldots, x_r]/(x_1, \ldots, x_r)^n$ has
tor dimension at most $r$ for all $n$ and hence the step using
Proposition \ref{proposition-uniform-artin-rees} can be avoided).
This case is discussed in the proof of \cite[Proposition 3.5.1]{BS}.
\end{remark}








\section{A bit of uniformity, III}
\label{section-uniform}

\noindent
In this section we fix a Noetherian ring $A$ and an ideal $I \subset A$.
Our goal is to prove Lemma \ref{lemma-bound-two-term-complex} which we will
use in a later chapter to solve a lifting problem, see
Restricted Power Series, Lemma
\ref{restricted-lemma-get-morphism-general-better}.

\medskip\noindent
Throughout this section we denote
$$
p : X \to \Spec(A)
$$
the blowing up of $\Spec(A)$ in the ideal $I$. In other words, $X$ is the
$\text{Proj}$ of the Rees algebra $\bigoplus_{n \geq 0} I^n$. We also consider
the fibre product
$$
\xymatrix{
Y \ar[r] \ar[d] & X \ar[d]^p \\
\Spec(A/I) \ar[r] & \Spec(A)
}
$$
Then $Y$ is the exceptional divisor of the blowup and hence
an effective Cartier divisor on $X$ such that
$\mathcal{O}_X(-1) = \mathcal{O}_X(Y)$. Since taking $\text{Proj}$
commutes with base change we have
$$
Y = \text{Proj}(\bigoplus\nolimits_{n \geq 0} I^n/I^{n + 1}) = \text{Proj}(S)
$$
where $S = \text{Gr}_I(A) = \bigoplus_{n \geq 0} I^n/I^{n + 1}$.

\medskip\noindent
We denote
$d = d(S) = d(\text{Gr}_I(A)) = d(\bigoplus_{n \geq 0} I^n/I^{n + 1})$
the maximum of the dimensions of the fibres of $p$
(and we set it equal to $0$ if $X = \emptyset$).
This is well defined. In fact, we have
\begin{enumerate}
\item $d \leq t - 1$ if $I = (a_1, \ldots, a_t)$ since then
$X \subset \mathbf{P}^{t - 1}_A$, and
\item $d$ is also the maximal dimension of the fibres of
$\text{Proj}(S) \to \Spec(S_0)$ provided that $Y$
is nonempty and $d = 0$ if $Y = \emptyset$
(equivalently $S = 0$, equivalently $I = A$).
\end{enumerate}
Hence $d$ only depends on the isomorphism class of $S = \text{Gr}_I(A)$.
Observe that $H^i(X, \mathcal{F}) = 0$ for every coherent
$\mathcal{O}_X$-module $\mathcal{F}$ and $i > d$ by
Cohomology of Schemes, Lemmas
\ref{coherent-lemma-higher-direct-images-zero-above-dimension-fibre} and
\ref{coherent-lemma-quasi-coherence-higher-direct-images-application}.
Of course the same is true for coherent modules on $Y$.

\medskip\noindent
We denote
$d = d(S) = d(\text{Gr}_I(A)) = d(\bigoplus_{n \geq 0} I^n/I^{n + 1})$
the integer defined as follows. Note that the algebra
$S = \bigoplus_{n \geq 0} I^n/I^{n + 1}$
is a Noetherian graded ring generated in degree $1$ over degree $0$.
Hence by
Cohomology of Schemes, Lemmas \ref{coherent-lemma-coherent-on-proj} and
\ref{coherent-lemma-recover-tail-graded-module} we can define $q(S)$
as the smallest integer $q(S) \geq 0$ such that for all $q \geq q(S)$ we have
$H^i(Y, \mathcal{O}_Y(q)) = 0$ for $1 \leq i \leq d$ and
$H^0(Y, \mathcal{O}_Y(q)) = I^q/I^{q + 1}$.
(If $S = 0$, then $q(S) = 0$.)

\medskip\noindent
For $n \geq 1$ we may consider the effective Cartier divisor $nY$
which we will denote $Y_n$.

\begin{lemma}
\label{lemma-bound-q-and-d}
With $q_0 = q(S)$ and $d = d(S)$ as above, we have
\begin{enumerate}
\item for $n \geq 1$, $q \geq q_0$, and $i > 0$ we have
$H^i(X, \mathcal{O}_{Y_n}(q)) = 0$,
\item for $n \geq 1$ and $q \geq q_0$ we have
$H^0(X, \mathcal{O}_{Y_n}(q)) = I^q/I^{q + n}$,
\item for $q \geq q_0$ and $i > 0$ we have
$H^i(X, \mathcal{O}_X(q)) = 0$,
\item for $q \geq q_0$ we have
$H^0(X, \mathcal{O}_X(q)) = I^q$.
\end{enumerate}
\end{lemma}

\begin{proof}
If $I = A$, then $X$ is affine and the statements are trivial.
Hence we may and do assume $I \not = A$.
Thus $Y$ and $X$ are nonempty schemes.

\medskip\noindent
Let us prove (1) and (2) by induction on $n$. The base case $n = 1$
is our definition of $q_0$ as $Y_1 = Y$.
Recall that $\mathcal{O}_X(1) = \mathcal{O}_X(-Y)$.
Hence we have a short exact sequence
$$
0 \to \mathcal{O}_{Y_n}(1) \to \mathcal{O}_{Y_{n + 1}} \to \mathcal{O}_Y \to 0
$$
Hence for $i > 0$ we find
$$
H^i(X, \mathcal{O}_{Y_n}(q + 1)) \to
H^i(X, \mathcal{O}_{Y_{n + 1}}(q)) \to
H^i(X, \mathcal{O}_{Y}(q))
$$
and we obtain the desired vanishing of the middle term from the given
vanishing of the outer terms. For $i = 0$ we obtain a commutative diagram
$$
\xymatrix{
0 \ar[r] &
I^{q + 1}/I^{q + 1 + n} \ar[d] \ar[r] &
I^q/I^{q + 1 + n} \ar[d] \ar[r] &
I^q/I^{q + 1} \ar[d] \ar[r] &
0 \\
0 \ar[r] &
H^0(X, \mathcal{O}_{Y_n}(q + 1)) \ar[r] &
H^0(X, \mathcal{O}_{Y_{n + 1}}(q)) \ar[r] &
H^0(Y, \mathcal{O}_Y(q)) \ar[r] &
0
}
$$
with exact rows for $q \geq q_0$ (for the bottom row observe that
the next term in the long exact cohomology sequence vanishes for
$q \geq q_0$). Since $q \geq q_0$ the left and right vertical arrows
are isomorphisms and we conclude the middle one is too.

\medskip\noindent
We omit the proofs of (3) and (4) which are similar.
In fact, one can deduce (3) and (4) from (1) and (2)
using the theorem on formal functors (but this would be
overkill).
\end{proof}

\noindent
Let us introduce a notation: given $n \geq c \geq 0$
{\it an $(A, n, c)$-module}
is a finite $A$-module $M$ which is annihilated by $I^n$
and which as an $A/I^n$-module is $I^c/I^n$-projective, see
More on Algebra, Section \ref{more-algebra-section-near-projective}.

\medskip\noindent
We will use the following abuse of notation: given an $A$-module $M$
we denote $p^*M$ the quasi-coherent module gotten by pulling back by $p$
the quasi-coherent module $\widetilde{M}$ on $\Spec(A)$ associated to $M$.
For example we have $\mathcal{O}_{Y_n} = p^*(A/I^n)$.
For a short exact sequence $0 \to K \to L \to M \to 0$ of $A$-modules
we obtain an exact sequence
$$
p^*K \to p^*L \to p^*M \to 0
$$
as $\widetilde{\ }$ is an exact functor and $p^*$ is a right exact functor.

\begin{lemma}
\label{lemma-almost-exactness}
Let $0 \to K \to L \to M \to 0$ be a short exact sequence of $A$-modules
such that $K$ and $L$ are annihilated by $I^n$ and $M$ is an
$(A, n, c)$-module. Then the kernel of $p^*K \to p^*L$
is scheme theoretically supported on $Y_c$.
\end{lemma}

\begin{proof}
Let $\Spec(B) \subset X$ be an affine open. The restriction of the exact
sequence over $\Spec(B)$ corresponds to the sequence of $B$-modules
$$
K \otimes_A B \to L \otimes_A B \to M \otimes_A B \to 0
$$
which is isomorphismic to the sequence
$$
K \otimes_{A/I^n} B/I^nB \to
L \otimes_{A/I^n} B/I^nB \to
M \otimes_{A/I^n} B/I^nB \to 0
$$
Hence the kernel of the first map is the image of the module
$\text{Tor}_1^{A/I^n}(M, B/I^nB)$. Recall that the exceptional
divisor $Y$ is cut out by $I\mathcal{O}_X$. 
Hence it suffices to show that
$\text{Tor}_1^{A/I^n}(M, B/I^nB)$ is annihilated by $I^c$. Since
multiplication by $a \in I^c$ on $M$ factors through a finite
free $A/I^n$-module, this is clear.
\end{proof}

\noindent
We have the canonical map $\mathcal{O}_X \to \mathcal{O}_X(1)$
which vanishes exactly along $Y$.
Hence for every coherent $\mathcal{O}_X$-module $\mathcal{F}$
we always have canonical maps
$\mathcal{F}(q) \to \mathcal{F}(q + n)$ for any $q \in \mathbf{Z}$
and $n \geq 0$.

\begin{lemma}
\label{lemma-annihilated}
Let $\mathcal{F}$ be a coherent $\mathcal{O}_X$-module.
Then $\mathcal{F}$ is scheme theoretically
supported on $Y_c$ if and only if the canonical map
$\mathcal{F} \to \mathcal{F}(c)$ is zero.
\end{lemma}

\begin{proof}
This is true because $\mathcal{O}_X \to \mathcal{O}_X(1)$
vanishes exactly along $Y$.
\end{proof}

\begin{lemma}
\label{lemma-vanishing-coh-almost-projective}
With $q_0 = q(S)$ and $d = d(S)$ as above, suppose we have
integers $n \geq c \geq 0$, an $(A, n, c)$-module $M$,
an index $i \in \{0, 1, \ldots, d\}$, and an integer $q$.
Then we distinguish the following cases
\begin{enumerate}
\item In the case $i = d \geq 1$ and $q \geq q_0$ we have
$H^d(X, p^*M(q)) = 0$.
\item In the case $i = d - 1 \geq 1$ and $q \geq q_0$ we have
$H^{d - 1}(X, p^*M(q)) = 0$.
\item In the case $d - 1 > i > 0$ and $q \geq q_0 + (d - 1 - i)c$
the map
$H^i(X, p^*M(q)) \to H^i(X, p^*M(q - (d - 1 - i)c))$
is zero.
\item In the case $i = 0$, $d \in \{0, 1\}$, and $q \geq q_0$, there
is a surjection
$$
I^qM \longrightarrow H^0(X, p^*M(q))
$$
\item In the case $i = 0$, $d > 1$, and $q \geq q_0 + (d - 1)c$ the map
$$
H^0(X, p^*M(q)) \to H^0(X, p^*M(q - (d - 1)c))
$$
has image contained in the image of the canonical map
$I^{q - (d - 1)c}M \to H^0(X, p^*M(q - (d - 1)c))$.
\end{enumerate}
\end{lemma}

\begin{proof}
Let $M$ be an $(A, n, c)$-module. Choose a short exact sequence
$$
0 \to K \to (A/I^n)^{\oplus r} \to M \to 0
$$
We will use below that $K$ is an $(A, n, c)$-module, see More on Algebra,
Lemma \ref{more-algebra-lemma-ses-near-projective}.
Consider the corresponding exact sequence
$$
p^*K \to (\mathcal{O}_{Y_n})^{\oplus r} \to p^*M \to 0
$$
We split this into short exact sequences
$$
0 \to \mathcal{F} \to p^*K \to \mathcal{G} \to 0
\quad\text{and}\quad
0 \to \mathcal{G} \to (\mathcal{O}_{Y_n})^{\oplus r} \to p^*M \to 0
$$
By Lemma \ref{lemma-almost-exactness} the coherent module $\mathcal{F}$
is scheme theoretically supported on $Y_c$.

\medskip\noindent
Proof of (1). Assume $d > 0$. We have to prove
$H^d(X, p^*M(q)) = 0$ for $q \geq q_0$.
By the vanishing of the cohomology of twists of $\mathcal{G}$ in degrees $> d$
and the long exact cohomology sequence associated to the second
short exact sequence above, it suffices to prove that
$H^d(X, \mathcal{O}_{Y_n}(q)) = 0$.
This is true by Lemma \ref{lemma-bound-q-and-d}.

\medskip\noindent
Proof of (2). Assume $d > 1$. We have to prove
$H^{d - 1}(X, p^*M(q)) = 0$ for $q \geq q_0$.
Arguing as in the previous paragraph, we see that it suffices
to show that $H^d(X, \mathcal{G}(q)) = 0$. Using the first
short exact sequence and the vanishing of the cohomology
of twists of $\mathcal{F}$ in degrees $> d$ we see that it suffices
to show $H^d(X, p^*K(q))$ is zero which is
true by (1) and the fact that $K$ is an $(A, n, c)$-module (see above).

\medskip\noindent
Proof of (3). Let $0 < i < d - 1$ and assume the statement holds for $i + 1$
except in the case $i = d - 2$ we have statement (2).
Using the long exact sequence of cohomology associated to the second
short exact sequence above we find an injection
$$
H^i(X, p^*M(q - (d - 1 - i)c)) \subset
H^{i + 1}(X, \mathcal{G}(q - (d - 1 - i)c))
$$
as $q - (d - 1 - i)c \geq q_0$ gives the vanishing of
$H^i(X, \mathcal{O}_{Y_n}(q - (d - 1 - i)c))$
(see above). Thus it suffices to show that the map
$H^{i + 1}(X, \mathcal{G}(q)) \to H^{i + 1}(X, \mathcal{G}(q - (d - 1 - i)c))$
is zero. To study this, we consider the maps of exact sequences
$$
\xymatrix{
H^{i + 1}(X, p^*K(q)) \ar[r] \ar[d] &
H^{i + 1}(X, \mathcal{G}(q)) \ar[r] \ar[d] \ar@{..>}[ld] &
H^{i + 2}(X, \mathcal{F}(q)) \ar[d] \\
H^{i + 1}(X, p^*K(q - c)) \ar[r] \ar[d] &
H^{i + 1}(X, \mathcal{G}(q - c)) \ar[r] \ar[d] &
H^{i + 2}(X, \mathcal{F}(q - c)) \\
H^{i + 1}(X, p^*K(q - (d - 1 - i)c)) \ar[r] &
H^{i + 1}(X, \mathcal{G}(q - (d - 1 - i)c))
}
$$
Since $\mathcal{F}$ is scheme theoretically supported on $Y_c$
we see that the canonical map
$\mathcal{G}(q) \to \mathcal{G}(q - c)$ factors through
$p^*K(q - c)$ by Lemma \ref{lemma-annihilated}.
This gives the dotted arrow in the diagram. (In fact, for the proof it
suffices to observe that the vertical arrow on the extreme right is
zero in order to get the dotted arrow as a map of sets.)
Thus it suffices to show that
$H^{i + 1}(X, p^*K(q - c)) \to
H^{i + 1}(X, p^*K(q - (d - 1 - i)c))$
is zero. If $i = d - 2$, then the source of this arrow
is zero by (2) as $q - c \geq q_0$ and $K$ is an $(A, n, c)$-module.
If $i < d - 2$, then as $K$ is an $(A, n, c)$-module, we get from the
induction hypothesis that the map is indeed zero
since $q - c - (q - (d - 1 - i)c) = (d - 2 - i)c = (d - 1 - (i + 1))c$
and since $q - c \geq q_0 + (d - 1 - (i + 1))c$.
In this way we conclude the proof of (3).

\medskip\noindent
Proof of (4). Assume $d \in \{0, 1\}$ and $q \geq q_0$.
Then the first short exact sequence gives a surjection
$H^1(X, p^*K(q)) \to H^1(X, \mathcal{G}(q))$
and the source of this arrow is zero by case (1). Hence
for all $q \in \mathbf{Z}$ we see that the map
$$
H^0(X, (\mathcal{O}_{Y_n})^{\oplus r}(q))
\longrightarrow
H^0(X, p^*M(q))
$$
is surjective. For $q \geq q_0$ the
source is equal to $(I^q/I^{q + n})^{\oplus r}$ by
Lemma \ref{lemma-bound-q-and-d} and this easily proves the statement.

\medskip\noindent
Proof of (5). Assume $d > 1$. Arguing as in the proof of (4) we see that
it suffices to show that the image of
$$
H^0(X, p^*M(q))
\longrightarrow
H^0(X, p^*M(q - (d - 1)c))
$$
is contained in the image of
$$
H^0(X, (\mathcal{O}_{Y_n})^{\oplus r}(q - (d - 1)c))
\longrightarrow
H^0(X, p^*M(q - (d - 1)c))
$$
To show the inclusion above, it suffices to show that for
$\sigma \in H^0(X, p^*M(q))$ with boundary
$\xi \in H^1(X, \mathcal{G}(q))$ the image of $\xi$ in
$H^1(X, \mathcal{G}(q - (d - 1)c))$ is zero. This follows by the
exact same arguments as in the proof of (3).
\end{proof}

\begin{remark}
\label{remark-duals}
Given a pair $(M, n)$ consisting of an integer $n \geq 0$
and a finite $A/I^n$-module $M$ we set $M^\vee = \Hom_{A/I^n}(M, A/I^n)$.
Given a pair $(\mathcal{F}, n)$ consisting of an integer $n$ and
a coherent $\mathcal{O}_{Y_n}$-module $\mathcal{F}$ we set
$$
\mathcal{F}^\vee =
\SheafHom_{\mathcal{O}_{Y_n}}(\mathcal{F}, \mathcal{O}_{Y_n})
$$
Given $(M, n)$ as above, there is a canonical map
$$
can : p^*(M^\vee) \longrightarrow (p^*M)^\vee
$$
Namely, if we choose a presentation
$(A/I^n)^{\oplus s} \to (A/I^n)^{\oplus r} \to M \to 0$
then we obtain a presentation
$\mathcal{O}_{Y_n}^{\oplus s} \to \mathcal{O}_{Y_n}^{\oplus r} \to
p^*M \to 0$. Taking duals we obtain exact sequences
$$
0 \to M^\vee \to (A/I^n)^{\oplus r} \to (A/I^n)^{\oplus s}
$$
and
$$
0 \to (p^*M)^\vee \to
\mathcal{O}_{Y_n}^{\oplus r} \to
\mathcal{O}_{Y_n}^{\oplus s}
$$
Pulling back the first sequence by $p$ we find the desired map $can$.
The construction of this map is functorial in the finite
$A/I^n$-module $M$. The kernel and cokernel of $can$
are scheme theoretically supported
on $Y_c$ if $M$ is an $(A, n, c)$-module. Namely, in that case for
$a \in I^c$ the map $a : M \to M$ factors through a finite free
$A/I^n$-module for which $can$ is an isomorphism. Hence $a$ annihilates
the kernel and cokernel of $can$.
\end{remark}

\begin{lemma}
\label{lemma-factor-hom}
With $q_0 = q(S)$ and $d = d(S)$ as above, let $M$ be an $(A, n, c)$-module
and let $\varphi : M \to I^n/I^{2n}$ be an $A$-linear map. Assume
$n \geq \max(q_0 + (1 + d)c, (2 + d)c)$ and if $d = 0$ assume
$n \geq q_0 + 2c$. Then the composition
$$
M \xrightarrow{\varphi} I^n/I^{2n} \to
I^{n - (1 + d)c}/I^{2n - (1 + d)c}
$$
is of the form $\sum a_i \psi_i$ with $a_i \in I^c$ and
$\psi_i : M \to I^{n - (2 + d)c}/I^{2n - (2 + d)c}$.
\end{lemma}

\begin{proof}
The case $d > 1$. Since we have a compatible system of maps
$p^*(I^q) \to \mathcal{O}_X(q)$ for $q \geq 0$ there are canonical maps
$p^*(I^q/I^{q + \nu}) \to \mathcal{O}_{Y_\nu}(q)$ for $\nu \geq 0$.
Using this and pulling back $\varphi$ we obtain a map
$$
\chi : p^*M \longrightarrow \mathcal{O}_{Y_n}(n)
$$
such that the composition
$M \to H^0(X, p^*M) \to H^0(X, \mathcal{O}_{Y_n}(n))$
is the given homomorphism $\varphi$ combined with the
map $I^n/I^{2n} \to H^0(X, \mathcal{O}_{Y_n}(n))$.
Since $\mathcal{O}_{Y_n}(n)$ is invertible on $Y_n$ the
linear map $\chi$ determines a section
$$
\sigma \in \Gamma(X, (p^*M)^\vee(n))
$$
with notation as in Remark \ref{remark-duals}.
The discussion in Remark \ref{remark-duals} shows
the cokernel and kernel of $can : p^*(M^\vee) \to (p^*M)^\vee$
are scheme theoretically supported on $Y_c$. By
Lemma \ref{lemma-annihilated} the map
$(p^*M)^\vee(n) \to (p^*M)^\vee(n - 2c)$ factors
through $p^*(M^\vee)(n - 2c)$; small detail omitted.
Hence the image of $\sigma$ in $\Gamma(X, (p^*M)^\vee(n - 2c))$
comes from an element
$$
\sigma' \in \Gamma(X, p^*(M^\vee)(n - 2c))
$$
By Lemma \ref{lemma-vanishing-coh-almost-projective} part (5),
the fact that $M^\vee$ is an $(A, n, c)$-module by
More on Algebra, Lemma \ref{more-algebra-lemma-dual-near-projective},
and the fact that $n \geq q_0 + (1 + d)c$ so $n - 2c \geq q_0 + (d - 1)c$
we see that the image of $\sigma'$ in $H^0(X, p^*M^\vee(n - (1 + d)c))$
is the image of an element $\tau$ in $I^{n - (1 + d)c}M^\vee$.
Write $\tau = \sum a_i \tau_i$ with $\tau_i \in I^{n - (2 + d)c}M^\vee$;
this makes sense as $n - (2 + d)c \geq 0$.
Then $\tau_i$ determines a homomorphism of modules
$\psi_i : M \to I^{n - (2 + d)c}/I^{2n - (2 + d)c}$
using the evaluation map $M \otimes M^\vee \to A/I^n$.

\medskip\noindent
Let us prove that this works\footnote{We hope some reader will suggest
a less dirty proof of this fact.}. Pick $z \in M$ and let us show that
$\varphi(z)$ and $\sum a_i \psi_i(z)$ have the same image in
$I^{n - (1 + d)c}/I^{2n - (1 + d)c}$.
First, the element $z$ determines a map
$p^*z : \mathcal{O}_{Y_n} \to p^*M$ whose composition with
$\chi$ is equal to the map $\mathcal{O}_{Y_n} \to \mathcal{O}_{Y_n}(n)$
corresponding to $\varphi(z)$ via the map
$I^n/I^{2n} \to \Gamma(\mathcal{O}_{Y_n}(n))$.
Next $z$ and $p^*z$ determine evaluation maps
$e_z : M^\vee \to A/I^n$ and $e_{p^*z} : (p^*M)^\vee \to \mathcal{O}_{Y_n}$.
Since $\chi(p^*z)$ is the section corresponding to $\varphi(z)$
we see that $e_{p^*z}(\sigma)$ is the section corresponding to $\varphi(z)$.
Here and below we abuse notation: for a map
$a : \mathcal{F} \to \mathcal{G}$ of modules on $X$
we also denote $a : \mathcal{F}(t) \to \mathcal{F}(t)$ the corresponding
map of twisted modules. The diagram
$$
\xymatrix{
p^*(M^\vee) \ar[d]_{can} \ar[r]_{p^*e_z} & \mathcal{O}_{Y_n} \ar@{=}[d] \\
(p^*M)^\vee \ar[r]^{e_{p^*z}} & \mathcal{O}_{Y_n}
}
$$
commutes by functoriality of the construction $can$. Hence
$(p^*e_z)(\sigma')$ in $\Gamma(Y_n, \mathcal{O}_{Y_n}(n - 2c))$
is the section corresponding to the image of $\varphi(z)$
in $I^{n - 2c}/I^{2n - 2c}$.
The next step is that $\sigma'$ maps to the image
of $\sum a_i \tau_i$ in $H^0(X, p^*M^\vee(n - (1 + d)c))$.
This implies that $(p^*e_z)(\sum a_i \tau_i) = \sum a_i p^*e_z(\tau_i)$
in $\Gamma(Y_n, \mathcal{O}_{Y_n}(n - (1 + d)c))$ is the section corresponding
to the image of $\varphi(z)$ in $I^{n - (1 + d)c}/I^{2n - (1 + d)c}$.
Recall that $\psi_i$ is defined from $\tau_i$ using an evaluation
map. Hence if we denote
$$
\chi_i : p^*M \longrightarrow \mathcal{O}_{Y_n}(n - (2 + d)c)
$$
the map we get from $\psi_i$, then we see by the same reasoning
as above that the section corresponding to $\psi_i(z)$ is
$\chi_i(p^*z) = e_{p^*z}(\chi_i) = p^*e_z(\tau_i)$. Hence we conclude that
the image of $\varphi(z)$ in $\Gamma(Y_n, \mathcal{O}_{Y_n}(n - (1 + d)c))$
is equal to the image of $\sum a_i\psi_i(z)$.
Since $n - (1 + d)c \geq q_0$ we have
$\Gamma(Y_n, \mathcal{O}_{Y_n}(n - (1 + d)c)) =
I^{n - (1 + d)c}/I^{2n - (1 + d)c}$ by Lemma \ref{lemma-bound-q-and-d} and
we conclude the desired compatibility is true.

\medskip\noindent
The case $d = 1$. Here we argue as above that we get
$$
\chi : p^*M \longrightarrow \mathcal{O}_{Y_n}(n),\quad
\sigma \in \Gamma(X, (p^*M)^\vee(n)),\quad
\sigma' \in \Gamma(X, p^*(M^\vee)(n - 2c)),
$$
and then we use
Lemma \ref{lemma-vanishing-coh-almost-projective} part (4)
to see that $\sigma'$ is the image of some element
$\tau \in I^{n - 2c}M^\vee$. The rest of the argument is the same.

\medskip\noindent
The case $d = 0$. Argument is exactly the same as in the
case $d = 1$.
\end{proof}

\begin{lemma}
\label{lemma-bound-two-term-complex}
With $d = d(S)$ and $q_0 = q(S)$ as above. Then
\begin{enumerate}
\item for integers $n \geq c \geq 0$ with
$n \geq \max(q_0 + (1 + d)c, (2 + d)c)$,
\item for $K$ of $D(A/I^n)$ with $H^i(K) = 0$ for $i \not = -1, 0$
and $H^i(K)$ finite for $i = -1, 0$ such that $\Ext^1_{A/I^c}(K, N)$
is annihilated by $I^c$ for all finite $A/I^c$-modules $N$
\end{enumerate}
the map
$$
\Ext^1_{A/I^n}(K, I^n/I^{2n})
\longrightarrow
\Ext^1_{A/I^n}(K, I^{n - (1 + d)c}/I^{2n - 2(1 + d)c})
$$
is zero.
\end{lemma}

\begin{proof}
The case $d > 0$. Let $K^{-1} \to K^0$ be a complex representing $K$ as in
More on Algebra, Lemma \ref{more-algebra-lemma-ext-1-annihilated-definite}
part (5) with respect to the ideal $I^c/I^n$ in the ring $A/I^n$.
In particular $K^{-1}$ is $I^c/I^n$-projective as multiplication
by elements of $I^c/I^n$ even factor through $K^0$. By
More on Algebra, Lemma \ref{more-algebra-lemma-map-out-of-almost-free} part (1)
we have
$$
\Ext^1_{A/I^n}(K, I^n/I^{2n}) =
\Coker(\Hom_{A/I^n}(K^0, I^n/I^{2n}) \to \Hom_{A/I^n}(K^{-1}, I^n/I^{2n}))
$$
and similarly for other Ext groups. Hence any class $\xi$ in
$\Ext^1_{A/I^n}(K, I^n/I^{2n})$
comes from an element $\varphi \in \Hom_{A/I^n}(K^{-1}, I^n/I^{2n})$.
Denote $\varphi'$ the image of $\varphi$ in
$\Hom_{A/I^n}(K^{-1}, I^{n - (1 + d)c}/I^{2n - (1 + d)c})$.
By Lemma \ref{lemma-factor-hom}
we can write $\varphi' = \sum a_i \psi_i$ with $a_i \in I^c$ and
$\psi_i \in \Hom_{A/I^n}(M, I^{n - (2 + d)c}/I^{2n - (2 + d)c})$.
Choose $h_i : K^0 \to K^{-1}$ such that
$a_i \text{id}_{K^{-1}} = h_i \circ d_K^{-1}$. Set
$\psi = \sum \psi_i \circ h_i : K^0 \to I^{n - (2 + d)c}/I^{2n - (2 + d)c}$.
Then $\varphi' = \psi \circ \text{d}_K^{-1}$ and we conclude that
$\xi$ already maps to zero in
$\Ext^1_{A/I^n}(K, I^{n - (1 + d)c}/I^{2n - (1 + d)c})$
and a fortiori in
$\Ext^1_{A/I^n}(K, I^{n - (1 + d)c}/I^{2n - 2(1 + d)c})$.

\medskip\noindent
The case $d = 0$\footnote{The argument given for $d > 0$ works but gives
a slightly weaker result.}. Let $\xi$ and $\varphi$ be as above.
We consider the diagram
$$
\xymatrix{
K^0 \\
K^{-1} \ar[u] \ar[r]^\varphi & I^n/I^{2n} \ar[r] & I^{n - c}/I^{2n - c}
}
$$
Pulling back to $X$ and using the map
$p^*(I^n/I^{2n}) \to \mathcal{O}_{Y_n}(n)$
we find a solid diagram
$$
\xymatrix{
p^*K^0 \ar@{..>}[rrd] \\
p^*K^{-1} \ar[u] \ar[r] &
\mathcal{O}_{Y_n}(n) \ar[r] &
\mathcal{O}_{Y_n}(n - c)
}
$$
We can cover $X$ by affine opens $U = \Spec(B)$ such that there
exists an $a \in I$ with the following property: $IB = aB$
and $a$ is a nonzerodivisor on $B$. Namely, we can cover
$X$ by spectra of affine blowup algebras, see
Divisors, Lemma \ref{divisors-lemma-blowing-up-affine}.
The restriction of $\mathcal{O}_{Y_n}(n) \to \mathcal{O}_{Y_n}(n - c)$
to $U$ is isomorphic to the map of quasi-coherent $\mathcal{O}_U$-modules
corresponding to the $B$-module map $a^c : B/a^nB \to B/a^nB$.
Since $a^c : K^{-1} \to K^{-1}$ factors through $K^0$ we see that
the dotted arrow exists over $U$. In other words,
locally on $X$ we can find the dotted arrow! Now the sheaf of dotted
arrows fitting into the diagram is principal homogeneous under
$$
\mathcal{F} = \SheafHom_{\mathcal{O}_X}(
\Coker(p^*K^{-1} \to p^*K^0), \mathcal{O}_{Y_n}(n - c))
$$
which is a coherent $\mathcal{O}_X$-module.
Hence the obstruction for finding the dotted arrow is an element
of $H^1(X, \mathcal{F})$. This cohomology group is zero as
$1 > d = 0$, see discussion following the definition of $d = d(S)$.
This proves that we can find a dotted arrow
$\psi : p^*K^0 \to \mathcal{O}_{Y_n}(n - c)$
fitting into the diagram. Since $n - c \geq q_0$
we find that $\psi$ induces a map $K^0 \to I^{n - c}/I^{2n - c}$.
Chasing the diagram we conclude that $\varphi' = \psi \circ \text{d}_K^{-1}$
and the proof is finished as before.
\end{proof}






\begin{multicols}{2}[\section{Other chapters}]
\noindent
Preliminaries
\begin{enumerate}
\item \hyperref[introduction-section-phantom]{Introduction}
\item \hyperref[conventions-section-phantom]{Conventions}
\item \hyperref[sets-section-phantom]{Set Theory}
\item \hyperref[categories-section-phantom]{Categories}
\item \hyperref[topology-section-phantom]{Topology}
\item \hyperref[sheaves-section-phantom]{Sheaves on Spaces}
\item \hyperref[sites-section-phantom]{Sites and Sheaves}
\item \hyperref[stacks-section-phantom]{Stacks}
\item \hyperref[fields-section-phantom]{Fields}
\item \hyperref[algebra-section-phantom]{Commutative Algebra}
\item \hyperref[brauer-section-phantom]{Brauer Groups}
\item \hyperref[homology-section-phantom]{Homological Algebra}
\item \hyperref[derived-section-phantom]{Derived Categories}
\item \hyperref[simplicial-section-phantom]{Simplicial Methods}
\item \hyperref[more-algebra-section-phantom]{More on Algebra}
\item \hyperref[smoothing-section-phantom]{Smoothing Ring Maps}
\item \hyperref[modules-section-phantom]{Sheaves of Modules}
\item \hyperref[sites-modules-section-phantom]{Modules on Sites}
\item \hyperref[injectives-section-phantom]{Injectives}
\item \hyperref[cohomology-section-phantom]{Cohomology of Sheaves}
\item \hyperref[sites-cohomology-section-phantom]{Cohomology on Sites}
\item \hyperref[dga-section-phantom]{Differential Graded Algebra}
\item \hyperref[dpa-section-phantom]{Divided Power Algebra}
\item \hyperref[sdga-section-phantom]{Differential Graded Sheaves}
\item \hyperref[hypercovering-section-phantom]{Hypercoverings}
\end{enumerate}
Schemes
\begin{enumerate}
\setcounter{enumi}{25}
\item \hyperref[schemes-section-phantom]{Schemes}
\item \hyperref[constructions-section-phantom]{Constructions of Schemes}
\item \hyperref[properties-section-phantom]{Properties of Schemes}
\item \hyperref[morphisms-section-phantom]{Morphisms of Schemes}
\item \hyperref[coherent-section-phantom]{Cohomology of Schemes}
\item \hyperref[divisors-section-phantom]{Divisors}
\item \hyperref[limits-section-phantom]{Limits of Schemes}
\item \hyperref[varieties-section-phantom]{Varieties}
\item \hyperref[topologies-section-phantom]{Topologies on Schemes}
\item \hyperref[descent-section-phantom]{Descent}
\item \hyperref[perfect-section-phantom]{Derived Categories of Schemes}
\item \hyperref[more-morphisms-section-phantom]{More on Morphisms}
\item \hyperref[flat-section-phantom]{More on Flatness}
\item \hyperref[groupoids-section-phantom]{Groupoid Schemes}
\item \hyperref[more-groupoids-section-phantom]{More on Groupoid Schemes}
\item \hyperref[etale-section-phantom]{\'Etale Morphisms of Schemes}
\end{enumerate}
Topics in Scheme Theory
\begin{enumerate}
\setcounter{enumi}{41}
\item \hyperref[chow-section-phantom]{Chow Homology}
\item \hyperref[intersection-section-phantom]{Intersection Theory}
\item \hyperref[pic-section-phantom]{Picard Schemes of Curves}
\item \hyperref[weil-section-phantom]{Weil Cohomology Theories}
\item \hyperref[adequate-section-phantom]{Adequate Modules}
\item \hyperref[dualizing-section-phantom]{Dualizing Complexes}
\item \hyperref[duality-section-phantom]{Duality for Schemes}
\item \hyperref[discriminant-section-phantom]{Discriminants and Differents}
\item \hyperref[derham-section-phantom]{de Rham Cohomology}
\item \hyperref[local-cohomology-section-phantom]{Local Cohomology}
\item \hyperref[algebraization-section-phantom]{Algebraic and Formal Geometry}
\item \hyperref[curves-section-phantom]{Algebraic Curves}
\item \hyperref[resolve-section-phantom]{Resolution of Surfaces}
\item \hyperref[models-section-phantom]{Semistable Reduction}
\item \hyperref[equiv-section-phantom]{Derived Categories of Varieties}
\item \hyperref[pione-section-phantom]{Fundamental Groups of Schemes}
\item \hyperref[etale-cohomology-section-phantom]{\'Etale Cohomology}
\item \hyperref[crystalline-section-phantom]{Crystalline Cohomology}
\item \hyperref[proetale-section-phantom]{Pro-\'etale Cohomology}
\item \hyperref[more-etale-section-phantom]{More \'Etale Cohomology}
\item \hyperref[trace-section-phantom]{The Trace Formula}
\end{enumerate}
Algebraic Spaces
\begin{enumerate}
\setcounter{enumi}{62}
\item \hyperref[spaces-section-phantom]{Algebraic Spaces}
\item \hyperref[spaces-properties-section-phantom]{Properties of Algebraic Spaces}
\item \hyperref[spaces-morphisms-section-phantom]{Morphisms of Algebraic Spaces}
\item \hyperref[decent-spaces-section-phantom]{Decent Algebraic Spaces}
\item \hyperref[spaces-cohomology-section-phantom]{Cohomology of Algebraic Spaces}
\item \hyperref[spaces-limits-section-phantom]{Limits of Algebraic Spaces}
\item \hyperref[spaces-divisors-section-phantom]{Divisors on Algebraic Spaces}
\item \hyperref[spaces-over-fields-section-phantom]{Algebraic Spaces over Fields}
\item \hyperref[spaces-topologies-section-phantom]{Topologies on Algebraic Spaces}
\item \hyperref[spaces-descent-section-phantom]{Descent and Algebraic Spaces}
\item \hyperref[spaces-perfect-section-phantom]{Derived Categories of Spaces}
\item \hyperref[spaces-more-morphisms-section-phantom]{More on Morphisms of Spaces}
\item \hyperref[spaces-flat-section-phantom]{Flatness on Algebraic Spaces}
\item \hyperref[spaces-groupoids-section-phantom]{Groupoids in Algebraic Spaces}
\item \hyperref[spaces-more-groupoids-section-phantom]{More on Groupoids in Spaces}
\item \hyperref[bootstrap-section-phantom]{Bootstrap}
\item \hyperref[spaces-pushouts-section-phantom]{Pushouts of Algebraic Spaces}
\end{enumerate}
Topics in Geometry
\begin{enumerate}
\setcounter{enumi}{79}
\item \hyperref[spaces-chow-section-phantom]{Chow Groups of Spaces}
\item \hyperref[groupoids-quotients-section-phantom]{Quotients of Groupoids}
\item \hyperref[spaces-more-cohomology-section-phantom]{More on Cohomology of Spaces}
\item \hyperref[spaces-simplicial-section-phantom]{Simplicial Spaces}
\item \hyperref[spaces-duality-section-phantom]{Duality for Spaces}
\item \hyperref[formal-spaces-section-phantom]{Formal Algebraic Spaces}
\item \hyperref[restricted-section-phantom]{Algebraization of Formal Spaces}
\item \hyperref[spaces-resolve-section-phantom]{Resolution of Surfaces Revisited}
\end{enumerate}
Deformation Theory
\begin{enumerate}
\setcounter{enumi}{87}
\item \hyperref[formal-defos-section-phantom]{Formal Deformation Theory}
\item \hyperref[defos-section-phantom]{Deformation Theory}
\item \hyperref[cotangent-section-phantom]{The Cotangent Complex}
\item \hyperref[examples-defos-section-phantom]{Deformation Problems}
\end{enumerate}
Algebraic Stacks
\begin{enumerate}
\setcounter{enumi}{91}
\item \hyperref[algebraic-section-phantom]{Algebraic Stacks}
\item \hyperref[examples-stacks-section-phantom]{Examples of Stacks}
\item \hyperref[stacks-sheaves-section-phantom]{Sheaves on Algebraic Stacks}
\item \hyperref[criteria-section-phantom]{Criteria for Representability}
\item \hyperref[artin-section-phantom]{Artin's Axioms}
\item \hyperref[quot-section-phantom]{Quot and Hilbert Spaces}
\item \hyperref[stacks-properties-section-phantom]{Properties of Algebraic Stacks}
\item \hyperref[stacks-morphisms-section-phantom]{Morphisms of Algebraic Stacks}
\item \hyperref[stacks-limits-section-phantom]{Limits of Algebraic Stacks}
\item \hyperref[stacks-cohomology-section-phantom]{Cohomology of Algebraic Stacks}
\item \hyperref[stacks-perfect-section-phantom]{Derived Categories of Stacks}
\item \hyperref[stacks-introduction-section-phantom]{Introducing Algebraic Stacks}
\item \hyperref[stacks-more-morphisms-section-phantom]{More on Morphisms of Stacks}
\item \hyperref[stacks-geometry-section-phantom]{The Geometry of Stacks}
\end{enumerate}
Topics in Moduli Theory
\begin{enumerate}
\setcounter{enumi}{105}
\item \hyperref[moduli-section-phantom]{Moduli Stacks}
\item \hyperref[moduli-curves-section-phantom]{Moduli of Curves}
\end{enumerate}
Miscellany
\begin{enumerate}
\setcounter{enumi}{107}
\item \hyperref[examples-section-phantom]{Examples}
\item \hyperref[exercises-section-phantom]{Exercises}
\item \hyperref[guide-section-phantom]{Guide to Literature}
\item \hyperref[desirables-section-phantom]{Desirables}
\item \hyperref[coding-section-phantom]{Coding Style}
\item \hyperref[obsolete-section-phantom]{Obsolete}
\item \hyperref[fdl-section-phantom]{GNU Free Documentation License}
\item \hyperref[index-section-phantom]{Auto Generated Index}
\end{enumerate}
\end{multicols}


\bibliography{my}
\bibliographystyle{amsalpha}

\end{document}
