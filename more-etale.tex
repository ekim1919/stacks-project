\IfFileExists{stacks-project.cls}{%
\documentclass{stacks-project}
}{%
\documentclass{amsart}
}

% For dealing with references we use the comment environment
\usepackage{verbatim}
\newenvironment{reference}{\comment}{\endcomment}
%\newenvironment{reference}{}{}
\newenvironment{slogan}{\comment}{\endcomment}
\newenvironment{history}{\comment}{\endcomment}

% For commutative diagrams we use Xy-pic
\usepackage[all]{xy}

% We use 2cell for 2-commutative diagrams.
\xyoption{2cell}
\UseAllTwocells

% We use multicol for the list of chapters between chapters
\usepackage{multicol}

% This is generall recommended for better output
\usepackage{lmodern}
\usepackage[T1]{fontenc}

% For cross-file-references
\usepackage{xr-hyper}

% Package for hypertext links:
\usepackage{hyperref}

% For any local file, say "hello.tex" you want to link to please
% use \externaldocument[hello-]{hello}
\externaldocument[introduction-]{introduction}
\externaldocument[conventions-]{conventions}
\externaldocument[sets-]{sets}
\externaldocument[categories-]{categories}
\externaldocument[topology-]{topology}
\externaldocument[sheaves-]{sheaves}
\externaldocument[sites-]{sites}
\externaldocument[stacks-]{stacks}
\externaldocument[fields-]{fields}
\externaldocument[algebra-]{algebra}
\externaldocument[brauer-]{brauer}
\externaldocument[homology-]{homology}
\externaldocument[derived-]{derived}
\externaldocument[simplicial-]{simplicial}
\externaldocument[more-algebra-]{more-algebra}
\externaldocument[smoothing-]{smoothing}
\externaldocument[modules-]{modules}
\externaldocument[sites-modules-]{sites-modules}
\externaldocument[injectives-]{injectives}
\externaldocument[cohomology-]{cohomology}
\externaldocument[sites-cohomology-]{sites-cohomology}
\externaldocument[dga-]{dga}
\externaldocument[dpa-]{dpa}
\externaldocument[sdga-]{sdga}
\externaldocument[hypercovering-]{hypercovering}
\externaldocument[schemes-]{schemes}
\externaldocument[constructions-]{constructions}
\externaldocument[properties-]{properties}
\externaldocument[morphisms-]{morphisms}
\externaldocument[coherent-]{coherent}
\externaldocument[divisors-]{divisors}
\externaldocument[limits-]{limits}
\externaldocument[varieties-]{varieties}
\externaldocument[topologies-]{topologies}
\externaldocument[descent-]{descent}
\externaldocument[perfect-]{perfect}
\externaldocument[more-morphisms-]{more-morphisms}
\externaldocument[flat-]{flat}
\externaldocument[groupoids-]{groupoids}
\externaldocument[more-groupoids-]{more-groupoids}
\externaldocument[etale-]{etale}
\externaldocument[chow-]{chow}
\externaldocument[intersection-]{intersection}
\externaldocument[pic-]{pic}
\externaldocument[weil-]{weil}
\externaldocument[adequate-]{adequate}
\externaldocument[dualizing-]{dualizing}
\externaldocument[duality-]{duality}
\externaldocument[discriminant-]{discriminant}
\externaldocument[derham-]{derham}
\externaldocument[local-cohomology-]{local-cohomology}
\externaldocument[algebraization-]{algebraization}
\externaldocument[curves-]{curves}
\externaldocument[resolve-]{resolve}
\externaldocument[models-]{models}
\externaldocument[equiv-]{equiv}
\externaldocument[pione-]{pione}
\externaldocument[etale-cohomology-]{etale-cohomology}
\externaldocument[proetale-]{proetale}
\externaldocument[more-etale-]{more-etale}
\externaldocument[trace-]{trace}
\externaldocument[crystalline-]{crystalline}
\externaldocument[spaces-]{spaces}
\externaldocument[spaces-properties-]{spaces-properties}
\externaldocument[spaces-morphisms-]{spaces-morphisms}
\externaldocument[decent-spaces-]{decent-spaces}
\externaldocument[spaces-cohomology-]{spaces-cohomology}
\externaldocument[spaces-limits-]{spaces-limits}
\externaldocument[spaces-divisors-]{spaces-divisors}
\externaldocument[spaces-over-fields-]{spaces-over-fields}
\externaldocument[spaces-topologies-]{spaces-topologies}
\externaldocument[spaces-descent-]{spaces-descent}
\externaldocument[spaces-perfect-]{spaces-perfect}
\externaldocument[spaces-more-morphisms-]{spaces-more-morphisms}
\externaldocument[spaces-flat-]{spaces-flat}
\externaldocument[spaces-groupoids-]{spaces-groupoids}
\externaldocument[spaces-more-groupoids-]{spaces-more-groupoids}
\externaldocument[bootstrap-]{bootstrap}
\externaldocument[spaces-pushouts-]{spaces-pushouts}
\externaldocument[spaces-chow-]{spaces-chow}
\externaldocument[groupoids-quotients-]{groupoids-quotients}
\externaldocument[spaces-more-cohomology-]{spaces-more-cohomology}
\externaldocument[spaces-simplicial-]{spaces-simplicial}
\externaldocument[spaces-duality-]{spaces-duality}
\externaldocument[formal-spaces-]{formal-spaces}
\externaldocument[restricted-]{restricted}
\externaldocument[spaces-resolve-]{spaces-resolve}
\externaldocument[formal-defos-]{formal-defos}
\externaldocument[defos-]{defos}
\externaldocument[cotangent-]{cotangent}
\externaldocument[examples-defos-]{examples-defos}
\externaldocument[algebraic-]{algebraic}
\externaldocument[examples-stacks-]{examples-stacks}
\externaldocument[stacks-sheaves-]{stacks-sheaves}
\externaldocument[criteria-]{criteria}
\externaldocument[artin-]{artin}
\externaldocument[quot-]{quot}
\externaldocument[stacks-properties-]{stacks-properties}
\externaldocument[stacks-morphisms-]{stacks-morphisms}
\externaldocument[stacks-limits-]{stacks-limits}
\externaldocument[stacks-cohomology-]{stacks-cohomology}
\externaldocument[stacks-perfect-]{stacks-perfect}
\externaldocument[stacks-introduction-]{stacks-introduction}
\externaldocument[stacks-more-morphisms-]{stacks-more-morphisms}
\externaldocument[stacks-geometry-]{stacks-geometry}
\externaldocument[moduli-]{moduli}
\externaldocument[moduli-curves-]{moduli-curves}
\externaldocument[examples-]{examples}
\externaldocument[exercises-]{exercises}
\externaldocument[guide-]{guide}
\externaldocument[desirables-]{desirables}
\externaldocument[coding-]{coding}
\externaldocument[obsolete-]{obsolete}
\externaldocument[fdl-]{fdl}
\externaldocument[index-]{index}

% Theorem environments.
%
\theoremstyle{plain}
\newtheorem{theorem}[subsection]{Theorem}
\newtheorem{proposition}[subsection]{Proposition}
\newtheorem{lemma}[subsection]{Lemma}

\theoremstyle{definition}
\newtheorem{definition}[subsection]{Definition}
\newtheorem{example}[subsection]{Example}
\newtheorem{exercise}[subsection]{Exercise}
\newtheorem{situation}[subsection]{Situation}

\theoremstyle{remark}
\newtheorem{remark}[subsection]{Remark}
\newtheorem{remarks}[subsection]{Remarks}

\numberwithin{equation}{subsection}

% Macros
%
\def\lim{\mathop{\mathrm{lim}}\nolimits}
\def\colim{\mathop{\mathrm{colim}}\nolimits}
\def\Spec{\mathop{\mathrm{Spec}}}
\def\Hom{\mathop{\mathrm{Hom}}\nolimits}
\def\Ext{\mathop{\mathrm{Ext}}\nolimits}
\def\SheafHom{\mathop{\mathcal{H}\!\mathit{om}}\nolimits}
\def\SheafExt{\mathop{\mathcal{E}\!\mathit{xt}}\nolimits}
\def\Sch{\mathit{Sch}}
\def\Mor{\mathop{\mathrm{Mor}}\nolimits}
\def\Ob{\mathop{\mathrm{Ob}}\nolimits}
\def\Sh{\mathop{\mathit{Sh}}\nolimits}
\def\NL{\mathop{N\!L}\nolimits}
\def\CH{\mathop{\mathrm{CH}}\nolimits}
\def\proetale{{pro\text{-}\acute{e}tale}}
\def\etale{{\acute{e}tale}}
\def\QCoh{\mathit{QCoh}}
\def\Ker{\mathop{\mathrm{Ker}}}
\def\Im{\mathop{\mathrm{Im}}}
\def\Coker{\mathop{\mathrm{Coker}}}
\def\Coim{\mathop{\mathrm{Coim}}}

% Boxtimes
%
\DeclareMathSymbol{\boxtimes}{\mathbin}{AMSa}{"02}

%
% Macros for moduli stacks/spaces
%
\def\QCohstack{\mathcal{QC}\!\mathit{oh}}
\def\Cohstack{\mathcal{C}\!\mathit{oh}}
\def\Spacesstack{\mathcal{S}\!\mathit{paces}}
\def\Quotfunctor{\mathrm{Quot}}
\def\Hilbfunctor{\mathrm{Hilb}}
\def\Curvesstack{\mathcal{C}\!\mathit{urves}}
\def\Polarizedstack{\mathcal{P}\!\mathit{olarized}}
\def\Complexesstack{\mathcal{C}\!\mathit{omplexes}}
% \Pic is the operator that assigns to X its picard group, usage \Pic(X)
% \Picardstack_{X/B} denotes the Picard stack of X over B
% \Picardfunctor_{X/B} denotes the Picard functor of X over B
\def\Pic{\mathop{\mathrm{Pic}}\nolimits}
\def\Picardstack{\mathcal{P}\!\mathit{ic}}
\def\Picardfunctor{\mathrm{Pic}}
\def\Deformationcategory{\mathcal{D}\!\mathit{ef}}


% OK, start here.
%
\begin{document}

\title{More \'Etale Cohomology}


\maketitle

\phantomsection
\label{section-phantom}

\tableofcontents


\section{Introduction}
\label{section-introduction}

\noindent
This chapter is the second in a series of chapter on the \'etale cohomology
of schemes. To read the first chapter, please visit
\'Etale Cohomology, Section \ref{etale-cohomology-section-introduction}.

\medskip\noindent
The split with the previous chapter is roughly speaking that anything
concerning ``shriek functors'' (cohomology with compact support and
its right adjoint) and anything using this material goes into this chapter.






\section{Growing sections}
\label{section-growing}

\noindent
In this section we discuss results of the following type.

\begin{lemma}
\label{lemma-section-support-in-locally-closed-pre}
Let $X$ be a scheme. Let $\mathcal{F}$ be an abelian sheaf on $X_\etale$.
Let $\varphi : U' \to U$ be a morphism of $X_\etale$. Let $Z' \subset U'$ be a
closed subscheme such that $Z' \to U' \to U$ is a closed immersion
with image $Z \subset U$. Then there is a canonical bijection
$$
\{s \in \mathcal{F}(U) \mid \text{Supp}(s) \subset Z\} =
\{s' \in \mathcal{F}(U') \mid \text{Supp}(s') \subset Z'\}
$$
which is given by restriction if $\varphi^{-1}(Z) = Z'$.
\end{lemma}

\begin{proof}
Consider the closed subscheme $Z'' = \varphi^{-1}(Z)$ of $U'$.
Then $Z' \subset Z''$ is closed because $Z'$ is closed in $U'$.
On the other hand, $Z' \to Z''$ is an \'etale morphism
(as a morphism between schemes \'etale over $Z$) and hence
open. Thus $Z'' = Z' \amalg T$ for some closed subset $T$.
The open covering $U' = (U' \setminus T) \cup (U' \setminus Z')$
shows that
$$
\{s' \in \mathcal{F}(U') \mid \text{Supp}(s') \subset Z'\} =
\{s' \in \mathcal{F}(U' \setminus T) \mid \text{Supp}(s') \subset Z'\}
$$
and the \'etale covering $\{U' \setminus T \to U, U \setminus Z \to U\}$
shows that
$$
\{s \in \mathcal{F}(U) \mid \text{Supp}(s) \subset Z\} =
\{s' \in \mathcal{F}(U' \setminus T) \mid \text{Supp}(s') \subset Z'\}
$$
This finishes the proof.
\end{proof}

\begin{lemma}
\label{lemma-section-support-in-locally-closed}
Let $X$ be a scheme. Let $Z \subset X$ be a locally closed subscheme.
Let $\mathcal{F}$ be an abelian sheaf on $X_\etale$. Given
$U, U' \subset X$ open containing $Z$ as a closed subscheme,
there is a canonical bijection
$$
\{s \in \mathcal{F}(U) \mid \text{Supp}(s) \subset Z\} =
\{s \in \mathcal{F}(U') \mid \text{Supp}(s) \subset Z\}
$$
which is given by restriction if $U' \subset U$.
\end{lemma}

\begin{proof}
Since $Z$ is a closed subscheme of $U \cap U'$, it suffices to
prove the lemma when $U' \subset U$. Then it is a special case
of Lemma \ref{lemma-section-support-in-locally-closed-pre}.
\end{proof}

\noindent
Let us introduce a bit of nonstandard notation which will stand us
in good stead later. Namely, in the situation of
Lemma \ref{lemma-section-support-in-locally-closed} above, let us denote
$$
H_Z(\mathcal{F}) = \{s \in \mathcal{F}(U) \mid \text{Supp}(s) \subset Z\}
$$
where $U \subset X$ is any choice of open subscheme containing $Z$ as a closed
subscheme. The reader who is troubled by the lack of precision this entails
may choose $U = X \setminus \partial Z$ where
$\partial Z = \overline{Z}\setminus Z$ is the ``boundary'' of $Z$ in $X$.
However, in many of the arguments below the flexibility of choosing
different opens will play a role. Here are some properties of this
construction:
\begin{enumerate}
\item
\label{item-inclusion}
If $Z \subset Z'$ are locally closed subschemes of $X$ and $Z$ is
closed in $Z'$, then there is a natural injective map
$$
H_Z(\mathcal{F}) \to H_{Z'}(\mathcal{F}).
$$
\item
\label{item-pullback}
If $f : Y \to X$ is a morphism of schemes and $Z \subset X$ is a locally
closed subscheme, then there is a natural
pullback map $f^* : H_Z(\mathcal{F}) \to H_{f^{-1}Z}(f^{-1}\mathcal{F})$.
\end{enumerate}
It will be convenient to extend our notation to the following situation:
suppose that we have $W \in X_\etale$ and a locally closed subscheme
$Z \subset W$. Then we will denote
$$
H_Z(\mathcal{F}) =
\{s \in \mathcal{F}(U) \mid \text{Supp}(s) \subset Z\} =
H_Z(\mathcal{F}|_{W_\etale})
$$
where $U \subset W$ is any choice of open subscheme containing $Z$
as a closed subscheme, exactly as above\footnote{In fact,
Lemma \ref{lemma-section-support-in-locally-closed-pre}
shows, given $Z$ over $X$ which is isomorphic to a locally closed
subscheme of some object $W$ of $X_\etale$, that
the choice of $W$ is irrelevant.}.







\section{Sections with compact support}
\label{section-compact-support}

\noindent
A reference for this section is \cite[Exposee XVII, Section 6]{SGA4}.
Let $f : X \to Y$ be a morphism of schemes which is separated and
locally of finite type. In this section we define a functor
$f_! : \textit{Ab}(X_\etale) \to \textit{Ab}(Y_\etale)$
by taking $f_!\mathcal{F} \subset f_*\mathcal{F}$
to be the subsheaf of sections which have proper support relative to $Y$
(suitably defined).

\medskip\noindent
Warning: The functor $f_!$ is the zeroth cohomology sheaf of a functor
$Rf_!$ on the derived category (insert future reference), but
$Rf_!$ is not the derived functor of $f_!$.

\begin{lemma}
\label{lemma-f-shriek-separated}
Let $f : X \to Y$ be a morphism of schemes which is locally of finite type.
Let $\mathcal{F}$ be an abelian sheaf on $X_\etale$. The rule
$$
Y_\etale \longrightarrow \textit{Ab},\quad
V \longmapsto \{s \in f_*\mathcal{F}(V) = \mathcal{F}(X_V) \mid
\text{Supp}(s) \subset X_V \text{ is proper over }V\}
$$
is an abelian subsheaf of $f_*\mathcal{F}$.
\end{lemma}

\noindent
Warning: This sheaf isn't the ``correct one'' if $f$ is not separated.

\begin{proof}
Recall that the support of a section is closed
(\'Etale Cohomology, Lemma \ref{etale-cohomology-lemma-support-section-closed})
hence the material in
Cohomology of Schemes, Section \ref{coherent-section-proper-over-base}
applies. By the lemma above and
Cohomology of Schemes, Lemma \ref{coherent-lemma-union-closed-proper-over-base}
we find that our subset of $f_*\mathcal{F}(V)$ is a subgroup.
By Cohomology of Schemes, Lemma
\ref{coherent-lemma-base-change-closed-proper-over-base}
we see that our rule defines a sub presheaf.
Finally, suppose that we have $s \in f_*\mathcal{F}(V)$
and an \'etale covering $\{V_i \to V\}$ such that
$s|_{V_i}$ has support proper over $V_i$.
Observe that the support of $s|_{V_i}$ is the inverse
image of the support of $s|_V$ (use the characterization
of the support in terms of stalks and
\'Etale Cohomology, Lemma \ref{etale-cohomology-lemma-stalk-pullback}).
Whence the support of $s$ is proper over $V$ by
Descent, Lemma \ref{descent-lemma-descending-property-proper-over-base}.
This proves that our rule satisfies the sheaf condition.
\end{proof}

\begin{lemma}
\label{lemma-separated-etale-shriek}
Let $j : U \to X$ be a separated \'etale morphism. Let $\mathcal{F}$
be an abelian sheaf on $U_\etale$. The image of the injective map
$j_!\mathcal{F} \to j_*\mathcal{F}$ of
\'Etale Cohomology, Lemma
\ref{etale-cohomology-lemma-shriek-into-star-separated-etale}
is the subsheaf of Lemma \ref{lemma-f-shriek-separated}.
\end{lemma}

\noindent
An alternative would be to move this lemma later and prove this
using the descrition of the stalks of both sheaves.

\begin{proof}
The construction of $j_!\mathcal{F} \to j_*\mathcal{F}$ in the proof of
\'Etale Cohomology, Lemma
\ref{etale-cohomology-lemma-shriek-into-star-separated-etale}
is via the construction of a map
$j_{p!}\mathcal{F} \to j_*\mathcal{F}$ of presheaves
whose image is clearly contained in the subsheaf of
Lemma \ref{lemma-f-shriek-separated}.
Hence since $j_!\mathcal{F}$ is the sheafification of
$j_{p!}\mathcal{F}$ we conclude the image of
$j_!\mathcal{F} \to j_*\mathcal{F}$ is contained in
this subsheaf. Conversely, let $s \in j_*\mathcal{F}(V)$
have support $Z$ proper over $V$. Then $Z \to V$ is
finite with closed image $Z' \subset V$, see
More on Morphisms, Lemma \ref{more-morphisms-lemma-characterize-finite}.
The restriction of $s$ to $V \setminus Z'$ is zero and the zero section is
contained in the image of $j_!\mathcal{F} \to j_*\mathcal{F}$.
On the other hand, if $v \in Z'$, then we can find
an \'etale neighbourhood
$(V', v') \to (V, v)$ such that we have a decomposition
$U_{V'} = W \amalg U'_1 \amalg \ldots \amalg U'_n$
into open and closed subschemes with $U'_i \to V'$ an isomorphism
and with $T_{V'} \subset U'_1 \amalg \ldots \amalg U'_n$, see
\'Etale Morphisms, Lemma \ref{etale-lemma-etale-etale-local-technical}.
Inverting the isomorphisms $U'_i \to V'$
we obtain $n$ morphisms $\varphi'_i : V' \to U$
and sections $s'_i$ over $V'$ by pulling back $s$.
Then the section $\sum (\varphi'_i, s'_i)$ of
$j_{p!}\mathcal{F}$ over $V'$, see formula for $j_{p!}\mathcal{F}(V')$
in proof of \'Etale Cohomology, Lemma
\ref{etale-cohomology-lemma-shriek-into-star-separated-etale},
maps to the restriction of $s$ to $V'$ by construction.
We conclude that $s$ is \'etale locally in the image
of $j_!\mathcal{F} \to j_*\mathcal{F}$ and the proof is complete.
\end{proof}

\begin{definition}
\label{definition-f-shriek-separated}
Let $f : X \to Y$ be a morphism of schemes which is separated (!) and
locally of finite type. Let $\mathcal{F}$ be an abelian sheaf on
$X_\etale$. The subsheaf $f_!\mathcal{F} \subset f_*\mathcal{F}$
constructed in Lemma \ref{lemma-f-shriek-separated} is called the
{\it direct image with compact support}.
\end{definition}

\noindent
By Lemma \ref{lemma-separated-etale-shriek} this does not conflict with
\'Etale Cohomology, Definition \ref{etale-cohomology-definition-extension-zero}
as we have agreement when both definitions apply. Here is a sanity check.

\begin{lemma}
\label{lemma-proper-f-shriek}
Let $f : X \to Y$ be a proper morphism of schemes.
Then $f_! = f_*$.
\end{lemma}

\begin{proof}
Immediate from the construction of $f_!$.
\end{proof}

\noindent
A very useful observation is the following.

\begin{remark}[Covariance with respect to open embeddings]
\label{remark-covariance-f-shriek-separated}
Let $f : X \to Y$ be morphism of schemes which is separated and
locally of finite type. Let $\mathcal{F}$ be an abelian sheaf on $X_\etale$.
Let $X' \subset X$ be an open subscheme. Denote $f' : X' \to Y$
the restriction of $f$. There is a canonical injective map
$$
f'_!(\mathcal{F}|_{X'}) \longrightarrow f_!\mathcal{F}
$$
Namely, let $V \in Y_\etale$ and consider a section
$s' \in f'_*(\mathcal{F}|_{X'})(V) = \mathcal{F}(X' \times_Y V)$
with support $Z'$ proper over $V$. Then $Z'$ is closed in $X \times_Y V$
as well, see Cohomology of Schemes, Lemma
\ref{coherent-lemma-functoriality-closed-proper-over-base}.
Thus there is a unique section
$s \in \mathcal{F}(X \times_Y V) = f_*\mathcal{F}(V)$
whose restriction to $X' \times_Y V$ is $s'$ and whose restriction
to $X \times_Y V \setminus Z'$ is zero, see
Lemma \ref{lemma-section-support-in-locally-closed}. This construction is
compatible with restriction maps and hence induces the desired map of
sheaves $f'_!(\mathcal{F}|_{X'}) \to f_!\mathcal{F}$ which is clearly
injective. By construction we obtain a commutative diagram
$$
\xymatrix{
f'_!(\mathcal{F}|_{X'}) \ar[r] \ar[d] &
f_!\mathcal{F} \ar[d] \\
f'_*(\mathcal{F}|_{X'}) &
f_*\mathcal{F} \ar[l]
}
$$
functorial in $\mathcal{F}$. It is clear that for $X'' \subset X'$ open
with $f'' = f|_{X''} : X'' \to Y$ the composition of the canonical maps
$f''_!\mathcal{F}|_{X''} \to f'_!\mathcal{F}|_{X'} \to f_!\mathcal{F}$
just constructed is the canonical map
$f''_!\mathcal{F}|_{X''} \to f_!\mathcal{F}$.
\end{remark}

\begin{lemma}
\label{lemma-compactify-f-shriek-separated}
Let $Y$ be a scheme. Let $j : X \to \overline{X}$ be an open
immersion of schemes over $Y$ with $\overline{X}$ proper over $Y$.
Denote $f : X \to Y$ and $\overline{f} : \overline{X} \to Y$
the structure morphisms. For $\mathcal{F} \in \textit{Ab}(X_\etale)$
there is a canonical isomorphism (see proof)
$$
f_!\mathcal{F} \longrightarrow \overline{f}_!j_!\mathcal{F}
$$
As we have $\overline{f}_! = \overline{f}_*$ by
Lemma \ref{lemma-proper-f-shriek} we obtain
$\overline{f}_* \circ j_! = f_!$ as functors
$\textit{Ab}(X_\etale) \to \textit{Ab}(Y_\etale)$.
\end{lemma}

\begin{proof}
We have $(j_!\mathcal{F})|_X = \mathcal{F}$, see
\'Etale Cohomology, Lemma \ref{etale-cohomology-lemma-jshriek-open}.
Thus the displayed arrow is the injective map
$f_!(\mathcal{G}|_X) \to \overline{f}_!\mathcal{G}$
of Remark \ref{remark-covariance-f-shriek-separated}
for $\mathcal{G} = j_!\mathcal{F}$. The explicit nature
of this map implies that it now suffices to show: if $V \in Y_\etale$ and
$s \in \overline{f}_!\mathcal{G}(V) = \overline{f}_*\mathcal{G}(V) =
\mathcal{G}(\overline{X}_V)$
is a section, then the support of $s$ is contained in the open
$X_V \subset \overline{X}_V$. This is immediate from the fact
that the stalks of $\mathcal{G}$ are zero at geometric
points of $\overline{X} \setminus X$.
\end{proof}

\noindent
We want to relate the stalks of $f_!\mathcal{F}$ to sections with
compact support on fibres. In order to state this, we need a definition.

\begin{definition}
\label{definition-compact-support}
Let $X$ be a separated scheme locally of finite type over a field $k$.
Let $\mathcal{F}$ be an abelian sheaf on $X_\etale$. We let
$H^0_c(X, \mathcal{F}) \subset H^0(X, \mathcal{F})$ be the
set of sections whose support is proper over $k$. Elements of
$H^0_c(X, \mathcal{F})$ are called {\it sections with compact support}.
\end{definition}

\noindent
Warning: This definition isn't the ``correct one'' if $X$ isn't
separated over $k$.

\begin{lemma}
\label{lemma-proper-compact-support}
Let $X$ be a proper scheme over a field $k$. Then
$H^0_c(X, \mathcal{F}) = H^0(X, \mathcal{F})$.
\end{lemma}

\begin{proof}
Immediate from the construction of $H^0_c$.
\end{proof}

\begin{remark}[Open embeddings and compactly supported sections]
\label{remark-covariance-compact-support}
Let $X$ be a separated scheme locally of finite type over a field $k$.
Let $\mathcal{F}$ be an abelian sheaf on $X_\etale$.
Exactly as in Remark \ref{remark-covariance-f-shriek-separated}
for $X' \subset X$ open there is an injective map
$$
H^0_c(X', \mathcal{F}|_{X'}) \longrightarrow H^0_c(X, \mathcal{F})
$$
and these maps turn $H^0_c$ into a ``cosheaf'' on the Zariski site of $X$.
\end{remark}

\begin{lemma}
\label{lemma-compactify-compact-support}
Let $k$ be a field. Let $j : X \to \overline{X}$ be an open
immersion of schemes over $k$ with $\overline{X}$ proper over $k$.
For $\mathcal{F} \in \textit{Ab}(X_\etale)$
there is a canonical isomorphism (see proof)
$$
H^0_c(X, \mathcal{F}) \longrightarrow
H^0_c(\overline{X}, j_!\mathcal{F}) =
H^0(\overline{X}, j_!\mathcal{F})
$$
where we have the equality on the right by
Lemma \ref{lemma-proper-compact-support}.
\end{lemma}

\begin{proof}
We have $(j_!\mathcal{F})|_X = \mathcal{F}$, see
\'Etale Cohomology, Lemma \ref{etale-cohomology-lemma-jshriek-open}.
Thus the displayed arrow is the injective map
$H^0_c(X, \mathcal{G}|_X) \to H^0_c(\overline{X}, \mathcal{G})$
of Remark \ref{remark-covariance-compact-support}
for $\mathcal{G} = j_!\mathcal{F}$. The explicit nature
of this map implies that it now suffices to show: if
$s \in H^0(\overline{X}, \mathcal{G})$ is a section, then the support of
$s$ is contained in the open $X$. This is immediate from the fact
that the stalks of $\mathcal{G}$ are zero at geometric
points of $\overline{X} \setminus X$.
\end{proof}

\begin{lemma}
\label{lemma-stalk-f-shriek-separated}
Let $f : X \to Y$ be a morphism of schemes which is separated and
locally of finite type. Let $\mathcal{F}$ be an abelian sheaf on
$X_\etale$. Then there is a canonical isomorphism
$$
(f_!\mathcal{F})_{\overline{y}}
\longrightarrow
H^0_c(X_{\overline{y}}, \mathcal{F}|_{X_{\overline{y}}})
$$
for any geometric point $\overline{y} : \Spec(k) \to Y$.
\end{lemma}

\begin{proof}
Recall that $(f_*\mathcal{F})_{\overline{y}} = \colim f_*\mathcal{F}(V)$
where the colimit is over the \'etale neighbourhoods $(V, \overline{v})$
of $\overline{y}$. If $s \in f_*\mathcal{F}(V) = \mathcal{F}(X_V)$,
then we can pullback $s$ to a section of $\mathcal{F}$ over
$(X_V)_{\overline{v}} = X_{\overline{y}}$. Thus we obtain a canonical map
$$
c_{\overline{y}} :
(f_*\mathcal{F})_{\overline{y}}
\longrightarrow 
H^0(X_{\overline{y}}, \mathcal{F}|_{X_{\overline{y}}})
$$
We claim that this map induces a bijection between the subgroups
$(f_!\mathcal{F})_{\overline{y}}$ and
$H^0_c(X_{\overline{y}}, \mathcal{F}|_{X_{\overline{y}}})$.
The claim implies the lemma, but is a little bit more precise
in that it describes the identification of the lemma as given
by pullbacks of sections of $\mathcal{F}$ to the geometric fibre of $f$.

\medskip\noindent
Observe that any element
$s \in (f_!\mathcal{F})_{\overline{y}} \subset (f_*\mathcal{F})_{\overline{y}}$
is mapped by $c_{\overline{y}}$ to an element of
$H^0_c(X_{\overline{y}}, \mathcal{F}|_{X_{\overline{y}}}) \subset
H^0(X_{\overline{y}}, \mathcal{F}|_{X_{\overline{y}}})$.
This is true because taking the support of a section
commutes with pullback and because properness is preserved by
base change. This at least produces the map in the statement of the lemma.
To prove that it is an isomorphism we may work Zariski
locally on $Y$ and hence we may and do assume $Y$ is affine.

\medskip\noindent
An observation that we will use below
is that given an open subscheme $X' \subset X$
and if $f' = f|_{X'}$, then we obtain a commutative diagram
$$
\xymatrix{
(f'_!(\mathcal{F}|_{X'}))_{\overline{y}} \ar[r] \ar[d] &
H^0_c(X'_{\overline{y}}, \mathcal{F}|_{X'_{\overline{y}}}) \ar[d] \\
(f_!\mathcal{F})_{\overline{y}} \ar[r] &
H^0_c(X_{\overline{y}}, \mathcal{F}|_{X_{\overline{y}}})
}
$$
where the horizontal arrows are the maps constructed above and
the vertical arrows are given in
Remarks \ref{remark-covariance-f-shriek-separated} and
\ref{remark-covariance-compact-support}.
The reason is that given an \'etale neighbourhood $(V, \overline{v})$
of $\overline{y}$ and a section $s \in f_*\mathcal{F}(V) = \mathcal{F}(X_V)$
whose support $Z$ happens to be contained in $X'_V$ and is proper over $V$,
so that $s$ gives rise to an element of both
$(f'_!(\mathcal{F}|_{X'}))_{\overline{y}}$ and
$(f_!\mathcal{F})_{\overline{y}}$ which correspond via
the vertical arrow of the diagram, then these elements are mapped via the
horizontal arrows to the pullback $s|_{X_{\overline{y}}}$ of $s$ to
$X_{\overline{y}}$ whose support $Z_{\overline{y}}$ is contained in
$X'_{\overline{y}}$ and hence this restriction gives rise to
a compatible pair of elements of
$H^0_c(X'_{\overline{y}}, \mathcal{F}|_{X'_{\overline{y}}})$
and
$H^0_c(X_{\overline{y}}, \mathcal{F}|_{X_{\overline{y}}})$.

\medskip\noindent
Suppose $s \in (f_!\mathcal{F})_{\overline{y}}$ maps to zero in
$H^0_c(X_{\overline{y}}, \mathcal{F}|_{X_{\overline{y}}})$.
Say $s$ corresponds to $s \in f_*\mathcal{F}(V) = \mathcal{F}(X_V)$
with support $Z$ proper over $V$. We may assume that $V$ is affine
and hence $Z$ is quasi-compact. Then we may choose a quasi-compact open
$X' \subset X$ containing the image of $Z$. Then $Z$ is contained in
$X'_V$ and hence $s$ is the image of an element
$s' \in f'_!(\mathcal{F}|_{X'})(V)$ where $f' = f|_{X'}$ as in
the previous paragraph. Then $s'$ maps to zero in
$H^0_c(X'_{\overline{y}}, \mathcal{F}|_{X'_{\overline{y}}})$.
Hence in order to prove injectivity, we may replace $X$ by
$X'$, i.e., we may assume $X$ is quasi-compact. We will prove
this case below.

\medskip\noindent
Suppose that
$t \in H^0_c(X_{\overline{y}}, \mathcal{F}|_{X_{\overline{y}}})$.
Then the support of $t$ is contained in a quasi-compact
open subscheme $W \subset X_{\overline{y}}$.
Hence we can find a quasi-compact open subscheme
$X' \subset X$ such that $X'_{\overline{y}}$ contains $W$.
Then it is clear that $t$ is contained in the image
of the injective map
$H^0_c(X'_{\overline{y}}, \mathcal{F}|_{X'_{\overline{y}}}) \to
H^0_c(X_{\overline{y}}, \mathcal{F}|_{X_{\overline{y}}})$.
Hence in order to show surjectivity, we may replace $X$
by $X'$, i.e., we may assume $X$ is quasi-compact.
We will prove this case below.

\medskip\noindent
In this last paragraph of the proof we prove the lemma in case
$X$ is quasi-compact and $Y$ is affine. By More on Flatness, Theorem
\ref{flat-theorem-nagata} there exists a compactification
$j : X \to \overline{X}$ over $Y$. Set $\mathcal{G} = j_!\mathcal{F}$
so that $\mathcal{F} = \mathcal{G}|_X$ by
\'Etale Cohomology, Lemma \ref{etale-cohomology-lemma-jshriek-open}.
By the disussion above we get a commutative diagram
$$
\xymatrix{
(f_!\mathcal{F})_{\overline{y}} \ar[r] \ar[d] &
H^0_c(X_{\overline{y}}, \mathcal{F}|_{X_{\overline{y}}}) \ar[d] \\
(\overline{f}_!\mathcal{G})_{\overline{y}} \ar[r] &
H^0_c(\overline{X}_{\overline{y}}, \mathcal{G}|_{\overline{X}_{\overline{y}}})
}
$$
By Lemmas \ref{lemma-compactify-f-shriek-separated} and
\ref{lemma-compactify-compact-support} the vertical maps
are isomorphisms. This reduces us to the case of the proper
morphism $\overline{X} \to Y$. For a proper morphism our map
is an isomorphism by
Lemmas \ref{lemma-proper-f-shriek} and \ref{lemma-proper-compact-support}
and proper base change for pushforwards, see
\'Etale Cohomology, Lemma
\ref{etale-cohomology-lemma-proper-pushforward-stalk}.
\end{proof}

\begin{lemma}
\label{lemma-base-change-f-shriek-separated}
Consider a cartesian square
$$
\xymatrix{
X' \ar[r]_{g'} \ar[d]_{f'} & X \ar[d]^f \\
Y' \ar[r]^g & Y
}
$$
of schemes with $f$ separated and locally of finite type.
For any abelian sheaf $\mathcal{F}$ on $X_\etale$ we have
$f'_!(g')^{-1}\mathcal{F} = g^{-1}f_!\mathcal{F}$.
\end{lemma}

\begin{proof}
In great generality there is a pullback map
$g^{-1}f_*\mathcal{F} \to f'_*(g')^{-1}\mathcal{F}$, see
Sites, Section \ref{sites-section-pullback}.
We claim that this map sends $g^{-1}f_!\mathcal{F}$
into the subsheaf $f'_!(g')^{-1}\mathcal{F}$
and induces the isomorphism in the lemma.

\medskip\noindent
Choose a geometric point $\overline{y}': \Spec(k) \to Y'$ and denote
$\overline{y} = g \circ \overline{y}'$ the image in $Y$. There is a
commutative diagram
$$
\xymatrix{
(f_*\mathcal{F})_{\overline{y}} \ar[r] \ar[d] &
H^0(X_{\overline{y}}, \mathcal{F}|_{X_{\overline{y}}}) \ar[d] \\
(f'_*(g')^{-1}\mathcal{F})_{\overline{y}'} \ar[r] &
H^0(X'_{\overline{y}'}, (g')^{-1}\mathcal{F}|_{X'_{\overline{y}'}})
}
$$
where the horizontal maps were used in the proof of
Lemma \ref{lemma-stalk-f-shriek-separated}
and the vertical maps are the pullback maps above.
The diagram commutes because each of the four maps
in question is given by pulling back local sections along
a morphism of schemes and the underlying diagram of morphisms
of schemes commutes. Since the diagram in the statement of the lemma
is cartesian we have $X'_{\overline{y}'} = X_{\overline{y}}$.
Hence by Lemma \ref{lemma-stalk-f-shriek-separated}
and its proof we obtain a commutative diagram
$$
\xymatrix{
(f_*\mathcal{F})_{\overline{y}} \ar[rrr] \ar[ddd] & & &
H^0(X_{\overline{y}}, \mathcal{F}|_{X_{\overline{y}}}) \ar[ddd] \\
& (f_!\mathcal{F})_{\overline{y}} \ar[r] \ar@{..>}[d] \ar[lu] &
H^0_c(X_{\overline{y}}, \mathcal{F}|_{X_{\overline{y}}}) \ar[d] \ar[ru] \\
& (f'_!(g')^{-1}\mathcal{F})_{\overline{y}'} \ar[r] \ar[ld] &
H^0_c(X'_{\overline{y}'}, (g')^{-1}\mathcal{F}|_{X'_{\overline{y}'}}) \ar[rd]\\
(f'_*(g')^{-1}\mathcal{F})_{\overline{y}'} \ar[rrr] & & &
H^0(X'_{\overline{y}'}, (g')^{-1}\mathcal{F}|_{X'_{\overline{y}'}})
}
$$
where the horizontal arrows of the inner square are isomorphisms
and the two right vertical arrows are equalities. Also, the
se, sw, ne, nw arrows are injective. It follows that there is a unique
bijective dotted arrow fitting into the diagram. We conclude that
$g^{-1}f_!\mathcal{F} \subset g^{-1}f_*\mathcal{F} \to f'_*(g')^{-1}\mathcal{F}$
is mapped into the subsheaf
$f'_!(g')^{-1}\mathcal{F} \subset f'_*(g')^{-1}\mathcal{F}$
because this is true on stalks, see
\'Etale Cohomology, Theorem \ref{etale-cohomology-theorem-exactness-stalks}.
The same theorem then implies that the induced map is an isomorphism
and the proof is complete.
\end{proof}

\begin{lemma}
\label{lemma-f-shriek-composition}
Let $f : X \to Y$ and $g : Y \to Z$ be composable morphisms of schemes which
are separated and locally of finite type. Let $\mathcal{F}$ be an abelian
sheaf on $X_\etale$. Then $g_!f_!\mathcal{F} = (g \circ f)_!\mathcal{F}$
as subsheaves of $(g \circ f)_*\mathcal{F}$.
\end{lemma}

\begin{proof}
We strongly urge the reader to prove this for themselves.
Let $W \in Z_\etale$ and
$s \in (g \circ f)_*\mathcal{F}(W) = \mathcal{F}(X_W)$.
Denote $T \subset X_W$ the support of $s$; this is a closed
subset. Observe that $s$ is a section of $(g \circ f)_!\mathcal{F}$
if and only if $T$ is proper over $W$. We have
$f_!\mathcal{F} \subset f_*\mathcal{F}$ and hence
$g_!f_!\mathcal{F} \subset g_!f_*\mathcal{F} \subset g_*f_*\mathcal{F}$.
On the other hand, $s$ is a section of $g_!f_!\mathcal{F}$ if and only
if (a) $T$ is proper over $Y_W$ and (b) the support $T'$ of $s$
viewed as section of $f_!\mathcal{F}$ is proper over $W$.
If (a) holds, then the image of $T$ in $Y_W$ is closed and since
$f_!\mathcal{F} \subset f_*\mathcal{F}$ we see that
$T' \subset Y_W$ is the image of $T$ (details omitted; look at stalks).

\medskip\noindent
The conclusion is that we have to show a closed subset $T \subset X_W$
is proper over $W$ if and only if $T$ is proper over $Y_W$
and the image of $T$ in $Y_W$ is proper over $W$. Let us endow $T$
with the reduced induced closed subscheme structure.
If $T$ is proper over $W$, then $T \to Y_W$ is proper by
Morphisms, Lemma \ref{morphisms-lemma-image-proper-scheme-closed}
and the image of $T$ in $Y_W$ is proper over $W$ by
Cohomology of Schemes, Lemma
\ref{coherent-lemma-functoriality-closed-proper-over-base}.
Conversely, if $T$ is proper over $Y_W$
and the image of $T$ in $Y_W$ is proper over $W$,
then the morphism $T \to W$ is proper as a composition
of proper morphisms (here we endow the closed image of $T$
in $Y_W$ with its reduced induced scheme structure to turn the
question into one about morphisms of schemes), see
Morphisms, Lemma \ref{morphisms-lemma-composition-proper}.
\end{proof}

\begin{remark}
\label{remark-f-shriek-base-change-composition}
The isomorphisms between functors
constructed above satisfy the following two properties:
\begin{enumerate}
\item Let $f : X \to Y$, $g : Y \to Z$, and $h : Z \to T$ be composable
morphisms of schemes which are separated and locally of finite type.
Then the diagram
$$
\xymatrix{
(h \circ g \circ f)_! \ar[r] \ar[d] &
(h \circ g)_! \circ f_! \ar[d] \\
h_! \circ (g \circ f)_! \ar[r] &
h_! \circ g_! \circ f_!
}
$$
commutes where the arrows are those of Lemma \ref{lemma-f-shriek-composition}.
\item Suppose that we have a diagram of schemes
$$
\xymatrix{
X' \ar[d]_{f'} \ar[r]_c & X \ar[d]^f \\
Y' \ar[d]_{g'} \ar[r]_b & Y \ar[d]^g \\
Z' \ar[r]^a & Z
}
$$
with both squares cartesian and $f$ and $g$ separated and
locally of finite type. Then the diagram
$$
\xymatrix{
a^{-1} \circ (g \circ f)_! \ar[d] \ar[rr] & &
(g' \circ f')_! \circ c^{-1} \ar[d] \\
a^{-1} \circ g_! \circ f_! \ar[r] &
g'_! \circ b^{-1} \circ f_! \ar[r] &
g'_! \circ f'_! \circ c^{-1}
}
$$
commutes where the horizontal arrows are those of
Lemma \ref{lemma-base-change-f-shriek-separated}
the arrows are those of Lemma \ref{lemma-f-shriek-composition}.
\end{enumerate}
Part (1) holds true because we have a similar commutative
diagram for pushforwards. Part (2) holds by the very general
compatibility of base change maps for pushforwards
(Sites, Remark \ref{sites-remark-compose-base-change})
and the fact that the isomorphisms in
Lemmas \ref{lemma-base-change-f-shriek-separated} and
\ref{lemma-f-shriek-composition}
are constructed using the corresponding maps fo pushforwards.
\end{remark}

\begin{lemma}
\label{lemma-colim-f-shriek-separated}
Let $f : X \to Y$ be morphism of schemes which is separated and
locally of finite type. Let $X = \bigcup_{i \in I} X_i$ be an
open covering such that for all $i, j \in I$ there exists a $k$
with $X_i \cup X_j \subset X_k$. Denote $f_i : X_i \to Y$
the restriction of $f$. Then
$$
f_!\mathcal{F} = \colim_{i \in I} f_{i, !}(\mathcal{F}|_{X_i})
$$
functorially in $\mathcal{F} \in \textit{Ab}(X_\etale)$
where the transition maps are the ones constructed in
Remark \ref{remark-covariance-f-shriek-separated}.
\end{lemma}

\begin{proof}
It suffices to show that the canonical map from
right to left is a bijection when evaluated on a quasi-compact
object $V$ of $Y_\etale$.
Observe that the colimit on the right hand side is directed
and has injective transition maps.
Thus we can use
Sites, Lemma \ref{sites-lemma-directed-colimits-sections}
to evaluate the colimit. Hence, the statement comes down
to the observation that a closed subset $Z \subset X_V$ proper over $V$
is quasi-compact and hence is contained in $X_{i, V}$ for some $i$.
\end{proof}

\begin{lemma}
\label{lemma-f-shriek-separated-direct-sums}
Let $f : X \to Y$ be a morphism of schemes which is separated and
locally of finite type. Then functor $f_!$ commutes with direct sums.
\end{lemma}

\begin{proof}
Let $\mathcal{F} = \bigoplus \mathcal{F}_i$. To show that the map
$\bigoplus f_!\mathcal{F}_i \to f_!\mathcal{F}$ is an isomorphism,
it suffices to show that these sheaves have the same sections over
a quasi-compact object $V$ of $Y_\etale$. Replacing $Y$ by $V$
it suffices to show
$H^0(Y, f_!\mathcal{F}) \subset H^0(X, \mathcal{F})$
is equal to
$\bigoplus H^0(Y, f_!\mathcal{F}_i)
\subset \bigoplus H^0(X, \mathcal{F}_i)
\subset H^0(X, \bigoplus \mathcal{F}_i)$.
In this case, by writing $X$ as the union of its quasi-compact opens
and using Lemma \ref{lemma-colim-f-shriek-separated}
we reduce to the case where $X$ is quasi-compact as well.
Then $H^0(X, \mathcal{F}) = \bigoplus H^0(X, \mathcal{F}_i)$
by \'Etale Cohomology, Theorem \ref{etale-cohomology-theorem-colimit}.
Looking at supports of sections the reader easily concludes.
\end{proof}

\begin{lemma}
\label{lemma-lqf-f-shriek-separated-colimits}
Let $f : X \to Y$ be a morphism of schemes which is separated and
locally quasi-finite. Then
\begin{enumerate}
\item for $\mathcal{F}$ in $\textit{Ab}(X_\etale)$ and a geometric
point $\overline{y} : \Spec(k) \to Y$ we have
$$
(f_!\mathcal{F})_{\overline{y}} =
\bigoplus\nolimits_{f(\overline{x}) = \overline{y}} \mathcal{F}_{\overline{x}}
$$
functorially in $\mathcal{F}$, and
\item the functor $f_!$ is exact.
\end{enumerate}
\end{lemma}

\begin{proof}
The functor $f_!$ is left exact by construction. Right exactness may
be checked on stalks
(\'Etale Cohomology, Theorem \ref{etale-cohomology-theorem-exactness-stalks}).
Thus it suffices to prove part (1).

\medskip\noindent
Let $\overline{y} : \Spec(k) \to Y$ be a geometric point.
The scheme $X_{\overline{y}}$ has a discrete underlying
topological space
(Morphisms, Lemma \ref{morphisms-lemma-locally-quasi-finite-fibres})
and all the residue fields at the points are equal to $k$
(as finite extensions of $k$). Hence
$\{\overline{x} : \Spec(k) \to X : f(\overline{x}) = \overline{y}\}$
is equal to the set of points of $X_{\overline{y}}$.
Thus the computation of the stalk follows from the more general
Lemma \ref{lemma-stalk-f-shriek-separated}.
\end{proof}









\section{Sections with finite support}
\label{section-finite-support}

\noindent
In this section we extend the construction of
Section \ref{section-compact-support} to not necessarily
separated locally quasi-finite morphisms.

\medskip\noindent
Let $f : X \to Y$ be a locally quasi-finite morphism of schemes.
Let $\mathcal{F}$ be an abelian sheaf on $X_\etale$. Given $V$ in
$Y_\etale$ denote $X_V = X \times_Y V$ the base change. We are going
to consider the group of finite formal sums
\begin{equation}
\label{equation-formal-sum}
s = \sum\nolimits_{i = 1, \ldots, n} (Z_i, s_i)
\end{equation}
where $Z_i \subset X_V$ is a locally closed subscheme such that the
morphism $Z_i \to V$ is finite\footnote{Since $f$ is locally quasi-finite,
the morphism $Z_i \to V$ is finite if and only if it is proper.}
and where $s_i \in H_{Z_i}(\mathcal{F})$. Here, as in
Section \ref{section-growing}, we set
$$
H_{Z_i}(\mathcal{F}) =
\{s_i \in \mathcal{F}(U_i) \mid \text{Supp}(s_i) \subset Z_i\}
$$
where $U_i \subset X_V$ is an open subscheme containing $Z_i$ as a
closed subscheme. We are going to consider these formal sums modulo the
following relations
\begin{enumerate}
\item
\label{item-sum}
$(Z, s) + (Z, s') = (Z, s + s')$,
\item
\label{item-sub}
$(Z, s) = (Z', s)$ if $Z \subset Z'$.
\end{enumerate}
Observe that the second relation makes sense: since $Z \to V$ is finite
and $Z' \to V$ is separated, the inclusion $Z \to Z'$ is closed and we
can use the map discussed in (\ref{item-inclusion}).

\medskip\noindent
Let us denote $f_{p!}\mathcal{F}(V)$ the quotient of the abelian
group of formal sums (\ref{equation-formal-sum}) by these relations.
The first relation tells us that $f_{p!}\mathcal{F}(V)$ is a quotient
of the direct sum of the abelian groups $H_Z(\mathcal{F})$
over all locally closed subschemes $Z \subset X_V$ finite over $V$.
The second relation tells us that we are really taking the colimit
\begin{equation}
\label{equation-colimit-definition}
f_{p!}\mathcal{F}(V) = \colim_Z H_Z(\mathcal{F})
\end{equation}
This formula will be a convenient abstract way to think about
our construction.

\medskip\noindent
Next, we observe that there is a natural way to turn this construction
into a presheaf $f_{p!}\mathcal{F}$ of abelian groups on $Y_\etale$.
Namely, given $V' \to V$ in $Y_\etale$ we obtain the base change morphism
$X_{V'} \to X_V$. If $Z \subset X_V$ is a locally closed subscheme
finite over $V$, then the scheme theoretic inverse image $Z' \subset X_{V'}$
is finite over $V'$. Moreover, if $U \subset X_V$ is an open such
that $Z$ is closed in $U$, then the inverse image $U' \subset X_{V'}$
is an open such that $Z'$ is closed in $U'$. Hence the restriction
mapping $\mathcal{F}(U) \to \mathcal{F}(U')$ of $\mathcal{F}$
sends $H_Z(\mathcal{F})$ into $H_{Z'}(\mathcal{F})$; this is a special
case of the functoriality discussed in (\ref{item-pullback}) above.
Clearly, these maps are compatible with inclusions
$Z_1 \subset Z_2$ of such locally closed subschemes of $X_V$ and
we obtain a map
$$
f_{p!}\mathcal{F}(V) = \colim_Z H_Z(\mathcal{F})
\longrightarrow
\colim_{Z'} H_{Z'}(\mathcal{F}) =
f_{p!}\mathcal{F}(V')
$$
These maps indeed turn $f_{p!}\mathcal{F}$ into a presheaf of abelian
groups on $Y_\etale$. We omit the details.

\medskip\noindent
A final observation is that the construction of $f_{p!}\mathcal{F}$
is functorial in $\mathcal{F}$ in $\textit{Ab}(X_\etale)$.
We conclude that given a locally quasi-finite morphism $f : X \to Y$
we have constructed a functor
$$
f_{p!} :
\textit{Ab}(X_\etale)
\longrightarrow
\textit{PAb}(Y_\etale)
$$
from the category of abelian sheaves on $X_\etale$ to the category
of abelian presheaves on $Y_\etale$. Before we define $f_!$ as the
sheafification of this functor, let us check that it agrees with
the construction in Section \ref{section-compact-support}
and with the construction in
\'Etale Cohomology, Section \ref{etale-cohomology-section-extension-by-zero}
when both apply.

\begin{lemma}
\label{lemma-finite-support-f-shriek-separated}
Let $f : X \to Y$ be a separated and locally quasi-finite morphism
of schemes. Functorially in $\mathcal{F} \in \textit{Ab}(X_\etale)$
there is a canonical isomorphism(!)
$$
f_{p!}\mathcal{F} \longrightarrow f_!\mathcal{F}
$$
of abelian presheaves which identifies the sheaf
$f_!\mathcal{F}$ of Definition \ref{definition-f-shriek-separated}
with the presheaf $f_{p!}\mathcal{F}$ constructed above.
\end{lemma}

\begin{proof}
Let $V$ be an object of $Y_\etale$. If $Z \subset X_V$ is locally closed
and finite over $V$, then, since $f$ is separated, we see that
the morphism $Z \to X_V$ is a closed immersion. Moreover, if
$Z_i$, $i = 1, \ldots, n$ are closed subschemes of $X_V$ finite
over $V$, then $Z_1 \cup \ldots \cup Z_n$ (scheme theoretic union)
is a closed subscheme finite over $V$. Hence in this case the colimit
(\ref{equation-colimit-definition}) defining $f_{p!}\mathcal{F}(V)$
is directed and we find that $f_{!p}\mathcal{F}(V)$ is simply equal
to the set of sections of $\mathcal{F}(X_V)$ whose support is finite over $V$.
Since any closed subset of $X_V$ which is proper over $V$ is
actually finite over $V$ (as $f$ is locally quasi-finite)
we conclude that this is equal to $f_!\mathcal{F}(V)$
by its very definition.
\end{proof}

\begin{lemma}
\label{lemma-finite-support-stalk}
Let $f : X \to Y$ be a morphism of schemes which is locally quasi-finite.
Let $\overline{y} : \Spec(k) \to Y$ be a geometric point.
Functorially in $\mathcal{F}$ in $\textit{Ab}(X_\etale)$ we have
$$
(f_{p!}\mathcal{F})_{\overline{y}} =
\bigoplus\nolimits_{f(\overline{x}) = \overline{y}} \mathcal{F}_{\overline{x}}
$$
\end{lemma}

\begin{proof}
Recall that the stalk at $\overline{y}$ of a presheaf is defined by the
usual colimit over \'etale neighbourhoods $(V, \overline{v})$
of $\overline{y}$, see \'Etale Cohomology, Definition
\ref{etale-cohomology-definition-stalk}. Accordingly
suppose $s = \sum_{i = 1, \ldots, n} (Z_i, s_i)$ as in
(\ref{equation-formal-sum}) is an element of $f_{p!}\mathcal{F}(V)$
where $(V, \overline{v})$ is an \'etale neighbourhood of $\overline{y}$.
Then since
$$
X_{\overline{y}} = (X_V)_{\overline{v}} \supset Z_{i, \overline{v}}
$$
and since $s_i$ is a section of $\mathcal{F}$ on an open neighbourhood
of $Z_i$ in $X_V$ we can send $s$ to
$$
\sum\nolimits_{i = 1, \ldots, n} 
\sum\nolimits_{\overline{x} \in Z_{i, \overline{v}}}
\left(\text{class of }s_i\text{ in }\mathcal{F}_{\overline{x}}\right)
\quad\in\quad
\bigoplus\nolimits_{f(\overline{x}) = \overline{y}} \mathcal{F}_{\overline{x}}
$$
We omit the verification that this is compatible with restriction
maps and that the relations (\ref{item-sum}) $(Z, s) + (Z, s') - (Z, s + s')$
and (\ref{item-sub}) $(Z, s) - (Z', s)$ if $Z \subset Z'$ are sent to zero.
Thus we obtain a map
$$
(f_{p!}\mathcal{F})_{\overline{y}}
\longrightarrow
\bigoplus\nolimits_{f(\overline{x}) = \overline{y}} \mathcal{F}_{\overline{x}}
$$

\medskip\noindent
Let us prove this arrow is surjective. For this it suffices to pick
an $\overline{x}$ with $f(\overline{x}) = \overline{y}$ and prove that
an element $s$ in the summand $\mathcal{F}_{\overline{x}}$ is in the
image. Let $s$ correspond to the element $s \in \mathcal{F}(U)$
where $(U, \overline{u})$ is an \'etale neighbourhood of $\overline{x}$.
Since $f$ is locally quasi-finite, the morphism $U \to Y$
is locally quasi-finite too. By More on Morphisms, Lemma
\ref{more-morphisms-lemma-etale-makes-quasi-finite-finite-multiple-points-var}
we can find an \'etale neighbourhood $(V, \overline{v})$ of
$\overline{y}$, an open subscheme
$$
W \subset U \times_Y V,
$$
and a geometric point $\overline{w}$ mapping to $\overline{u}$ and
$\overline{v}$ such that $W \to V$ is finite and $\overline{w}$ is the
only geometric point of $W$ mapping to $\overline{v}$. (We omit the translation
between the language of geometric points we are currently using and the
language of points and residue field extensions used in the
statement of the lemma.) Observe that $W \to X_V = X \times_Y V$
is \'etale. Choose an affine open neighbourhood $W' \subset X_V$
of the image $\overline{w}'$ of $\overline{w}$. Since $\overline{w}$
is the only point of $W$ over $\overline{v}$ and since $W \to V$
is closed, after replacing $V$ by an open neighbourhood of $\overline{v}$,
we may assume $W \to X_V$ maps into $W'$. Then $W \to W'$ is finite and
\'etale and there is a unique geometric point $\overline{w}$ of $W$
lying over $\overline{w}'$. It follows that $W \to W'$ is an open immersion
over an open neighbourhood of $\overline{w}'$ in $W'$, see
\'Etale Morphisms, Lemma \ref{etale-lemma-finite-etale-one-point}.
Shrinking $V$ and $W'$ we may assume $W \to W'$ is an isomorphism.
Thus $s$ may be viewed as a section $s'$ of $\mathcal{F}$ over
the open subscheme $W' \subset X_V$ which is finite over $V$.
Hence by definition $(W', s')$ defines an element of $j_{p!}\mathcal{F}(V)$
which maps to $s$ as desired.

\medskip\noindent
Let us prove the arrow is injective. To do this, let
$s = \sum_{i = 1, \ldots, n} (Z_i, s_i)$ as in (\ref{equation-formal-sum})
be an element of $f_{p!}\mathcal{F}(V)$ where $(V, \overline{v})$ is an
\'etale neighbourhood of $\overline{y}$. Assume $s$ maps to zero
under the map constructed above. First, after replacing
$(V, \overline{v})$ by an \'etale neighbourhood of itself,
we may assume there exist decompositions
$Z_i = Z_{i, 1} \amalg \ldots \amalg Z_{i, m_i}$ into open and closed
subschemes such that each $Z_{i, j}$ has exactly one geometric point
over $\overline{v}$. Say under the obvious direct sum decomposition
$$
H_{Z_i}(\mathcal{F}) = \bigoplus H_{Z_{i, j}}(\mathcal{F})
$$
the element $s_i$ corresponds to $\sum s_{i, j}$. We may use relations
(\ref{item-sum}) and (\ref{item-sub}) to replace $s$ by
$\sum_{i = 1, \ldots, n} \sum_{j = 1, \ldots, m_i} (Z_{i, j}, s_{i, j})$.
In other words, we may assume $Z_i$ has a unique geometric point
lying over $\overline{v}$. Let $\overline{x}_1, \ldots, \overline{x}_m$
be the geometric points of $X$ over $\overline{y}$ corresponding to
the geometric points of our $Z_i$ over $\overline{v}$; note that for
one $j \in \{1, \ldots, m\}$ there may be multiple indices $i$ for which
$\overline{x}_j$ corresponds to a point of $Z_i$.
By More on Morphisms, Lemma
\ref{more-morphisms-lemma-etale-makes-quasi-finite-finite-multiple-points-var}
applied to both $X_V \to V$
after replacing $(V, \overline{v})$ by an \'etale neighbourhood of itself
we may assume there exist open subschemes
$$
W_j \subset X \times_Y V,\quad j = 1, \ldots, m
$$
and a geometric point $\overline{w}_j$ of $W_j$ mapping to $\overline{x}_j$ and
$\overline{v}$ such that $W_j \to V$ is finite and $\overline{w}_j$ is the
only geometric point of $W_j$ mapping to $\overline{v}$.
After shrinking $V$ we may assume $Z_i \subset W_j$ for some $j$
and we have the map $H_{Z_i}(\mathcal{F}) \to H_{W_j}(\mathcal{F})$.
Thus by the relation (\ref{item-sub})
we see that our element is equivalent to an element of the form
$$
\sum\nolimits_{j = 1, \ldots, m} (W_j, t_j)
$$
for some $t_j \in H_{W_j}(\mathcal{F})$. Clearly, this element is mapped
simply to the class of $t_j$ in the summand $\mathcal{F}_{\overline{x}_j}$.
Since $s$ maps to zero, we find that $t_j$ maps to zero in
$\mathcal{F}_{\overline{x}_j}$. This implies that $t_j$ restricts
to zero on an open neighbourhood of $\overline{w}_j$ in $W_j$, see
\'Etale Cohomology, Lemma \ref{etale-cohomology-lemma-zero-over-image}.
Shrinking $V$ once more we obtain $t_j = 0$ for all $j$ as desired.
\end{proof}

\begin{lemma}
\label{lemma-finite-support-etale-shriek}
Let $f = j : U \to X$ be an \'etale of schemes. Denote $j_{p!}$
the construction of \'Etale Cohomology, Equation
(\ref{etale-cohomology-equation-j-p-shriek})
and denote $f_{p!}$ the construction above. Functorially in
$\mathcal{F} \in \textit{Ab}(X_\etale)$ there is a canonical map
$$
j_{p!}\mathcal{F} \longrightarrow f_{p!}\mathcal{F}
$$
of abelian presheaves which identifies the sheaf
$j_!\mathcal{F} = (j_{p!}\mathcal{F})^\#$ of \'Etale Cohomology,
Definition \ref{etale-cohomology-definition-extension-zero}
with $(f_{p!}\mathcal{F})^\#$.
\end{lemma}

\begin{proof}
Please read the proof of \'Etale Cohomology, Lemma
\ref{etale-cohomology-lemma-shriek-into-star-separated-etale}
before reading the proof of this lemma.
Let $V$ be an object of $X_\etale$. Recall that
$$
j_{p!}\mathcal{F}(V) =
\bigoplus\nolimits_{\varphi : V \to U} \mathcal{F}(V \xrightarrow{\varphi} U)
$$
Given $\varphi$ we obtain an open subscheme
$Z_\varphi \subset U_V = U \times_X V$, namely,
the image of the graph of $\varphi$. Via $\varphi$
we obtain an isomorphism $V \to Z_\varphi$ over $U$
and we can think of an element
$$
s_\varphi \in \mathcal{F}(V \xrightarrow{\varphi} U) =
\mathcal{F}(Z_\varphi) = H_{Z_\varphi}(\mathcal{F})
$$
as a section of $\mathcal{F}$ over $Z_{\varphi}$. Since
$Z_\varphi \subset U_V$ is open, we actually have
$H_{Z_\varphi}(\mathcal{F}) = \mathcal{F}(Z_\varphi)$
and we can think of $s_\varphi$ as an element of $H_{Z_\varphi}(\mathcal{F})$.
Having said this, our map $j_{p!}\mathcal{F} \to f_{p!}\mathcal{F}$
is defined by the rule
$$
\sum\nolimits_{i = 1, \ldots, n} s_{\varphi_i}
\longmapsto
\sum\nolimits_{i = 1, \ldots, n} (Z_{\varphi_i}, s_{\varphi_i})
$$
with right hand side a sum as in (\ref{equation-formal-sum}).
We omit the verification that this is compatible with restriction
mappings and functorial in $\mathcal{F}$.

\medskip\noindent
To finish the proof, we claim that given a geometric point
$\overline{y} : \Spec(k) \to Y$ there is a commutative diagram
$$
\xymatrix{
(j_{p!}\mathcal{F})_{\overline{y}} \ar[r] \ar[d] &
\bigoplus_{j(\overline{x}) = \overline{y}} \mathcal{F}_{\overline{x}}
\ar@{=}[d] \\
(f_{p!}\mathcal{F})_{\overline{y}} \ar[r] &
\bigoplus_{f(\overline{x}) = \overline{y}} \mathcal{F}_{\overline{x}}
}
$$
where the top horizontal arrow is constructed in the proof of
\'Etale Cohomology, Proposition
\ref{etale-cohomology-proposition-describe-jshriek},
the bottom horizontal arrow is constructed in the proof of
Lemma \ref{lemma-finite-support-stalk},
the right vertical arrow is the obvious equality, and
the left veritical arrow is the map defined in the previous
paragraph on stalks. The claim follows in a straightforward manner
from the explicit description of all of the arrows involved
here and in the references given.
Since the horizontal arrows are isomorphisms
we conclude so is the left vertical arrow. Hence we find that
our map induces an isomorphism on sheafifications by
\'Etale Cohomology, Theorem \ref{etale-cohomology-theorem-exactness-stalks}.
\end{proof}

\begin{definition}
\label{definition-f-shriek-lqf}
Let $f : X \to Y$ be a locally quasi-finite morphism of schemes.
We define the {\it direct image with compact support} to be the
functor
$$
f_! : \textit{Ab}(X_\etale) \longrightarrow \textit{Ab}(Y_\etale)
$$
defined by the formula $f_!\mathcal{F} = (f_{p!}\mathcal{F})^\#$,
i.e., $f_!\mathcal{F}$ is the sheafification of the presheaf
$f_{p!}\mathcal{F}$ constructed above.
\end{definition}

\noindent
By Lemma \ref{lemma-finite-support-f-shriek-separated}
this does not conflict with Definition \ref{definition-f-shriek-separated}
(when both definitions apply) and by
Lemma \ref{lemma-finite-support-etale-shriek}
this does not conflict with
\'Etale Cohomology, Definition \ref{etale-cohomology-definition-extension-zero}
(when both definitions apply).

\begin{lemma}
\label{lemma-lqf-f-shriek-stalk}
Let $f : X \to Y$ be a locally quasi-finite morphism of schemes. Then
\begin{enumerate}
\item for $\mathcal{F}$ in $\textit{Ab}(X_\etale)$ and a geometric
point $\overline{y} : \Spec(k) \to Y$ we have
$$
(f_!\mathcal{F})_{\overline{y}} =
\bigoplus\nolimits_{f(\overline{x}) = \overline{y}} \mathcal{F}_{\overline{x}}
$$
functorially in $\mathcal{F}$, and
\item the functor $f_! : \textit{Ab}(X_\etale) \to \textit{Ab}(Y_\etale)$
is exact and commutes with direct sums.
\end{enumerate}
\end{lemma}

\begin{proof}
The formula for the stalks is immediate (and in fact equivalent) to
Lemma \ref{lemma-finite-support-stalk}.
The exactness of the functor follows immediately from this
and the fact that exactness may be checked on stalks, see
\'Etale Cohomology, Theorem \ref{etale-cohomology-theorem-exactness-stalks}.
\end{proof}

\begin{remark}[Covariance with respect to open embeddings]
\label{remark-covariance-lqf-f-shriek}
Let $f : X \to Y$ be locally quasi-finite morphism of schemes. Let
$\mathcal{F}$ be an abelian sheaf on $X_\etale$.
Let $X' \subset X$ be an open subscheme and denote $f' : X' \to Y$
the restriction of $f$.
We claim there is a canonical map
$$
f'_!(\mathcal{F}|_{X'}) \longrightarrow f_!\mathcal{F}
$$
Namely, this map will be the sheafification of a canonical map
$$
f'_{p!}(\mathcal{F}|_{X'}) \to f_{p!}\mathcal{F}
$$
constructed as follows. Let $V \in Y_\etale$ and consider a section
$s' = \sum_{i = 1, \ldots, n} (Z'_i, s'_i)$ as in
(\ref{equation-formal-sum}) defining an element of
$f'_{p!}(\mathcal{F}|_{X'})(V)$.
Then $Z'_i \subset X'_V$ may also be viewed as a locally closed subscheme
of $X_V$ and we have $H_{Z'_i}(\mathcal{F}|_{X'}) = H_{Z'_i}(\mathcal{F})$.
We will map $s'$ to the exact same sum
$s = \sum_{i = 1, \ldots, n} (Z'_i, s'_i)$
but now viewed as an element of $f_{p!}\mathcal{F}(V)$.
We omit the verification that this construction is compatible with
restriction mappings and functorial in $\mathcal{F}$.
This construction has the following properties:
\begin{enumerate}
\item The maps $f'_{p!}\mathcal{F}' \to f_{p!}\mathcal{F}$ and
$f'_!\mathcal{F}' \to f_!\mathcal{F}$ are compatible with
the description of stalks given in Lemmas
\ref{lemma-finite-support-stalk} and \ref{lemma-lqf-f-shriek-stalk}.
\item If $f$ is separated, then the map
$f'_{p!}\mathcal{F}' \to f_{p!}\mathcal{F}$ is the same as the map
constructed in Remark \ref{remark-covariance-f-shriek-separated}
via the isomorphism in Lemma \ref{lemma-finite-support-f-shriek-separated}.
\item If $X'' \subset X'$ is another open, then the composition of
$f''_{p!}(\mathcal{F}|_{X''}) \to f'_{p!}(\mathcal{F}|_{X'}) \to
f_{p!}\mathcal{F}$ is the map
$f''_{p!}(\mathcal{F}|_{X''}) \to f_{p!}\mathcal{F}$ for the
inclusion $X'' \subset X$. Sheafifying we conclude
the same holds true for
$f''_!(\mathcal{F}|_{X''}) \to f'_!(\mathcal{F}|_{X'}) \to f_!\mathcal{F}$.
\item The map $f'_!\mathcal{F}' \to f_!\mathcal{F}$ is injective
because we can check this on stalks.
\end{enumerate}
All of these statements are easily proven by representing elements
as finite sums as above and considering what happens to these elements.
\end{remark}

\begin{lemma}
\label{lemma-lqf-colimit-f-shriek}
Let $f : X \to Y$ be a locally quasi-finite morphism of schemes.
Let $X = \bigcup_{i \in I} X_i$ be an open covering. Then there
exists an exact complex
$$
\ldots \to
\bigoplus\nolimits_{i_0, i_1, i_2} f_{i_0i_1i_2, !}
\mathcal{F}|_{X_{i_0i_1i_2}} \to
\bigoplus\nolimits_{i_0, i_1} f_{i_0i_1, !} \mathcal{F}|_{X_{i_0i_1}} \to
\bigoplus\nolimits_{i_0} f_{i_0, !} \mathcal{F}|_{X_{i_0}}
\to f_!\mathcal{F} \to 0
$$
functorial in $\mathcal{F} \in \textit{Ab}(X_\etale)$, see
proof for details.
\end{lemma}

\begin{proof}
Here as usual we set $X_{i_0 \ldots i_p} = X_{i_0} \cap \ldots \cap X_{i_p}$
and we denote $f_{i_0 \ldots i_p}$ the restriction of $f$ to
$X_{i_0 \ldots i_p}$. The maps in the complex are the maps
constructed in Remark \ref{remark-covariance-lqf-f-shriek}
with sign rules as in the {\v C}ech complex.
Exactness follows easily from the description of stalks in
Lemma \ref{lemma-lqf-f-shriek-stalk}. Details omitted.
\end{proof}

\begin{remark}[Alternative construction]
\label{remark-alternative-lqf-f-shriek}
Lemma \ref{lemma-lqf-colimit-f-shriek}
gives an alternative construction of the functor $f_!$
for locally quasi-finite morphisms $f$.
Namely, given a locally quasi-finite morphism $f : X \to Y$ of schemes
we can choose an open covering $X = \bigcup_{i \in I} X_i$
such that each $f_i : X_i \to Y$ is separated. For example choose
an affine open covering of $X$. Then we
can define $f_!\mathcal{F}$ as the cokernel of the penultimate map
of the complex of the lemma, i.e.,
$$
f_!\mathcal{F} = \Coker\left(
\bigoplus\nolimits_{i_0, i_1} f_{i_0i_1, !} \mathcal{F}|_{X_{i_0i_1}} \to
\bigoplus\nolimits_{i_0} f_{i_0, !} \mathcal{F}|_{X_{i_0}}
\right)
$$
where we can use the construction of $f_{i_0, !}$ and
$f_{i_0i_1, !}$ in Section \ref{section-compact-support}
because the morphisms $f_{i_0}$ and $f_{i_0 i_1}$ are separated.
One can then compute the stalks of $f_!$ (using the separated
case, namely Lemma \ref{lemma-lqf-f-shriek-separated-colimits})
and obtain the result of Lemma \ref{lemma-lqf-f-shriek-stalk}.
Having done so all the other results of this section can be
deduced from this as well.
\end{remark}

\begin{remark}
\label{remark-construct-map-presheaves-downstairs}
Let $g : Y' \to Y$ be a morphism of schemes.
For an abelian presheaf $\mathcal{G}'$ on $Y'_\etale$ let us denote
$g_*\mathcal{G}'$ the presheaf $V \mapsto \mathcal{G}'(Y' \times_Y V)$.
If $\alpha : \mathcal{G} \to g_*\mathcal{G}'$ is a map of abelian presheaves
on $Y_\etale$, then there is a unique map
$\alpha^\# : \mathcal{G}^\# \to g_*((\mathcal{G}')^\#)$
of abelian sheaves on $Y_\etale$ such that the diagram
$$
\xymatrix{
\mathcal{G} \ar[d] \ar[r]_\alpha & g_*\mathcal{G}' \ar[d] \\
\mathcal{G}^\# \ar[r]^-{\alpha^\#} & g_*((\mathcal{G}')^\#)
}
$$
is commutative where the vertical maps come from the canonical maps
$\mathcal{G} \to \mathcal{G}^\#$ and $\mathcal{G}' \to (\mathcal{G}')^\#$. If
$\alpha' : g^{-1}\mathcal{G}^\# \to (\mathcal{G}')^\#$
is the map adjoint to $\alpha^\#$, then for a geometric point
$\overline{y}' : \Spec(k) \to Y'$ with image
$\overline{y} = g \circ \overline{y}'$ in $Y$, the map
$$
\alpha'_{\overline{y}'} :
\mathcal{G}_{\overline{y}} =
(\mathcal{G}^\#)_{\overline{y}} =
(g^{-1}\mathcal{G}^\#)_{\overline{y}'}
\longrightarrow
(\mathcal{G}')^\#_{\overline{y}'} =
\mathcal{G}'_{\overline{y}'}
$$
is given by mapping the class in the stalk of a section $s$ of $\mathcal{G}$
over an \'etale neighbourhood $(V, \overline{v})$ to the class of the section
$\alpha(s)$ in $g_*\mathcal{G}'(V) = \mathcal{G}'(Y' \times_Y V)$
over the \'etale neighbourhood $(Y' \times_Y V, (\overline{y}', \overline{v}))$
in the stalk of $\mathcal{G}'$ at $\overline{y}'$.
\end{remark}

\begin{lemma}
\label{lemma-lqf-base-change-f-shriek}
Consider a cartesian square
$$
\xymatrix{
X' \ar[r]_{g'} \ar[d]_{f'} & X \ar[d]^f \\
Y' \ar[r]^g & Y
}
$$
of schemes with $f$ locally quasi-finite. There is an isomorphism
$g^{-1}f_!\mathcal{F} \to f'_!(g')^{-1}\mathcal{F}$ functorial for
$\mathcal{F}$ in $\textit{Ab}(X_\etale)$ which is compatible with
the descriptions of stalks given in Lemma \ref{lemma-lqf-f-shriek-stalk}
(see proof for the precise statement).
\end{lemma}

\begin{proof}
With conventions as in Remark \ref{remark-construct-map-presheaves-downstairs}
we will explicitly construct a map
$$
c : f_{p!}\mathcal{F} \longrightarrow g_*f'_{p!}(g')^{-1}\mathcal{F}
$$
of abelian presheaves on $Y_\etale$. By the discussion in
Remark \ref{remark-construct-map-presheaves-downstairs}
this will determine a canonical map
$g^{-1}f_!\mathcal{F} \to f'_!(g')^{-1}\mathcal{F}$. Finally, we
will show this map induces isomorphisms on stalks and conclude by
\'Etale Cohomology, Theorem \ref{etale-cohomology-theorem-exactness-stalks}.

\medskip\noindent
Construction of the map $c$. Let $V \in Y_\etale$ and consider a section
$s = \sum_{i = 1, \ldots, n} (Z_i, s_i)$ as in
(\ref{equation-formal-sum}) defining an element of $f_{p!}\mathcal{F}(V)$.
The value of $g_*f'_{p!}(g')^{-1}\mathcal{F}$ at $V$ is
$f'_{p!}(g')^{-1}\mathcal{F}(V')$ where $V' = V \times_Y Y'$.
Denote $Z'_i \subset X'_{V'}$ the base change of $Z_i$ to $V'$.
By (\ref{item-pullback}) there is a pullback map
$H_{Z_i}(\mathcal{F}) \to H_{Z'_i}((g')^{-1}\mathcal{F})$.
Denoting $s'_i \in H_{Z'_i}((g')^{-1}\mathcal{F})$ the image of $s_i$
under pullback, we set $c(s) = \sum_{i = 1, \ldots, n} (Z'_i, s'_i)$ as in
(\ref{equation-formal-sum}) defining an element of
$f'_{p!}(g')^{-1}\mathcal{F}(V')$. We omit the verification
that this construction is compatible the relations
(\ref{item-sum}) and (\ref{item-sub}) and compatible
with restriction mappings. The construction is clearly
functorial in $\mathcal{F}$.

\medskip\noindent
Let $\overline{y}' : \Spec(k) \to Y'$ be a geometric point with image
$\overline{y} = g \circ \overline{y}'$ in $Y$. Observe that
$X'_{\overline{y}'} = X_{\overline{y}}$ by transitivity of
fibre products. Hence $g'$ produces a bijection
$\{f'(\overline{x}') = \overline{y}'\} \to \{f(\overline{x}) = \overline{y}\}$
and if $\overline{x}'$ maps to $\overline{x}$, then
$((g')^{-1}\mathcal{F})_{\overline{x}'} = \mathcal{F}_{\overline{x}}$
by \'Etale Cohomology, Lemma \ref{etale-cohomology-lemma-stalk-pullback}.
Now we claim that the diagram
$$
\xymatrix{
(g^{-1}f_!\mathcal{F})_{\overline{y}'} \ar@{=}[r] \ar[d] &
(f_!\mathcal{F})_{\overline{y}} \ar[r] \ar[ld] &
\bigoplus\nolimits_{f(\overline{x}) = \overline{y}}
\mathcal{F}_{\overline{x}} \ar[d]
\\
(f'_!(g')^{-1}\mathcal{F})_{\overline{y}'} \ar[rr] & &
\bigoplus\nolimits_{f'(\overline{x}') = \overline{y}'}
(g')^{-1}\mathcal{F}_{\overline{x}'}
}
$$
commutes where the horizontal arrows are given in the proof of
Lemma \ref{lemma-finite-support-stalk} and where the right vertical
arrow is an equality by what we just said above. The southwest arrow is
described in Remark \ref{remark-construct-map-presheaves-downstairs}
as the pullback map, i.e.,
simply given by our construction $c$ above. Then the simple
description of the image of a sum $\sum (Z_i, z_i)$ in the
stalk at $\overline{x}$ given in the proof of
Lemma \ref{lemma-finite-support-stalk} immediately shows the
diagram commutes. This finishes the proof of the lemma.
\end{proof}

\begin{lemma}
\label{lemma-lqf-separated-shriek-composition}
Let $f' : X \to Y'$ and $g : Y' \to Y$ be composable morphisms of schemes
with $f'$ and $f = g \circ f'$ locally quasi-finite and $g$ separated and
locally of finite type. Then there is a canonical isomorphism of functors
$g_! \circ f'_! = f_!$. This isomorphism is compatible with
\begin{enumerate}
\item[(a)] covariance with respect to open embeddings as in
Remarks \ref{remark-covariance-f-shriek-separated} and
\ref{remark-covariance-lqf-f-shriek},
\item[(b)] the base change isomorphisms of
Lemmas \ref{lemma-lqf-base-change-f-shriek}
and \ref{lemma-base-change-f-shriek-separated}, and
\item[(c)] equal to the isomorphism of Lemma \ref{lemma-f-shriek-composition}
via the identifications of Lemma \ref{lemma-finite-support-f-shriek-separated}
in case $f'$ is separated.
\end{enumerate}
\end{lemma}

\begin{proof}
Let $\mathcal{F}$ be an abelian sheaf on $X_\etale$. With conventions as in
Remark \ref{remark-construct-map-presheaves-downstairs} we will explicitly
construct a map
$$
c : f_{p!}\mathcal{F} \longrightarrow g_*f'_{p!}\mathcal{F}
$$
of abelian presheaves on $Y_\etale$. By the discussion in
Remark \ref{remark-construct-map-presheaves-downstairs}
this will determine a canonical map
$c^\# : f_!\mathcal{F} \to g_*f'_!\mathcal{F}$.
We will show that $c^\#$ has image contained in the subsheaf
$g_!f'_!\mathcal{F}$, thereby obtaining a map
$c' : f_!\mathcal{F} \to g_!f'_!\mathcal{F}$. Next, we will prove
(a), (b), and (c) that. Finally, part (b)
will allow us to show that $c'$ is an isomorphism.

\medskip\noindent
Construction of the map $c$. Let $V \in Y_\etale$ and
let $s = \sum (Z_i, s_i)$ be a sum as in (\ref{equation-formal-sum})
defining an element of $f_{p!}\mathcal{F}(V)$.
Recall that $Z_i \subset X_V = X \times_Y V$
is a locally closed subscheme finite over $V$.
Setting $V' = Y' \times_Y V$ we get $X_{V'} = X \times_{Y'} V' = X_V$.
Hence $Z_i \subset X_{V'}$ is locally closed and
$Z_i$ is finite over $V'$ because $g$ is separated
(Morphisms, Lemma \ref{morphisms-lemma-finite-permanence}).
Hence we may set $c(s) = \sum (Z_i, s_i)$ but now viewed
as an element of $f'_{p!}\mathcal{F}(V') = (g_*f'_{p!}\mathcal{F})(V)$.
The construction is clearly compatible with relations
(\ref{item-sum}) and (\ref{item-sub})
and compatible with restriction mappings and hence we obtain the map $c$.

\medskip\noindent
Observe that in the discussion above our section $c(s) = \sum (Z_i, s_i)$ of
$f'_!\mathcal{F}$ over $V'$ restricts to zero on
$V' \setminus \Im(\coprod Z_i \to V')$. Since $\Im(\coprod Z_i \to V')$
is proper over $V$ (for example by Morphisms, Lemma
\ref{morphisms-lemma-scheme-theoretic-image-is-proper})
we conclude that $c(s)$ defines a section of
$g_!f'_!\mathcal{F} \subset g_*f'_!\mathcal{F}$ over $V$.
Since every local section of $f_!\mathcal{F}$ locally comes from a
local section of $f_{p!}\mathcal{F}$ we conclude that the image
of $c^\#$ is contained in $g_!f'_!\mathcal{F}$.
Thus we obtain an induced map $c' : f_!\mathcal{F} \to g_!f'_!\mathcal{F}$
factoring $c^\#$ as predicted in the first paragraph of the proof.

\medskip\noindent
Proof of (a). Let $Y'_1 \subset Y'$ be an open subscheme
and set $X_1 = (f')^{-1}(W')$. We obtain a diagram
$$
\xymatrix{
X_1 \ar[d]_{f'_1} \ar[r]_a \ar@/_2em/[dd]_{f_1} &
X \ar[d]^{f'} \ar@/^2em/[dd]^f \\
Y'_1 \ar[d]_{g_1} \ar[r]_{b'} &
Y' \ar[d]^g \\
Y \ar@{=}[r] &
Y
}
$$
where the horizontal arrows are open immersions. Then our claim is that
the diagram
$$
\xymatrix{
f_{1, !}\mathcal{F}|_{X_1} \ar[r]_{c'_1} \ar[dd] &
g_{1, !}f'_{1, !}\mathcal{F}|_{X_1} \ar@{=}[d] \\
& g_{1, !}(f'_!\mathcal{F})|_{Y'_1} \ar[d] \\
f_!\mathcal{F} \ar[r]^{c'} &
g_!f'_!\mathcal{F} \ar[r] & g_*f'_!\mathcal{F}
}
$$
commutes where the left vertical arrow is
Remark \ref{remark-covariance-lqf-f-shriek} and
the right vertical arrow is Remark \ref{remark-covariance-f-shriek-separated}.
The equality sign in the diagram comes about because $f'_1$
is the restriction of $f'$ to $Y'_1$ and our construction
of $f'_!$ is local on the base.
Finally, to prove the commutativity we choose an object $V$ of
$Y_\etale$ and a formal sum $s_1 = \sum (Z_{1, i}, s_{1, i})$ as in
(\ref{equation-formal-sum}) defining an element of
$f_{1, p!}\mathcal{F}|_{X_1}(V)$. Recall this means
$Z_{1, i} \subset X_1 \times_Y V$ is locally closed finite over $V$
and $s_{1, i} \in H_{Z_{1, i}}(\mathcal{F})$.
Then we chase this section
across the maps involved, but we only need to show we
end up with the same element of
$g_*f'_!\mathcal{F}(V) = f'_!\mathcal{F}(Y' \times_Y V)$.
Going around both sides of the diagram the reader immediately
sees we end up with the element $\sum (Z_{1, i}, s_{1, i})$
where now $Z_{1, i}$ is viewed as a locally closed subscheme
of $X \times_{Y'} (Y' \times_Y V) = X \times_Y V$ finite over
$Y' \times_Y V$.

\medskip\noindent
Proof of (b). Let $b : Y_1 \to Y$ be a morphism of schemes. Let us form the
commutative diagram
$$
\xymatrix{
X_1 \ar[d]_{f'_1} \ar[r]_a \ar@/_2em/[dd]_{f_1} &
X \ar[d]^{f'} \ar@/^2em/[dd]^f \\
Y'_1 \ar[d]_{g_1} \ar[r]_{b'} &
Y' \ar[d]^g \\
Y_1 \ar[r]^b &
Y
}
$$
with cartesian squares. We claim that our construction is compatible
with the base change maps of Lemmas \ref{lemma-lqf-base-change-f-shriek}
and \ref{lemma-base-change-f-shriek-separated}, i.e.,
that the top rectangle of the diagram
$$
\xymatrix{
b^{-1}f_!\mathcal{F} \ar[rr] \ar[d]_{b^{-1}c'} & &
f_{1, !}a^{-1}\mathcal{F} \ar[d]^{c_1'} \\
b^{-1}g_!f'_!\mathcal{F} \ar[r] \ar[d] &
g_{1, !}(b')^{-1}f'_!\mathcal{F} \ar[r] \ar[d] &
g_{1, !}f'_{1, !}a^{-1}\mathcal{F} \ar[d] \\
b^{-1}g_*f'_!\mathcal{F} \ar[r] &
g_{1, *}(b')^{-1}f'_!\mathcal{F} \ar[r] &
g_{1, *}f'_{1, !}a^{-1}\mathcal{F}
}
$$
commutes. The verification of this is completely routine and we
urge the reader to skip it. Since the arrows going from the middle
row down to the bottom row are injective, it suffices to show that
the outer diagram commutes.
To show this it suffices to take a local section of
$b^{-1}f_!\mathcal{F}$ and show we end up with the same local
section of $g_{1, *}f'_{1, !}a^{-1}\mathcal{F}$
going around either way. However, in fact it suffices to check
this for local sections which are of the the pullback by $b$ of
a section $s = \sum (Z_i, s_i)$ of $f_{p!}\mathcal{F}(V)$
as above (since such pullbacks generate the abelian sheaf
$b^{-1}f_!\mathcal{F}$). Denote $V_1$, $V'_1$, and $Z_{1, i}$
the base change of $V$, $V' = Y' \times_Y V$, $Z_i$ by $Y_1 \to Y$.
Recall that $Z_i$ is a locally closed subscheme of $X_V = X_{V'}$
and hence $Z_{1, i}$ is a locally closed subscheme
of $(X_1)_{V_1} = (X_1)_{V'_1}$. Then $b^{-1}c'$ sends the pullback
of $s$ to the pullback of the local section $c(s) \sum (Z_i, s_i)$ viewed
as an element of $f'_{p!}\mathcal{F}(V') = (g_*f'_{p!}\mathcal{F})(V)$.
The composition of the bottom two base change maps
simply maps this to $\sum (Z_{i, 1}, s_{1, i})$ viewed as an
element of $f'_{1, p!}a^{-1}\mathcal{F}(V'_1) =
g_{1, *}f'_{1, p!}a^{-1}\mathcal{F}(V_1)$.
On the other hand, the base change map at the top of the diagram
sends the pullback of $s$ to $\sum (Z_{1, i}, s_{1, i})$ viewed
as an element of $f_{1, !}a^{-1}\mathcal{F}(V_1)$.
Then finally $c'_1$ by its very construction does indeed
map this to $\sum (Z_{i, 1}, s_{1, i})$ viewed as an
element of $f'_{1, p!}a^{-1}\mathcal{F}(V'_1) =
g_{1, *}f'_{1, p!}a^{-1}\mathcal{F}(V_1)$ and the commutativity
has been verified.

\medskip\noindent
Proof of (c). This follows from comparing the definitions
for both maps; we omit the details.

\medskip\noindent
To finish the proof it suffices to show that the pullback of
$c'$ via any geometric point $\overline{y} : \Spec(k) \to Y$
is an isomorphism. Namely, pulling back by $\overline{y}$
is the same thing as taking stalks and $\overline{y}$
(\'Etale Cohomology, Remark \ref{etale-cohomology-remark-stalk-pullback})
and hence we can invoke
\'Etale Cohomology, Theorem \ref{etale-cohomology-theorem-exactness-stalks}.
By the compatibility (b) just shown, we 
conclude that we may assume $Y$ is the spectrum of $k$
and we have to show that $c'$ is an isomorphism.
To do this it suffices to show that the induced map
$$
\bigoplus\nolimits_{x \in X} \mathcal{F}_x = H^0(Y, f_!\mathcal{F})
\longrightarrow
H^0(Y, g_!f'_!\mathcal{F}) = H^0_c(Y', f'_!\mathcal{F})
$$
is an isomorphism. The equalities hold by
Lemmas \ref{lemma-lqf-f-shriek-stalk} and
\ref{lemma-stalk-f-shriek-separated}.
Recall that $X$ is a disjoint union of
spectra of Artinian local rings with residue field $k$, see
Varieties, Lemma \ref{varieties-lemma-algebraic-scheme-dim-0}.
Since the left and right hand side commute with direct
sums (details omitted) we may assume that $\mathcal{F}$ is a skyscraper
sheaf $x_*A$ supported at some $x \in X$.
Then $f'_!\mathcal{F}$ is the skyscraper sheaf at the
image $y'$ of $x$ in $Y$ by Lemma \ref{lemma-lqf-f-shriek-stalk}.
In this case it is obvious that our construction
produces the identity map $A \to H^0_c(Y', y'_*A) = A$
as desired.
\end{proof}

\begin{lemma}
\label{lemma-lqf-shriek-composition}
Let $f : X \to Y$ and $g : Y \to Z$ be composable locally quasi-finite
morphisms of schemes. Then there is a canonical isomorphism of functors
$$
(g \circ f)_! \longrightarrow g_! \circ f_!
$$
These isomorphisms satisfy the following properties:
\begin{enumerate}
\item If $f$ and $g$ are separated, then the isomorphism agrees
with Lemma \ref{lemma-f-shriek-composition}.
\item If $g$ is separated, then the isomorphism agrees with
Lemma \ref{lemma-lqf-separated-shriek-composition}.
\item For a geometric point $\overline{z} : \Spec(k) \to Z$ the diagram
$$
\xymatrix{
((g \circ f)_!\mathcal{F})_{\overline{z}} \ar[d] \ar[rr] & &
\bigoplus\nolimits_{g(f(\overline{x})) = \overline{z}}
\mathcal{F}_{\overline{x}} \ar@{=}[d] \\
(g_!f_!\mathcal{F})_{\overline{z}} \ar[r] &
\bigoplus\nolimits_{g(\overline{y}) = \overline{z}}
(f_!\mathcal{F})_{\overline{y}} \ar[r] &
\bigoplus\nolimits_{g(f(\overline{x})) = \overline{z}}
\mathcal{F}_{\overline{x}}
}
$$
is commutative where the horizontal arrows are given by
Lemma \ref{lemma-lqf-f-shriek-stalk}.
\item Let $h : Z \to T$ be a third locally quasi-finite
morphism of schemes. Then the diagram
$$
\xymatrix{
(h \circ g \circ f)_! \ar[r] \ar[d] &
(h \circ g)_! \circ f_! \ar[d] \\
h_! \circ (g \circ f)_! \ar[r] &
h_! \circ g_! \circ f_!
}
$$
commutes.
\item Suppose that we have a diagram of schemes
$$
\xymatrix{
X' \ar[d]_{f'} \ar[r]_c & X \ar[d]^f \\
Y' \ar[d]_{g'} \ar[r]_b & Y \ar[d]^g \\
Z' \ar[r]^a & Z
}
$$
with both squares cartesian and $f$ and $g$
locally quasi-finite. Then the diagram
$$
\xymatrix{
a^{-1} \circ (g \circ f)_! \ar[d] \ar[rr] & &
(g' \circ f')_! \circ c^{-1} \ar[d] \\
a^{-1} \circ g_! \circ f_! \ar[r] &
g'_! \circ b^{-1} \circ f_! \ar[r] &
g'_! \circ f'_! \circ c^{-1}
}
$$
commutes where the horizontal arrows are those of
Lemma \ref{lemma-lqf-base-change-f-shriek}.
\end{enumerate}
\end{lemma}

\begin{proof}
If $f$ and $g$ are separated, then this is a special case of
Lemma \ref{lemma-f-shriek-composition}.
If $g$ is separated, then this is a special case of
Lemma \ref{lemma-lqf-separated-shriek-composition}
which moreover agrees with the case where $f$ and $g$ are separated.

\medskip\noindent
Construction in the general case. Choose an open covering $Y = \bigcup Y_i$
such that the restriction $g_i : Y_i \to Z$ of $g$ is separated.
Set $X_i = f^{-1}(Y_i)$ and denote $f_i : X_i \to Y_i$ the restriction
of $f$. Also denote $h = g \circ f$ and $h_i : X_i \to Z$ the restriction
of $h$. Consider the following diagram
$$
\xymatrix{
\bigoplus\nolimits_{i_0, i_1}
h_{i_0i_1, !}\mathcal{F}|_{X_{i_0i_1}} \ar[r] \ar[d] &
\bigoplus\nolimits_{i_0} h_{i_0, !}\mathcal{F}|_{X_{i_0}} \ar[r] \ar[d] &
h_!\mathcal{F} \ar[r] \ar@{..>}[dd] &
0 \\
\bigoplus\nolimits_{i_0, i_1}
g_{i_0i_1, !} f_{i_0i_1, !}\mathcal{F}|_{X_{i_0i_1}} \ar[r] \ar[d] &
\bigoplus\nolimits_{i_0}
g_{i_0, !} f_{i_0, !}\mathcal{F}|_{X_{i_0}} \ar[d] \\
\bigoplus\nolimits_{i_0, i_1}
g_{i_0i_1, !} (f_!\mathcal{F})|_{Y_{i_0i_1}} \ar[r] &
\bigoplus\nolimits_{i_0}
g_{i_0, !} (f_!\mathcal{F})|_{Y_{i_0}} \ar[r] &
g_!f_!\mathcal{F} \ar[r] &
0
}
$$
By Lemma \ref{lemma-lqf-colimit-f-shriek} the top and bottom row
in the diagram are exact. By Lemma \ref{lemma-lqf-separated-shriek-composition}
the top left square commutes. The vertical arrows in the
lower left square come about because
$(f_!\mathcal{F})|_{Y_{i_0i_1}} = f_{i_0i_1, !}\mathcal{F}|_{X_{i_0i_1}}$ and
$(f_!\mathcal{F})|_{Y_{i_0}} = f_{i_0, !}\mathcal{F}|_{X_{i_0}}$
as the construction of $f_!$ is local on the base. Moreover, these
equalities are (of course) compatible with the identifications
$((f_!\mathcal{F})|_{Y_{i_0}})|_{Y_{i_0i_1}} =
(f_!\mathcal{F})|_{Y_{i_0i_1}}$ and
$(f_{i_0, !}\mathcal{F}|_{X_{i_0}})|_{Y_{i_0i_1}} =
f_{i_0i_1, !}\mathcal{F}|_{X_{i_0i_1}}$
which are used (together with the covariance for open embeddings
for $Y_{i_0i_1} \subset Y_{i_0}$)
to define the horizontal maps of the lower left square.
Thus this square commutes as well.
In this way we conclude there is a unique
dotted arrow as indicated in the diagram and
moreover this arrow is an isomorphism.

\medskip\noindent
Proof of properties (1) -- (5). Fix the open covering $Y = \bigcup Y_i$.
Observe that if $Y \to Z$ happens to be separated, then we get a dotted
arrow fitting into the huge diagram above by using the map of
Lemma \ref{lemma-lqf-separated-shriek-composition}
(by the very properties of that lemma).
This proves (2) and hence also (1) by the compatibility of the
maps of Lemma \ref{lemma-lqf-separated-shriek-composition}
and Lemma \ref{lemma-f-shriek-composition}.
Next, for any scheme $Z'$ over $Z$, we obtain the compatibility in (5)
for the map $(g' \circ f')_! \to g'_! \circ f'_!$
constructed using the open covering $Y' = \bigcup b^{-1}(Y_i)$.
This is clear from the corresponding compatibility of the maps
constructed in Lemma \ref{lemma-lqf-separated-shriek-composition}.
In particular, we can consider a geometric point
$\overline{z} : \Spec(k) \to Z$. Since
$X_{\overline{z}} \to Y_{\overline{z}} \to \Spec(k)$
are separated maps, we find that the base change of
$(g \circ f)_!\mathcal{F} \to g_! f_! \mathcal{F}$
by $\overline{z}$ is equal to the map of
Lemma \ref{lemma-f-shriek-composition}.
The reader then immediately sees that we obtain property (3).
Of course, property (3) guarantees that our transformation of functors
$(g \circ f)_! \to g_! \circ f_!$ constructed using the open covering
$Y = \bigcup Y_i$ doesn't depend on the choice of this open covering.
Finally, property (4) follows by looking at what happens on stalks
using the already proven property (3).
\end{proof}








\section{Weightings and trace maps for locally quasi-finite morphisms}
\label{section-weightings}

\noindent
A reference for this section is
\cite[Exposee XVII, Proposition 6.2.5]{SGA4}.

\medskip\noindent
Let $f : X \to Y$ be a locally quasi-finite morphism of schemes.
Let $w : X \to \mathbf{Z}$ be a weighting of $f$, see
More on Morphisms, Definition \ref{more-morphisms-definition-weighting}.
Let $\mathcal{F}$ be an abelian sheaf on $Y_\etale$.
In this section we will show that there exists map
$$
\text{Tr}_{f, w, \mathcal{F}} :
f_!f^{-1}\mathcal{F}
\longrightarrow
\mathcal{F}
$$
of abelian sheaves on $Y_\etale$ characterized by the following property:
on stalks at a geometric point $\overline{y}$ of $Y$ we obtain the map
$$
\bigoplus\nolimits_{f(\overline{x}) = \overline{y}} w(\overline{x}) :
(f_!f^{-1}\mathcal{F})_{\overline{y}} =
\bigoplus\nolimits_{f(\overline{x}) = \overline{y}}
\mathcal{F}_{\overline{y}}
\longrightarrow
\mathcal{F}_{\overline{y}}
$$
Here as indicated the arrow is given by multiplication by the integer
$w(\overline{x})$ on the summand corresponding to $\overline{x}$.
The equality on the left of the arrow follows from
Lemma \ref{lemma-lqf-f-shriek-stalk} combined with
\'Etale Cohomology, Lemma \ref{etale-cohomology-lemma-stalk-pullback}.

\medskip\noindent
If the morphism $f : X \to Y$ is flat, locally quasi-finite, and locally of
finite presentation, then there exists a canonical weighting
and we obtain a canonical trace map whose formation is
compatible with base change, see
Example \ref{example-trace-for-flat-quasi-finite}.
If $Y$ is a locally Noetherian unibranch scheme and $f : X \to Y$
is locally quasi-finite, then we can also define a (natural)
weighting for $f$ and we have trace maps in this case as well, see
Example \ref{example-trace-for-quasi-finite-over-normal}.

\begin{lemma}
\label{lemma-f-shriek-projection}
Let $f : X \to Y$ be a locally quasi-finite morphism of schemes.
Let $\Lambda$ be a ring.
Let $\mathcal{F}$ be a sheaf of $\Lambda$-modules on $X_\etale$
and let $\mathcal{G}$ be a sheaf of $\Lambda$-modues on $Y_\etale$.
There is a canonical isomorphism
$$
can :
f_!\mathcal{F} \otimes_\Lambda \mathcal{G}
\longrightarrow
f_!(\mathcal{F} \otimes_\Lambda f^{-1}\mathcal{G})
$$
of sheaves of $\Lambda$-modules on $Y_\etale$.
\end{lemma}

\begin{proof}
Recall that $f_!\mathcal{F} = (f_{p!}\mathcal{F})^\#$
by Definition \ref{definition-f-shriek-lqf} where
$f_{p!}\mathcal{F}$ is the presheaf
constructed in Section \ref{section-finite-support}.
Thus in order to construct the arrow it suffices to construct a map
$$
f_{p!}\mathcal{F} \otimes_{p, \Lambda} \mathcal{G}
\longrightarrow
f_{p!}(\mathcal{F} \otimes_\Lambda f^{-1}\mathcal{G})
$$
of presheaves on $Y_\etale$. Here the symbol $\otimes_{p, \Lambda}$
denotes the presheaf tensor product, see
Modules on Sites, Section \ref{sites-modules-section-tensor-product}.
Let $V$ be an object of $Y_\etale$. Recall that
$$
f_{p!}\mathcal{F}(V) = \colim_Z H_Z(\mathcal{F})
\quad\text{and}\quad
f_{p!}(\mathcal{F} \otimes_\Lambda f^{-1}\mathcal{G})(V) =
\colim_Z H_Z(\mathcal{F} \otimes_\Lambda f^{-1}\mathcal{G})
$$
See Section \ref{section-finite-support}. Our map will be defined
on pure tensors by the rule
$$
(Z, s) \otimes t \longmapsto (Z, s \otimes f^{-1}t)
$$
(for notation see below) and extended by linearity to all of
$(f_{p!}\mathcal{F} \otimes_{p, \Lambda} \mathcal{G})(V) =
f_{p!}\mathcal{F}(V) \otimes_\Lambda \mathcal{G}(V)$.
Here the notation used is as follows
\begin{enumerate}
\item $Z \subset X_V$ is a locally closed subscheme finite over $V$,
\item $s \in H_Z(\mathcal{F})$ which means that $s \in \mathcal{F}(U)$
with $\text{Supp}(s) \subset Z$ for some $U \subset X_V$ open such that
$Z \subset U$ is closed, and
\item $t \in \mathcal{G}(V)$ with image $f^{-1}t \in f^{-1}\mathcal{G}(U)$.
\end{enumerate}
Since the support of $s \in \mathcal{F}(U)$ is contained in $Z$ it is clear
that the support of $s \otimes f^{-1}t$ is contained in $Z$ as well.
Thus considering the pair $(Z, s \otimes f^{-1}t)$ makes sense.
It is immediate that the construction commutes with the transition
maps in the colimit $\colim_Z H_Z(\mathcal{F})$ and that it is
compatible with restriction mappings. Finally, it is equally clear
that the construction is compatible with the identifications of
stalks of $f_!$ in Lemma \ref{lemma-lqf-f-shriek-stalk}.
In other words, the map $can$ we've produced on stalks at a geometric
point $\overline{y}$ fits into a commutative diagram
$$
\xymatrix{
(f_!\mathcal{F} \otimes_\Lambda \mathcal{G})_{\overline{y}}
\ar[r]_-{can_{\overline{y}}} \ar[d] &
f_!(\mathcal{F} \otimes_\Lambda f^{-1}\mathcal{G})_{\overline{y}} \ar[d] \\
(\bigoplus \mathcal{F}_{\overline{x}})
\otimes_\Lambda \mathcal{G}_{\overline{y}} \ar[r] &
\bigoplus
(\mathcal{F}_{\overline{x}} \otimes_\Lambda \mathcal{G}_{\overline{y}})
}
$$
where the direct sums are over the geometric points $\overline{x}$
lying over $\overline{y}$, where the vertical arrows are the identifications
of Lemma \ref{lemma-lqf-f-shriek-stalk}, and where the lower horizontal arrow
is the obvious isomorphism.
We conclude that $can$ is an isomorphism as desired.
\end{proof}

\begin{lemma}
\label{lemma-trace-map-exists}
Let $f : X \to Y$ be a locally quasi-finite morphism of schemes.
Let $w : X \to \mathbf{Z}$ be a weighting of $f$. For any abelian sheaf
$\mathcal{F}$ on $Y$ there exists a unique trace map
$\text{Tr}_{f, w, \mathcal{F}} : f_!f^{-1}\mathcal{F} \to \mathcal{F}$
having the prescribed behaviour on stalks.
\end{lemma}

\begin{proof}
By Lemma \ref{lemma-f-shriek-projection} we have an identification
$f_!f^{-1}\mathcal{F} = f_!\underline{\mathbf{Z}} \otimes \mathcal{F}$
compatible with the description of stalks of these sheaves at
geometric points. Hence it suffices to produce the map
$$
\text{Tr}_{f, w, \underline{\mathbf{Z}}} :
f_!\underline{\mathbf{Z}}
\longrightarrow
\underline{\mathbf{Z}}
$$
having the prescribed behaviour on stalks. By
Definition \ref{definition-f-shriek-lqf} we have
$f_!\underline{\mathbf{Z}} = (f_{p!}\underline{\mathbf{Z}})^\#$
where $f_{p!}\underline{\mathbf{Z}}$ is the presheaf
constructed in Section \ref{section-finite-support}.
Thus it suffices to construct a map
$$
f_{p!}\underline{\mathbf{Z}} \longrightarrow \underline{\mathbf{Z}}
$$
of presheaves on $Y_\etale$. Let $V$ be an object of $Y_\etale$.
Recall from Section \ref{section-finite-support} that
$$
f_{p!}\underline{\mathbf{Z}}(V) = \colim_Z H_Z(\underline{\mathbf{Z}})
$$
Here the colimit is over the (partially ordered) collection of locally closed
subschemes $Z \subset X_V$ which are finite over $V$. For each such
$Z$ we will define a map
$$
H_Z(\underline{\mathbf{Z}}) \longrightarrow \underline{\mathbf{Z}}(V)
$$
compatible with the maps defining the colimit.

\medskip\noindent
Let $Z \subset X_V$ be locally closed and finite over $V$.
Choose an open $U \subset X_V$ containing $Z$ as a closed subset.
An element $s$ of $H_Z(\underline{\mathbf{Z}})$ is a section
$s \in \underline{\mathbf{Z}}(U)$ whose support is contained in $Z$.
Let $U_n \subset U$ be the open and closed subset where the value
of $s$ is $n \in \mathbf{Z}$. By the support condition we see that
$Z \cap U_n = U_n$ for $n \not = 0$. Hence for $n \not = 0$, the open
$U_n$ is also closed in $Z$ (as the complement of all the others)
and we conclude that $U_n \to V$ is finite as $Z$ is finite over $V$.
By the very definition of a weighting this means the
function $\int_{U_n \to V} w|_{U_n}$ is locally constant on $V$
and we may view it as an element of $\underline{\mathbf{Z}}(V)$.
Our construction sends $(Z, s)$ to the element
$$
\sum\nolimits_{n \in \mathbf{Z},\ n \not = 0}
n \left(\int_{U_n \to V} w|_{U_n}\right)
\quad \in \quad \underline{\mathbf{Z}}(V)
$$
The sum is locally finite on $V$ and hence makes sense; details omitted
(in the whole discussion the reader may first choose affine opens and
make sure all the schemes occuring in the argument are quasi-compact so
the sum is finite). We omit the verification that this construction
is compatible with the maps in the colimit and with the restriction
mappings defining $f_{p!}\underline{\mathbf{Z}}$.

\medskip\noindent
Let $\overline{y}$ be a geometric point of $Y$ lying over the point $y \in Y$.
Taking stalks at $\overline{y}$ the construction above determines a map
$$
(f_!\underline{\mathbf{Z}})_{\overline{y}}
= \bigoplus\nolimits_{f(\overline{x}) = \overline{y}} \mathbf{Z}
\longrightarrow
\mathbf{Z} = \underline{\mathbf{Z}}_{\overline{y}}
$$
To finish the proof we will show this map is given by multiplication by
$w(\overline{x})$ on the summand corresponding to $\overline{x}$.
Namely, pick $\overline{x}$ lying over $\overline{y}$.
We can find an \'etale neighbourhood $(V, \overline{v}) \to (Y, \overline{y})$
such that $X_V$ contains an open $U$ finite over $V$
such that only the geometric point $\overline{x}$ is in $U$
and not the other geometric points of $X$ lifting $\overline{y}$.
This follows from More on Morphisms, Lemma
\ref{more-morphisms-lemma-etale-makes-quasi-finite-finite-multiple-points-var};
some details omitted. Then $(U, 1)$ defines a section of
$f_!\underline{\mathbf{Z}}$ over $V$ which maps to $1$ in the summand
corresponding to $\overline{x}$ and zero in the other summands
(see proof of Lemma \ref{lemma-finite-support-stalk}) and
our construction above sends $(U, 1)$ to $\int_{U \to V} w|_U$
which is constant with value $w(\overline{x})$ in a neighbourhood
of $\overline{v}$ as desired.
\end{proof}

\begin{lemma}
\label{lemma-properties-trace-map}
Let $f : X \to Y$ be a locally quasi-finite morphism of schemes.
Let $w : X \to \mathbf{Z}$ be a weighting of $f$. The trace maps
constructed above have the following properties:
\begin{enumerate}
\item $\text{Tr}_{f, w, \mathcal{F}}$ is functorial in $\mathcal{F}$,
\item $\text{Tr}_{f, w, \mathcal{F}}$ is compatible with arbitrary base change,
\item given a ring $\Lambda$ and $K$ in $D(Y_\etale, \Lambda)$
we obtain $\text{Tr}_{f, w, K} : f_!f^{-1}K \to K$ functorial in $K$
and compatible with arbitrary base change.
\end{enumerate}
\end{lemma}

\begin{proof}
Part (1) either follows from the construction of the trace map
in the proof of Lemma \ref{lemma-trace-map-exists}
or more simply because the characterization of the map
forces it to be true on all stalks.
Let
$$
\xymatrix{
X' \ar[r]_{g'} \ar[d]_{f'} & X \ar[d]^f \\
Y' \ar[r]^g & Y
}
$$
be a cartesian diagram of schemes. Then the function
$w' = w \circ g' : X' \to \mathbf{Z}$ is a weighting of $f'$ by
More on Morphisms, Lemma \ref{more-morphisms-lemma-weighting-base-change}.
Statement (2) means that the diagram
$$
\xymatrix{
g^{-1}f_!f^{-1}\mathcal{F}
\ar[rr]_-{g^{-1}\text{Tr}_{f, w, \mathcal{F}}} \ar@{=}[d] & &
g^{-1}\mathcal{F} \ar@{=}[d] \\
f'_!(f')^{-1}g^{-1}\mathcal{F}
\ar[rr]^-{\text{Tr}_{f', w', g^{-1}\mathcal{F}}} & &
g^{-1}\mathcal{F}
}
$$
is commutative where the left vertical equality is given by
$$
g^{-1}f_!f^{-1}\mathcal{F} =
f'_!(g')^{-1}f^{-1}\mathcal{F} =
f'_!(f')^{-1}g^{-1}\mathcal{F}
$$
with first equality sign given by Lemma \ref{lemma-lqf-base-change-f-shriek}
(base change for lower shriek). The commutativity of this diagram
follows from the characterization of the action of our trace maps
on stalks and the fact that the base change map of
Lemma \ref{lemma-lqf-base-change-f-shriek} respects the descriptions of stalks.

\medskip\noindent
Given parts (1) and (2), part (3) follows as the functors
$f^{-1} : D(Y_\etale, \Lambda) \to D(X_\etale, \Lambda)$ and
$f_! : D(X_\etale, \Lambda) \to D(Y_\etale, \Lambda)$
are obtained by applying $f^{-1}$ and $f_!$ to any complexes
of modules representing the objects in question.
\end{proof}

\begin{lemma}
\label{lemma-trace-composition}
Let $f : X \to Y$ and $g : Y \to Z$ be locally quasi-finite morphisms.
Let $w_f : X \to \mathbf{Z}$ be a weighting of $f$ and let
$w_g : Y \to \mathbf{Z}$ be a weighting of $g$. For
$K \in D(Z_\etale, \Lambda)$ the composition
$$
(g \circ f)_!(g \circ f)^{-1}K =
g_! f_! f^{-1} g^{-1}K
\xrightarrow{g_! \text{Tr}_{f, w_f, g^{-1}K}}
g_!g^{-1}K
\xrightarrow{\text{Tr}_{g, w_g, K}}
K
$$
is equal to $\text{Tr}_{g \circ f, w_{g \circ f}, K}$ where
$w_{g \circ f}(x) = w_f(x) w_g(f(x))$.
\end{lemma}

\begin{proof}
We have $(g \circ f)_! = g_! \circ f_!$ by
Lemma \ref{lemma-lqf-shriek-composition}.
In More on Morphisms, Lemma \ref{more-morphisms-lemma-weighting-composition}
we have seen that $w_{g \circ f}$ is a weighting for $g \circ f$
so the statement makes sense. To check equality compute on stalks.
Details omitted.
\end{proof}

\begin{example}[Trace for flat quasi-finite]
\label{example-trace-for-flat-quasi-finite}
Let $f : X \to Y$ be a morphism of schemes which is flat,
locally quasi-finite, and locally of finite presentation.
Then we obtain a canonical positive weighting $w : X \to \mathbf{Z}$
by setting
$$
w(x) = \text{length}_{\mathcal{O}_{X, x}}
(\mathcal{O}_{X, x}/\mathfrak m_{f(x)} \mathcal{O}_{X, x})
[\kappa(x) : \kappa(f(x))]_i
$$
See More on Morphisms, Lemma
\ref{more-morphisms-lemma-weighting-flat-quasi-finite}.
Thus by Lemmas \ref{lemma-trace-map-exists} and
\ref{lemma-properties-trace-map} for $f$ we obtain trace maps
$$
\text{Tr}_{f, K} : f_!f^{-1}K \longrightarrow K
$$
functorial for $K$ in $D(Y_\etale, \Lambda)$ and compatible
with arbitrary base change. Note that any base change
$f' : X' \to Y'$ of $f$ satisfies the same properties and that $w$
restricts to the canonical weighting for $f'$.
\end{example}

\begin{remark}
\label{remark-trace-etale-counit}
Let $j : U \to X$ be an \'etale morphism of schemes. Then the trace
map $\text{Tr} : j_!j^{-1}K \to K$ of
Example \ref{example-trace-for-flat-quasi-finite} is equal to the
counit for the adjunction between $j_!$ and $j^{-1}$.
We already used the terminology ``trace'' for this counit in
\'Etale cohomology, Section \ref{etale-cohomology-section-trace-method}.
\end{remark}

\begin{example}[Trace for quasi-finite over normal]
\label{example-trace-for-quasi-finite-over-normal}
Let $Y$ be a geometrically unibranch and locally Noetherian scheme,
for example $Y$ could be a normal variety. Let $f : X \to Y$ be a locally
quasi-finite morphism of schemes. Then there exists a
positive weighting $w : X \to \mathbf{Z}$ for $f$
which is roughly defined by sending $x$ to the
``generic separable degree''
of $\mathcal{O}_{X, x}^{sh}$ over $\mathcal{O}_{Y, f(x)}^{sh}$.
See More on Morphisms, Lemma
\ref{more-morphisms-lemma-weighting-quasi-finite-Noetherian}.
Thus by Lemmas \ref{lemma-trace-map-exists} and
\ref{lemma-properties-trace-map} for $f$ and $w$ we obtain trace maps
$$
\text{Tr}_{f, w, K} : f_!f^{-1}K \longrightarrow K
$$
functorial for $K$ in $D(Y_\etale, \Lambda)$ and compatible
with arbitrary base change. However, in this case, given a base change
$f' : X' \to Y'$ of $f$ the restriction of $w$ to $X'$ in general
does not have a ``natural'' interpretation in terms of the
morphism $f'$.
\end{example}
















\section{Upper shriek for locally quasi-finite morphisms}
\label{section-duality-locally-quasi-finite}

\noindent
For a locally quasi-finite morphism $f : X \to Y$ of schemes, the
functor $f_! : \textit{Ab}(X_\etale) \to \textit{Ab}(Y_\etale)$ commutes
with direct sums and is exact, see Lemma \ref{lemma-lqf-f-shriek-stalk}.
This suggests that it has a right adjoint which we will denote $f^!$.

\medskip\noindent
Warning: This functor is the non-derived version!

\begin{lemma}
\label{lemma-lqf-f-upper-shriek}
Let $f : X \to Y$ be a locally quasi-finite morphism of schemes.
\begin{enumerate}
\item The functor $f_! : \textit{Ab}(X_\etale) \to \textit{Ab}(Y_\etale)$
has a right adjoint $f^! : \textit{Ab}(Y_\etale) \to \textit{Ab}(X_\etale)$.
\item We have
$f^!(\overline{y}_*A) = \prod_{f(\overline{x}) = \overline{y}} \overline{x}_*A$.
\item If $\Lambda$ is a ring, then the functor
$f_! : \textit{Mod}(X_\etale, \Lambda) \to \textit{Mod}(Y_\etale, \Lambda)$
has a right adjoint
$f^! : \textit{Mod}(Y_\etale, \Lambda) \to \textit{Mod}(X_\etale, \Lambda)$
which agrees with $f^!$ on underlying abelian sheaves.
\end{enumerate}
\end{lemma}

\begin{proof}
Proof of (1). Let $E \subset \Ob(\textit{Ab}(Y_\etale))$ be the class
consisting of products of skyscraper sheaves. We claim that
\begin{enumerate}
\item[(a)] every $\mathcal{G}$ in $\textit{Ab}(Y_\etale)$ is a subsheaf
of an element of $E$, and
\item[(b)] for every $\mathcal{G} \in E$ there exists an object
$\mathcal{H}$ of $\textit{Ab}(X_\etale)$ such that
$\Hom(f_!\mathcal{F}, \mathcal{G}) = \Hom(\mathcal{F}, \mathcal{H})$
functorially in $\mathcal{F}$.
\end{enumerate}
Once the claim has been verified, the dual of
Homology, Lemma \ref{homology-lemma-partially-defined-adjoint}
produces the adjoint functor $f^!$.

\medskip\noindent
Part (a) is true because we can map $\mathcal{G}$ to the sheaf
$\prod \overline{y}_*\mathcal{G}_{\overline{y}}$ where the
product is over all geometric points of $Y$. This is an injection by
\'Etale Cohomology, Theorem \ref{etale-cohomology-theorem-exactness-stalks}.
(This is the first step in the Godement resolution when
done in the setting of abelian sheaves on topological spaces.)

\medskip\noindent
Part (b) and part (2) of the lemma can be seen as follows.
Suppose that $\mathcal{G} = \prod \overline{y}_*A_{\overline{y}}$
for some abelian groups $A_{\overline{y}}$. Then
$$
\Hom(f_!\mathcal{F}, \mathcal{G}) =
\prod \Hom(f_!\mathcal{F}, \overline{y}_*A_{\overline{y}})
$$
Thus it suffices to find abelian sheaves $\mathcal{H}_{\overline{y}}$
on $X_\etale$ representing the functors
$\mathcal{F} \mapsto \Hom(f_!\mathcal{F}, \overline{y}_*A_{\overline{y}})$
and to take $\mathcal{H} = \prod \mathcal{H}_{\overline{y}}$.
This reduces us to the case $\mathcal{H} = \overline{y}_*A$
for some fixed geometric point $\overline{y} : \Spec(k) \to Y$
and some fixed abelian group $A$. We claim that in this case
$\mathcal{H} = \prod_{f(\overline{x}) = \overline{y}} \overline{x}_*A$ works.
This will finish the proof of parts (1) and (2) of the lemma.
Namely, we have
$$
\Hom(f_!\mathcal{F}, \overline{y}_*A) =
\Hom_{\textit{Ab}}((f_!\mathcal{F})_{\overline{y}}, A) =
\Hom_{\textit{Ab}}(\bigoplus\nolimits_{f(\overline{x}) = \overline{y}}
\mathcal{F}_{\overline{x}}, A)
$$
by the description of stalks in
Lemma \ref{lemma-lqf-f-shriek-stalk}
on the one hand and on the other hand we have
$$
\Hom(\mathcal{F}, \mathcal{H}) =
\prod\nolimits_{f(\overline{x}) = \overline{y}}
\Hom(\mathcal{F}, \overline{x}_*A) =
\prod\nolimits_{f(\overline{x}) = \overline{y}}
\Hom_{\textit{Ab}}(\mathcal{F}_{\overline{x}}, A)
$$
We leave it to the reader to identify these as functors of $\mathcal{F}$.

\medskip\noindent
Proof of part (3). Observe that an object $\textit{Mod}(X_\etale, \Lambda)$
is the same thing as an object $\mathcal{F}$ of $\textit{Ab}(X_\etale)$
together with a map $\Lambda \to \text{End}(\mathcal{F})$. Hence the
functors $f_!$ and $f^!$ in (1) define functors $f_!$ and $f^!$ as in (3).
A straightforward computation shows that they are adjoints.
\end{proof}

\begin{lemma}
\label{lemma-etale-upper-shriek}
Let $j : U \to X$ be an \'etale morphism. Then $j^! = j^{-1}$.
\end{lemma}

\begin{proof}
This is true because $j_!$ as defined in
Section \ref{section-finite-support}
agrees with $j_!$ as defined in \'Etale Cohomology, Section
\ref{etale-cohomology-section-extension-by-zero}, see
Lemma \ref{lemma-finite-support-etale-shriek}. Finally, in
\'Etale Cohomology, Section \ref{etale-cohomology-section-extension-by-zero}
the functor $j_!$ is defined
as the left adjoint of $j^{-1}$ and hence we conclude by
uniqueness of adjoint functors.
\end{proof}

\begin{lemma}
\label{lemma-upper-shriek-restriction}
Let $f : X \to Y$ and $g : Y \to Z$ be separated and locally quasi-finite
morphisms. There is a canonical isomorphism $(g \circ f)^! \to f^! \circ g^!$.
Given a third locally quasi-finite morphism $h : Z \to T$
the diagram
$$
\xymatrix{
(h \circ g \circ f)^! \ar[r] \ar[d] &
f^! \circ (h \circ g)^! \ar[d] \\
(g \circ f)^! \circ h^! \ar[r] & f^! \circ g^! \circ h^!
}
$$
commutes.
\end{lemma}

\begin{proof}
By uniqueness of adjoint functors, this immediately translates
into the corresponding (dual) statement for the functors $f_!$.
See Lemma \ref{lemma-lqf-shriek-composition}.
\end{proof}

\begin{lemma}
\label{lemma-upper-shriek-restriction-etale}
Let $j : U \to X$ and $j' : V \to U$ be \'etale morphisms.
The isomorphism $(j \circ j')^{-1} = (j')^{-1} \circ j^{-1}$
and the isomorphism $(j \circ j')^! = (j')^! \circ j^!$ of
Lemma \ref{lemma-upper-shriek-restriction}
agree via the isomorphism of Lemma \ref{lemma-etale-upper-shriek}.
\end{lemma}

\begin{proof}
Omitted.
\end{proof}

\begin{lemma}
\label{lemma-lqf-base-change-upper-shriek}
Consider a cartesian square
$$
\xymatrix{
X' \ar[r]_{g'} \ar[d]_{f'} & X \ar[d]^f \\
Y' \ar[r]^g & Y
}
$$
of schemes with $f$ locally quasi-finite. For any abelian sheaf $\mathcal{F}$
on $Y'_\etale$ we have $(g')_*(f')^!\mathcal{F} = f^!g_*\mathcal{F}$.
\end{lemma}

\begin{proof}
By uniqueness of adjoint functors, this follows from
the corresponding (dual) statement for the functors $f_!$.
See Lemma \ref{lemma-lqf-base-change-f-shriek}.
\end{proof}

\begin{remark}
\label{remark-pointed-sets}
The material in this section can be generalized to sheaves of pointed sets.
Namely, for a site $\mathcal{C}$ denote $\Sh^*(\mathcal{C})$ the category of
sheaves of pointed sets. The constructions in this and the preceding section
apply, mutatis mutandis, to sheaves of pointed sets. Thus given a locally
quasi-finite morphism $f : X \to Y$ of schemes we obtain
an adjoint pair of functors
$$
f_! : \Sh^*(X_\etale) \longrightarrow \Sh^*(Y_\etale)
\quad\text{and}\quad
f^! : \Sh^*(Y_\etale) \longrightarrow \Sh^*(X_\etale)
$$
such that for every geometric point $\overline{y}$ of $Y$ there are
isomorphisms
$$
(f_!\mathcal{F})_{\overline{y}} =
\coprod\nolimits_{f(\overline{x}) = \overline{y}}
\mathcal{F}_{\overline{x}}
$$
(coproduct taken in the category of pointed sets) functorial in
$\mathcal{F} \in \Sh^*(X_\etale)$ and isomorphisms
$$
f^!(\overline{y}_*S) =
\prod\nolimits_{f(\overline{x}) = \overline{y}}
\overline{x}_*S
$$
functorial in the pointed set $S$. If
$F : \textit{Ab}(X_\etale) \to \Sh^*(X_\etale)$ and
$F : \textit{Ab}(Y_\etale) \to \Sh^*(Y_\etale)$
denote the forgetful functors, compatibility between the constructions
will guarantee the existence of canonical maps
$$
f_!F(\mathcal{F}) \longrightarrow F(f_!\mathcal{F})
$$
functorial in $\mathcal{F} \in \textit{Ab}(X_\etale)$ and
$$
F(f^!\mathcal{G}) \longrightarrow f^!F(\mathcal{G})
$$
functorial in $\mathcal{G} \in \textit{Ab}(Y_\etale)$
which produce the obvious maps on stalks, resp.\ skyscraper sheaves.
In fact, the transformation $F \circ f^! \to f^! \circ F$ is an isomorphism
(because $f^!$ commutes with products).
\end{remark}







\section{Derived upper shriek for locally quasi-finite morphisms}
\label{section-derived-duality-locally-quasi-finite}

\noindent
We can take the derived versions of the functors in
Section \ref{section-duality-locally-quasi-finite}
and obtain the following.

\begin{lemma}
\label{lemma-lqf-shriek-derived}
Let $f : X \to Y$ be a locally quasi-finite morphism of schemes.
Let $\Lambda$ be a ring. The functors $f_!$ and $f^!$ of
Definition \ref{definition-f-shriek-lqf} and
Lemma \ref{lemma-lqf-f-upper-shriek}
induce adjoint functors $f_! : D(X_\etale, \Lambda) \to D(Y_\etale, \Lambda)$
and $Rf^! : D(Y_\etale, \Lambda) \to D(X_\etale, \Lambda)$
on derived categories.
\end{lemma}

\noindent
In the separated case the functor $f_!$ is defined in
Section \ref{section-compact-support}.

\begin{proof}
This follows immediately from
Derived Categories, Lemma \ref{derived-lemma-derived-adjoint-functors},
the fact that $f_!$ is exact (Lemma \ref{lemma-lqf-f-shriek-stalk})
and hence $Lf_! = f_!$
and the fact that we have enough K-injective complexes of $\Lambda$-modules
on $Y_\etale$ so that $Rf^!$ is defined.
\end{proof}

\begin{remark}
\label{remark-lqf-shriek-derived-torsion}
Let $f : X \to Y$ be a locally quasi-finite morphism of schemes. Let
$\Lambda$ be a ring. The functor
$f_! : D(X_\etale, \Lambda) \to D(Y_\etale, \Lambda)$ of
Lemma \ref{lemma-lqf-shriek-derived}
sends complexes with torsion cohomology sheaves
to complexes with torsion cohomology sheaves.
This is immediate from the description of the stalks of $f_!$,
see Lemma \ref{lemma-lqf-f-shriek-stalk}.
\end{remark}

\begin{lemma}
\label{lemma-mayer-vietoris-shriek}
Let $X$ be a scheme. Let $X = U \cup V$ with $U$ and $V$ open.
Let $\Lambda$ be a ring. Let $K \in D(X_\etale, \Lambda)$. There
is a distinguished triangle
$$
j_{U \cap V!}K|_{U \cap V} \to
j_{U!}K|_U \oplus j_{V!}K|_V \to K \to 
j_{U \cap V!}K|_{U \cap V}[1]
$$
in $D(X_\etale, \Lambda)$ with obvious notation.
\end{lemma}

\begin{proof}
Since the restriction functors and the lower shriek functors we use
are exact, it suffices to show for any abelian sheaf $\mathcal{F}$ on
$X_\etale$ the sequence
$$
0 \to j_{U \cap V!}\mathcal{F}|_{U \cap V} \to
j_{U!}\mathcal{F}|_U \oplus j_{V!}\mathcal{F}|_V \to \mathcal{F} \to 0
$$
is exact. This can be seen by looking at stalks.
\end{proof}

\begin{lemma}
\label{lemma-triangle-associated-to-open}
Let $X$ be a scheme. Let $Z \subset X$ be a closed subscheme and let
$U \subset X$ be the complement. Denote $i : Z \to X$ and $j : U \to X$
the inclusion morphisms. Let $\Lambda$ be a ring.
Let $K \in D(X_\etale, \Lambda)$. There is a distinguished triangle
$$
j_!j^{-1}K \to K \to i_*i^{-1}K \to j_!j^{-1}K[1]
$$
in $D(X_\etale, \Lambda)$.
\end{lemma}

\begin{proof}
Immediate consequence of \'Etale Cohomology, Lemma
\ref{etale-cohomology-lemma-ses-associated-to-open}
and the fact that the functors $j_!$, $j^{-1}$, $i_*$, $i^{-1}$
are exact and hence their derived versions are computed by
applying these functors to any complex of sheaves representing $K$.
\end{proof}





\section{Preliminaries to derived lower shriek via compactifications}
\label{section-prelim}

\noindent
In this section we prove some lemmas on the existence of
certain natural isomorphisms of functors which follow immediately
from proper base change.

\begin{lemma}
\label{lemma-shriek-proper-and-open}
Consider a commutative diagram of schemes
$$
\xymatrix{
X' \ar[r]_{g'} \ar[d]_{f'} & X \ar[d]^f \\
Y' \ar[r]^g & Y
}
$$
with $f$ and $f'$ proper and $g$ and $g'$ separated and locally quasi-finite.
Let $\Lambda$ be a ring. Functorially in $K \in D(X'_\etale, \Lambda)$
there is a canonical map
$$
g_!Rf'_*K \longrightarrow Rf_*(g'_!K)
$$
in $D(Y_\etale, \Lambda)$. This map is an isomorphism if
(a) $K$ is bounded below and has torsion cohomology sheaves, or
(b) $\Lambda$ is a torsion ring.
\end{lemma}

\begin{proof}
Represent $K$ by a K-injective complex $\mathcal{J}^\bullet$ of
sheaves of $\Lambda$-modules on $X'_\etale$. Choose a quasi-isomorphism
$g'_!\mathcal{J}^\bullet \to \mathcal{I}^\bullet$ to a
K-injective complex $\mathcal{I}^\bullet$ of
sheaves of $\Lambda$-modules on $X_\etale$.
Then we can consider the map
$$
g_!f'_*\mathcal{J}^\bullet =
g_!f'_!\mathcal{J}^\bullet =
f_!g'_!\mathcal{J}^\bullet =
f_*g'_!\mathcal{J}^\bullet \to
f_*\mathcal{I}^\bullet
$$
where the first and third equality come from
Lemma \ref{lemma-proper-f-shriek}
and the second equality comes from
Lemma \ref{lemma-f-shriek-composition} which tells us that both
$g_! \circ f'_!$ and $f_! \circ g'_!$ are equal to
$(g \circ f')_! = (f \circ g')_!$ as subsheaves of
$(g \circ f')_* = (f \circ g')_*$.

\medskip\noindent
Assume $\Lambda$ is torsion, i.e., we are in case (b).
With notation as above, it suffices to show that
$f_*g'_!\mathcal{J}^\bullet \to f_*\mathcal{I}^\bullet$
is an isomorphism. The question is local on $Y$.
Hence we may assume that the dimension
of fibres of $f$ is bounded, see Morphisms, Lemma
\ref{morphisms-lemma-morphism-finite-type-bounded-dimension}.
Then we see that $Rf_*$ has finite cohomological dimension, see
\'Etale Cohomology, Lemma
\ref{etale-cohomology-lemma-cohomological-dimension-proper}.
Hence by Derived Categories, Lemma \ref{derived-lemma-unbounded-right-derived},
if we show that $R^qf_*(g'_!\mathcal{J}) = 0$ for $q > 0$
and any injective sheaf of $\Lambda$-modules $\mathcal{J}$
on $X'_\etale$, then the result follows.

\medskip\noindent
The stalk of $R^qf_*(g'_!\mathcal{J})$ at a geometric point
$\overline{y}$ is equal to
$H^q(X_{\overline{y}}, (g'_!\mathcal{J})|_{X_{\overline{y}}})$ by
\'Etale Cohomology, Lemma
\ref{etale-cohomology-lemma-proper-base-change-stalk}.
Since formation of $g'_!$ commutes with base change
(Lemma \ref{lemma-base-change-f-shriek-separated})
this is equal to
$$
H^q(X_{\overline{y}}, g'_{\overline{y}, !}(\mathcal{J}|_{X'_{\overline{y}}}))
$$
where $g'_{\overline{y}} : X'_{\overline{y}} \to X_{\overline{y}}$
is the induced morphism between geometric fibres.
Since $Y' \to Y$ is locally quasi-finite, we see that
$X'_{\overline{y}}$ is a disjoint union of the fibres
$X'_{\overline{y}'}$ at geometric points $\overline{y}'$ of $Y'$
lying over $\overline{y}$. Denote
$g'_{\overline{y}'} : X'_{\overline{y}'} \to X_{\overline{y}}$
the restriction of $g'_{\overline{y}}$ to $X'_{\overline{y}'}$.
Thus the previous cohomology group is equal to
$$
H^q(X_{\overline{y}},
\bigoplus\nolimits_{\overline{y}'/\overline{y}}
g'_{\overline{y}', !}(\mathcal{J}|_{X'_{\overline{y}'}}))
$$
for example by Lemma \ref{lemma-colim-f-shriek-separated} (but it
is also obvious from the definition of $g'_{\overline{y}, !}$
in Section \ref{section-compact-support}).
Since taking \'etale cohomology over $X_{\overline{y}}$
commutes with direct sums
(\'Etale Cohomology, Theorem \ref{etale-cohomology-theorem-colimit})
we conclude it suffices to show that
$$
H^q(X_{\overline{y}}, g'_{\overline{y}', !}(\mathcal{J}|_{X'_{\overline{y}'}}))
$$
is zero. Observe that
$g_{\overline{y}'} : X'_{\overline{y}'} \to X_{\overline{y}}$
is a morphism between proper scheme over $\overline{y}$ and hence is
proper itself. As it is locally quasi-finite as well we conclude that
$g_{\overline{y}'}$ is finite. Thus we see that
$g'_{\overline{y}', !} = g'_{\overline{y}', *} = Rg'_{\overline{y}', *}$.
By Leray we conlude that we have to show
$$
H^q(X'_{\overline{y}'}, \mathcal{J}|_{X'_{\overline{y}'}})
$$
is zero. As $\Lambda$ is torsion, this follows from proper base change
(\'Etale Cohomology, Lemma
\ref{etale-cohomology-lemma-proper-base-change-stalk})
as the higher direct images of $\mathcal{J}$ under $f'$ are zero.

\medskip\noindent
Proof in case (a). We will deduce this from case (b) by standard arguments.
We will show that the induced map $g_! R^pf'_* K \to R^pf_*(g'_!K)$
is an isomorphism for all $p \in \mathbf{Z}$. Fix an integer
$p_0 \in \mathbf{Z}$. Let $a$ be an integer such that $H^j(K) = 0$
for $j < a$. We will prove $g_! R^pf'_* K \to R^pf_*(g'_!K)$
is an isomorphism for $p \leq p_0$ by descending induction on $a$. If
$a > p_0$, then we see that the left and right hand side of
the map are zero for $p \leq p_0$ by trivial vanishing, see
Derived Categories, Lemma \ref{derived-lemma-negative-vanishing}
(and use that $g_!$ and $g'_!$ are exact functors).
Assume $a \leq p_0$. Consider the distinguished triangle
$$
H^a(K)[-a] \to K \to \tau_{\geq a + 1}K
$$
By induction we have the result for $\tau_{\geq a + 1}K$.
In the next paragraph, we will prove the result for $H^a(K)[-a]$.
Then five lemma applied to the map between
long exact sequence of cohomology sheaves
associated to the map of distinguished triangles
$$
\xymatrix{
g_! Rf'_*(H^a(K)[-a]) \ar[d] \ar[r] &
g_! Rf'_* K \ar[r] \ar[d] &
g_! Rf'_* \tau_{\geq a + 1} K \ar[d] \\
Rf_*(g'_!(H^a(K)[-a])) \ar[r] &
Rf_*(g'_!K) \ar[r] &
Rf_*(g'_!\tau_{|geq a + 1}K)
}
$$
gives the result for $K$. Some details omitted.

\medskip\noindent
Let $\mathcal{F}$ be a torsion abelian sheaf on $X'_\etale$.
To finish the proof we show that
$g_! Rf'_*\mathcal{F} \to R^pf_*(g'_!\mathcal{F})$
is an isomorphism for all $p$.
We can write $\mathcal{F} = \bigcup \mathcal{F}[n]$ where
$\mathcal{F}[n] = \Ker(n : \mathcal{F} \to \mathcal{F})$.
We have the isomorphism for $\mathcal{F}[n]$ by case (b).
Since the functors $g_!$, $g'_!$, $R^pf_*$, $R^pf'_*$ commute
with filtered colimits (follows from
Lemma \ref{lemma-lqf-f-shriek-separated-colimits} and
\'Etale Cohomology, Lemma
\ref{etale-cohomology-lemma-relative-colimit-general})
the proof is complete.
\end{proof}

\begin{lemma}
\label{lemma-shriek-proper-and-open-compose}
Consider a commutative diagram of schemes
$$
\xymatrix{
X' \ar[r]_k \ar[d]_{f'} & X \ar[d]^f \\
Y' \ar[r]_l \ar[d]_{g'} & Y \ar[d]^g \\
Z' \ar[r]^m & Z
}
$$
with $f$, $f'$, $g$ and $g'$ proper and
$k$, $l$, and $m$ separated and locally quasi-finite.
Then the isomorphisms of Lemma \ref{lemma-shriek-proper-and-open}
for the two squares compose to give the isomorphism
for the outer rectangle (see proof for a precise statement).
\end{lemma}

\begin{proof}
The statement means that if we write
$R(g \circ f)_* = Rg_* \circ Rf_*$ and
$R(g' \circ f')_* = Rg'_* \circ Rf'_*$,
then the isomorphism
$m_! \circ Rg'_* \circ Rf'_* \to Rg_* \circ Rf_* \circ k_!$
of the outer rectangle is equal to the composition
$$
m_! \circ Rg'_* \circ Rf'_* \to
Rg_* \circ l_! \circ Rf'_* \to
Rg_* \circ Rf_* \circ k_!
$$
of the two maps of the squares in the diagram. To prove this choose
a K-injective complex $\mathcal{J}^\bullet$ of $\Lambda$-modules
on $X'_\etale$ and a quasi-isomorphism
$k_!\mathcal{J}^\bullet \to \mathcal{I}^\bullet$
to a K-injective complex $\mathcal{I}^\bullet$ of $\Lambda$-modules
on $X_\etale$. The proof of Lemma \ref{lemma-shriek-proper-and-open}
shows that the canonical map
$$
a : l_!f'_*\mathcal{J}^\bullet \to f_*\mathcal{I}^\bullet
$$
is a quasi-isomorphism and this quasi-isomorphism produces the
second arrow on applying $Rg_*$. By
Cohomology on Sites, Lemma \ref{sites-cohomology-lemma-K-injective-flat}
the complex $f_*\mathcal{I}^\bullet$, resp.\ $f'_*\mathcal{J}^\bullet$
is a K-injective complex of $\Lambda$-modules on 
$Y_\etale$, resp.\ $Y'_\etale$.
(Using this is cheating and could be avoided.)
In particular, the same reasoning gives that the canonical map
$$
b : m_!g'_*f'_*\mathcal{J}^\bullet \to g_*f_*\mathcal{I}^\bullet
$$
is a quasi-isomorphism and this quasi-isomorphism represents
the first arrow. Finally, the proof of Lemma \ref{lemma-shriek-proper-and-open}
show that $g_*l_!f'_!\mathcal{J}^\bullet$ represents
$Rg_*(l_!f'_*\mathcal{J}^\bullet)$ because $f'_*\mathcal{J}^\bullet$
is K-injective. Hence $Rg_*(a) = g_*(a)$ and the composition
$g_*(a) \circ b$ is the arrow of Lemma \ref{lemma-shriek-proper-and-open}
for the rectangle.
\end{proof}

\begin{lemma}
\label{lemma-shriek-proper-and-open-compose-horizontal}
Consider a commutative diagram of schemes
$$
\xymatrix{
X'' \ar[r]_{g'} \ar[d]_{f''} &
X' \ar[r]_g \ar[d]_{f'} &
X \ar[d]^f \\
Y'' \ar[r]^{h'} &
Y' \ar[r]^h &
Y
}
$$
with $f$, $f'$, and $f''$ proper and
$g$, $g'$, $h$, and $h'$ separated and locally quasi-finite.
Then the isomorphisms of Lemma \ref{lemma-shriek-proper-and-open}
for the two squares compose to give the isomorphism
for the outer rectangle (see proof for a precise statement).
\end{lemma}

\begin{proof}
The statement means that if we write
$(h \circ h')_! = h_! \circ h'_!$ and
$(g \circ g')_! = g_! \circ g'_!$
using the equalities of Lemma \ref{lemma-f-shriek-composition},
then the isomorphism
$h_! \circ h'_! \circ Rf''_* \to Rf_* \circ g_! \circ g'_!$
of the outer rectangle is equal to the composition
$$
h_! \circ h'_! \circ Rf''_* \to
h_! \circ Rf'_* \circ g'_! \to
Rf_* \circ g_! \circ g'_!
$$
of the two maps of the squares in the diagram. To prove this choose
a K-injective complex $\mathcal{I}^\bullet$ of $\Lambda$-modules
on $X''_\etale$ and a quasi-isomorphism
$g'_!\mathcal{I}^\bullet \to \mathcal{J}^\bullet$
to a K-injective complex $\mathcal{J}^\bullet$ of $\Lambda$-modules
on $X'_\etale$. Next, choose a quasi-isomorphism
$g_!\mathcal{J}^\bullet \to \mathcal{K}^\bullet$
to a K-injective complex $\mathcal{K}^\bullet$ of $\Lambda$-modules
on $X_\etale$.
The proof of Lemma \ref{lemma-shriek-proper-and-open} shows that the canonical
maps
$$
h'_!f''_*\mathcal{I}^\bullet \to f'_*\mathcal{J}^\bullet
\quad\text{and}\quad
h_!f'_*\mathcal{J}^\bullet \to f_*\mathcal{K}^\bullet
$$
are quasi-isomorphisms and these quasi-isomorphisms define the
first and second arrow above. Since $g_!$ is an exact functor
(Lemma \ref{lemma-lqf-f-shriek-separated-colimits})
we find that $g_!g'_!\mathcal{I}^\bullet \to \mathcal{K}^\bullet$
is a quasi-ismorphism and hence the canonical map
$$
h_!h'_!f''_*\mathcal{I}^\bullet \to f_*\mathcal{K}^\bullet
$$
is a quasi-isomorphism and represents the map for the outer
rectangle in the derived category. Clearly this map is the
composition of the other two and the proof is complete.
\end{proof}

\begin{remark}
\label{remark-going-around}
Consider a commutative diagram
$$
\xymatrix{
X'' \ar[r]_{k'} \ar[d]_{f''} & X' \ar[r]_k \ar[d]_{f'} & X \ar[d]^f \\
Y'' \ar[r]^{l'} \ar[d]_{g''} & Y' \ar[r]^l \ar[d]_{g'} & Y \ar[d]^g \\
Z'' \ar[r]^{m'} & Z' \ar[r]^m & Z
}
$$
of schemes whose vertical arrows are proper and whose horizontal
arrows are separated and locally quasi-finite.
Let us label the squares of the diagram $A$, $B$, $C$, $D$
as follows
$$
\begin{matrix}
A & B \\
C & D
\end{matrix}
$$
Then the maps of Lemma \ref{lemma-shriek-proper-and-open}
for the squares are (where we use $Rf_* = f_*$, etc)
$$
\begin{matrix}
\gamma_A : l'_! \circ f''_* \to f'_* \circ k'_! &
\gamma_B : l_! \circ f'_* \to f_* \circ k_! \\
\gamma_C : m'_! \circ g''_* \to g'_* \circ l'_! &
\gamma_D : m_! \circ g'_* \to g_* \circ l_!
\end{matrix}
$$
For the $2 \times 1$ and $1 \times 2$ rectangles we have four further
maps
$$
\begin{matrix}
\gamma_{A + B} :
(l \circ l')_! \circ f''_* \to f_* \circ (k \circ k')_*  \\
\gamma_{C + D} :
(m \circ m')_! \circ g''_* \to g_* \circ (l \circ l')_! \\
\gamma_{A + C} :
m'_! \circ (g'' \circ f'')_* \to (g' \circ f')_* \circ k'_! \\
\gamma_{B + D} :
m_! \circ (g' \circ f')_* \to (g \circ f)_* \circ k_!
\end{matrix}
$$
By Lemma \ref{lemma-shriek-proper-and-open-compose-horizontal} we have
$$
\gamma_{A + B} = \gamma_B \circ \gamma_A, \quad
\gamma_{C + D} = \gamma_D \circ \gamma_C
$$
and by Lemma \ref{lemma-shriek-proper-and-open-compose} we have
$$
\gamma_{A + C} = \gamma_A \circ \gamma_C, \quad
\gamma_{B + D} = \gamma_B \circ \gamma_D
$$
Here it would be more correct to write
$\gamma_{A + B} = (\gamma_B \star \text{id}_{k'_!}) \circ
(\text{id}_{l_!} \star \gamma_A)$ with notation as in
Categories, Section \ref{categories-section-formal-cat-cat}
and similarly for the others.
Having said all of this we find (a priori) two transformations
$$
m_! \circ m'_! \circ g''_* \circ f''_*
\longrightarrow
g_* \circ f_* \circ k_! \circ k'_!
$$
namely
$$
\gamma_B \circ \gamma_D \circ \gamma_A \circ \gamma_C =
\gamma_{B + D} \circ \gamma_{A + C}
$$
and
$$
\gamma_B \circ \gamma_A \circ \gamma_D \circ \gamma_C =
\gamma_{A + B} \circ \gamma_{C + D}
$$
The point of this remark is to point out that these transformations
are equal. Namely, to see this it suffices to show that
$$
\xymatrix{
m_! \circ g'_* \circ l'_! \circ f''_* \ar[r]_{\gamma_D} \ar[d]_{\gamma_A} &
g_* \circ l_! \circ l'_! \circ f''_* \ar[d]^{\gamma_A} \\
m_! \circ g'_* \circ f'_* \circ k'_! \ar[r]^{\gamma_D} &
g_* \circ l_! \circ f'_* \circ k'_!
}
$$
commutes. This is true because the squares $A$ and $D$ meet in only
one point, more precisely by
Categories, Lemma \ref{categories-lemma-properties-2-cat-cats}
or more simply the discussion preceding
Categories, Definition \ref{categories-definition-horizontal-composition}.
\end{remark}

\begin{lemma}
\label{lemma-shriek-proper-and-open-base-change}
Let $b : Y_1 \to Y$ be a morphism of schemes. Consider a commutative diagram
of schemes
$$
\vcenter{
\xymatrix{
X' \ar[r]_{g'} \ar[d]_{f'} & X \ar[d]^f \\
Y' \ar[r]^g & Y
}
}
\quad\text{and let}\quad
\vcenter{
\xymatrix{
X'_1 \ar[r]_{g'_1} \ar[d]_{f'_1} & X_1 \ar[d]^{f_1} \\
Y'_1 \ar[r]^{g_1} & Y_1
}
}
$$
be the base change by $b$. Assume $f$ and $f'$ proper and
$g$ and $g'$ separated and locally quasi-finite.
For a ring $\Lambda$ and $K$ in $D(X'_\etale, \Lambda)$
there is commutative diagram
$$
\xymatrix{
b^{-1}g_!Rf'_*K \ar[d] \ar[r] &
g_{1, !}(b')^{-1}Rf'_*K \ar[r] &
g_{1, !}Rf'_{1, *}(a')^{-1}K \ar[d] \\
b^{-1}Rf_*g'_!K \ar[r] &
Rf_{1, *}a^{-1}g'_!K \ar[r] &
Rf_{1, *}g'_{1, !}(a')^{-1}K
}
$$
in $D(Y_{1, \etale}, \Lambda)$ where $a : X_1 \to X$, $a' : X'_1 \to X'$,
$b' : Y'_1 \to Y'$ are the projections, the vertical maps are the arrows
of Lemma \ref{lemma-shriek-proper-and-open}
and the horizontal arrows are the base change map
(from \'Etale Cohomology, Section
\ref{etale-cohomology-section-base-change-preliminaries})
and the base change map of Lemma \ref{lemma-base-change-f-shriek-separated}.
\end{lemma}

\begin{proof}
Represent $K$ by a K-injective complex $\mathcal{J}^\bullet$ of
sheaves of $\Lambda$-modules on $X'_\etale$. Choose a quasi-isomorphism
$g'_!\mathcal{J}^\bullet \to \mathcal{I}^\bullet$ to a
K-injective complex $\mathcal{I}^\bullet$ of
sheaves of $\Lambda$-modules on $X_\etale$.
The proof of Lemma \ref{lemma-shriek-proper-and-open}
constructs $g_!Rf'_*K \to Rf_*g'_!K$ as
$$
g_!f'_*\mathcal{J}^\bullet =
g_!f'_!\mathcal{J}^\bullet =
f_!g'_!\mathcal{J}^\bullet =
f_*g'_!\mathcal{J}^\bullet \to
f_*\mathcal{I}^\bullet
$$
Choose a quasi-isomorphism
$(a')^{-1}\mathcal{J}^\bullet \to \mathcal{J}_1^\bullet$
to a K-injective complex $\mathcal{J}_1^\bullet$ of
sheaves of $\Lambda$-modules on $X'_{1, \etale}$.
Then we can pick a diagram of complexes
$$
\xymatrix{
g'_{1, !}\mathcal{J}_1^\bullet \ar[rr] & &
\mathcal{I}_1^\bullet \\
g'_{1, !}(a')^{-1}\mathcal{J}^\bullet \ar[u] \ar@{=}[r] &
a^{-1}g'_!\mathcal{J}^\bullet \ar[r] &
a^{-1}\mathcal{I}^\bullet \ar[u]
}
$$
commuting up to homotopy where all arrows are quasi-isomorphisms, the
equality comes from Lemma \ref{lemma-proper-f-shriek},
and $\mathcal{I}_1^\bullet$ is a K-injective complex of sheaves of
$\Lambda$-modules on $X_{1, \etale}$. The map
$g_{1, !}Rf'_{1, *}(a')^{-1}K \to Rf_{1, *}g'_{1, !}(a')^{-1}K$ is given by
$$
g_{1, !}f'_{1, *}\mathcal{J}_1^\bullet =
g_{1, !}f'_{1, !}\mathcal{J}_1^\bullet =
f_{1, !}g'_{1, !}\mathcal{J}_1^\bullet =
f_{1, *}g'_{1, !}\mathcal{J}_1^\bullet \to
f_{1, *}\mathcal{I}_1^\bullet
$$
The identifications across the $3$ equal signs in both arrows
are compatible with pullback maps, i.e., the diagram
$$
\xymatrix{
b^{-1}g_!f'_*\mathcal{J}^\bullet \ar@{=}[d] \ar[r] &
g_{1, !}(b')^{-1}f'_*\mathcal{J}^\bullet \ar[r] &
g_{1, !}f'_{1, *}(a')^{-1}\mathcal{J}^\bullet \ar@{=}[d] \\
b^{-1}f_*g'_!\mathcal{J}^\bullet \ar[r] &
f_{1, *}a^{-1}g'_!\mathcal{J}^\bullet \ar[r] &
f_{1, *}g'_{1, !}(a')^{-1}\mathcal{J}^\bullet
}
$$
of complexes of abelian sheaves commutes. To show this it is enough to
show the diagram commutes with $g_!, g_{1, !}, g'_!, g'_{1, !}$
replaced by $g_*, g_{1, *}, g'_*, g'_{1, *}$ (because the shriek
functors are defined as subfunctors of the $*$ functors and the
base change maps are defined in a manner compatible with this, see
proof of Lemma \ref{lemma-base-change-f-shriek-separated}).
For this new diagram the commutativity follows from the compatibility
of pullback maps with horizontal and vertical stacking of diagrams, see
Sites, Remarks \ref{sites-remark-compose-base-change} and
\ref{sites-remark-compose-base-change-horizontal}
so that going around the diagram in either direction is the pullback
map for the base change of $f \circ g' = g \circ f'$ by $b$.
Since of course
$$
\xymatrix{
g_{1, !}f'_{1, *}(a')^{-1}\mathcal{J}^\bullet \ar@{=}[d] \ar[r] &
g_{1, !}f'_{1, *}\mathcal{J}_1^\bullet \ar@{=}[d] \\
f_{1, *}g'_{1, !}(a')^{-1}\mathcal{J}^\bullet \ar[r] &
f_{1, *}g'_{1, !}\mathcal{J}_1^\bullet 
}
$$
commutes, to finish the proof it suffices to show that
$$
\xymatrix{
b^{-1}f_*g'_!\mathcal{J}^\bullet \ar[r] \ar[d] &
f_{1, *}a^{-1}g'_!\mathcal{J}^\bullet \ar[r] \ar[d] &
f_{1, *}g'_{1, !}(a')^{-1}\mathcal{J}^\bullet \ar[r] &
f_{1, *}g'_{1, !}\mathcal{J}_1^\bullet \ar[d] \\
b^{-1}f_*\mathcal{I}^\bullet \ar[r] &
f_{1, *}a^{-1}\mathcal{I}^\bullet \ar[rr] & &
f_{1, *}\mathcal{I}_1^\bullet
}
$$
commutes in the derived category, which holds by our choice of
maps earlier.
\end{proof}

\begin{lemma}
\label{lemma-shriek-lqf-and-proper}
Consider a commutative diagram of schemes
$$
\xymatrix{
X \ar[r]_f \ar[rd]_g & Y \ar[d]^h \\
& Z
}
$$
with $f$ and $g$ locally quasi-finite and $h$ proper. Let $\Lambda$ be a ring.
Funtorially in $K \in D(X_\etale, \Lambda)$ there is a canonical map
$$
g_!K \longrightarrow Rh_*(f_!K)
$$
in $D(Z_\etale, \Lambda)$. This map is an isomorphism if
(a) $K$ is bounded below and has torsion cohomology sheaves, or
(b) $\Lambda$ is a torsion ring.
\end{lemma}

\begin{proof}
This is a special case of Lemma \ref{lemma-shriek-proper-and-open}
if $f$ and $g$ are separated. We urge the reader to skip the proof
in the general case
as we'll mainly use the case where $f$ and $g$ are separated.

\medskip\noindent
Represent $K$ by a complex $\mathcal{K}^\bullet$ of sheaves of
$\Lambda$-modules on $X_\etale$. Choose a quasi-isomorphism
$f_!\mathcal{K}^\bullet \to \mathcal{I}^\bullet$ into a K-injective
complex $\mathcal{I}^\bullet$ of sheaves of $\Lambda$-modules on $Y_\etale$.
Consider the map
$$
g_!\mathcal{K}^\bullet =
h_!f_!\mathcal{K}^\bullet =
h_*f_!\mathcal{K}^\bullet
\longrightarrow
h_*\mathcal{I}^\bullet
$$
where the equalities are
Lemmas \ref{lemma-lqf-separated-shriek-composition}
and \ref{lemma-proper-f-shriek}. This map of complexes determines the map
$g_!K \to Rh_*(f_!K)$ of the statement of the lemma.

\medskip\noindent
Assume $\Lambda$ is torsion, i.e., we are in case (b). To check the map
is an isomorphism we may work locally on $Z$.
Hence we may assume that the dimension of fibres of $h$ is bounded,
see Morphisms, Lemma
\ref{morphisms-lemma-morphism-finite-type-bounded-dimension}.
Then we see that $Rh_*$ has finite cohomological dimension, see
\'Etale Cohomology, Lemma
\ref{etale-cohomology-lemma-cohomological-dimension-proper}.
Hence by Derived Categories, Lemma \ref{derived-lemma-unbounded-right-derived},
if we show that $R^qh_*(f_!\mathcal{F}) = 0$ for $q > 0$
and any sheaf $\mathcal{F}$ of $\Lambda$-modules on $X_\etale$, then
$h_*f_!\mathcal{K}^\bullet \to h_*\mathcal{I}^\bullet$ is a quasi-isomorphism.

\medskip\noindent
Observe that $\mathcal{G} = f_!\mathcal{F}$ is a sheaf of $\Lambda$-modules
on $Y$ whose stalks are nonzero only at points $y \in Y$ such that
$\kappa(y)/\kappa(h(y))$ is a finite extension. This follows from the
description of stalks of $f_!\mathcal{F}$ in
Lemma \ref{lemma-lqf-f-shriek-stalk}
and the fact that both $f$ and $g$ are locally quasi-finite.
Hence by the proper base change theorem (\'Etale Cohomology, Lemma
\ref{etale-cohomology-lemma-proper-base-change-stalk})
it suffices to show that $H^q(Y_{\overline{z}}, \mathcal{H}) = 0$
where $\mathcal{H}$ is a sheaf on the proper scheme $Y_{\overline{z}}$
over $\kappa(\overline{z})$ whose support is contained in the set
of closed points. Thus the required vanishing by \'Etale Cohomology, Lemma
\ref{etale-cohomology-lemma-supported-in-closed-points}.

\medskip\noindent
Case (a) follows from case (b) by the exact same argument as used
in the proof of Lemma \ref{lemma-shriek-proper-and-open}
(using Lemma \ref{lemma-lqf-f-shriek-stalk} instead of
Lemma \ref{lemma-lqf-f-shriek-separated-colimits}).
\end{proof}







\section{Derived lower shriek via compactifications}
\label{section-derived-lower-shriek-compactification}

\noindent
Let $f : X \to Y$ be a finite type separated morphism of schemes
with $Y$ quasi-compact and quasi-separated. Choose a compactification
$j : X \to \overline{X}$ over $Y$, see
More on Flatness, Theorem \ref{flat-theorem-nagata}.
Let $\Lambda$ be a ring. Denote $D^+_{tors}(X_\etale, \Lambda)$
the strictly full saturated triangulated subcategory of
$D(X_\etale, \Lambda)$ consisting of objects $K$ which are bounded below
and whose cohomology sheaves are torsion. We will consider the
functor
$$
Rf_! = R\overline{f}_* \circ j_! :
D^+_{tors}(X_\etale, \Lambda)
\longrightarrow
D^+_{tors}(Y_\etale, \Lambda)
$$
This makes sense: the functor $j_!$ sends $D^+_{tors}(X_\etale, \Lambda)$ into
$D^+_{tors}(\overline{X}_\etale, \Lambda)$ by
Remark \ref{remark-lqf-shriek-derived-torsion} and $R\overline{f}_*$ sends
$D^+_{tors}(\overline{X}_\etale, \Lambda)$ into $D^+_{tors}(Y_\etale, \Lambda)$
by \'Etale Cohomology, Lemma \ref{etale-cohomology-lemma-torsion-direct-image}.
If $\Lambda$ is a torsion ring, then we define
$$
Rf_! = R\overline{f}_* \circ j_! :
D(X_\etale, \Lambda)
\longrightarrow
D(Y_\etale, \Lambda)
$$
Here is the obligatory lemma.

\begin{lemma}
\label{lemma-shriek-well-defined}
Let $f : X \to Y$ be a finite type separated morphism of quasi-compact
and quasi-separated schemes. The functors $Rf_!$ constructed above
are, up to canonical isomorphism, independent of the choice of the
compactification.
\end{lemma}

\begin{proof}
We will prove this for the functor
$Rf_! : D(X_\etale, \Lambda) \to D(Y_\etale, \Lambda)$
when $\Lambda$ is a torsion ring; the case of the functor
$Rf_! : D^+_{tors}(X_\etale, \Lambda) \to D^+_{tors}(Y_\etale, \Lambda)$
is proved in exactly the same way.

\medskip\noindent
Consider the category of compactifications of $X$ over $Y$, which is
cofiltered according to More on Flatness, Theorem \ref{flat-theorem-nagata} and
Lemmas \ref{flat-lemma-compactifications-cofiltered} and
\ref{flat-lemma-compactifyable}.
To every choice of a compactification
$$
j : X \to \overline{X},\quad \overline{f} : \overline{X} \to Y
$$
the construction above associates the functor $R\overline{f}_* \circ j_! :
D(X_\etale, \Lambda) \to D(Y_\etale, \Lambda)$.
Let's be a little more explicit. Given a complex $\mathcal{K}^\bullet$
of sheaves of $\Lambda$-modules on $X_\etale$, we choose a quasi-isomorphism
$j_!\mathcal{K}^\bullet \to \mathcal{I}^\bullet$ into a K-injective
complex of sheaves of $\Lambda$-modules on $\overline{X}_\etale$.
Then our functor sends $\mathcal{K}^\bullet$ to
$\overline{f}_*\mathcal{I}^\bullet$.

\medskip\noindent
Suppose given a morphism $g : \overline{X}_1 \to \overline{X}_2$
between compactifications $j_i : X \to \overline{X}_i$ over $Y$.
Then we get an isomorphism
$$
R\overline{f}_{2, *} \circ j_{2, !} =
R\overline{f}_{2, *} \circ Rg_* \circ j_{1, !} =
R\overline{f}_{1, *} \circ j_{1, !}
$$
using Lemma \ref{lemma-shriek-lqf-and-proper} in the first equality.

\medskip\noindent
To finish the proof, since the category of compactifications of $X$ over $Y$
is cofiltered, it suffices to show compositions of morphisms of
compactifications of $X$ over $Y$ are turned into compositions of
isomorphisms of functors\footnote{Namely, if $\alpha, \beta : F \to G$
are morphisms of functors and $\gamma : G \to H$ is an isomorphism
of functors such that $\gamma \circ \alpha = \gamma \circ \beta$, then
we conclude $\alpha = \beta$.}. To do this, suppose that
$j_3 : X \to \overline{X}_3$
is a third compactification and that $h : \overline{X}_2 \to \overline{X}_3$
is a morphism of compactifications. Then we have to show that the
composition
$$
R\overline{f}_{3, *} \circ j_{3, !} =
R\overline{f}_{3, *} \circ Rh_* \circ j_{2, !} =
R\overline{f}_{2, *} \circ j_{2, !} =
R\overline{f}_{2, *} \circ Rg_* \circ j_{1, !} =
R\overline{f}_{1, *} \circ j_{1, !}
$$
is equal to the isomorphism of functors constructed using simply
$j_3$, $g \circ h$, and $j_1$. A calculation shows that it suffices to
prove that the composition of the maps
$$
j_{3, !} \to Rh_* \circ j_{2, !} \to Rh_* \circ Rg_* \circ j_{1, !}
$$
of Lemma \ref{lemma-shriek-lqf-and-proper} agrees with the corresponding
map $j_{3, !} \to R(h \circ g)_* \circ j_{1, !}$
via the identification $R(h \circ g)_* = Rh_* \circ Rg_*$.
Since the map of Lemma \ref{lemma-shriek-lqf-and-proper}
is a special case of the map of Lemma \ref{lemma-shriek-proper-and-open}
(as $j_1$ and $j_2$ are separated) this follows immediately from
Lemma \ref{lemma-shriek-proper-and-open-compose}.
\end{proof}

\begin{lemma}
\label{lemma-shriek-composition}
Let $f : X \to Y$ and $g : Y \to Z$ be separated morphisms of finite type
of quasi-compact and quasi-separated schemes. Then there is a canonical
isomorphism $Rg_! \circ Rf_! \to R(g \circ f)_!$.
\end{lemma}

\begin{proof}
Choose a compactification $i : Y \to \overline{Y}$ of $Y$ over $Z$.
Choose a compactification $X \to \overline{X}$ of $X$ over
$\overline{Y}$. This uses More on Flatness, Theorem \ref{flat-theorem-nagata}
and Lemma \ref{flat-lemma-compactifyable} twice.
Let $U$ be the inverse image of $Y$ in $\overline{X}$
so that we get the commutative diagram
$$
\xymatrix{
X \ar[r]_j \ar[d]_f &
U \ar[dl]^{f'} \ar[r]_{j'} &
\overline{X} \ar[dl]^{\overline{f}} \\
Y \ar[r]_i \ar[d]_g &
\overline{Y} \ar[dl]^{\overline{g}} \\
Z
}
$$
Then we have
\begin{align*}
R(g \circ f)_!
& =
R(\overline{g} \circ \overline{f})_* \circ (j' \circ j)_! \\
& =
R\overline{g}_* \circ R\overline{f}_* \circ j'_! \circ j_! \\
& =
R\overline{g}_* \circ i_! \circ Rf'_* \circ j_! \\
& =
Rg_! \circ Rf_!
\end{align*}
The first equality is the definition of $R(g \circ f)_!$.
The second equality uses the identifications
$R(\overline{g} \circ \overline{f})_* = R\overline{g}_* \circ R\overline{f}_*$
and $(j' \circ j)_! = j'_! \circ j_!$ of Lemma \ref{lemma-f-shriek-composition}.
The identification $i_! \circ Rf'_* \to R\overline{f}_* \circ j_!$
used in the third equality is Lemma \ref{lemma-shriek-proper-and-open}.
The final fourth equality is the definition of $Rg_!$ and $Rf_!$.
To finish the proof we show that
this isomorphism is independent of choices made.

\medskip\noindent
Suppose we have two diagrams
$$
\vcenter{
\xymatrix{
X \ar[r]_{j_1} \ar[d] &
U_1 \ar[dl]^{f_1} \ar[r]_{j'_1} &
\overline{X}_1 \ar[dl]^{\overline{f}_1} \\
Y \ar[r]_{i_1} \ar[d] &
\overline{Y}_1 \ar[dl]^{\overline{g}_1} \\
Z
}
}
\quad\text{and}\quad
\vcenter{
\xymatrix{
X \ar[r]_{j_2} \ar[d] &
U_2 \ar[dl]^{f_2} \ar[r]_{j'_2} &
\overline{X}_2 \ar[dl]^{\overline{f}_2} \\
Y \ar[r]_{i_2} \ar[d] &
\overline{Y}_2 \ar[dl]^{\overline{g}_2} \\
Z
}
}
$$
We can first choose a compactification $i : Y \to \overline{Y}$
of $Y$ over $Z$ which dominates both $\overline{Y}_1$ and $\overline{Y}_2$,
see More on Flatness, Lemma \ref{flat-lemma-compactifications-cofiltered}.
By More on Flatness, Lemma \ref{flat-lemma-right-multiplicative-system} and
Categories, Lemmas \ref{categories-lemma-morphisms-right-localization} and
\ref{categories-lemma-equality-morphisms-right-localization}
we can choose a compactification $X \to \overline{X}$ of
$X$ over $\overline{Y}$ with morphisms $\overline{X} \to \overline{X}_1$
and $\overline{X} \to \overline{X}_2$ and such that the composition
$\overline{X} \to \overline{Y} \to \overline{Y}_1$ is equal to
the composition $\overline{X} \to \overline{X}_1 \to \overline{Y}_1$
and such that the composition
$\overline{X} \to \overline{Y} \to \overline{Y}_2$ is equal to
the composition $\overline{X} \to \overline{X}_2 \to \overline{Y}_2$.
Thus we see that it suffices to compare the maps
determined by our diagrams when we have a commutative diagram
as follows
$$
\xymatrix{
X \ar[rr]_{j_1} \ar@{=}[d] & &
U_1 \ar[d]^{h'} \ar[ddll] \ar[rr]_{j'_1} & &
\overline{X}_1 \ar[d]^h \ar[ddll] \\
X \ar'[r][rr]^-{j_2} \ar[d] & &
U_2 \ar'[dl][ddll] \ar'[r][rr]^-{j'_2} & &
\overline{X}_2 \ar[ddll] \\
Y \ar[rr]^{i_1} \ar@{=}[d] & & \overline{Y}_1 \ar[d]^k \\
Y \ar[rr]^{i_2} \ar[d] & & \overline{Y}_2 \ar[dll] \\
Z
}
$$
Each of the squares
$$
\xymatrix{
X \ar[r]_{j_1} \ar[d]_{\text{id}} \ar@{}[dr]|A &
U_1 \ar[d]^{h'} \\
X \ar[r]^{j_2} &
U_2
}
\quad
\xymatrix{
U_2 \ar[r]_{j_2'} \ar[d]_{f_2} \ar@{}[dr]|B &
\overline{X}_2 \ar[d]^{\overline{f}_2} \\
Y \ar[r]^{i_2} &
\overline{Y}_2
}
\quad
\xymatrix{
U_1 \ar[r]_{j_1'} \ar[d]_{f_1} \ar@{}[dr]|C &
\overline{X}_1 \ar[d]^{\overline{f}_1} \\
Y \ar[r]^{i_1} &
\overline{Y}_1
}
\quad
\xymatrix{
Y \ar[r]_{i_1} \ar[d]_{\text{id}} \ar@{}[dr]|D &
\overline{Y}_1 \ar[d]^k \\
Y \ar[r]^{i_2} &
\overline{Y}_2
}
\quad
\xymatrix{
X \ar[r]_{j_1' \circ j_1} \ar[d]_{\text{id}} \ar@{}[dr]|E &
\overline{X}_1 \ar[d]^h \\
X \ar[r]^{j_2} &
\overline{X}_2
}
$$
gives rise to an isomorphism as follows
\begin{align*}
\gamma_A & :
j_{2, !} \to Rh'_* \circ j_{1, !}  \\
\gamma_B & :
i_{2, !} \circ Rf_{2, *} \to R\overline{f}_{2, *} \circ j'_{2, !} \\
\gamma_C & :
i_{1, !} \circ Rf_{1, *} \to R\overline{f}_{1, *} \circ j'_{1, !} \\
\gamma_D & :
i_{2, !} \to Rk_* \circ i_{1, !} \\
\gamma_E & :
j_{2, !} \to Rh_* \circ (j'_1 \circ j_1)_!
\end{align*}
by applying the map from Lemma \ref{lemma-shriek-proper-and-open}
(which is the same as the map in Lemma \ref{lemma-shriek-lqf-and-proper}
in case the left vertical arrow is the identity). Let us write
\begin{align*}
F_1 & = Rf_{1, *} \circ j_{1, !} \\
F_2 & = Rf_{2, *} \circ j_{2, !} \\
G_1 & = R\overline{g}_{1, *} \circ i_{1, !} \\
G_2 & = R\overline{g}_{2, *} \circ i_{2, !} \\
C_1 & = R(\overline{g}_1 \circ \overline{f}_1)_* \circ (j'_1 \circ j_1)_! \\
C_2 & = R(\overline{g}_2 \circ \overline{f}_2)_* \circ (j'_2 \circ j_2)_!
\end{align*}
The construction given in the first paragraph of the proof
and in Lemma \ref{lemma-shriek-well-defined} uses
\begin{enumerate}
\item $\gamma_C$ for the map $G_1 \circ F_1 \to C_1$,
\item $\gamma_B$ for the map $G_2 \circ F_2 \to C_2 $,
\item $\gamma_A$ for the map $F_2 \to F_1$,
\item $\gamma_D$ for the map $G_2 \to G_1$, and
\item $\gamma_E$ for the map $C_2 \to C_1$.
\end{enumerate}
This implies that we have to show that the diagram
$$
\xymatrix{
C_2 \ar[rr]_{\gamma_E} & &
C_1 \\
G_2 \circ F_2 \ar[rr]^{\gamma_D \circ \gamma_A} \ar[u]^{\gamma_B} & &
G_1 \circ F_1 \ar[u]_{\gamma_C}
}
$$
is commutative. We will use
Lemmas \ref{lemma-shriek-proper-and-open-compose} and
\ref{lemma-shriek-proper-and-open-compose-horizontal}
and with (abuse of) notation as in
Remark \ref{remark-going-around} (in particular
dropping $\star$ products with identity transformations
from the notation).
We can write $\gamma_E = \gamma_F \circ \gamma_A$ where
$$
\xymatrix{
U_1 \ar[r]_{j'_1} \ar[d]_{h'} \ar@{}[rd]|F &
\overline{X}_1 \ar[d]^h \\
U_2 \ar[r]^{j'_2} &
\overline{X}_2
}
$$
Thus we see that
$$
\gamma_E \circ \gamma_B = \gamma_F \circ \gamma_A  \circ \gamma_B
= \gamma_F \circ \gamma_B \circ \gamma_A
$$
the last equality because the two squares $A$ and $B$ only
intersect in one point (similar to the last argument in
Remark \ref{remark-going-around}). Thus it suffices to prove that
$\gamma_C \circ \gamma_D = \gamma_F \circ \gamma_B$.
Since both of these are equal to the map for the square
$$
\xymatrix{
U_1 \ar[r] \ar[d] & \overline{X}_1 \ar[d] \\
Y \ar[r] & \overline{Y}_2
}
$$
we conclude.
\end{proof}

\begin{lemma}
\label{lemma-pseudo-functor}
Let $f : X \to Y$, $g : Y \to Z$, $h : Z \to T$ be separated morphisms of
finite type of quasi-compact and quasi-separated schemes. Then
the diagram
$$
\xymatrix{
Rh_! \circ Rg_! \circ Rf_! \ar[r]_{\gamma_C} \ar[d]^{\gamma_A} &
R(h \circ g)_! \circ Rf_! \ar[d]_{\gamma_{A + B}} \\
Rh_! \circ R(g \circ f)_! \ar[r]^{\gamma_{B + C}} &
R(h \circ g \circ f)_!
}
$$
of isomorphisms of Lemma \ref{lemma-shriek-composition} commutes
(for the meaning of the $\gamma$'s see proof).
\end{lemma}

\begin{proof}
To do this we choose a compactification $\overline{Z}$
of $Z$ over $T$, then a compactification $\overline{Y}$ of $Y$ over
$\overline{Z}$, and then a compactification $\overline{X}$ of
$X$ over $\overline{Y}$. This uses
More on Flatness, Theorem \ref{flat-theorem-nagata} and
Lemma \ref{flat-lemma-compactifyable}.
Let $W \subset \overline{Y}$ be the inverse image of $Z$ under
$\overline{Y} \to \overline{Z}$ and let $U \subset V \subset \overline{X}$
be the inverse images of $Y \subset W$ under $\overline{X} \to \overline{Y}$.
This produces the following diagram
$$
\xymatrix{
X \ar[d]_f \ar[r] & U \ar[r] \ar[d] \ar@{}[dr]|A &
V \ar[d] \ar[r] \ar@{}[rd]|B & \overline{X} \ar[d] \\
Y \ar[d]_g \ar[r] & Y \ar[r] \ar[d] & W \ar[r] \ar[d] \ar@{}[rd]|C &
\overline{Y} \ar[d] \\
Z \ar[d]_h \ar[r] & Z \ar[d] \ar[r] & Z \ar[d] \ar[r] & \overline{Z} \ar[d] \\
T \ar[r] & T \ar[r] & T \ar[r] & T
}
$$
Without introducing tons of notation but arguing exactly
as in the proof of Lemma \ref{lemma-shriek-composition}
we see that the maps in the first displayed diagram use the
maps of Lemma \ref{lemma-shriek-proper-and-open} for the rectangles
$A + B$, $B + C$, $A$, and $C$ as indicated in the diagram in
the statement of the lemma. Since by
Lemmas \ref{lemma-shriek-proper-and-open-compose} and
\ref{lemma-shriek-proper-and-open-compose-horizontal}
we have $\gamma_{A + B} = \gamma_B \circ \gamma_A$ and
$\gamma_{B + C} = \gamma_B \circ \gamma_C$ we conclude
that the desired equality holds provided
$\gamma_A \circ \gamma_C = \gamma_C \circ \gamma_A$.
This is true because the two squares $A$ and $C$ only
intersect in one point (similar to the last argument in
Remark \ref{remark-going-around}).
\end{proof}

\begin{lemma}
\label{lemma-base-change-shriek}
Consider a cartesian square
$$
\xymatrix{
X' \ar[r]_{g'} \ar[d]_{f'} & X \ar[d]^f \\
Y' \ar[r]^g & Y
}
$$
of quasi-compact and quasi-separated schemes with $f$ separated and
of finite type. Then there is a canonical isomorphism
$$
g^{-1} \circ Rf_! \to Rf'_! \circ (g')^{-1}
$$
Moreover, these isomorphisms are compatible with the isomorphisms
of Lemma \ref{lemma-shriek-composition}.
\end{lemma}

\begin{proof}
Choose a compactification $j : X \to \overline{X}$ over $Y$
and denote $\overline{f} : \overline{X} \to Y$ the structure morphism.
Let $j' : X' \to \overline{X}'$ and $\overline{f}' : \overline{X}' \to Y'$
denote the base changes of $j$ and $\overline{f}$.
Since $Rf_! = R\overline{f} \circ j_!$ and $Rf'_! = R\overline{f}' \circ j'_!$
the isomorphism can be constructed via
$$
g^{-1} \circ R\overline{f} \circ j_! \to
R\overline{f}' \circ (\overline{g}')^{-1} \circ j_! \to
R\overline{f}' \circ j'_! \circ (g')^{-1} 
$$
where the first arrow is the isomorphism given to us by the
proper base change theorem (\'Etale Cohomology, Lemma
\ref{etale-cohomology-lemma-proper-base-change} in the bounded below torsion
case and \'Etale Cohomology, Lemma
\ref{etale-cohomology-lemma-proper-base-change-mod-n} in the
case that $\Lambda$ is torsion) and the second arrow is the isomorphism of
Lemma \ref{lemma-base-change-f-shriek-separated}.

\medskip\noindent
To finish the proof we have to show two things: first we have to show
that the isomorphism of functors so obtained does not depend on
the choice of the compactification and second we have to show that
if we vertically stack two base change diagrams as in the lemma, then
these base change isomorphisms are compatible with the isomorphisms
of Lemma \ref{lemma-shriek-composition}.
A straightforward argument which we omit shows that both follow
if we can show that the isomorphisms
\begin{enumerate}
\item $Rg_* \circ Rf_* = R(g \circ f)_*$ for $f : X \to Y$ and $g : Y \to Z$
proper,
\item $g_! \circ f_! = (g \circ f)_!$ for $f : X \to Y$ and $g : Y \to Z$
separated and quasi-finite, and
\item $g_! \circ Rf'_* = Rf_* \circ g'_!$ for  $f : X \to Y$ and
$f' : X' \to Y'$ proper and $g : Y' \to Y$ and $g' : X' \to X$
separated and quasi-finite with $f \circ g' = g \circ f'$
\end{enumerate}
are compatible with base change. This holds for (1) by
Cohomology on Sites, Remark \ref{sites-cohomology-remark-compose-base-change},
for (2) by Remark \ref{remark-f-shriek-base-change-composition}, and
(3) by Lemma \ref{lemma-shriek-proper-and-open-base-change}.
\end{proof}















\section{Properties of derived lower shriek}
\label{section-derived-lower-shriek-properties}

\noindent
Here are some properties of derived lower shriek.

\begin{lemma}
\label{lemma-derived-lower-shriek-commute-direct-sums}
Let $f : X \to Y$ be a finite type separated morphism of quasi-compact
and quasi-separated schemes. Let $\Lambda$ be a ring.
\begin{enumerate}
\item Let $K_i \in D^+_{tors}(X_\etale, \Lambda)$, $i \in I$ be a family
of objects. Assume given $a \in \mathbf{Z}$ such that
$H^n(K_i) = 0$ for $n < a$ and $i \in I$. Then $Rf_!(\bigoplus_i K_i) =
\bigoplus_i Rf_!K_i$.
\item If $\Lambda$ is torsion, then the functor
$Rf_! : D(X_\etale, \Lambda) \to D(Y_\etale, \Lambda)$
commutes with direct sums.
\end{enumerate}
\end{lemma}

\begin{proof}
By construction it suffices to prove this when $f$ is an open immersion
and when $f$ is a proper morphism. For any open immersion $j : U \to X$
of schemes, the functor $j_! : D(U_\etale) \to D(X_\etale)$ is a left
adjoint to pullback $j^{-1} : D(X_\etale) \to D(U_\etale)$
and hence commutes with direct sums, see Cohomology on Sites, Lemma
\ref{sites-cohomology-lemma-adjoint-lower-shriek-restrict}.
In the proper case we have $Rf_! = Rf_*$ and we get the result from
\'Etale Cohomology, Lemma
\ref{etale-cohomology-lemma-direct-sum-bounded-below-pushforward}
in the bounded belo case and from
\'Etale Cohomology, Lemma \ref{etale-cohomology-lemma-proper-mod-n-direct-sums}
in the case that our coefficient ring $\Lambda$ is a torsion ring.
\end{proof}

\begin{lemma}
\label{lemma-derived-lower-shriek-bounded}
Let $f : X \to Y$ be a finite type separated morphism of quasi-compact
and quasi-separated schemes. Let $\Lambda$ be a ring. The functors $Rf_!$
constructed in Section \ref{section-derived-lower-shriek-compactification}
are bounded in the following sense: There exists an integer $N$ such that
for $E \in D^+_{tors}(X_\etale, \Lambda)$ or $E \in D(X_\etale, \Lambda)$
if $\Lambda$ is torsion, we have
\begin{enumerate}
\item $H^i(Rf_!(\tau_{\leq a}E) \to H^i(Rf_!(E))$ is an isomorphism
for $i \leq a$,
\item $H^i(Rf_!(E)) \to H^i(Rf_!(\tau_{\geq b - N}E))$ is an isomorphism
for $i \geq b$,
\item if $H^i(E) = 0$ for $i \not \in [a, b]$ for some
$-\infty \leq a \leq b \leq \infty$, then $H^i(Rf_!(E)) = 0$
for $i \not \in [a, b + N]$.
\end{enumerate}
\end{lemma}

\begin{proof}
Assume $\Lambda$ is torsion and consider the functor
$Rf_! : D(X_\etale, \Lambda) \to D(Y_\etale, \Lambda)$.
By construction it suffices to prove this when $f$ is an open immersion
and when $f$ is a proper morphism. For any open immersion $j : U \to X$
of schemes, the functor $j_! : D(U_\etale) \to D(X_\etale)$ is exact
and hence the statement holds with $N = 0$ in this case.
If $f$ is proper then $Rf_! = Rf_*$, i.e., it is a right derived
functor. Hence the bound on the left by
Derived Categories, Lemma \ref{derived-lemma-negative-vanishing}.
Moreover in this case $f_* : \textit{Mod}(X_\etale, \Lambda)
\to \textit{Mod}(Y_\etale, \Lambda)$ has bounded cohomological dimension by
Morphisms, Lemma \ref{morphisms-lemma-morphism-finite-type-bounded-dimension}
and \'Etale Cohomology, Lemma
\ref{etale-cohomology-lemma-cohomological-dimension-proper}.
Thus we conclude by
Derived Categories, Lemma \ref{derived-lemma-unbounded-right-derived}.

\medskip\noindent
Next, assume $\Lambda$ is arbitrary and let us consider the functor
$Rf_! : D^+_{tors}(X_\etale, \Lambda) \to D^+_{tors}(Y_\etale, \Lambda)$.
Again we immediately reduce to the case where $f$ is proper and
$Rf_! = Rf_*$. Again part (1) is immediate. To show part (3)
we can use induction on $b - a$, the distinguished
triangles of trunctions, and \'Etale Cohomology, Lemma
\ref{etale-cohomology-lemma-cohomological-dimension-proper}.
Part (2) follows from (3). Details omitted.
\end{proof}

\begin{lemma}
\label{lemma-Rf-shriek-for-quasi-finite}
Let $f : X \to Y$ be a quasi-finite separated morphism of quasi-compact and
quasi-separated schemes. Then the functors $Rf_!$ constructed in
Section \ref{section-derived-lower-shriek-compactification}
agree with the restriction of the functor
$f_! : D(X_\etale, \Lambda) \to D(Y_\etale, \Lambda)$
constructed in Section \ref{section-derived-duality-locally-quasi-finite}
to their common domains of definition.
\end{lemma}

\begin{proof}
By Zariski's main theorem (More on Morphisms, Lemma
\ref{more-morphisms-lemma-quasi-finite-separated-pass-through-finite})
we can find an open immersion $j : X \to \overline{X}$ and a finite morphism
$\overline{f} : \overline{X} \to Y$ with $f = \overline{f} \circ j$.
By construction we have $Rf_! = R\overline{f}_* \circ j_!$.
Since $\overline{f}$ is finite, we have $R\overline{f}_* = \overline{f}_*$
by \'Etale Cohomology, Proposition
\ref{etale-cohomology-proposition-finite-higher-direct-image-zero}.
The lemma follows because $\overline{f}_* \circ j_! = f_!$
for example by Lemma \ref{lemma-compactify-f-shriek-separated}.
\end{proof}

\begin{lemma}
\label{lemma-relative-mayer-vietoris}
Let $f : X \to Y$ be a finite type separated morphism of quasi-compact and
quasi-separated schemes. Let $U$ and $V$ be quasi-compact opens of $X$
such that $X = U \cup V$. Denote $a : U \to Y$, $b : V \to Y$ and
$c : U \cap V \to Y$ the restrictions of $f$. Let $\Lambda$ be a ring.
For $K$ in $D^+_{tors}(X_\etale, \Lambda)$ or $K \in D(X_\etale, \Lambda)$
if $\Lambda$ is torsion, we have a distinguished triangle
$$
Rc_!(K|_{U \cap V}) \to
Ra_!(K|_U) \oplus Rb_!(K|_V) \to
Rf_!K \to 
Rc_!(K|_{U \cap V})[1]
$$
in $D(Y_\etale, \Lambda)$.
\end{lemma}

\begin{proof}
This follows from Lemma \ref{lemma-mayer-vietoris-shriek},
the fact that $Rf_! \circ Rj_{U!} = Ra_!$ by
Lemma \ref{lemma-shriek-composition}, and the fact that
$Rj_{U!} = j_{U!}$ by Lemma \ref{lemma-Rf-shriek-for-quasi-finite}.
\end{proof}

\begin{lemma}
\label{lemma-relative-triangle-associated-to-open}
Let $f : X \to Y$ be a finite type separated morphism of quasi-compact and
quasi-separated schemes. Let $U$ be a quasi-compact open of $X$ with
complement $Z \subset X$.  Denote $g : U \to Y$ and $h : Z \to Y$
the restrictions of $f$. Let $\Lambda$ be a ring.
For $K$ in $D^+_{tors}(X_\etale, \Lambda)$ or $K \in D(X_\etale, \Lambda)$
if $\Lambda$ is torsion, we have a distinguished triangle
$$
Rg_!(K|_U) \to
Rf_!K \to
Rh_!(K|_Z) \to
Rg_!(K|_U)[1]
$$
in $D(Y_\etale, \Lambda)$.
\end{lemma}

\begin{proof}
This follows from
Lemma \ref{lemma-triangle-associated-to-open},
the fact that $Rf_! \circ Rj_! = Rg_!$ and $Rf_! \circ Ri_!$ by
Lemma \ref{lemma-shriek-composition}, and the fact that
$Rj_! = j_!$ and $Ri_! = i_! = i_*$ by
Lemma \ref{lemma-Rf-shriek-for-quasi-finite}.
\end{proof}

\begin{lemma}
\label{lemma-shriek-and-thickening}
Let $f' : X' \to Y$ be a finite type separated morphism of quasi-compact and
quasi-separated schemes. Let $i : X \to X'$ be a thickening and
denote $f = f' \circ i$. Let $\Lambda$ be a ring.
For $K'$ in $D^+_{tors}(X'_\etale, \Lambda)$ or $K' \in D(X'_\etale, \Lambda)$
if $\Lambda$ is torsion, we have $Rf_!i^{-1}K' = Rf'_!K'$.
\end{lemma}

\begin{proof}
This is true because $i^{-1}$ and $i_* = i_!$ inverse equivalences
of categories by the topological invariance of the small \'etale topos
(\'Etale Cohomology, Theorem
\ref{etale-cohomology-theorem-topological-invariance})
and we can apply
Lemma \ref{lemma-shriek-composition}.
\end{proof}

\begin{lemma}
\label{lemma-projection-formula-Rf-lower-shriek}
Let $f : X \to Y$ be a separated finite type morphism of quasi-compact
and quasi-separated schemes. Let $\Lambda$ be a torsion ring. Let
$E \in D(X_\etale, \Lambda)$ and $K \in D(Y_\etale, \Lambda)$. Then
$$
Rf_!E \otimes_\Lambda^\mathbf{L} K =
Rf_!(E \otimes_\Lambda^\mathbf{L} f^{-1}K)
$$
in $D(Y_\etale, \Lambda)$.
\end{lemma}

\begin{proof}
Choose $j : X \to \overline{X}$ and $\overline{f} : \overline{X} \to Y$
as in the construction of $Rf_!$. We have
$j_!E \otimes_\Lambda^\mathbf{L} \overline{f}^{-1}K =
j_!(E \otimes_\Lambda^\mathbf{L} f^{-1}K)$ by
Cohomology on Sites, Lemma \ref{sites-cohomology-lemma-j-shriek-and-tensor}.
Then we get the result by applying
\'Etale Cohomology, Lemma
\ref{etale-cohomology-lemma-projection-formula-proper-mod-n}
and using that $f^{-1} = j^{-1} \circ \overline{f}^{-1}$ and
$Rf_! = R\overline{f}_*j_!$.
\end{proof}

\begin{remark}
\label{remark-Rf-lower-shriek-change-of-rings}
Let $\Lambda_1 \to \Lambda_2$ be a homomorphism of torsion rings.
Let $f : X \to Y$ be a separated finite type morphism of quasi-compact
and quasi-separated schemes. The diagram
$$
\xymatrix{
D(X_\etale, \Lambda_2) \ar[r]_{res} \ar[d]_{Rf_!} &
D(X_\etale, \Lambda_1) \ar[d]^{Rf_!} \\
D(Y_\etale, \Lambda_2) \ar[r]^{res} &
D(Y_\etale, \Lambda_1)
}
$$
commutes where $res$ is the ``restriction'' functor which turns a
$\Lambda_2$-module into a $\Lambda_1$-module using the given ring map.
Writing $Rf_! = R\overline{f}_* \circ j_!$ for a factorization
$f = \overline{f} \circ j$ as in
Section \ref{section-derived-lower-shriek-compactification}, we see that
the result holds for $j_!$ by inspection and for $R\overline{f}_*$
by Cohomology on Sites, Lemma
\ref{sites-cohomology-lemma-modules-abelian-unbounded}.
On the other hand, also the diagram
$$
\xymatrix{
D(X_\etale, \Lambda_1)
\ar[r]_{- \otimes_{\Lambda_1}^\mathbf{L} \Lambda_2} \ar[d]_{Rf_!} &
D(X_\etale, \Lambda_2) \ar[d]^{Rf_!} \\
D(Y_\etale, \Lambda_1)
\ar[r]^{- \otimes_{\Lambda_1}^\mathbf{L} \Lambda_2} &
D(Y_\etale, \Lambda_2)
}
$$
is commutative as follows from
Lemma \ref{lemma-projection-formula-Rf-lower-shriek}.
\end{remark}

\begin{remark}
\label{remark-projection-formula-for-internal-hom}
Let $f : X \to Y$ be a separated finite type morphism of quasi-compact
and quasi-separated schemes. Let $\Lambda$ be a torsion coefficient ring
and let $K$ and $L$ be objects of $D(X_\etale, \Lambda)$. We claim
there is a canonical map
$$
\alpha :
Rf_*R\SheafHom_\Lambda(K, L)
\longrightarrow
R\SheafHom_\Lambda(Rf_!K, Rf_!L)
$$
functorial in $K$ and $L$. Namely, choose $j : X \to \overline{X}$ and
$\overline{f} : \overline{X} \to Y$ as in the construction of $Rf_!$.
We first define a map
$$
\beta :
Rj_*R\SheafHom_\Lambda(K, L)
\longrightarrow
R\SheafHom_\Lambda(j_!K, j_!L)
$$
By the construction of internal hom in the derived category, this is
the same thing as defining a map
$$
\beta' :
Rj_*R\SheafHom_\Lambda(K, L) \otimes_\Lambda^\mathbf{L} j_!K
\longrightarrow
j_!L
$$
See Cohomology on Sites, Section \ref{sites-cohomology-section-internal-hom}.
The source of $\beta'$ is equal to
$$
j_!\left(R\SheafHom_\Lambda(K, L) \otimes_\Lambda^\mathbf{L} K\right)
$$
by Cohomology on Sites, Lemma \ref{sites-cohomology-lemma-j-shriek-and-tensor}.
Hence we can set $\beta' = j_!\beta''$ where
$\beta'' : R\SheafHom_\Lambda(K, L) \otimes_\Lambda^\mathbf{L} K \to L$
corresponds to the identity on $R\SheafHom_\Lambda(K, L)$
via the universal property of internal hom mentioned above. By
Cohomology on Sites, Remark
\ref{sites-cohomology-remark-projection-formula-for-internal-hom}
we have a canonical map
$$
\gamma :
R\overline{f}_*R\SheafHom_\Lambda(j_!K, j_!L)
\longrightarrow
R\SheafHom_\Lambda(R\overline{f}_*j_!K, R\overline{f}_*j_!L)
$$
Since $Rf_! = R\overline{f}_*j_!$ and $Rf_* = R\overline{f}_* Rj_*$
(by Leray) we obtain the desired map
$\alpha = \gamma \circ R\overline{f}_*\beta$.
\end{remark}









\section{Derived upper shriek}
\label{section-derived-upper-shriek}

\noindent
We obtain $Rf^!$ by a Brown representability theorem.

\begin{lemma}
\label{lemma-upper-shriek-derived}
Let $f : X \to Y$ be a finite type separated morphism of quasi-compact
and quasi-separated schemes. Let $\Lambda$ be a torsion coefficient ring.
The functor
$Rf_! : D(X_\etale, \Lambda) \to D(Y_\etale, \Lambda)$
has a right adjoint
$Rf^! : D(Y_\etale, \Lambda) \to D(X_\etale, \Lambda)$.
\end{lemma}

\begin{proof}
This follows from
Injectives, Proposition \ref{injectives-proposition-brown}
and Lemma \ref{lemma-derived-lower-shriek-commute-direct-sums} above.
\end{proof}

\begin{lemma}
\label{lemma-Rf-upper-shriek-for-quasi-finite}
Let $f : X \to Y$ be a separated quasi-finite morphism of quasi-compact
and quasi-separated schemes. Let $\Lambda$ be a torsion coefficient ring.
The functor $Rf^! : D(Y_\etale, \Lambda) \to D(X_\etale, \Lambda)$
of Lemma \ref{lemma-upper-shriek-derived} is the same as the functor
$Rf^!$ of Lemma \ref{lemma-lqf-shriek-derived}.
\end{lemma}

\begin{proof}
Follows from uniqueness of adjoints as $Rf_! = f_!$ by
Lemma \ref{lemma-Rf-shriek-for-quasi-finite}.
\end{proof}

\begin{lemma}
\label{lemma-Rf-upper-shriek-for-etale}
Let $j : U \to X$ be a separated \'etale morphism of quasi-compact
and quasi-separated schemes. Let $\Lambda$ be a torsion coefficient ring.
The functor $Rj^! : D(X_\etale, \Lambda) \to D(U_\etale, \Lambda)$
is equal to $j^{-1}$.
\end{lemma}

\begin{proof}
This is true because both $Rj^!$ and $j^{-1}$ are right adjoints to
$Rj_! = j_!$. See for example Lemmas
\ref{lemma-Rf-upper-shriek-for-quasi-finite} and
\ref{lemma-etale-upper-shriek}.
\end{proof}

\begin{lemma}
\label{lemma-twisted-inverse-image-bounded-below}
Let $f : X \to Y$ be a finite type separated morphism of
quasi-compact and quasi-separated schemes. Let $\Lambda$ be a torsion
ring. The functor $Rf^!$ sends $D^+(Y_\etale, \Lambda)$ into
$D^+(X_\etale, \Lambda)$. More precisely, there exists an integer
$N \geq 0$ such that if $K \in D(Y_\etale, \Lambda)$ has $H^i(K) = 0$
for $i < a$ then $H^i(Rf^!K) = 0$ for $i < a - N$.
\end{lemma}

\begin{proof}
Let $N$ be the integer found in Lemma \ref{lemma-derived-lower-shriek-bounded}.
By construction, for $K \in D(Y_\etale, \Lambda)$ and
$L \in \in D(X_\etale, \Lambda)$ we have
$\Hom_X(L, Rf^!K) = \Hom_Y(Rf_!L, K)$. Suppose $H^i(K) = 0$
for $i < a$. Then we take $L = \tau_{\leq a - N - 1}Rf^!K$. By
Lemma \ref{lemma-derived-lower-shriek-bounded}
the complex $Rf_!L$ has vanishing cohomology sheaves in
degrees $\leq a - 1$. Hence $\Hom_Y(Rf_!L, K) = 0$ by
Derived Categories, Lemma \ref{derived-lemma-negative-exts}.
Hence the canonical map $\tau_{\leq a - N - 1}Rf^!K \to Rf^!K$
is zero which implies $H^i(Rf^!K) = 0$ for $i \leq a - N - 1$.
\end{proof}

\noindent
Let $f : X \to Y$ be a separated finite type morphism of quasi-separated
and quasi-compact schemes. Let $\Lambda$ be a torsion coefficient ring.
For every $K \in D(Y_\etale, \Lambda)$ and $L \in D(X_\etale, \Lambda)$
we obtain a canonical map
\begin{equation}
\label{equation-sheafy-trace}
Rf_*R\SheafHom_\Lambda(L, Rf^!K) \longrightarrow R\SheafHom_\Lambda(Rf_!L, K)
\end{equation}
Namely, this map is constructed as the composition
$$
Rf_*R\SheafHom_\Lambda(L, Rf^!K) \to
R\SheafHom_\Lambda(Rf_!L, Rf_!Rf^!K) \to
R\SheafHom_\Lambda(Rf_!L, K)
$$
where the first arrow is
Remark \ref{remark-projection-formula-for-internal-hom}
and the second arrow is the counit $Rf_!Rf^!K \to K$ of the adjunction.

\begin{lemma}
\label{lemma-iso-on-RSheafHom}
Let $f : X \to Y$ be a separated finite type morphism of quasi-compact and
quasi-separated schemes. Let $\Lambda$ be a torsion ring.
For every $K \in D(Y_\etale, \Lambda)$ and $L \in D(X_\etale, \Lambda)$
the map (\ref{equation-sheafy-trace})
$$
Rf_*R\SheafHom_\Lambda(L, Rf^!K) \longrightarrow R\SheafHom_\Lambda(Rf_!L, K)
$$
is an isomorphism.
\end{lemma}

\begin{proof}
To prove the lemma we have to show that for any $M \in D(Y_\etale, \Lambda)$
the map (\ref{equation-sheafy-trace}) induces an bijection
$$
\Hom_Y(M, Rf_*R\SheafHom_\Lambda(L, Rf^!K))
\longrightarrow
\Hom_Y(M, R\SheafHom_\Lambda(Rf_!L, K))
$$
To see this we use the following string of equalities
\begin{align*}
\Hom_Y(M, Rf_*R\SheafHom_\Lambda(L, Rf^!K))
& =
\Hom_X(f^{-1}M, R\SheafHom_\Lambda(L, Rf^!K)) \\
& =
\Hom_X(f^{-1}M \otimes_\Lambda^\mathbf{L} L, Rf^!K) \\
& =
\Hom_Y(Rf_!(f^{-1}M \otimes_\Lambda^\mathbf{L} L), K) \\
& =
\Hom_Y(M \otimes_\Lambda^\mathbf{L} Rf_!L, K) \\
& =
\Hom_Y(M, R\SheafHom_\Lambda(Rf_!L, K))
\end{align*}
The first equality holds by
Cohomology on Sites, Lemma \ref{sites-cohomology-lemma-adjoint}.
The second equality by
Cohomology on Sites, Lemma \ref{sites-cohomology-lemma-internal-hom}.
The third equality by construction of $Rf^!$.
The fourth equality by Lemma \ref{lemma-projection-formula-Rf-lower-shriek}
(this is the important step).
The fifth by
Cohomology on Sites, Lemma \ref{sites-cohomology-lemma-internal-hom}.
\end{proof}

\begin{lemma}
\label{lemma-iso-global-hom}
Let $f : X \to Y$ be a separated finite type morphism of
quasi-separated and quasi-compact schemes. Let $\Lambda$ be a torsion ring.
For every $K \in D(Y_\etale, \Lambda)$ and $L \in D(X_\etale, \Lambda)$ the map
(\ref{equation-sheafy-trace}) induces an isomorphism
$$
R\Hom_X(L, Rf^!K) \longrightarrow R\Hom_Y(Rf_!L, K)
$$
of global derived homs.
\end{lemma}

\begin{proof}
By the construction in
Cohomology on Sites, Section \ref{sites-cohomology-section-global-RHom}
we have
$$
R\Hom_X(L, Rf^!K) =
R\Gamma(X, R\SheafHom_\Lambda(L, Rf^!K)) =
R\Gamma(Y, Rf_*R\SheafHom_\Lambda(L, Rf^!K))
$$
(the second equality by Leray) and
$$
R\Hom_Y(Rf_!L, K) = R\Gamma(Y, R\SheafHom_\Lambda(Rf_!L, K))
$$
Thus the lemma is a consequence of Lemma \ref{lemma-iso-on-RSheafHom}.
\end{proof}

\begin{lemma}
\label{lemma-cartesian-square-Rf-upper-shriek}
Consider a cartesian square
$$
\xymatrix{
X' \ar[r]_{g'} \ar[d]_{f'} & X \ar[d]^f \\
Y' \ar[r]^g & Y
}
$$
of quasi-compact and quasi-separated schemes with $f$ separated and
of finite type. Then we have $Rf^! \circ Rg_* = Rg'_* \circ R(f')^!$.
\end{lemma}

\begin{proof}
By uniqueness of adjoint functors this follows from base change
for derived lower shriek: we have
$g^{-1} \circ Rf_! = Rf'_! \circ (g')^{-1}$ by
Lemma \ref{lemma-base-change-shriek}.
\end{proof}

\begin{remark}
\label{remark-Rf-upper-shriek-change-of-rings}
Let $\Lambda_1 \to \Lambda_2$ be a homomorphism of torsion rings.
Let $f : X \to Y$ be a separated finite type morphism of quasi-compact
and quasi-separated schemes. The diagram
$$
\xymatrix{
D(X_\etale, \Lambda_2) \ar[r]_{res} &
D(X_\etale, \Lambda_1) \\
D(Y_\etale, \Lambda_2) \ar[r]^{res} \ar[u]^{Rf^!} &
D(Y_\etale, \Lambda_1) \ar[u]_{Rf^!}
}
$$
commutes where $res$ is the ``restriction'' functor which turns a
$\Lambda_2$-module into a $\Lambda_1$-module using the given ring map.
This holds by uniquenss of adjoints, the second commutative diagram
of Remark \ref{remark-Rf-lower-shriek-change-of-rings} and because we have
$$
\Hom_{\Lambda_2}(K_1 \otimes_{\Lambda_1}^\mathbf{L} \Lambda_2, K_2) =
\Hom_{\Lambda_1}(K_1, res(K_2))
$$
This equality either for objects living over $X_\etale$ or on
$Y_\etale$ is a very special case of
Cohomology on Sites, Lemma \ref{sites-cohomology-lemma-adjoint}.
\end{remark}











\section{Compactly supported cohomology}
\label{section-compactly-supported-cohomology}

\noindent
Let $k$ be a field. Let $\Lambda$ be a ring. Let $X$ be a separated scheme
of finite type over $k$ with structure morphism $f : X \to \Spec(k)$.
In Section \ref{section-derived-lower-shriek-compactification}
we have defined the functor
$Rf_! : D^+_{tors}(X_\etale, \Lambda) \to D^+_{tors}(\Spec(k), \Lambda)$
and the functor $Rf_! : D(X_\etale, \Lambda) \to D(\Spec(k), \Lambda)$
if $\Lambda$ is a torsion ring. Composing with the global sections functor
on $\Spec(k)$ we obtain what we will call the compactly supported cohomology.

\begin{definition}
\label{definition-cohomology-compact-support}
Let $X$ be a separated scheme of finite type over a field $k$.
Let $\Lambda$ be a ring. Let $K$ be an object of
$D^+_{tors}(X_\etale, \Lambda)$
or of $D(X_\etale, \Lambda)$ in case $\Lambda$ is torsion.
The {\it cohomology of $K$ with compact support} or the
{\it compactly supported cohomology of $K$} is
$$
R\Gamma_c(X, K) = R\Gamma(\Spec(k), Rf_!K)
$$
where $f : X \to \Spec(k)$ is the structure morphism. We will
write  $H^i_c(X, K) = H^i(R\Gamma_c(X, K))$.
\end{definition}

\noindent
We will check that this definition doesn't conflict with
Definition \ref{definition-compact-support} by
Lemma \ref{lemma-compact-support-h0}.
The utility of this definition lies in the following result.

\begin{lemma}
\label{lemma-stalk-R-f-shriek}
Let $f : X \to Y$ be a finite type separated morphism of schemes
with $Y$ quasi-compact and quasi-separated. Let $K$ be an object of
$D^+_{tors}(X_\etale, \Lambda)$ or of $D(X_\etale, \Lambda)$ in case
$\Lambda$ is torsion. Then there is a canonical isomorphism
$$
(Rf_!K)_{\overline{y}}
\longrightarrow
R\Gamma_c(X_{\overline{y}}, K|_{X_{\overline{y}}})
$$
in $D(\Lambda)$ for any geometric point $\overline{y} : \Spec(k) \to Y$.
\end{lemma}

\begin{proof}
Immediate consequence of Lemma \ref{lemma-base-change-shriek} and the
definitions.
\end{proof}

\begin{lemma}
\label{lemma-compact-support-h0}
Let $X$ be a separated scheme of finite type over a field $k$.
If $\mathcal{F}$ is a torsion abelian sheaf, then the abelian group
$H^0_c(X, \mathcal{F})$ defined in Definition \ref{definition-compact-support}
agrees with the abelian group $H^0_c(X, \mathcal{F})$ defined in
Definition \ref{definition-cohomology-compact-support}.
\end{lemma}

\begin{proof}
Choose a compactification $j : X \to \overline{X}$ over $k$.
In both cases the group is defined as $H^0(\overline{X}, j_!\mathcal{F})$.
This is true for the first version by
Lemma \ref{lemma-compactify-compact-support}
and for the second version by construction.
\end{proof}

\begin{lemma}
\label{lemma-finiteness-curves}
Let $k$ be an algebraically closed field. Let $X$ be a separated
scheme of finite type type over $k$ of dimension $\leq 1$.
Let $\Lambda$ be a Noetherian ring.
Let $\mathcal{F}$ be a constructible sheaf of $\Lambda$-modules
on $X$ which is torsion. Then $H^q_c(X, \mathcal{F})$ is a
finite $\Lambda$-module.
\end{lemma}

\begin{proof}
This is a consequence of \'Etale Cohomology, Theorem
\ref{etale-cohomology-theorem-vanishing-affine-curves-coefficients}.
Namely, choose a compactification $j : X \to \overline{X}$.
After replacing $\overline{X}$ by the scheme theoretic closure
of $X$, we see that we may assume $\dim(\overline{X}) \leq 1$.
Then $H^q_c(X, \mathcal{F}) = H^q(\overline{X}, j_!\mathcal{F})$
and the theorem applies.
\end{proof}

\begin{remark}[Covariance of compactly supported cohomology]
\label{remark-covariance-compactly-supported}
Let $k$ be a field. Let $f : X \to Y$ be a morphism of separated
schemes of finite type over $k$. If $X$, $Y$, and $f$ satisfies one
of the following conditions
\begin{enumerate}
\item $f$ is \'etale, or
\item $f$ is flat and quasi-finite, or
\item $f$ is quasi-finite and $Y$ is geometrically unibranch, or
\item there exists a weighting $w : X \to \mathbf{Z}$ of $f$
\end{enumerate}
then compactly supported cohomology is covariant with respect to $f$.
More precisely, let $\Lambda$ be a ring. Let $K$ be an object of
$D^+_{tors}(Y_\etale, \Lambda)$ or of $D(Y_\etale, \Lambda)$ in case
$\Lambda$ is torsion. Under one of the assumptions (1) -- (4)
there is a canonical map
$$
\text{Tr}_{f, w, K} : f_!f^{-1}K \longrightarrow K
$$
See Section \ref{section-weightings} for the existence of the
trace map and Examples \ref{example-trace-for-flat-quasi-finite} and
\ref{example-trace-for-quasi-finite-over-normal} for cases (2) and (3).
If $p : X \to \Spec(k)$ and $q : Y \to \Spec(k)$ denote
the structure morphisms, then we have $Rq_! \circ f_! = Rp_!$
by Lemma \ref{lemma-shriek-composition} and the fact that
$Rf_! = f_!$ for the quasi-finite separated morphism $f$ by
Lemma \ref{lemma-Rf-shriek-for-quasi-finite}. Hence we can look
at the map
\begin{align*}
R\Gamma_c(X, f^{-1}K)
& =
R\Gamma(\Spec(k), Rp_!f^{-1}K) \\
& =
R\Gamma(\Spec(k), Rq_!f_!f^{-1}K) \\
& \xrightarrow{Rq_!\text{Tr}_{f, w, K}}
R\Gamma(\Spec(k), Rq_!K) \\
& =
R\Gamma_c(Y, K)
\end{align*}
In particular, if $\Lambda$ is a torsion ring, then we obtain an arrow
$$
\text{Tr}_f : R\Gamma_c(X, \Lambda) \longrightarrow R\Gamma_c(Y, \Lambda)
$$
This map has lots of additional properties, for example it is
compatible with taking ground field extensions.
\end{remark}










\section{A constructibility result}
\label{section-family-smooth-curves-cohomology}

\noindent
We ``compute'' the cohomology of a smooth projective
family of curves with constant coefficients.

\begin{lemma}
\label{lemma-frobenius-linear-on-vb}
Let $p$ be a prime number. Let $S$ be a scheme over $\mathbf{F}_p$.
Let $\mathcal{E}$ be a finite locally free
$\mathcal{O}_S$-module viewed as an $\mathcal{O}_S$-module on $S_\etale$.
Let $F : \mathcal{E} \to \mathcal{E}$ be a homomorphism of abelian sheaves
on $S_\etale$ such that $F(a e) = a^pF(e)$ for local sections $a$, $e$
of $\mathcal{O}_S$, $\mathcal{E}$ on $S_\etale$. Then
$$
\Coker(F - 1 : \mathcal{E} \to \mathcal{E})
$$
is zero and
$$
\Ker(F - 1 : \mathcal{E} \to \mathcal{E})
$$
is a constructible abelian sheaf on $S_\etale$.
\end{lemma}

\noindent
This lemma is a generalization of \'Etale Cohomology, Lemma
\ref{etale-cohomology-lemma-F-1}.

\begin{proof}
We may assume $S = \Spec(A)$ where $A$ is an $\mathbf{F}_p$-algebra
and that $\mathcal{E}$ is the quasi-coherent module associated
to the free $A$-module $Ae_1 \oplus \ldots \oplus Ae_n$.
We write $F(e_i) = \sum a_{ij} e_j$.

\medskip\noindent
Surjectivity of $F - 1$. It suffices to show that any element
$\sum a_i e_i$, $a_i \in A$ is in the image of $F - 1$
after replacing $A$ by a faithfully flat \'etale extension.
Observe that
$$
F(\sum x_ie_i) - \sum x_i e_i = \sum x_i^p a_{ij} e_j - \sum x_i e_i
$$
Consider the $A$-algebra
$$
A' = A[x_1, \ldots, x_n]/(a_i + x_i - \sum\nolimits_j a_{ji} x_j^p)
$$
A computation shows that $\text{d}x_i$ is zero in $\Omega_{A'/A}$
and hence $\Omega_{A'/A} = 0$. Since $A'$ is of finite type over
$A$, this implies that $\Spec(A') \to \Spec(A)$ is unramified
and hence is quasi-finite. Since $A'$ is generated by $n$ elements
and cut out by $n$ equations, we conclude that $A'$
is a global relative complete intersection over $A$.
Thus $A'$ is flat over $A$ and we conclude that $A \to A'$
is \'etale (as a flat and unramified ring map). Finally,
the reader can show that $A \to A'$ is faithfully flat
by verifying directly that all geometric fibres of
$\Spec(A') \to \Spec(A)$ are nonempty, however this also
follows from \'Etale Cohomology, Lemma
\ref{etale-cohomology-lemma-F-1}.
Finally, the element $\sum x_i e_i \in A'e_1 \oplus \ldots \oplus A'e_n$
maps to $\sum a_i e_i$ by $F - 1$.

\medskip\noindent
Constructibility of the kernel. The calculations above show that
$\Ker(F - 1)$ is represented by the scheme
$$
\Spec(A[x_1, \ldots, x_n]/(x_i - \sum\nolimits_j a_{ji} x_j^p))
$$
over $S = \Spec(A)$. Since this is a scheme affine and \'etale
over $S$ we obtain the result from
\'Etale Cohomology, Lemma \ref{etale-cohomology-lemma-jshriek-constructible}.
\end{proof}

\begin{lemma}
\label{lemma-proper-smooth-family-curves}
Let $f : X \to S$ be a proper smooth morphism of schemes with
geometrically connected fibres of dimension $1$. Let $\ell$
be a prime number. Then $R^qf_*\underline{\mathbf{Z}/\ell\mathbf{Z}}$
is a constructible.
\end{lemma}

\begin{proof}
We may assume $S$ is affine. Say $S = \Spec(A)$. Then, if we write
$A = \bigcup A_i$ as the union of its finite type $\mathbf{Z}$-subalgebras,
we can find an $i$ and a morphism $f_i : X_i \to S_i = \Spec(A_i)$ of finite
type whose base change to $S$ is $f : X \to S$, see
Limits, Lemma \ref{limits-lemma-descend-finite-presentation}.
After increasing $i$ we may assume $f_i : X_i \to S_i$ is
smooth, proper, and of relative dimension $1$, see
Limits, Lemmas
\ref{limits-lemma-eventually-proper}
\ref{limits-lemma-descend-smooth}, and
\ref{limits-lemma-descend-dimension-d}.
By More on Morphisms, Lemma \ref{more-morphisms-lemma-proper-flat-geom-red}
we obtain an open subscheme $U_i \subset S_i$ such that the fibres
of $f_i : X_i \to S_i$ over $U_i$ are geometrically connected.
Then $S \to S_i$ maps into $U_i$. We may replace $X \to S$
by $f_i : f_i^{-1}(U_i) \to U_i$ to reduce to the case
discussed in the next paragraph.

\medskip\noindent
Assume $S$ is Noetherian. We may write $S = U \cup Z$ where $U$ is the
open subscheme defined by the nonvanishing of $\ell$ and
$Z = V(\ell) \subset S$. Since the formation of
$R^qf_*\underline{\mathbf{Z}/\ell\mathbf{Z}}$ commutes with arbtrary
base change (\'Etale Cohomology, Theorem
\ref{etale-cohomology-theorem-proper-base-change}),
it suffices to prove the result over $U$ and over $Z$.
Thus we reduce to the following two cases: (a) $\ell$
is invertible on $S$ and (b) $\ell$ is zero on $S$.

\medskip\noindent
Case (a). We claim that in this case the sheaves
$R^qf_*\underline{\mathbf{Z}/\ell\mathbf{Z}}$ are
finite locally constant on $S$. First, by proper base change
(in the form of \'Etale Cohomology, Lemma
\ref{etale-cohomology-lemma-proper-base-change-stalk})
and by finiteness 
(\'Etale Cohomology, Theorem
\ref{etale-cohomology-theorem-vanishing-affine-curves})
we see that the stalks of $R^qf_*\underline{\mathbf{Z}/\ell\mathbf{Z}}$
are finite. By \'Etale Cohomology, Lemma
\ref{etale-cohomology-lemma-sp-isom-proper-loc-cst-torsion}
all specialization maps are isomorphisms.
We conclude the claim holds by
\'Etale Cohomology, Lemma
\ref{etale-cohomology-lemma-characterize-locally-constant-module}.

\medskip\noindent
Case (b). Here $\ell = p$ is a prime and $S$ is a scheme over
$\Spec(\mathbf{F}_p)$. By the same references as above we already
know that the stalks of $R^qf_*\underline{\mathbf{Z}/p\mathbf{Z}}$
are finite and zero for $q \geq 2$. It follows from
\'Etale Cohomology, Lemma \ref{etale-cohomology-lemma-sections-upstairs}
that $f_*\underline{\mathbf{Z}/p\mathbf{Z}} =
\underline{\mathbf{Z}/p\mathbf{Z}}$. It remains to prove
that $R^1f_*\underline{\mathbf{Z}/p\mathbf{Z}}$ is constructible.
Consider the Artin-Schreyer sequence
$$
0 \to \underline{\mathbf{Z}/p\mathbf{Z}} \to \mathcal{O}_X
\xrightarrow{F - 1} \mathcal{O}_X
\to 0
$$
See \'Etale Cohomology, Section \ref{etale-cohomology-section-artin-schreier}.
Recall that $f_*\mathcal{O}_X = \mathcal{O}_S$ and
$R^1f_*\mathcal{O}_X$ is a finite locally free $\mathcal{O}_S$-module
of rank equal to the genera of the fibres of $X \to S$, see
Algebraic Curves, Lemma \ref{curves-lemma-genus-in-nodal-family-of-curves}.
We conclude that we have a short exact sequence
$$
0 \to
\Coker(F - 1 : \mathcal{O}_S \to \mathcal{O}_S)
\to R^1f_*\underline{\mathbf{Z}/p\mathbf{Z}} \to
\Ker(F - 1 : R^1f_*\mathcal{O}_X \to R^1f_*\mathcal{O}_X)
\to 0
$$
Applying Lemma \ref{lemma-frobenius-linear-on-vb} we win.
\end{proof}

\begin{lemma}
\label{lemma-proper-smooth-family-curves-modules}
Let $f : X \to S$ be a proper smooth morphism of schemes with
geometrically connected fibres of dimension $1$. Let $\Lambda$
be a Noetherian ring. Let $M$ be a finite $\Lambda$-module
annihilated by an integer $n > 0$. Then $R^qf_*\underline{M}$
is a constructible sheaf of $\Lambda$-modules on $S$.
\end{lemma}

\begin{proof}
If $n = \ell n'$ for some prime number $\ell$, then
we get a short exact sequence $0 \to M[\ell] \to M \to M' \to 0$
of finite $\Lambda$-modules and $M'$ is annihilated by $n'$.
This produces a corresponding short exact sequence of constant
sheaves, which in turn gives rise to an exact sequence
$$
R^{q - 1}f_*\underline{M'} \to
R^qf_*\underline{M[n]} \to
R^qf_*\underline{M} \to
R^qf_*\underline{M'} \to
R^{q + 1}f_*\underline{M[n]}
$$
Thus, if we can show the result in case $M$ is annihilated by
a prime number, then by induction on $n$ we win by
\'Etale Cohomology, Lemma \ref{etale-cohomology-lemma-constructible-abelian}.

\medskip\noindent
Let $\ell$ be a prime number such that $\ell$ annihilates $M$.
Then we can replace $\Lambda$ by the $\mathbf{F}_\ell$-algebra
$\Lambda/\ell \Lambda$. Namely, the sheaf $R^qf_*\underline{M}$
where $\underline{M}$ is viewed as a sheaf of $\Lambda$-modules
is the same as the sheaf $R^qf_*\underline{M}$ computed by
viewing $\underline{M}$ as a sheaf of $\Lambda/\ell \Lambda$-modules, see
Cohomology on Sites, Lemma
\ref{sites-cohomology-lemma-modules-abelian-unbounded}.

\medskip\noindent
Assume $\ell$ be a prime number such that $\ell$ annihilates $M$ and $\Lambda$.
Let us reduce to the case where $M$ is a finite free $\Lambda$-module.
Namely, choose a resolution
$$
\ldots \to
\Lambda^{\oplus m_2} \to
\Lambda^{\oplus m_1} \to
\Lambda^{\oplus m_0} \to
M \to 0
$$
Recall that $f_*$ has finite cohomological dimension on sheaves of
$\Lambda$-modules, see
\'Etale Cohomology, Lemma
\ref{etale-cohomology-lemma-cohomological-dimension-proper}
and
Derived Categories, Lemma \ref{derived-lemma-unbounded-right-derived}.
Thus we see that $R^qf_*\underline{M}$ is the $q$th cohomology
sheaf of the object
$$
Rf_*(\underline{\Lambda^{\oplus m_a}} \to \ldots \to
\underline{\Lambda^{\oplus m_0}})
$$
in $D(S_\etale, \Lambda)$ for some integer $a$ large enough.
Using the first spectral sequence of
Derived Categories, Lemma \ref{derived-lemma-two-ss-complex-functor}
(or alternatively using an argument with truncations)
we conclude that it suffices to prove that $R^qf_*\underline{\Lambda})$
is constructible.

\medskip\noindent
At this point we can finally use that
$$
(Rf_*\underline{\mathbf{Z}/\ell\mathbf{Z}})
\otimes_{\mathbf{Z}/\ell\mathbf{Z}}^\mathbf{L} \underline{\Lambda} =
Rf_*\underline{\Lambda}
$$
by \'Etale Cohomology, Lemma
\ref{etale-cohomology-lemma-projection-formula-proper-mod-n}.
Since any module over the field $\mathbf{Z}/\ell\mathbf{Z}$
is flat we obtain
$$
(R^qf_*\underline{\mathbf{Z}/\ell\mathbf{Z}})
\otimes_{\mathbf{Z}/\ell\mathbf{Z}} \underline{\Lambda} =
R^qf_*\underline{\Lambda}
$$
Hence it suffices to prove the result for
$R^qf_*\underline{\mathbf{Z}/\ell\mathbf{Z}}$
by \'Etale Cohomology, Lemma \ref{etale-cohomology-lemma-tensor-constructible}.
This case is Lemma \ref{lemma-proper-smooth-family-curves}.
\end{proof}






\section{Complexes with constructible cohomology}
\label{section-Dc}

\noindent
We continue the discussion started in \'Etale Cohomology, Section
\ref{etale-cohomology-section-Dc}. In particular, for a scheme
$X$ and a Noetherian ring $\Lambda$ we denote
$D_c(X_\etale, \Lambda)$ the strictly full saturated triangulated
subcategory of $D(X_\etale, \Lambda)$ consisting of objects
whose cohomology sheaves are constructible sheaves of $\Lambda$-modules.

\begin{lemma}
\label{lemma-qf-f-shriek-constructible}
Let $f : X \to Y$ be a morphism of schemes which is
locally quasi-finite and of finite presentation.
The functor $f_! : D(X_\etale, \Lambda) \to D(Y_\etale, \Lambda)$
of Lemma \ref{lemma-lqf-shriek-derived}
sends $D_c(X_\etale, \Lambda)$ into $D_c(Y_\etale, \Lambda)$.
\end{lemma}

\begin{proof}
Since the functor $f_!$ is exact, it suffices to show that
$f_!\mathcal{F}$ is constructible for any constructible sheaf
$\mathcal{F}$ of $\Lambda$-modules on $X_\etale$.
The question is local on $Y$ and hence
we may and do assume $Y$ is affine.
Then $X$ is quasi-compact and quasi-separated, see
Morphisms, Definition \ref{morphisms-definition-finite-presentation}.
Say $X = \bigcup_{i = 1, \ldots, n} X_i$ is a finite affine open
covering. By Lemma \ref{lemma-lqf-colimit-f-shriek}
we see that it suffices to show that
$f_{i, !}\mathcal{F}|_{X_i}$ and $f_{ii', !}\mathcal{F}|_{X_i \cap X_{i'}}$
are constructible where $f_i : X_i \to Y$ and
$f_{ii'} : X_i \cap X_{i'} \to Y$ are the restrictions of $f$.
Since $X_i$ and $X_i \cap X_{i'}$ are quasi-compact and separated
this means we may assume $f$ is separated.
By Zariski's main theorem (in the form of
More on Morphisms, Lemma
\ref{more-morphisms-lemma-quasi-finite-separated-pass-through-finite-addendum})
we can choose a factorization $f = g \circ j$ where
$j : X \to X'$ is an open immersion and $g : X' \to Y$ is
finite and of finite presentation. Then $f_! = g_! \circ j_!$ by
Lemma \ref{lemma-f-shriek-composition}.
By \'Etale Cohomology, Lemma \ref{etale-cohomology-lemma-jshriek-constructible}
we see that $j_!\mathcal{F}$ is constructible on $X'$.
The morphism $g$ is finite hence $g_! = g_*$
by Lemma \ref{lemma-proper-f-shriek}.
Thus $f_!\mathcal{F} = g_!j_!\mathcal{F} = g_*j_!\mathcal{F}$
is constructible by
\'Etale Cohomology, Lemma
\ref{etale-cohomology-lemma-finite-pushforward-constructible}.
\end{proof}

\begin{lemma}
\label{lemma-constant-shriek-rel-dim-1}
Let $S$ be a Noetherian affine scheme of finite dimension.
Let $f : X \to S$ be a separated, affine, smooth morphism
of relative dimension $1$. Let $\Lambda$ be a Noetherian ring
which is torsion. Let $M$ be a finite $\Lambda$-module.
Then $Rf_!\underline{M}$ has constructible cohomology sheaves.
\end{lemma}

\begin{proof}
We will prove the result by induction on $d = \dim(S)$.

\medskip\noindent
Base case. If $d = 0$, then the only thing to show is that the stalks
of $R^qf_!\underline{M}$ are finite $\Lambda$-modules.
If $\overline{s}$ is a geometric point of $S$, then we have
$(R^qf_!\underline{M})_{\overline{s}} = H^q_c(X_{\overline{s}}, \underline{M})$
by Lemma \ref{lemma-stalk-R-f-shriek}.
This is a finite $\Lambda$-module by Lemma \ref{lemma-finiteness-curves}.

\medskip\noindent
Induction step. It suffices to find a dense open $U \subset S$ such
that $Rf_!\underline{M}|_U$ has constructible cohomology sheaves.
Namely, the restriction of $Rf_!\underline{M}$ to the complement
$S \setminus U$ will have constructible cohomology sheaves by induction
and the fact that formation of $Rf_!\underline{M}$ commutes
with all base change (Lemma \ref{lemma-base-change-shriek}).
In fact, let $\eta \in S$ be a generic point of an irreducible
component of $S$. Then it suffices to find an open neighbourhood $U$
of $\eta$ such that the restriction of $Rf_!\underline{M}$
to $U$ is constructible. This is what we will do in the next paragraph.

\medskip\noindent
Given a generic point $\eta \in S$ we choose a diagram
$$
\xymatrix{
\overline{Y}_1 \amalg \ldots \amalg \overline{Y}_n \ar[rd] &
Y_1 \amalg \ldots \amalg Y_n \ar[r]_-\nu \ar[d] \ar[l]^j &
X_V \ar[r] \ar[d] &
X_U \ar[r] \ar[d] &
X \ar[d]^f \\
& T_1 \amalg \ldots \amalg T_n \ar[r] &
V \ar[r] &
U \ar[r] &
S
}
$$
as in More on Morphisms, Lemma \ref{more-morphisms-lemma-make-good-curves}.
We will show that $Rf_!\underline{M}|_U$ is constructible.
First, since $V \to U$ is finite and surjective, it suffices
to show that the pullback to $V$ is constructible, see
\'Etale Cohomology, Lemma \ref{etale-cohomology-lemma-check-constructible}.
Since formation of $Rf_!$ commutes with base change, we
see that it suffices to show that
$R(X_V \to V)_!\underline{M}$ is constructible.
Let $W \subset X_V$ be the open subscheme given to us by
More on Morphisms, Lemma \ref{more-morphisms-lemma-make-good-curves} part (4).
Let $Z \subset X_V$ be the reduced induced scheme structure on
the complement of $W$ in $X_V$. Then the fibres of $Z \to V$
have dimension $0$ (as $W$ is dense in the fibres) and hence
$Z \to V$ is quasi-finite. From the distinguished triangle
$$
R(W \to V)_!\underline{M} \to
R(X_V \to V)_!\underline{M} \to
R(Z \to V)_!\underline{M} \to
\ldots
$$
of Lemma \ref{lemma-relative-triangle-associated-to-open} and
from Lemma \ref{lemma-qf-f-shriek-constructible}
we conclude that it suffices to show that
$R(W \to V)_!\underline{M}$ has constructible cohomology sheaves.
Next, we have
$$
R(W \to V)_!\underline{M} = R(\nu^{-1}(W) \to V)_!\underline{M}
$$
because the morphism $\nu : \nu^{-1}(W) \to W$ is a thickening
and we may apply Lemma \ref{lemma-shriek-and-thickening}.
Next, we let $Z' \subset \coprod \overline{Y}_i$ denote the
complement of the open $j(\nu^{-1}(W))$. Again $Z' \to V$ is quasi-finite.
Again use the distinguished triangle
$$
R(\nu^{-1}(W) \to V)_!\underline{M} \to
R(\coprod \overline{Y}_i \to V)_!\underline{M} \to
R(Z' \to V)_!\underline{M} \to
\ldots
$$
to conclude that it suffices to prove
$$
R(\coprod \overline{Y}_i \to V)_!\underline{M} =
\bigoplus\nolimits_i R(\overline{Y}_i \to V)_!\underline{M} =
\bigoplus\nolimits_i R(T_i \to V)_!R(\overline{Y}_i \to T_i)_!\underline{M}
$$
has constructible cohomology sheaves (second equality by
Lemma \ref{lemma-shriek-composition}).
The result for $R(\overline{Y}_i \to T_i)_!\underline{M}$
is Lemma \ref{lemma-proper-smooth-family-curves-modules}
and we win because $T_i \to V$ is finite \'etale
and we can apply Lemma \ref{lemma-qf-f-shriek-constructible}.
\end{proof}

\begin{lemma}
\label{lemma-loc-constant-shriek-rel-dim-1}
Let $Y$ be a Noetherian affine scheme of finite dimension.
Let $\Lambda$ be a Noetherian ring which is torsion.
Let $\mathcal{F}$ be a finite type, locally constant sheaf of
$\Lambda$-modules on an open subscheme $U \subset \mathbf{A}^1_Y$.
Then $Rf_!\mathcal{F}$ has constructible cohomology sheaves where
$f : U \to Y$ is the structure morphism.
\end{lemma}

\begin{proof}
We may decompose $\Lambda$ as a product
$\Lambda = \Lambda_1 \times \ldots \times \Lambda_r$
where $\Lambda_i$ is $\ell_i$-primary for some prime $\ell_i$.
Thus we may assume there exists a prime $\ell$ and an integer $n > 0$
such that $\ell^n$ annihilates $\Lambda$ (and hence $\mathcal{F}$).

\medskip\noindent
Since $U$ is Noetherian, we see that $U$ has finitely many connected
components. Thus we may assume $U$ is connected.
Let $g : U' \to U$ be the finite \'etale covering
constructed in \'Etale Cohomology, Lemma
\ref{etale-cohomology-lemma-pullback-filtered-modules}.
The discussion in
\'Etale Cohomology, Section \ref{etale-cohomology-section-trace-method}
gives maps
$$
\mathcal{F} \to g_*g^{-1}\mathcal{F} \to \mathcal{F}
$$
whose composition is an isomorphism. Hence it suffices to
prove the result for $g_*g^{-1}\mathcal{F}$. On the other hand,
we have $Rf_!g_*g^{-1}\mathcal{F} = R(f \circ g)_!g^{-1}\mathcal{F}$
by Lemma \ref{lemma-shriek-composition}.
Since $g^{-1}\mathcal{F}$ has a finite filtration by
constant sheaves of $\Lambda$-modules of the form
$\underline{M}$ for some finite $\Lambda$-module $M$
(by our choice of $g$) this reduces us to the case proved in
Lemma \ref{lemma-constant-shriek-rel-dim-1}.
\end{proof}

\begin{lemma}
\label{lemma-constructible-shriek-rel-dim-1}
Let $Y$ be an affine scheme. Let $\Lambda$ be a Noetherian ring.
Let $\mathcal{F}$ be a constructible sheaf of $\Lambda$-modules on
$\mathbf{A}^1_Y$ which is torsion. Then $Rf_!\mathcal{F}$ has constructible
cohomology sheaves where $f : \mathbf{A}^1_Y \to Y$ is the structure morphism.
\end{lemma}

\begin{proof}
Say $\mathcal{F}$ is annihilated by $n > 0$. Then we can replace
$\Lambda$ by $\Lambda/n\Lambda$ without changing $Rf_!\mathcal{F}$.
Thus we may and do assume $\Lambda$ is a torsion ring.

\medskip\noindent
Say $Y = \Spec(R)$. Then, if we write
$R = \bigcup R_i$ as the union of its finite type $\mathbf{Z}$-subalgebras,
we can find an $i$ such that $\mathcal{F}$ is the pullback of
a constructible sheaf of $\Lambda$-modules on $\mathbf{A}^1_{R_i}$, see
\'Etale Cohomology, Lemma \ref{etale-cohomology-lemma-category-is-colimit}.
Hence we may assume $Y$ is a Noetherian scheme of finite dimension.

\medskip\noindent
Assume $Y$ is a Noetherian scheme of finite dimension $d = \dim(Y)$
and $\Lambda$ is torsion. We will prove the result by induction on $d$.

\medskip\noindent
Base case. If $d = 0$, then the only thing to show is that the stalks
of $R^qf_!\mathcal{F}$ are finite $\Lambda$-modules.
If $\overline{y}$ is a geometric point of $Y$, then we have
$(R^qf_!\mathcal{F})_{\overline{y}} = H^q_c(X_{\overline{y}}, \mathcal{F})$
by Lemma \ref{lemma-stalk-R-f-shriek}.
This is a finite $\Lambda$-module by Lemma \ref{lemma-finiteness-curves}.

\medskip\noindent
Induction step. It suffices to find a dense open $V \subset Y$ such
that $Rf_!\mathcal{F}|_V$ has constructible cohomology sheaves. Namely, the
restriction of $Rf_!\mathcal{F}$ to the complement $Y \setminus V$
will have constructible cohomology sheaves by induction
and the fact that formation of $Rf_!\mathcal{F}$ commutes
with all base change (Lemma \ref{lemma-base-change-shriek}).
By definition of constructible sheaves of $\Lambda$-modules, there is a dense
open subscheme $U \subset \mathbf{A}^1_Y$ such that $\mathcal{F}|_U$
is a finite type, locally constant sheaf of $\Lambda$-modules.
Denote $Z \subset \mathbf{A}^1_Y$ the complement (viewed as a reduced
closed subscheme). Note that $U$ contains all the generic points of the
fibres of $\mathbf{A}^1_Y \to Y$ over the generic points
$\xi_1, \ldots, \xi_n$ of the irreducible components of $Y$.
Hence $Z \to Y$ has finite fibres over $\xi_1, \ldots, \xi_n$.
After replacing $Y$ by a dense open (which is allowed), we may
assume $Z \to Y$ is finite, see
Morphisms, Lemma \ref{morphisms-lemma-generically-finite}.
By the distinguished triangle of
Lemma \ref{lemma-relative-triangle-associated-to-open}
and the result for $Z \to Y$ (Lemma \ref{lemma-qf-f-shriek-constructible})
we reduce to showing that
$R(U \to Y)_!\mathcal{F}$ has constructible cohomology sheaves.
This is Lemma \ref{lemma-loc-constant-shriek-rel-dim-1}.
\end{proof}

\begin{theorem}
\label{theorem-constructible-shriek}
Let $f : X \to Y$ be a separated morphism of finite presentation
of quasi-compact and quasi-separated schemes. Let $\Lambda$ be a
Noetherian ring. Let $K$ be an object of $D^+_{tors, c}(X_\etale, \Lambda)$
or of $D_c(X_\etale, \Lambda)$ in case $\Lambda$ is torsion.
Then $Rf_!K$ has constructible cohomology sheaves, i.e.,
$Rf_!K$ is in $D^+_{tors, c}(Y_\etale, \Lambda)$
or in $D_c(Y_\etale, \Lambda)$ in case $\Lambda$ is torsion.
\end{theorem}

\begin{proof}
The question is local on $Y$ hence we may and do assume $Y$ is affine.
By the induction principle and Lemma \ref{lemma-relative-mayer-vietoris}
we reduce to the case where $X$ is also affine.

\medskip\noindent
Assume $X$ and $Y$ are affine. Since $X$ is of finite presentation,
we can choose a closed immersion $i : X \to \mathbf{A}^n_Y$ which is
of finite presentation. If $p : \mathbf{A}^n_Y \to Y$ denotes
the structure morphism, then we see that $Rf_! = Rp_! \circ Ri_!$
by Lemma \ref{lemma-shriek-composition}. By
Lemma \ref{lemma-qf-f-shriek-constructible}
we have the result for $Ri_! = i_!$. Hence we may assume $f$
is the projection morphism $\mathbf{A}^n_Y \to Y$.
Since we can view $f$ as the composition
$$
X = \mathbf{A}^n_Y \to \mathbf{A}^{n - 1}_Y \to
\mathbf{A}^{n - 2}_S \to \ldots \to
\mathbf{A}^1_Y \to Y
$$
we may assume $n = 1$.

\medskip\noindent
Assume $Y$ is affine and $X = \mathbf{A}^1_Y$. Since $Rf_!$ has
finite cohomological dimension
(Lemma \ref{lemma-derived-lower-shriek-bounded})
we may assume $K$ is bounded below. Using the first spectral sequence of
Derived Categories, Lemma \ref{derived-lemma-two-ss-complex-functor}
(or alternatively using an argument with truncations),
we reduce to showing the result of
Lemma \ref{lemma-constructible-shriek-rel-dim-1}.
\end{proof}








\section{Applications}
\label{section-applications}

\noindent
In this section we give some applications of
Theorem \ref{theorem-constructible-shriek}.

\begin{lemma}
\label{lemma-finiteness-compactly-supported}
Let $k$ be an algebraically closed field.
Let $X$ be a finite type separated scheme over $k$.
Let $\Lambda$ be a Noetherian ring. Let $K$ be an object of
$D^+_{tors, c}(X_\etale, \Lambda)$ or of $D_c(X_\etale, \Lambda)$
in case $\Lambda$ is torsion. Then
$H^i_c(X, K)$ is a finite $\Lambda$-module for all
$i \in \mathbf{Z}$.
\end{lemma}

\begin{proof}
Immediate consequence of Theorem \ref{theorem-constructible-shriek}
and the definition of compactly supported cohomology in
Section \ref{section-compactly-supported-cohomology}.
\end{proof}

\begin{proposition}
\label{proposition-loc-cst-torsion}
Let $f : X \to S$ be a smooth proper morphism of schemes.
Let $\Lambda$ be a Noetherian ring. Let $\mathcal{F}$ be a
finite type, locally constant sheaf of $\Lambda$-modules
on $X_\etale$ such that for every geometric point $\overline{x}$ of $X$
the stalk $\mathcal{F}_{\overline{x}}$ is annihilated by an integer
$n > 0$ prime to the residue characteristic of $\overline{x}$.
Then $R^if_*\mathcal{F}$ is a finite type, locally constant sheaf of
$\Lambda$-modules on $S_\etale$ for all $i \in \mathbf{Z}$.
\end{proposition}

\begin{proof}
The question is local on $S$ and hence we may assume $S$ is affine.
For a point $x$ of $X$ denote $n_x \geq 1$
the smallest integer annihilating $\mathcal{F}_{\overline{x}}$
for some (equivalently any) geometric point $\overline{x}$ of $X$
lying over $x$. Since $X$ is quasi-compact (being proper over affine)
there exists a finite \'etale covering $\{U_j \to X\}_{j = 1, \ldots, m}$
such that $\mathcal{F}|_{U_j}$ is constant. Since $U_j \to X$
is open, we conclude that the function $x \mapsto n_x$ is locally
constant and takes finitely many values. Accordingly we
obtain a finite decomposition $X = X_1 \amalg \ldots \amalg X_N$
into open and closed subschemes such that $n_x = n$ if and only if $x \in X_n$.
Then it suffices to prove the lemma for the induced
morphisms $X_n \to S$ and the restriction of $\mathcal{F}$ to $X_n$.
Thus we may and do assume there exists an integer $n > 0$ such
that $\mathcal{F}$ is annihilated by $n$ and such that $n$
is prime to the residue characteristics of all residue fields of $X$.

\medskip\noindent
Since $f$ is smooth and proper the image $f(X) \subset S$ is open and closed.
Hence we may replace $S$ by $f(X)$ and assume $f(X) = S$. In particular,
we see that we may assume $n$ is invertible in the ring defining the
affine scheme $S$.

\medskip\noindent
In this paragraph we reduce to the case where $S$ is Noetherian.
Write $S = \Spec(A)$ for some $\mathbf{Z}[1/n]$-algebra $A$.
Write $A = \bigcup A_i$ as the union of its finite type
$\mathbf{Z}[1/n]$-subalgebras. We can find an $i$ and a morphism
$f_i : X_i \to S_i = \Spec(A_i)$ of finite
type whose base change to $S$ is $f : X \to S$, see
Limits, Lemma \ref{limits-lemma-descend-finite-presentation}.
After increasing $i$ we may assume $f_i : X_i \to S_i$ is
smooth and proper, see Limits, Lemmas
\ref{limits-lemma-eventually-proper}
\ref{limits-lemma-descend-smooth}, and
\ref{limits-lemma-descend-dimension-d}.
By \'Etale Cohomology, Lemma
\ref{etale-cohomology-lemma-category-loc-cst-is-colimit}
we see that there exists an $i$ and a finite type,
locally constant sheaf of $\Lambda$-modules
$\mathcal{F}_i$ whose pullback to $X$ is isomorphic to $\mathcal{F}$.
As $\mathcal{F}$ is annihilated by $n$, we may replace
$\mathcal{F}_i$ by $\Ker(n : \mathcal{F}_i \to \mathcal{F}_i)$
and assume the same thing is true for $\mathcal{F}_i$.
This reduces us to the case discussed in the next paragraph.

\medskip\noindent
Assume we have an integer $n \geq 1$, the base scheme $S$ is Noetherian
and lives over $\mathbf{Z}[1/n]$, and $\mathcal{F}$ is $n$-torsion.
By Theorem \ref{theorem-constructible-shriek}
the sheaves $R^if_*\mathcal{F}$ are constructible sheaves of $\Lambda$-modules.
By \'Etale Cohomology, Lemma
\ref{etale-cohomology-lemma-sp-isom-proper-torsion-loc-ac}
the specialization maps of $R^if_*\mathcal{F}$ are always isomorphisms.
We conclude by \'Etale Cohomology, Lemma
\ref{etale-cohomology-lemma-characterize-locally-constant-module}.
\end{proof}







\section{More on derived upper shriek}
\label{section-more-derived-upper-shriek}

\noindent
Let $\Lambda$ be a torsion ring. Consider a commutative diagram
$$
\xymatrix{
U \ar[rr]_j \ar[rd]_g & & U' \ar[ld]^{g'} \\
& Y
}
$$
of quasi-compact and quasi-separated schemes with $g$ and $g'$
separated and of finite type and with $j$ \'etale. This induces
a canonical map
$$
Rg_!\Lambda \longrightarrow Rg'_!\Lambda
$$
in $D(Y_\etale, \Lambda)$. Namely, by Lemmas \ref{lemma-shriek-composition}
and \ref{lemma-Rf-shriek-for-quasi-finite} we have $Rg_! = Rg'_! \circ j_!$.
On the other hand, since $j_!$ is left adjoint to $j^{-1}$ we have
the counit $\text{Tr}_j : j_!\Lambda = j_!j^{-1}\Lambda \to \Lambda$;
we also call this the trace map for $j$, see
Remark \ref{remark-trace-etale-counit}.
The map above is constructed as the composition
$$
Rg_!\Lambda = Rg'_!j_!\Lambda
\xrightarrow{Rg'_! \text{Tr}_j}
Rg'_!\Lambda
$$
Given a second \'etale morphism $j' : U' \to U''$
for some $g'' : U'' \o Y$ separated and of finite type
the composition
$$
Rg_!\Lambda \longrightarrow Rg'_!\Lambda \longrightarrow Rg''_!\Lambda
$$
of the maps for $j$ and $j'$ is equal to the map
$Rg_!\Lambda \longrightarrow Rg''_!\Lambda$ constructed for $j' \circ j$.
This follows from the corresponding statement on trace maps,
see Lemma \ref{lemma-trace-composition} for a more general case.

\medskip\noindent
Let $f : X \to Y$ be a separated finite type morphism of quasi-compact
and quasi-separated schemes. Then we obtain a functor
$$
X_{affine, \etale}
\longrightarrow
\left\{
\begin{matrix}
\text{schemes separated of finite type over }Y\\
\text{with \'etale morphisms between them}
\end{matrix}
\right\}
$$
Thus the construction above determines a functor
$X_{affine, \etale}^{opp} \to D(Y_\etale, \Lambda)$
sending $U$ to $R(U \to Y)_!\Lambda$.

\begin{lemma}
\label{lemma-describe-Rf-upper-shriek}
Let $f : X \to Y$ be a separated finite type morphism of quasi-compact
and quasi-separated schemes. Let $\Lambda$ be a torsion ring.
Let $K \in D(Y_\etale, \Lambda)$. For $n \in \mathbf{Z}$ the
cohomology sheaf $H^n(Rf^!K)$ restricted to $X_{affine, \etale}$
is the sheaf associated to the presheaf
$$
U \longmapsto \Hom_Y(R(U \to Y)_!\Lambda, K[n])
$$
See discussion above for the functorial nature of $R(U \to Y)_!\Lambda$.
\end{lemma}

\begin{proof}
Let $j : U \to X$ be an object of $X_{affine, \etale}$ and set $g = f \circ j$.
Recall that $\Hom_X(j_!\Lambda, M[n]) = H^n(U, M)$ for any $M$ in
$D(X_\etale, \Lambda)$. Then $H^n(Rf^!K)$ is the sheaf associated to the
presheaf
$$
U \mapsto H^n(U, Rf^!K) = \Hom_X(j_!\Lambda, Rf^!K[n]) =
\Hom_Y(Rf_!j_!\Lambda, K[n] = \Hom_Y(Rg_!\Lambda, K[n])
$$
We omit the verification that the transition maps are given by the transition
maps between the objects $Rg_!\Lambda = R(U \to Y)_!\Lambda$ we constructed
above.
\end{proof}
















\begin{multicols}{2}[\section{Other chapters}]
\noindent
Preliminaries
\begin{enumerate}
\item \hyperref[introduction-section-phantom]{Introduction}
\item \hyperref[conventions-section-phantom]{Conventions}
\item \hyperref[sets-section-phantom]{Set Theory}
\item \hyperref[categories-section-phantom]{Categories}
\item \hyperref[topology-section-phantom]{Topology}
\item \hyperref[sheaves-section-phantom]{Sheaves on Spaces}
\item \hyperref[sites-section-phantom]{Sites and Sheaves}
\item \hyperref[stacks-section-phantom]{Stacks}
\item \hyperref[fields-section-phantom]{Fields}
\item \hyperref[algebra-section-phantom]{Commutative Algebra}
\item \hyperref[brauer-section-phantom]{Brauer Groups}
\item \hyperref[homology-section-phantom]{Homological Algebra}
\item \hyperref[derived-section-phantom]{Derived Categories}
\item \hyperref[simplicial-section-phantom]{Simplicial Methods}
\item \hyperref[more-algebra-section-phantom]{More on Algebra}
\item \hyperref[smoothing-section-phantom]{Smoothing Ring Maps}
\item \hyperref[modules-section-phantom]{Sheaves of Modules}
\item \hyperref[sites-modules-section-phantom]{Modules on Sites}
\item \hyperref[injectives-section-phantom]{Injectives}
\item \hyperref[cohomology-section-phantom]{Cohomology of Sheaves}
\item \hyperref[sites-cohomology-section-phantom]{Cohomology on Sites}
\item \hyperref[dga-section-phantom]{Differential Graded Algebra}
\item \hyperref[dpa-section-phantom]{Divided Power Algebra}
\item \hyperref[sdga-section-phantom]{Differential Graded Sheaves}
\item \hyperref[hypercovering-section-phantom]{Hypercoverings}
\end{enumerate}
Schemes
\begin{enumerate}
\setcounter{enumi}{25}
\item \hyperref[schemes-section-phantom]{Schemes}
\item \hyperref[constructions-section-phantom]{Constructions of Schemes}
\item \hyperref[properties-section-phantom]{Properties of Schemes}
\item \hyperref[morphisms-section-phantom]{Morphisms of Schemes}
\item \hyperref[coherent-section-phantom]{Cohomology of Schemes}
\item \hyperref[divisors-section-phantom]{Divisors}
\item \hyperref[limits-section-phantom]{Limits of Schemes}
\item \hyperref[varieties-section-phantom]{Varieties}
\item \hyperref[topologies-section-phantom]{Topologies on Schemes}
\item \hyperref[descent-section-phantom]{Descent}
\item \hyperref[perfect-section-phantom]{Derived Categories of Schemes}
\item \hyperref[more-morphisms-section-phantom]{More on Morphisms}
\item \hyperref[flat-section-phantom]{More on Flatness}
\item \hyperref[groupoids-section-phantom]{Groupoid Schemes}
\item \hyperref[more-groupoids-section-phantom]{More on Groupoid Schemes}
\item \hyperref[etale-section-phantom]{\'Etale Morphisms of Schemes}
\end{enumerate}
Topics in Scheme Theory
\begin{enumerate}
\setcounter{enumi}{41}
\item \hyperref[chow-section-phantom]{Chow Homology}
\item \hyperref[intersection-section-phantom]{Intersection Theory}
\item \hyperref[pic-section-phantom]{Picard Schemes of Curves}
\item \hyperref[weil-section-phantom]{Weil Cohomology Theories}
\item \hyperref[adequate-section-phantom]{Adequate Modules}
\item \hyperref[dualizing-section-phantom]{Dualizing Complexes}
\item \hyperref[duality-section-phantom]{Duality for Schemes}
\item \hyperref[discriminant-section-phantom]{Discriminants and Differents}
\item \hyperref[derham-section-phantom]{de Rham Cohomology}
\item \hyperref[local-cohomology-section-phantom]{Local Cohomology}
\item \hyperref[algebraization-section-phantom]{Algebraic and Formal Geometry}
\item \hyperref[curves-section-phantom]{Algebraic Curves}
\item \hyperref[resolve-section-phantom]{Resolution of Surfaces}
\item \hyperref[models-section-phantom]{Semistable Reduction}
\item \hyperref[equiv-section-phantom]{Derived Categories of Varieties}
\item \hyperref[pione-section-phantom]{Fundamental Groups of Schemes}
\item \hyperref[etale-cohomology-section-phantom]{\'Etale Cohomology}
\item \hyperref[crystalline-section-phantom]{Crystalline Cohomology}
\item \hyperref[proetale-section-phantom]{Pro-\'etale Cohomology}
\item \hyperref[more-etale-section-phantom]{More \'Etale Cohomology}
\item \hyperref[trace-section-phantom]{The Trace Formula}
\end{enumerate}
Algebraic Spaces
\begin{enumerate}
\setcounter{enumi}{62}
\item \hyperref[spaces-section-phantom]{Algebraic Spaces}
\item \hyperref[spaces-properties-section-phantom]{Properties of Algebraic Spaces}
\item \hyperref[spaces-morphisms-section-phantom]{Morphisms of Algebraic Spaces}
\item \hyperref[decent-spaces-section-phantom]{Decent Algebraic Spaces}
\item \hyperref[spaces-cohomology-section-phantom]{Cohomology of Algebraic Spaces}
\item \hyperref[spaces-limits-section-phantom]{Limits of Algebraic Spaces}
\item \hyperref[spaces-divisors-section-phantom]{Divisors on Algebraic Spaces}
\item \hyperref[spaces-over-fields-section-phantom]{Algebraic Spaces over Fields}
\item \hyperref[spaces-topologies-section-phantom]{Topologies on Algebraic Spaces}
\item \hyperref[spaces-descent-section-phantom]{Descent and Algebraic Spaces}
\item \hyperref[spaces-perfect-section-phantom]{Derived Categories of Spaces}
\item \hyperref[spaces-more-morphisms-section-phantom]{More on Morphisms of Spaces}
\item \hyperref[spaces-flat-section-phantom]{Flatness on Algebraic Spaces}
\item \hyperref[spaces-groupoids-section-phantom]{Groupoids in Algebraic Spaces}
\item \hyperref[spaces-more-groupoids-section-phantom]{More on Groupoids in Spaces}
\item \hyperref[bootstrap-section-phantom]{Bootstrap}
\item \hyperref[spaces-pushouts-section-phantom]{Pushouts of Algebraic Spaces}
\end{enumerate}
Topics in Geometry
\begin{enumerate}
\setcounter{enumi}{79}
\item \hyperref[spaces-chow-section-phantom]{Chow Groups of Spaces}
\item \hyperref[groupoids-quotients-section-phantom]{Quotients of Groupoids}
\item \hyperref[spaces-more-cohomology-section-phantom]{More on Cohomology of Spaces}
\item \hyperref[spaces-simplicial-section-phantom]{Simplicial Spaces}
\item \hyperref[spaces-duality-section-phantom]{Duality for Spaces}
\item \hyperref[formal-spaces-section-phantom]{Formal Algebraic Spaces}
\item \hyperref[restricted-section-phantom]{Algebraization of Formal Spaces}
\item \hyperref[spaces-resolve-section-phantom]{Resolution of Surfaces Revisited}
\end{enumerate}
Deformation Theory
\begin{enumerate}
\setcounter{enumi}{87}
\item \hyperref[formal-defos-section-phantom]{Formal Deformation Theory}
\item \hyperref[defos-section-phantom]{Deformation Theory}
\item \hyperref[cotangent-section-phantom]{The Cotangent Complex}
\item \hyperref[examples-defos-section-phantom]{Deformation Problems}
\end{enumerate}
Algebraic Stacks
\begin{enumerate}
\setcounter{enumi}{91}
\item \hyperref[algebraic-section-phantom]{Algebraic Stacks}
\item \hyperref[examples-stacks-section-phantom]{Examples of Stacks}
\item \hyperref[stacks-sheaves-section-phantom]{Sheaves on Algebraic Stacks}
\item \hyperref[criteria-section-phantom]{Criteria for Representability}
\item \hyperref[artin-section-phantom]{Artin's Axioms}
\item \hyperref[quot-section-phantom]{Quot and Hilbert Spaces}
\item \hyperref[stacks-properties-section-phantom]{Properties of Algebraic Stacks}
\item \hyperref[stacks-morphisms-section-phantom]{Morphisms of Algebraic Stacks}
\item \hyperref[stacks-limits-section-phantom]{Limits of Algebraic Stacks}
\item \hyperref[stacks-cohomology-section-phantom]{Cohomology of Algebraic Stacks}
\item \hyperref[stacks-perfect-section-phantom]{Derived Categories of Stacks}
\item \hyperref[stacks-introduction-section-phantom]{Introducing Algebraic Stacks}
\item \hyperref[stacks-more-morphisms-section-phantom]{More on Morphisms of Stacks}
\item \hyperref[stacks-geometry-section-phantom]{The Geometry of Stacks}
\end{enumerate}
Topics in Moduli Theory
\begin{enumerate}
\setcounter{enumi}{105}
\item \hyperref[moduli-section-phantom]{Moduli Stacks}
\item \hyperref[moduli-curves-section-phantom]{Moduli of Curves}
\end{enumerate}
Miscellany
\begin{enumerate}
\setcounter{enumi}{107}
\item \hyperref[examples-section-phantom]{Examples}
\item \hyperref[exercises-section-phantom]{Exercises}
\item \hyperref[guide-section-phantom]{Guide to Literature}
\item \hyperref[desirables-section-phantom]{Desirables}
\item \hyperref[coding-section-phantom]{Coding Style}
\item \hyperref[obsolete-section-phantom]{Obsolete}
\item \hyperref[fdl-section-phantom]{GNU Free Documentation License}
\item \hyperref[index-section-phantom]{Auto Generated Index}
\end{enumerate}
\end{multicols}


\bibliography{my}
\bibliographystyle{amsalpha}

\end{document}
