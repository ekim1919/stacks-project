\IfFileExists{stacks-project.cls}{%
\documentclass{stacks-project}
}{%
\documentclass{amsart}
}

% For dealing with references we use the comment environment
\usepackage{verbatim}
\newenvironment{reference}{\comment}{\endcomment}
%\newenvironment{reference}{}{}
\newenvironment{slogan}{\comment}{\endcomment}
\newenvironment{history}{\comment}{\endcomment}

% For commutative diagrams we use Xy-pic
\usepackage[all]{xy}

% We use 2cell for 2-commutative diagrams.
\xyoption{2cell}
\UseAllTwocells

% We use multicol for the list of chapters between chapters
\usepackage{multicol}

% This is generall recommended for better output
\usepackage{lmodern}
\usepackage[T1]{fontenc}

% For cross-file-references
\usepackage{xr-hyper}

% Package for hypertext links:
\usepackage{hyperref}

% For any local file, say "hello.tex" you want to link to please
% use \externaldocument[hello-]{hello}
\externaldocument[introduction-]{introduction}
\externaldocument[conventions-]{conventions}
\externaldocument[sets-]{sets}
\externaldocument[categories-]{categories}
\externaldocument[topology-]{topology}
\externaldocument[sheaves-]{sheaves}
\externaldocument[sites-]{sites}
\externaldocument[stacks-]{stacks}
\externaldocument[fields-]{fields}
\externaldocument[algebra-]{algebra}
\externaldocument[brauer-]{brauer}
\externaldocument[homology-]{homology}
\externaldocument[derived-]{derived}
\externaldocument[simplicial-]{simplicial}
\externaldocument[more-algebra-]{more-algebra}
\externaldocument[smoothing-]{smoothing}
\externaldocument[modules-]{modules}
\externaldocument[sites-modules-]{sites-modules}
\externaldocument[injectives-]{injectives}
\externaldocument[cohomology-]{cohomology}
\externaldocument[sites-cohomology-]{sites-cohomology}
\externaldocument[dga-]{dga}
\externaldocument[dpa-]{dpa}
\externaldocument[sdga-]{sdga}
\externaldocument[hypercovering-]{hypercovering}
\externaldocument[schemes-]{schemes}
\externaldocument[constructions-]{constructions}
\externaldocument[properties-]{properties}
\externaldocument[morphisms-]{morphisms}
\externaldocument[coherent-]{coherent}
\externaldocument[divisors-]{divisors}
\externaldocument[limits-]{limits}
\externaldocument[varieties-]{varieties}
\externaldocument[topologies-]{topologies}
\externaldocument[descent-]{descent}
\externaldocument[perfect-]{perfect}
\externaldocument[more-morphisms-]{more-morphisms}
\externaldocument[flat-]{flat}
\externaldocument[groupoids-]{groupoids}
\externaldocument[more-groupoids-]{more-groupoids}
\externaldocument[etale-]{etale}
\externaldocument[chow-]{chow}
\externaldocument[intersection-]{intersection}
\externaldocument[pic-]{pic}
\externaldocument[weil-]{weil}
\externaldocument[adequate-]{adequate}
\externaldocument[dualizing-]{dualizing}
\externaldocument[duality-]{duality}
\externaldocument[discriminant-]{discriminant}
\externaldocument[derham-]{derham}
\externaldocument[local-cohomology-]{local-cohomology}
\externaldocument[algebraization-]{algebraization}
\externaldocument[curves-]{curves}
\externaldocument[resolve-]{resolve}
\externaldocument[models-]{models}
\externaldocument[equiv-]{equiv}
\externaldocument[pione-]{pione}
\externaldocument[etale-cohomology-]{etale-cohomology}
\externaldocument[proetale-]{proetale}
\externaldocument[more-etale-]{more-etale}
\externaldocument[trace-]{trace}
\externaldocument[crystalline-]{crystalline}
\externaldocument[spaces-]{spaces}
\externaldocument[spaces-properties-]{spaces-properties}
\externaldocument[spaces-morphisms-]{spaces-morphisms}
\externaldocument[decent-spaces-]{decent-spaces}
\externaldocument[spaces-cohomology-]{spaces-cohomology}
\externaldocument[spaces-limits-]{spaces-limits}
\externaldocument[spaces-divisors-]{spaces-divisors}
\externaldocument[spaces-over-fields-]{spaces-over-fields}
\externaldocument[spaces-topologies-]{spaces-topologies}
\externaldocument[spaces-descent-]{spaces-descent}
\externaldocument[spaces-perfect-]{spaces-perfect}
\externaldocument[spaces-more-morphisms-]{spaces-more-morphisms}
\externaldocument[spaces-flat-]{spaces-flat}
\externaldocument[spaces-groupoids-]{spaces-groupoids}
\externaldocument[spaces-more-groupoids-]{spaces-more-groupoids}
\externaldocument[bootstrap-]{bootstrap}
\externaldocument[spaces-pushouts-]{spaces-pushouts}
\externaldocument[spaces-chow-]{spaces-chow}
\externaldocument[groupoids-quotients-]{groupoids-quotients}
\externaldocument[spaces-more-cohomology-]{spaces-more-cohomology}
\externaldocument[spaces-simplicial-]{spaces-simplicial}
\externaldocument[spaces-duality-]{spaces-duality}
\externaldocument[formal-spaces-]{formal-spaces}
\externaldocument[restricted-]{restricted}
\externaldocument[spaces-resolve-]{spaces-resolve}
\externaldocument[formal-defos-]{formal-defos}
\externaldocument[defos-]{defos}
\externaldocument[cotangent-]{cotangent}
\externaldocument[examples-defos-]{examples-defos}
\externaldocument[algebraic-]{algebraic}
\externaldocument[examples-stacks-]{examples-stacks}
\externaldocument[stacks-sheaves-]{stacks-sheaves}
\externaldocument[criteria-]{criteria}
\externaldocument[artin-]{artin}
\externaldocument[quot-]{quot}
\externaldocument[stacks-properties-]{stacks-properties}
\externaldocument[stacks-morphisms-]{stacks-morphisms}
\externaldocument[stacks-limits-]{stacks-limits}
\externaldocument[stacks-cohomology-]{stacks-cohomology}
\externaldocument[stacks-perfect-]{stacks-perfect}
\externaldocument[stacks-introduction-]{stacks-introduction}
\externaldocument[stacks-more-morphisms-]{stacks-more-morphisms}
\externaldocument[stacks-geometry-]{stacks-geometry}
\externaldocument[moduli-]{moduli}
\externaldocument[moduli-curves-]{moduli-curves}
\externaldocument[examples-]{examples}
\externaldocument[exercises-]{exercises}
\externaldocument[guide-]{guide}
\externaldocument[desirables-]{desirables}
\externaldocument[coding-]{coding}
\externaldocument[obsolete-]{obsolete}
\externaldocument[fdl-]{fdl}
\externaldocument[index-]{index}

% Theorem environments.
%
\theoremstyle{plain}
\newtheorem{theorem}[subsection]{Theorem}
\newtheorem{proposition}[subsection]{Proposition}
\newtheorem{lemma}[subsection]{Lemma}

\theoremstyle{definition}
\newtheorem{definition}[subsection]{Definition}
\newtheorem{example}[subsection]{Example}
\newtheorem{exercise}[subsection]{Exercise}
\newtheorem{situation}[subsection]{Situation}

\theoremstyle{remark}
\newtheorem{remark}[subsection]{Remark}
\newtheorem{remarks}[subsection]{Remarks}

\numberwithin{equation}{subsection}

% Macros
%
\def\lim{\mathop{\mathrm{lim}}\nolimits}
\def\colim{\mathop{\mathrm{colim}}\nolimits}
\def\Spec{\mathop{\mathrm{Spec}}}
\def\Hom{\mathop{\mathrm{Hom}}\nolimits}
\def\Ext{\mathop{\mathrm{Ext}}\nolimits}
\def\SheafHom{\mathop{\mathcal{H}\!\mathit{om}}\nolimits}
\def\SheafExt{\mathop{\mathcal{E}\!\mathit{xt}}\nolimits}
\def\Sch{\mathit{Sch}}
\def\Mor{\mathop{\mathrm{Mor}}\nolimits}
\def\Ob{\mathop{\mathrm{Ob}}\nolimits}
\def\Sh{\mathop{\mathit{Sh}}\nolimits}
\def\NL{\mathop{N\!L}\nolimits}
\def\CH{\mathop{\mathrm{CH}}\nolimits}
\def\proetale{{pro\text{-}\acute{e}tale}}
\def\etale{{\acute{e}tale}}
\def\QCoh{\mathit{QCoh}}
\def\Ker{\mathop{\mathrm{Ker}}}
\def\Im{\mathop{\mathrm{Im}}}
\def\Coker{\mathop{\mathrm{Coker}}}
\def\Coim{\mathop{\mathrm{Coim}}}

% Boxtimes
%
\DeclareMathSymbol{\boxtimes}{\mathbin}{AMSa}{"02}

%
% Macros for moduli stacks/spaces
%
\def\QCohstack{\mathcal{QC}\!\mathit{oh}}
\def\Cohstack{\mathcal{C}\!\mathit{oh}}
\def\Spacesstack{\mathcal{S}\!\mathit{paces}}
\def\Quotfunctor{\mathrm{Quot}}
\def\Hilbfunctor{\mathrm{Hilb}}
\def\Curvesstack{\mathcal{C}\!\mathit{urves}}
\def\Polarizedstack{\mathcal{P}\!\mathit{olarized}}
\def\Complexesstack{\mathcal{C}\!\mathit{omplexes}}
% \Pic is the operator that assigns to X its picard group, usage \Pic(X)
% \Picardstack_{X/B} denotes the Picard stack of X over B
% \Picardfunctor_{X/B} denotes the Picard functor of X over B
\def\Pic{\mathop{\mathrm{Pic}}\nolimits}
\def\Picardstack{\mathcal{P}\!\mathit{ic}}
\def\Picardfunctor{\mathrm{Pic}}
\def\Deformationcategory{\mathcal{D}\!\mathit{ef}}


% OK, start here.
%
\begin{document}

\title{Schemes}


\maketitle

\phantomsection
\label{section-phantom}

\tableofcontents

\section{Introduction}
\label{section-introduction}

\noindent
In this document we define schemes.
A basic reference is \cite{EGA}.









\section{Locally ringed spaces}
\label{section-locally-ringed-spaces}

\noindent
Recall that we defined ringed spaces
in Sheaves, Section \ref{sheaves-section-ringed-spaces}.
Briefly, a ringed space is a pair $(X, \mathcal{O}_X)$ consisting
of a topological space $X$ and a sheaf of rings $\mathcal{O}_X$.
A morphism of ringed spaces $f : (X, \mathcal{O}_X) \to (Y, \mathcal{O}_Y)$
is given by a continuous map $f : X \to Y$ and an $f$-map of sheaves
of rings $f^\sharp : \mathcal{O}_Y \to \mathcal{O}_X$. You can
think of $f^\sharp$ as a map $\mathcal{O}_Y \to f_*\mathcal{O}_X$, see
Sheaves, Definition \ref{sheaves-definition-f-map} and
Lemma \ref{sheaves-lemma-f-map}.

\medskip\noindent
A good geometric example of this to keep in mind is
$\mathcal{C}^\infty$-manifolds and morphisms of
$\mathcal{C}^\infty$-manifolds. Namely, if $M$ is a
$\mathcal{C}^\infty$-manifold, then the sheaf $\mathcal{C}^\infty_M$
of smooth functions is a sheaf of rings on $M$. And any
map $f : M \to N$ of manifolds is smooth if and only if
for every local section $h$ of $\mathcal{C}^\infty_N$
the composition $h \circ f$ is a local section of $\mathcal{C}^\infty_M$.
Thus a smooth map $f$ gives rise in a natural way to a morphism
of ringed spaces
$$
f : (M , \mathcal{C}^\infty_M) \longrightarrow (N, \mathcal{C}^\infty_N)
$$
see Sheaves, Example \ref{sheaves-example-continuous-map-ringed}.
It is instructive to consider what happens to stalks. Namely, let
$m \in M$ with image $f(m) = n \in N$. Recall that the stalk
$\mathcal{C}^\infty_{M, m}$ is the ring of germs of smooth
functions at $m$, see
Sheaves, Example \ref{sheaves-example-germs-functions}.
The algebra of germs of functions on $(M, m)$ is a local ring with maximal
ideal the functions which vanish at $m$.
Similarly for $\mathcal{C}^\infty_{N, n}$. The map on stalks
$f^\sharp : \mathcal{C}^\infty_{N, n} \to \mathcal{C}^\infty_{M, m}$
maps the maximal ideal into the maximal ideal, simply
because $f(m) = n$.

\medskip\noindent
In algebraic geometry we study schemes. On a scheme the sheaf of rings
is not determined by an intrinsic property of the space.
The spectrum of a ring $R$
(see Algebra, Section \ref{algebra-section-spectrum-ring}) endowed
with a sheaf of rings constructed out of $R$ (see below),
will be our basic building block. It will turn out that
the stalks of $\mathcal{O}$ on $\Spec(R)$ are the local
rings of $R$ at its primes. There are two reasons to introduce
locally ringed spaces in this setting: (1) There is in general
no mechanism that assigns to a continuous map of spectra a
map of the corresponding rings. This is why we add as an extra datum
the map $f^\sharp$. (2) If we consider morphisms
of these spectra in the category of ringed spaces, then the
maps on stalks may not be local homomorphisms. Since our
geometric intuition says it should we introduce locally ringed spaces
as follows.

\begin{definition}
\label{definition-locally-ringed-space}
Locally ringed spaces.
\begin{enumerate}
\item A {\it locally ringed space $(X, \mathcal{O}_X)$}
is a pair consisting of a
topological space $X$ and a sheaf of rings $\mathcal{O}_X$ all of whose stalks
are local rings.
\item Given a locally ringed space $(X, \mathcal{O}_X)$ we say that
$\mathcal{O}_{X, x}$ is the {\it local ring of $X$ at $x$}.
We denote $\mathfrak{m}_{X, x}$ or simply $\mathfrak{m}_x$
the maximal ideal of $\mathcal{O}_{X, x}$. Moreover, the
{\it residue field of $X$ at $x$} is the residue field
$\kappa(x) = \mathcal{O}_{X, x}/\mathfrak{m}_x$.
\item A {\it morphism of locally ringed spaces}
$(f, f^\sharp) : (X, \mathcal{O}_X) \to (Y, \mathcal{O}_Y)$
is a morphism of ringed spaces such that for all $x \in X$
the induced ring map $\mathcal{O}_{Y, f(x)} \to \mathcal{O}_{X, x}$ is a
local ring map.
\end{enumerate}
\end{definition}

\noindent
We will usually suppress the sheaf of rings $\mathcal{O}_X$
in the notation when discussing locally ringed spaces. We will
simply refer to ``the locally ringed space $X$''.
We will by abuse of notation think of $X$ also as
the underlying topological space. Finally we will denote
the corresponding sheaf of rings
$\mathcal{O}_X$ as the {\it structure sheaf of $X$}.
In addition, it is customary to denote the maximal ideal
of the local ring $\mathcal{O}_{X, x}$ by
$\mathfrak{m}_{X, x}$ or simply $\mathfrak{m}_x$.
We will say ``let $f : X \to Y$ be a morphism of locally ringed
spaces'' thereby suppressing the structure sheaves even further.
In this case, we will by abuse of notation think of $f : X\to Y$
also as the underlying continuous map of topological spaces.
The $f$-map corresponding to $f$ will customarily
be denoted $f^\sharp$. The condition that $f$ is a morphism
of locally ringed spaces can then be expressed by saying that
for every $x\in X$ the map on stalks
$$
f^\sharp_x : \mathcal{O}_{Y, f(x)} \longrightarrow \mathcal{O}_{X, x}
$$
maps the maximal ideal $\mathfrak m_{Y, f(x)}$ into
$\mathfrak m_{X, x}$.

\medskip\noindent
Let us use these notational conventions to show that the
collection of locally ringed spaces and morphisms of locally
ringed spaces forms a category. In order to see this we have
to show that the composition of morphisms of locally ringed
spaces is a morphism of locally ringed spaces. OK, so let
$f : X \to Y$ and $g : Y \to Z$ be morphism of locally
ringed spaces. The composition of $f$ and $g$ is defined in
Sheaves, Definition \ref{sheaves-definition-composition-maps-ringed-spaces}.
Let $x \in X$. By
Sheaves, Lemma \ref{sheaves-lemma-compose-f-maps-stalks}
the composition
$$
\mathcal{O}_{Z, g(f(x))}
\xrightarrow{g^\sharp}
\mathcal{O}_{Y, f(x)}
\xrightarrow{f^\sharp}
\mathcal{O}_{X, x}
$$
is the associated map on stalks for the morphism $g \circ f$.
The result follows since a composition of local ring
homomorphisms is a local ring homomorphism.

\medskip\noindent
A pleasing feature of the definition is the fact that the functor
$$
\textit{Locally ringed spaces}
\longrightarrow
\textit{Ringed spaces}
$$
reflects isomorphisms (plus more).
Here is a less abstract statement.

\begin{lemma}
\label{lemma-isomorphism-locally-ringed}
\begin{slogan}
An isomorphism of ringed spaces between locally ringed spaces is an
isomorphism of locally ringed spaces.
\end{slogan}
Let $X$, $Y$ be locally ringed spaces.
If $f : X \to Y$ is an isomorphism of
ringed spaces, then $f$ is an isomorphism
of locally ringed spaces.
\end{lemma}

\begin{proof}
This follows trivially from the corresponding fact in algebra:
Suppose $A$, $B$ are local rings. Any isomorphism of rings
$A \to B$ is a local ring homomorphism.
\end{proof}













\section{Open immersions of locally ringed spaces}
\label{section-open-immersion}

\begin{definition}
\label{definition-immersion-locally-ringed-spaces}
Let $f : X \to Y$ be a morphism of locally ringed spaces.
We say that $f$ is an {\it open immersion} if
$f$ is a homeomorphism of $X$ onto an open subset
of $Y$, and the map $f^{-1}\mathcal{O}_Y \to \mathcal{O}_X$
is an isomorphism.
\end{definition}

\noindent
The following construction is parallel to
Sheaves, Definition \ref{sheaves-definition-restriction} (3).

\begin{example}
\label{example-open-subspace}
Let $X$ be a locally ringed space.
Let $U \subset X$ be an open subset.
Let $\mathcal{O}_U = \mathcal{O}_X|_U$
be the restriction of $\mathcal{O}_X$ to $U$.
For $u \in U$ the stalk $\mathcal{O}_{U, u}$
is equal to the stalk $\mathcal{O}_{X, u}$, and hence is a
local ring. Thus $(U, \mathcal{O}_U)$ is a locally ringed
space and the morphism $j : (U, \mathcal{O}_U) \to (X, \mathcal{O}_X)$
is an open immersion.
\end{example}

\begin{definition}
\label{definition-open-subspace}
Let $X$ be a locally ringed space.
Let $U \subset X$ be an open subset.
The locally ringed space $(U, \mathcal{O}_U)$
of Example \ref{example-open-subspace} above
is the {\it open subspace of $X$ associated to $U$}.
\end{definition}

\begin{lemma}
\label{lemma-open-immersion}
Let $f : X \to Y$ be an open immersion of
locally ringed spaces. Let $j : V = f(X) \to Y$
be the open subspace of $Y$ associated to the image of $f$.
There is a unique isomorphism $f' : X \cong V$ of
locally ringed spaces such that $f = j \circ f'$.
\end{lemma}

\begin{proof}
Let $f'$ be the homeomorphism between $X$ and $V$ induced by $f$. Then
$f = j \circ f'$ as maps of topological spaces. Since there
is an isomorphism of sheaves
$f^\sharp : f^{-1}(\mathcal{O}_Y) \to \mathcal{O}_X$, there is an isomorphism
of rings
$f^\sharp : \Gamma(U, f^{-1}(\mathcal{O}_Y)) \to \Gamma(U, \mathcal{O}_X)$
for each open subset $U \subset X$. Since
$\mathcal{O}_V = j^{-1}\mathcal{O}_Y$ and $f^{-1} = f'^{-1} j^{-1}$
(Sheaves, Lemma \ref{sheaves-lemma-pullback-composition}) we
see that $f^{-1}\mathcal{O}_Y = f'^{-1}\mathcal{O}_V$, hence
$\Gamma(U, f'^{-1}(\mathcal{O}_V)) \to \Gamma(U, f^{-1}(\mathcal{O}_Y))$
is an isomorphism
for every $U \subset X$ open. By composing these we get an isomorphism of rings
$$
\Gamma(U, f'^{-1}(\mathcal{O}_V)) \to \Gamma(U, \mathcal{O}_X)
$$
for each open subset $U \subset X$, and therefore an isomorphism of sheaves
$f^{-1}(\mathcal{O}_V) \to \mathcal{O}_X$. In other words, we have an
isomorphism $f'^{\sharp} : f'^{-1}(\mathcal{O}_V) \to \mathcal{O}_X$ and
therefore an isomorphism of locally ringed spaces
$(f', f'^{\sharp}) : (X, \mathcal{O}_X) \to (V, \mathcal{O}_V)$
(use Lemma \ref{lemma-isomorphism-locally-ringed}).
Note that $f = j \circ f'$ as morphisms of locally ringed spaces
by construction.

\medskip\noindent
Suppose we have another morphism
$f'' : (X, \mathcal{O}_X) \to (V, \mathcal{O}_V)$ such that $f = j \circ f''$.
At any point $x \in X$, we have $j(f'(x)) = j(f''(x))$ from which it follows
that $f'(x) = f''(x)$ since $j$ is the inclusion map; therefore $f'$ and $f''$
are the same as morphisms of topological spaces. On structure sheaves,
for each open subset $U \subset X$ we have a commutative diagram
$$
\xymatrix @R=5em{
\Gamma(U, f^{-1}(\mathcal{O}_Y)) \ar[d]_\cong\ar[r]^\cong &
\Gamma(U, \mathcal{O}_X) \\
\Gamma(U, f'^{-1}(\mathcal{O}_V)) \ar@/^/[ru]^{f'^\sharp}
\ar@/_/[ru]_{f''^\sharp} &
}
$$
from which we see that $f'^\sharp$ and $f''^\sharp$ define
the same morphism of sheaves.
\end{proof}

\noindent
From now on we do not distinguish between open subsets
and their associated subspaces.

\begin{lemma}
\label{lemma-restrict-map-to-opens}
Let $f : X \to Y$ be a morphism of locally ringed spaces.
Let $U \subset X$, and $V \subset Y$ be open subsets.
Suppose that $f(U) \subset V$. There exists a unique
morphism of locally ringed spaces $f|_U : U \to V$ such
that the following diagram is a commutative square of
locally ringed spaces
$$
\xymatrix{
U \ar[d]_{f|_U} \ar[r] & X \ar[d]^f \\
V \ar[r] & Y
}
$$
\end{lemma}

\begin{proof}
Omitted.
\end{proof}

\noindent
In the following we will use without further mention the following
fact which follows from the lemma above. Given any morphism
$f : Y \to X$ of locally ringed spaces, and any open subset
$U \subset X$ such that $f(Y) \subset U$, then there exists
a unique morphism of locally ringed spaces $Y \to U$ such that
the composition $Y \to U \to X$ is equal to $f$. In fact, we will
even by abuse of notation write $f : Y \to U$ since this rarely
gives rise to confusion.









\section{Closed immersions of locally ringed spaces}
\label{section-closed-immersion}

\noindent
We follow our conventions introduced in
Modules, Definition \ref{modules-definition-closed-immersion}.

\begin{definition}
\label{definition-closed-immersion-locally-ringed-spaces}
Let $i : Z \to X$ be a morphism of locally ringed spaces.
We say that $i$ is a {\it closed immersion} if:
\begin{enumerate}
\item The map $i$ is a homeomorphism of $Z$ onto a closed subset of $X$.
\item The map $\mathcal{O}_X \to i_*\mathcal{O}_Z$ is surjective;
let $\mathcal{I}$ denote the kernel.
\item The $\mathcal{O}_X$-module $\mathcal{I}$
is locally generated by sections.
\end{enumerate}
\end{definition}

\begin{lemma}
\label{lemma-closed-local-target}
Let $f : Z \to X$ be a morphism of locally ringed spaces.
In order for $f$ to be a closed immersion it suffices
that there exists an open covering $X = \bigcup U_i$ such
that each $f : f^{-1}U_i \to U_i$ is a closed immersion.
\end{lemma}

\begin{proof}
Omitted.
\end{proof}

\begin{example}
\label{example-closed-subspace}
Let $X$ be a locally ringed space.
Let $\mathcal{I} \subset \mathcal{O}_X$ be a sheaf
of ideals which is locally generated by sections as a sheaf
of $\mathcal{O}_X$-modules. Let $Z$ be the support of
the sheaf of rings $\mathcal{O}_X/\mathcal{I}$.
This is a closed subset of $X$, by
Modules, Lemma \ref{modules-lemma-support-sheaf-rings-closed}.
Denote $i : Z \to X$ the inclusion map.
By Modules, Lemma \ref{modules-lemma-i-star-exact}
there is a unique sheaf of rings $\mathcal{O}_Z$ on $Z$
with $i_*\mathcal{O}_Z = \mathcal{O}_X/\mathcal{I}$.
For any $z \in Z$ the stalk $\mathcal{O}_{Z, z}$
is equal to a quotient $\mathcal{O}_{X, i(z)}/\mathcal{I}_{i(z)}$
of a local ring and nonzero, hence a local ring.
Thus $i : (Z, \mathcal{O}_Z) \to (X, \mathcal{O}_X)$ is
a closed immersion of locally ringed spaces.
\end{example}

\begin{definition}
\label{definition-closed-subspace}
Let $X$ be a locally ringed space.
Let $\mathcal{I}$ be a sheaf of ideals on $X$
which is locally generated by sections.
The locally ringed space $(Z, \mathcal{O}_Z)$
of Example \ref{example-closed-subspace} above
is the {\it closed subspace of $X$ associated to
the sheaf of ideals $\mathcal{I}$}.
\end{definition}

\begin{lemma}
\label{lemma-closed-immersion}
Let $f : X \to Y$ be a closed immersion of
locally ringed spaces. Let $\mathcal{I}$ be the
kernel of the map $\mathcal{O}_Y \to f_*\mathcal{O}_X$.
Let $i : Z \to Y$ be the closed subspace of $Y$
associated to $\mathcal{I}$.
There is a unique isomorphism $f' : X \cong Z$ of
locally ringed spaces such that $f = i \circ f'$.
\end{lemma}

\begin{proof}
Omitted.
\end{proof}

\begin{lemma}
\label{lemma-characterize-closed-subspace}
Let $X$, $Y$ be locally ringed spaces. Let
$\mathcal{I} \subset \mathcal{O}_X$ be a sheaf of ideals locally generated
by sections. Let $i : Z \to X$ be the associated closed subspace.
A morphism $f : Y \to X$ factors through $Z$ if and only if the map
$f^*\mathcal{I} \to f^*\mathcal{O}_X = \mathcal{O}_Y$
is zero. If this is the case the morphism $g : Y \to Z$
such that $f = i \circ g$ is unique.
\end{lemma}

\begin{proof}
Clearly if $f$ factors as $Y \to Z \to X$ then the map
$f^*\mathcal{I} \to \mathcal{O}_Y$ is zero. Conversely
suppose that $f^*\mathcal{I} \to \mathcal{O}_Y$ is zero.
Pick any $y \in Y$, and consider the ring map
$f^\sharp_y : \mathcal{O}_{X, f(y)} \to \mathcal{O}_{Y, y}$.
Since the composition
$\mathcal{I}_{f(y)} \to \mathcal{O}_{X, f(y)} \to \mathcal{O}_{Y, y}$
is zero by assumption and since $f^\sharp_y(1) = 1$
we see that $1 \not \in \mathcal{I}_{f(y)}$, i.e.,
$\mathcal{I}_{f(y)} \not = \mathcal{O}_{X, f(y)}$. We conclude that
$f(Y) \subset Z = \text{Supp}(\mathcal{O}_X/\mathcal{I})$.
Hence $f = i \circ g$ where $g : Y \to Z$ is continuous.
Consider the map $f^\sharp : \mathcal{O}_X \to f_*\mathcal{O}_Y$.
The assumption $f^*\mathcal{I} \to \mathcal{O}_Y$ is zero implies that
the composition $\mathcal{I} \to \mathcal{O}_X \to f_*\mathcal{O}_Y$ is
zero by adjointness of $f_*$ and $f^*$.
In other words, we obtain a morphism of sheaves of rings
$\overline{f^\sharp} : \mathcal{O}_X/\mathcal{I} \to f_*\mathcal{O}_Y$.
Note that $f_*\mathcal{O}_Y = i_*g_*\mathcal{O}_Y$ and
that $\mathcal{O}_X/\mathcal{I} = i_*\mathcal{O}_Z$.
By Sheaves, Lemma \ref{sheaves-lemma-equivalence-categories-closed-structures}
we obtain a unique morphism of sheaves of rings
$g^\sharp : \mathcal{O}_Z \to g_*\mathcal{O}_Y$ whose pushforward
under $i$ is $\overline{f^\sharp}$. We omit the verification that
$(g, g^\sharp)$ defines a morphism of locally ringed spaces
and that $f = i \circ g$ as a morphism of locally ringed spaces.
The uniqueness of $(g, g^\sharp)$ was pointed out above.
\end{proof}

\begin{lemma}
\label{lemma-restrict-map-to-closed}
Let $f : X \to Y$ be a morphism of locally ringed spaces.
Let $\mathcal{I} \subset \mathcal{O}_Y$ be a sheaf of
ideals which is locally generated by sections.
Let $i : Z \to Y$ be the closed subspace associated to the
sheaf of ideals $\mathcal{I}$.
Let $\mathcal{J}$ be the image of the map
$f^*\mathcal{I} \to f^*\mathcal{O}_Y = \mathcal{O}_X$.
Then this ideal is locally generated by sections.
Moreover, let $i' : Z' \to X$ be the associated closed
subspace of $X$. There exists a unique
morphism of locally ringed spaces $f' : Z' \to Z$ such
that the following diagram is a commutative square of
locally ringed spaces
$$
\xymatrix{
Z' \ar[d]_{f'} \ar[r]_{i'} & X \ar[d]^f \\
Z \ar[r]^{i} & Y
}
$$
Moreover, this diagram is a fibre square in the category of
locally ringed spaces.
\end{lemma}

\begin{proof}
The ideal $\mathcal{J}$ is locally generated by sections
by Modules, Lemma \ref{modules-lemma-pullback-locally-generated}.
The rest of the lemma follows from the characterization,
in Lemma \ref{lemma-characterize-closed-subspace} above,
of what it means for a morphism to factor through a closed
subspace.
\end{proof}














\section{Affine schemes}
\label{section-affine-schemes}

\noindent
Let $R$ be a ring. Consider the topological space $\Spec(R)$
associated to $R$, see Algebra, Section \ref{algebra-section-spectrum-ring}.
We will endow this space with a sheaf of rings $\mathcal{O}_{\Spec(R)}$
and the resulting pair $(\Spec(R), \mathcal{O}_{\Spec(R)})$
will be an affine scheme.

\medskip\noindent
Recall that $\Spec(R)$ has a basis of open sets $D(f)$,
$f \in R$ which we call standard opens, see Algebra,
Definition \ref{algebra-definition-Zariski-topology}.
In addition, the intersection of two standard opens is another:
$D(f) \cap D(g) = D(fg)$, $f, g\in R$.

\begin{lemma}
\label{lemma-standard-open}
Let $R$ be a ring. Let $f \in R$.
\begin{enumerate}
\item If $g\in R$ and $D(g) \subset D(f)$, then
\begin{enumerate}
\item $f$ is invertible in $R_g$,
\item $g^e = af$ for some $e \geq 1$ and $a \in R$,
\item there is a canonical ring map $R_f \to R_g$, and
\item there is a canonical $R_f$-module map
$M_f \to M_g$ for any $R$-module $M$.
\end{enumerate}
\item Any open covering of $D(f)$ can be refined to a finite
open covering of the form $D(f) = \bigcup_{i = 1}^n D(g_i)$.
\item If $g_1, \ldots, g_n \in R$, then $D(f) \subset \bigcup D(g_i)$
if and only if $g_1, \ldots, g_n$ generate the unit ideal in $R_f$.
\end{enumerate}
\end{lemma}

\begin{proof}
Recall that $D(g) = \Spec(R_g)$ (see
Algebra, Lemma \ref{algebra-lemma-standard-open}).
Thus (a) holds because $f$
maps to an element of $R_g$ which is not
contained in any prime ideal, and hence invertible,
see Algebra, Lemma \ref{algebra-lemma-Zariski-topology}.
Write the inverse of $f$ in $R_g$ as $a/g^d$.
This means $g^d - af$ is annihilated by a power of $g$, whence (b).
For (c), the map $R_f \to R_g$ exists by (a) from the universal property
of localization, or we can define it by mapping $b/f^n$
to $a^nb/g^{ne}$. The equality $M_f = M \otimes_R R_f$
can be used to obtain the map on modules, or
we can define $M_f \to M_g$ by mapping
$x/f^n$ to $a^nx/g^{ne}$.

\medskip\noindent
Recall that $D(f)$ is quasi-compact, see
Algebra, Lemma \ref{algebra-lemma-qc-open}.
Hence the second statement follows directly
from the fact that the standard opens form
a basis for the topology.

\medskip\noindent
The third statement follows directly from
Algebra, Lemma \ref{algebra-lemma-Zariski-topology}.
\end{proof}

\noindent
In Sheaves, Section \ref{sheaves-section-bases} we defined
the notion of a sheaf on a basis, and we showed that it is
essentially equivalent to the notion of a sheaf on the space,
see Sheaves, Lemmas \ref{sheaves-lemma-extend-off-basis} and
\ref{sheaves-lemma-extend-off-basis-structures}. Moreover,
we showed in
Sheaves, Lemma \ref{sheaves-lemma-cofinal-systems-coverings-standard-case}
that it is sufficient to check the sheaf
condition on a cofinal system of open coverings for each
standard open. By the lemma above it suffices to check
on the finite coverings by standard opens.

\begin{definition}
\label{definition-standard-covering}
Let $R$ be a ring.
\begin{enumerate}
\item A {\it standard open covering} of $\Spec(R)$
is a covering $\Spec(R) = \bigcup_{i = 1}^n D(f_i)$,
where $f_1, \ldots, f_n \in R$.
\item Suppose that $D(f) \subset \Spec(R)$ is a standard
open. A {\it standard open covering} of $D(f)$
is a covering $D(f) = \bigcup_{i = 1}^n D(g_i)$,
where $g_1, \ldots, g_n \in R$.
\end{enumerate}
\end{definition}

\noindent
Let $R$ be a ring. Let $M$ be an $R$-module. We will define
a presheaf $\widetilde M$ on the basis of standard opens.
Suppose that $U \subset \Spec(R)$ is a standard open.
If $f, g \in R$ are such that $D(f) = D(g)$, then
by Lemma \ref{lemma-standard-open} above there are canonical
maps $M_f \to M_g$ and $M_g \to M_f$ which are mutually inverse.
Hence we may choose any $f$ such that $U = D(f)$
and define
$$
\widetilde M(U) = M_f.
$$
Note that if $D(g) \subset D(f)$, then by
Lemma \ref{lemma-standard-open} above we have
a canonical map
$$
\widetilde M(D(f)) = M_f \longrightarrow M_g = \widetilde M(D(g)).
$$
Clearly, this defines a presheaf of abelian groups on the basis
of standard opens. If $M = R$, then $\widetilde R$ is a presheaf
of rings on the basis of standard opens.

\medskip\noindent
Let us compute the stalk of $\widetilde M$ at a point $x \in \Spec(R)$.
Suppose that $x$ corresponds to the prime $\mathfrak p \subset R$.
By definition of the stalk we see that
$$
\widetilde M_x = \colim_{f\in R, f\not\in \mathfrak p} M_f
$$
Here the set $\{f \in R, f \not \in \mathfrak p\}$ is preordered by
the rule $f \geq f' \Leftrightarrow D(f) \subset D(f')$.
If $f_1, f_2 \in R \setminus \mathfrak p$, then we have
$f_1f_2 \geq f_1$ in this ordering. Hence by
Algebra, Lemma \ref{algebra-lemma-localization-colimit}
we conclude that
$$
\widetilde M_x = M_{\mathfrak p}.
$$

\medskip\noindent
Next, we check the sheaf condition for the standard open coverings.
If $D(f) = \bigcup_{i = 1}^n D(g_i)$, then the sheaf condition
for this covering is equivalent with the exactness of the
sequence
$$
0 \to M_f \to \bigoplus M_{g_i} \to \bigoplus M_{g_ig_j}.
$$
Note that $D(g_i) = D(fg_i)$, and hence we can rewrite this
sequence as the sequence
$$
0 \to M_f \to \bigoplus M_{fg_i} \to \bigoplus M_{fg_ig_j}.
$$
In addition, by Lemma \ref{lemma-standard-open} above
we see that $g_1, \ldots, g_n$ generate the unit ideal
in $R_f$. Thus we may apply
Algebra, Lemma \ref{algebra-lemma-cover-module}
to the module $M_f$ over $R_f$ and the elements $g_1, \ldots, g_n$.
We conclude that the sequence is exact. By the remarks
made above, we see that $\widetilde M$ is a sheaf
on the basis of standard opens.

\medskip\noindent
Thus we conclude from the material in
Sheaves, Section \ref{sheaves-section-bases}
that there exists a
unique sheaf of rings $\mathcal{O}_{\Spec(R)}$
which agrees with $\widetilde R$ on the standard opens.
Note that by our computation of stalks above, the
stalks of this sheaf of rings are all local rings.

\medskip\noindent
Similarly, for any $R$-module $M$ there exists
a unique sheaf of $\mathcal{O}_{\Spec(R)}$-modules
$\mathcal{F}$ which agrees with $\widetilde M$ on the
standard opens, see
Sheaves, Lemma \ref{sheaves-lemma-extend-off-basis-module}.

\begin{definition}
\label{definition-structure-sheaf}
Let $R$ be a ring.
\begin{enumerate}
\item The {\it structure sheaf $\mathcal{O}_{\Spec(R)}$ of the
spectrum of $R$} is the unique sheaf of rings $\mathcal{O}_{\Spec(R)}$
which agrees with $\widetilde R$ on the basis of standard opens.
\item The locally ringed space
$(\Spec(R), \mathcal{O}_{\Spec(R)})$ is called
the {\it spectrum} of $R$ and denoted $\Spec(R)$.
\item The sheaf of $\mathcal{O}_{\Spec(R)}$-modules
extending $\widetilde M$ to all opens of $\Spec(R)$
is called the sheaf of $\mathcal{O}_{\Spec(R)}$-modules
associated to $M$. This sheaf is denoted $\widetilde M$ as
well.
\end{enumerate}
\end{definition}

\noindent
We summarize the results obtained so far.

\begin{lemma}
\label{lemma-spec-sheaves}
Let $R$ be a ring. Let $M$ be an $R$-module. Let $\widetilde M$
be the sheaf of $\mathcal{O}_{\Spec(R)}$-modules
associated to $M$.
\begin{enumerate}
\item We have $\Gamma(\Spec(R), \mathcal{O}_{\Spec(R)}) = R$.
\item We have $\Gamma(\Spec(R), \widetilde M) = M$ as an $R$-module.
\item For every $f \in R$ we have
$\Gamma(D(f), \mathcal{O}_{\Spec(R)}) = R_f$.
\item For every $f\in R$ we have $\Gamma(D(f), \widetilde M) = M_f$
as an $R_f$-module.
\item Whenever $D(g) \subset D(f)$ the restriction mappings
on $\mathcal{O}_{\Spec(R)}$ and $\widetilde M$
are the maps
$R_f \to R_g$ and $M_f \to M_g$ from Lemma
\ref{lemma-standard-open}.
\item Let $\mathfrak p$ be a prime of $R$, and let $x \in \Spec(R)$
be the corresponding point. We have
$\mathcal{O}_{\Spec(R), x} = R_{\mathfrak p}$.
\item Let $\mathfrak p$ be a prime of $R$, and let $x \in \Spec(R)$
be the corresponding point. We have $\widetilde M_x = M_{\mathfrak p}$
as an $R_{\mathfrak p}$-module.
\end{enumerate}
Moreover, all these identifications are functorial in the $R$
module $M$. In particular, the functor $M \mapsto \widetilde M$
is an exact functor from the category of $R$-modules
to the category of $\mathcal{O}_{\Spec(R)}$-modules.
\end{lemma}

\begin{proof}
Assertions (1) - (7) are clear from the discussion above.
The exactness of the functor $M \mapsto \widetilde M$
follows from the fact that the functor $M \mapsto M_{\mathfrak p}$
is exact and the fact that exactness of short exact sequences
may be checked on stalks, see
Modules, Lemma \ref{modules-lemma-abelian}.
\end{proof}

\begin{definition}
\label{definition-affine-scheme}
An {\it affine scheme} is a locally ringed space isomorphic
as a locally ringed space to $\Spec(R)$ for some ring $R$.
A {\it morphism of affine schemes} is a morphism in the category
of locally ringed spaces.
\end{definition}

\noindent
It turns out that affine schemes play a special role
among all locally ringed spaces, which is what the next
section is about.


















\section{The category of affine schemes}
\label{section-category-affine-schemes}

\noindent
Note that if $Y$ is an affine scheme, then its
points are in canonical $1-1$ bijection with
prime ideals in $\Gamma(Y, \mathcal{O}_Y)$.

\begin{lemma}
\label{lemma-morphism-into-affine-where-point-goes}
Let $X$ be a locally ringed space.
Let $Y$ be an affine scheme.
Let $f \in \Mor(X, Y)$ be a morphism
of locally ringed spaces. Given a point $x \in X$
consider the ring maps
$$
\Gamma(Y, \mathcal{O}_Y) \xrightarrow{f^\sharp}
\Gamma(X, \mathcal{O}_X) \to \mathcal{O}_{X, x}
$$
Let $\mathfrak p \subset \Gamma(Y, \mathcal{O}_Y)$ denote
the inverse image of $\mathfrak m_x$. Let $y \in Y$ be the
corresponding point. Then $f(x) = y$.
\end{lemma}

\begin{proof}
Consider the commutative diagram
$$
\xymatrix{
\Gamma(X, \mathcal{O}_X) \ar[r] &
\mathcal{O}_{X, x} \\
\Gamma(Y, \mathcal{O}_Y) \ar[r] \ar[u] &
\mathcal{O}_{Y, f(x)} \ar[u]
}
$$
(see the discussion of $f$-maps below
Sheaves, Definition \ref{sheaves-definition-f-map}).
Since the right vertical arrow is local
we see that $\mathfrak m_{f(x)}$ is the
inverse image of $\mathfrak m_x$. The result
follows.
\end{proof}

\begin{lemma}
\label{lemma-f-open}
Let $X$ be a locally ringed space.
Let $f \in \Gamma(X, \mathcal{O}_X)$.
The set
$$
D(f) = \{x \in X \mid \text{image }f \not\in \mathfrak m_x\}
$$
is open. Moreover $f|_{D(f)}$ has an inverse.
\end{lemma}

\begin{proof}
This is a special case of Modules, Lemma \ref{modules-lemma-s-open}, but
we also give a direct proof.
Suppose that $U \subset X$ and $V \subset X$ are
two open subsets such that $f|_U$ has an inverse
$g$ and $f|_V$ has an inverse $h$. Then clearly
$g|_{U\cap V} = h|_{U\cap V}$. Thus it suffices
to show that $f$ is invertible in an open neighbourhood
of any $x \in D(f)$. This is clear because
$f \not \in \mathfrak m_x$ implies that $f \in \mathcal{O}_{X, x}$
has an inverse $g \in \mathcal{O}_{X, x}$ which means there
is some open neighbourhood $x \in U \subset X$ so
that $g \in \mathcal{O}_X(U)$ and $g\cdot f|_U = 1$.
\end{proof}

\begin{lemma}
\label{lemma-f-open-affine}
In Lemma \ref{lemma-f-open} above, if $X$ is an affine scheme,
then the open $D(f)$ agrees with the standard open $D(f)$
defined previously (in
Algebra, Definition \ref{algebra-definition-spectrum-ring}).
\end{lemma}

\begin{proof}
Omitted.
\end{proof}

\begin{lemma}
\label{lemma-morphism-into-affine}
\begin{reference}
A reference for this fact is \cite[II, Err 1, Prop. 1.8.1]{EGA}
where it is attributed to J. Tate.
\end{reference}
Let $X$ be a locally ringed space.
Let $Y$ be an affine scheme.
The map
$$
\Mor(X, Y)
\longrightarrow
\Hom(\Gamma(Y, \mathcal{O}_Y), \Gamma(X, \mathcal{O}_X))
$$
which maps $f$ to $f^\sharp$ (on global sections) is bijective.
\end{lemma}

\begin{proof}
Since $Y$ is affine we have
$(Y, \mathcal{O}_Y) \cong (\Spec(R), \mathcal{O}_{\Spec(R)})$
for some ring $R$.
During the proof we will use facts about $Y$ and
its structure sheaf which are direct consequences of things
we know about the spectrum of a ring, see e.g.\ Lemma
\ref{lemma-spec-sheaves}.

\medskip\noindent
Motivated by the lemmas above we construct the inverse map.
Let $\psi_Y : \Gamma(Y, \mathcal{O}_Y) \to \Gamma(X, \mathcal{O}_X)$
be a ring map. First, we define the corresponding map of
spaces
$$
\Psi : X \longrightarrow Y
$$
by the rule of
Lemma \ref{lemma-morphism-into-affine-where-point-goes}.
In other words, given $x \in X$ we define $\Psi(x)$
to be the point of $Y$ corresponding to the prime
in $\Gamma(Y, \mathcal{O}_Y)$ which is the inverse
image of $\mathfrak m_x$ under the composition
$
\Gamma(Y, \mathcal{O}_Y) \xrightarrow{\psi_Y}
\Gamma(X, \mathcal{O}_X) \to
\mathcal{O}_{X, x}
$.

\medskip\noindent
We claim that the map $\Psi : X \to Y$ is continuous.
The standard opens $D(g)$, for $g \in \Gamma(Y, \mathcal{O}_Y)$
are a basis for the topology of $Y$. Thus it suffices to prove
that $\Psi^{-1}(D(g))$ is open. By construction of $\Psi$
the inverse image $\Psi^{-1}(D(g))$ is exactly the set
$D(\psi_Y(g)) \subset X$ which is open by Lemma \ref{lemma-f-open}.
Hence $\Psi$ is continuous.

\medskip\noindent
Next we construct a $\Psi$-map of sheaves from
$\mathcal{O}_Y$ to $\mathcal{O}_X$. By
Sheaves, Lemma \ref{sheaves-lemma-f-map-basis-below-structures}
it suffices to define ring maps
$\psi_{D(g)} : \Gamma(D(g), \mathcal{O}_Y) \to
\Gamma(\Psi^{-1}(D(g)), \mathcal{O}_X)$
compatible with restriction maps.
We have a canonical isomorphism
$\Gamma(D(g), \mathcal{O}_Y) = \Gamma(Y, \mathcal{O}_Y)_g$,
because $Y$ is an affine scheme.
Because $\psi_Y(g)$ is invertible on $D(\psi_Y(g))$
we see that there is a canonical map
$$
\Gamma(Y, \mathcal{O}_Y)_g
\longrightarrow
\Gamma(\Psi^{-1}(D(g)), \mathcal{O}_X)
=
\Gamma(D(\psi_Y(g)), \mathcal{O}_X)
$$
extending the map $\psi_Y$
by the universal property of localization.
Note that there is no choice but to take the canonical map here!
And we take this, combined
with the canonical identification
$\Gamma(D(g), \mathcal{O}_Y) = \Gamma(Y, \mathcal{O}_Y)_g$, to
be $\psi_{D(g)}$. This is compatible with localization since the
restriction mapping on the affine schemes are defined in terms
of the universal properties of localization also, see
Lemmas \ref{lemma-spec-sheaves} and \ref{lemma-standard-open}.

\medskip\noindent
Thus we have defined a morphism of ringed spaces
$(\Psi, \psi) : (X, \mathcal{O}_X) \to (Y, \mathcal{O}_Y)$
recovering $\psi_Y$ on global sections. To see that it is
a morphism of locally ringed spaces we have to show that
the induced maps on local rings
$$
\psi_x : \mathcal{O}_{Y, \Psi(x)} \longrightarrow \mathcal{O}_{X, x}
$$
are local. This follows immediately from the commutative diagram
of the proof of Lemma \ref{lemma-morphism-into-affine-where-point-goes}
and the definition of $\Psi$.

\medskip\noindent
Finally, we have to show that the constructions
$(\Psi, \psi) \mapsto \psi_Y$ and the construction
$\psi_Y \mapsto (\Psi, \psi)$ are inverse to each other.
Clearly, $\psi_Y \mapsto (\Psi, \psi) \mapsto \psi_Y$.
Hence the only thing to prove is that given $\psi_Y$
there is at most one pair $(\Psi, \psi)$ giving rise
to it. The uniqueness of $\Psi$ was shown in Lemma
\ref{lemma-morphism-into-affine-where-point-goes} and
given the uniqueness of $\Psi$ the uniqueness of the
map $\psi$ was pointed out during the course of the proof
above.
\end{proof}

\begin{lemma}
\label{lemma-category-affine-schemes}
The category of affine schemes is equivalent to the opposite of the
category of rings. The equivalence is given by the functor that associates
to an affine scheme the global sections of its structure sheaf.
\end{lemma}

\begin{proof}
This is now clear from Definition \ref{definition-affine-scheme}
and Lemma \ref{lemma-morphism-into-affine}.
\end{proof}

\begin{lemma}
\label{lemma-standard-open-affine}
Let $Y$ be an affine scheme.
Let $f \in \Gamma(Y, \mathcal{O}_Y)$.
The open subspace $D(f)$ is an affine scheme.
\end{lemma}

\begin{proof}
We may assume that $Y = \Spec(R)$ and $f \in R$.
Consider the morphism of affine schemes
$\phi : U = \Spec(R_f) \to \Spec(R) = Y$ induced by the ring
map $R \to R_f$. By Algebra, Lemma \ref{algebra-lemma-standard-open}
we know that it is a homeomorphism onto $D(f)$.
On the other hand, the map $\phi^{-1}\mathcal{O}_Y \to \mathcal{O}_U$
is an isomorphism on stalks, hence an isomorphism. Thus we see
that $\phi$ is an open immersion. We conclude that $D(f)$
is isomorphic to $U$ by Lemma \ref{lemma-open-immersion}.
\end{proof}

\begin{lemma}
\label{lemma-fibre-product-affine-schemes}
The category of affine schemes has finite products, and fibre products.
In other words, it has finite limits. Moreover, the products
and fibre products in the category of affine schemes
are the same as in the category of locally ringed spaces.
In a formula, we have (in the category of locally ringed spaces)
$$
\Spec(R) \times \Spec(S) =
\Spec(R \otimes_{\mathbf{Z}} S)
$$
and given ring maps $R \to A$, $R \to B$ we have
$$
\Spec(A) \times_{\Spec(R)} \Spec(B)
=
\Spec(A \otimes_R B).
$$
\end{lemma}

\begin{proof}
This is just an application of Lemma \ref{lemma-morphism-into-affine}.
First of all, by that lemma, the affine scheme
$\Spec(\mathbf{Z})$ is the final object in the category
of locally ringed spaces. Thus the first displayed formula
follows from the second. To prove the second note that
for any locally ringed space $X$ we have
\begin{eqnarray*}
\Mor(X, \Spec(A \otimes_R B))
& = &
\Hom(A \otimes_R B, \mathcal{O}_X(X)) \\
& = &
\Hom(A, \mathcal{O}_X(X))
\times_{\Hom(R, \mathcal{O}_X(X))}
\Hom(B, \mathcal{O}_X(X)) \\
& = &
\Mor(X, \Spec(A))
\times_{\Mor(X, \Spec(R))}
\Mor(X, \Spec(B))
\end{eqnarray*}
which proves the formula.
See Categories, Section \ref{categories-section-fibre-products} for the
relevant definitions.
\end{proof}

\begin{lemma}
\label{lemma-disjoint-union-affines}
Let $X$ be a locally ringed space.
Assume $X = U \amalg V$ with $U$ and $V$ open and
such that $U$, $V$ are affine schemes. Then $X$ is an affine scheme.
\end{lemma}

\begin{proof}
Set $R = \Gamma(X, \mathcal{O}_X)$.
Note that $R = \mathcal{O}_X(U) \times \mathcal{O}_X(V)$
by the sheaf property. By Lemma \ref{lemma-morphism-into-affine}
there is a canonical morphism of locally ringed spaces
$X \to \Spec(R)$. By Algebra, Lemma \ref{algebra-lemma-spec-product}
we see that as a topological space
$\Spec(\mathcal{O}_X(U)) \amalg \Spec(\mathcal{O}_X(V)) =
\Spec(R)$
with the maps coming from the ring homomorphisms
$R \to \mathcal{O}_X(U)$ and $R \to \mathcal{O}_X(V)$.
This of course means that $\Spec(R)$ is the coproduct
in the category of locally ringed spaces as well.
By assumption the morphism $X \to \Spec(R)$ induces an isomorphism
of $\Spec(\mathcal{O}_X(U))$ with $U$ and similarly
for $V$. Hence $X \to \Spec(R)$ is an isomorphism.
\end{proof}















\section{Quasi-coherent sheaves on affines}
\label{section-quasi-coherent-affine}

\noindent
Recall that we have defined the abstract notion of a quasi-coherent
sheaf in Modules, Definition \ref{modules-definition-quasi-coherent}.
In this section we show that any quasi-coherent sheaf on an affine
scheme $\Spec(R)$ corresponds to the sheaf $\widetilde M$ associated to
an $R$-module $M$.

\begin{lemma}
\label{lemma-compare-constructions}
Let $(X, \mathcal{O}_X) = (\Spec(R), \mathcal{O}_{\Spec(R)})$
be an affine scheme. Let $M$ be an $R$-module. There exists a canonical
isomorphism between the sheaf $\widetilde M$ associated to the $R$-module
$M$ (Definition \ref{definition-structure-sheaf}) and the sheaf
$\mathcal{F}_M$ associated to the $R$-module $M$
(Modules, Definition \ref{modules-definition-sheaf-associated}).
This isomorphism is functorial in $M$. In particular,
the sheaves $\widetilde M$ are quasi-coherent. Moreover, they
are characterized by the following mapping property
$$
\Hom_{\mathcal{O}_X}(\widetilde M, \mathcal{F})
=
\Hom_R(M, \Gamma(X, \mathcal{F}))
$$
for any sheaf of $\mathcal{O}_X$-modules $\mathcal{F}$.
Here a map $\alpha : \widetilde M \to \mathcal{F}$ corresponds
to its effect on global sections.
\end{lemma}

\begin{proof}
By Modules, Lemma \ref{modules-lemma-construct-quasi-coherent-sheaves}
we have a morphism $\mathcal{F}_M \to \widetilde M$ corresponding
to the map $M \to \Gamma(X, \widetilde M) = M$. Let $x \in X$
correspond to the prime $\mathfrak p \subset R$.
The induced map on stalks are the maps
$\mathcal{O}_{X, x} \otimes_R M \to M_{\mathfrak p}$
which are isomorphisms because
$R_{\mathfrak p} \otimes_R M = M_{\mathfrak p}$.
Hence the map $\mathcal{F}_M \to \widetilde M$ is an isomorphism.
The mapping property follows from the mapping property of
the sheaves $\mathcal{F}_M$.
\end{proof}

\begin{lemma}
\label{lemma-widetilde-constructions}
Let $(X, \mathcal{O}_X) = (\Spec(R), \mathcal{O}_{\Spec(R)})$
be an affine scheme. There are canonical isomorphisms
\begin{enumerate}
\item
$
\widetilde{M \otimes_R N}
\cong
\widetilde M \otimes_{\mathcal{O}_X} \widetilde N
$,
see Modules, Section \ref{modules-section-tensor-product}.
\item
$
\widetilde{\text{T}^n(M)}
\cong
\text{T}^n(\widetilde M)
$,
$
\widetilde{\text{Sym}^n(M)}
\cong
\text{Sym}^n(\widetilde M)
$, and
$
\widetilde{\wedge^n(M)}
\cong
\wedge^n(\widetilde M)
$,
see
Modules, Section \ref{modules-section-symmetric-exterior}.
\item if $M$ is a finitely presented $R$-module, then
$
\SheafHom_{\mathcal{O}_X}(\widetilde M, \widetilde N)
\cong
\widetilde{\Hom_R(M,  N)}
$,
see
Modules, Section \ref{modules-section-internal-hom}.
\end{enumerate}
\end{lemma}

\begin{proof}[First proof]
Using Lemma \ref{lemma-compare-constructions} and
Modules, Lemma \ref{modules-lemma-construct-quasi-coherent-sheaves}
we see that the functor $M \mapsto \widetilde M$ can be viewed
as $\pi^*$ for a morphism $\pi$ of ringed spaces.
And pulling back modules commutes with tensor constructions by
Modules, Lemmas \ref{modules-lemma-tensor-product-pullback}
and \ref{modules-lemma-pullback-tensor-algebra}.
The morphism $\pi : (X, \mathcal{O}_X) \to (\{*\}, R)$ is
flat for example because the stalks of $\mathcal{O}_X$ are
localizations of $R$ (Lemma \ref{lemma-spec-sheaves})
and hence flat over $R$. Thus pullback by $\pi$ commutes
with internal hom if the first module is finitely presented by
Modules, Lemma \ref{modules-lemma-pullback-internal-hom}.
\end{proof}

\begin{proof}[Second proof]
Proof of (1). By Lemma \ref{lemma-compare-constructions} to give a map
$\widetilde{M \otimes_R N}$ into
$\widetilde M \otimes_{\mathcal{O}_X} \widetilde N$
we have to give a map on global sections
$M \otimes_R N \to
\Gamma(X, \widetilde M \otimes_{\mathcal{O}_X} \widetilde N)$
which exists by definition of the tensor product of sheaves
of modules. To see that this map is an isomorphism it
suffices to check that it is an isomorphism on stalks.
And this follows from the description of the stalks
of $\widetilde{M}$ (either in Lemma \ref{lemma-spec-sheaves} or in
Modules, Lemma \ref{modules-lemma-construct-quasi-coherent-sheaves}),
the fact that tensor product commutes with localization
(Algebra, Lemma \ref{algebra-lemma-tensor-product-localization}) and
Modules, Lemma \ref{modules-lemma-stalk-tensor-product}.

\medskip\noindent
Proof of (2). This is similar to the proof of (1), using
Algebra, Lemma \ref{algebra-lemma-tensor-algebra-localization} and
Modules, Lemma \ref{modules-lemma-stalk-tensor-algebra}.

\medskip\noindent
Proof of (3). Since the construction $M \mapsto \widetilde{M}$ is functorial
there is an $R$-linear map
$\Hom_R(M, N) \to \Hom_{\mathcal{O}_X}(\widetilde{M}, \widetilde{N})$.
The target of this map is the global sections of
$\SheafHom_{\mathcal{O}_X}(\widetilde M, \widetilde N)$.
Hence by Lemma \ref{lemma-compare-constructions} we obtain a map
of $\mathcal{O}_X$-modules $\widetilde{\Hom_R(M,  N)} \to
\SheafHom_{\mathcal{O}_X}(\widetilde M, \widetilde N)$.
We check that this is an isomorphism by comparing stalks.
If $M$ is finitely presented as an $R$-module
then $\widetilde M$ has a global finite presentation as an
$\mathcal{O}_X$-module. Hence we conclude using
Algebra, Lemma \ref{algebra-lemma-hom-from-finitely-presented} and
Modules, Lemma \ref{modules-lemma-stalk-internal-hom}.
\end{proof}

\begin{proof}[Third proof of part (1)]
For any $\mathcal{O}_X$-module $\mathcal{F}$
we have the following isomorphisms functorial in $M$, $N$, and $\mathcal{F}$
\begin{align*}
\Hom_{\mathcal{O}_X}(\widetilde{M} \otimes _{\mathcal{O} _X} \widetilde{N},
\mathcal{F})
& =
\Hom_{\mathcal{O}_X}(\widetilde{M},
\SheafHom_{\mathcal{O} _X} (\widetilde{N}, \mathcal{F})) \\
& =
\Hom_R(M, \Gamma(X,
\SheafHom_{\mathcal{O}_X}(\widetilde{N}, \mathcal{F})) \\
& =
\Hom_R(M, \Hom_{\mathcal{O}_X}(\widetilde{N}, \mathcal{F})) \\
& =
\Hom_R(M, \Hom_R(N, \Gamma(X,\mathcal{F}))) \\
& =
\Hom_R(M \otimes_R N, \Gamma(X, \mathcal{F})) \\
& =
\Hom_{\mathcal{O}_X}(\widetilde{M \otimes_R N}, \mathcal{F})
\end{align*}
The first equality is Modules, Lemma \ref{modules-lemma-internal-hom}.
The second equality is the universal property of $\widetilde{M}$, see
Lemma \ref{lemma-compare-constructions}. The third equality holds by
definition of $\SheafHom$. The fourth equality is the universal property
of $\widetilde{N}$. Then fifth equality is
Algebra, Lemma \ref{algebra-lemma-hom-from-tensor-product}.
The final equality is the universal property of
$\widetilde{M \otimes_R N}$.
By the Yoneda lemma (Categories, Lemma \ref{categories-lemma-yoneda})
we obtain (1).
\end{proof}

\begin{lemma}
\label{lemma-widetilde-pullback}
Let
$(X, \mathcal{O}_X) = (\Spec(S), \mathcal{O}_{\Spec(S)})$,
$(Y, \mathcal{O}_Y) = (\Spec(R), \mathcal{O}_{\Spec(R)})$
be affine schemes.
Let $\psi : (X, \mathcal{O}_X) \to (Y, \mathcal{O}_Y)$ be a
morphism of affine schemes, corresponding to the ring map
$\psi^\sharp : R \to S$ (see Lemma \ref{lemma-category-affine-schemes}).
\begin{enumerate}
\item We have $\psi^* \widetilde M = \widetilde{S \otimes_R M}$
functorially in the $R$-module $M$.
\item We have $\psi_* \widetilde N = \widetilde{N_R}$ functorially
in the $S$-module $N$.
\end{enumerate}
\end{lemma}

\begin{proof}
The first assertion follows from the identification in
Lemma \ref{lemma-compare-constructions}
and the result of Modules, Lemma \ref{modules-lemma-restrict-quasi-coherent}.
The second assertion follows from the fact
that $\psi^{-1}(D(f)) = D(\psi^\sharp(f))$ and hence
$$
\psi_* \widetilde N(D(f)) = \widetilde N(D(\psi^\sharp(f))) =
N_{\psi^\sharp(f)} = (N_R)_f = \widetilde{N_R}(D(f))
$$
as desired.
\end{proof}

\noindent
Lemma \ref{lemma-widetilde-pullback} above says in particular
that if you restrict
the sheaf $\widetilde M$ to a standard affine open subspace
$D(f)$, then you get $\widetilde{M_f}$. We will use this from
now on without further mention.

\begin{lemma}
\label{lemma-quasi-coherent-affine}
Let $(X, \mathcal{O}_X) = (\Spec(R), \mathcal{O}_{\Spec(R)})$
be an affine scheme. Let $\mathcal{F}$ be a
quasi-coherent $\mathcal{O}_X$-module. Then
$\mathcal{F}$ is isomorphic to the sheaf associated to
the $R$-module $\Gamma(X, \mathcal{F})$.
\end{lemma}

\begin{proof}
Let $\mathcal{F}$ be a quasi-coherent $\mathcal{O}_X$-module.
Since every standard open $D(f)$ is quasi-compact we see that
$X$ is a locally quasi-compact, i.e., every point has a fundamental
system of quasi-compact neighbourhoods, see Topology,
Definition \ref{topology-definition-locally-quasi-compact}.
Hence by Modules, Lemma \ref{modules-lemma-quasi-coherent-module}
for every prime $\mathfrak p \subset R$ corresponding to $x \in X$
there exists an open neighbourhood $x \in U \subset X$ such that
$\mathcal{F}|_U$ is isomorphic to the quasi-coherent
sheaf associated to some $\mathcal{O}_X(U)$-module $M$.
In other words, we get an open covering by $U$'s with this property.
By Lemma \ref{lemma-standard-open} for example we can refine this
covering to  a standard open covering.
Thus we get a covering $\Spec(R) = \bigcup D(f_i)$
and $R_{f_i}$-modules $M_i$ and isomorphisms
$\varphi_i : \mathcal{F}|_{D(f_i)} \to \mathcal{F}_{M_i}$
for some $R_{f_i}$-module $M_i$. On the overlaps
we get isomorphisms
$$
\xymatrix{
\mathcal{F}_{M_i}|_{D(f_if_j)}
\ar[rr]^{\varphi_i^{-1}|_{D(f_if_j)}}
& &
\mathcal{F}|_{D(f_if_j)}
\ar[rr]^{\varphi_j|_{D(f_if_j)}}
& &
\mathcal{F}_{M_j}|_{D(f_if_j)}.
}
$$
Let us denote these $\psi_{ij}$. It is clear that
we have the cocycle condition
$$
\psi_{jk}|_{D(f_if_jf_k)}
\circ
\psi_{ij}|_{D(f_if_jf_k)}
=
\psi_{ik}|_{D(f_if_jf_k)}
$$
on triple overlaps.

\medskip\noindent
Recall that each of the open subspaces $D(f_i)$, $D(f_if_j)$,
$D(f_if_jf_k)$ is an affine scheme. Hence the sheaves $\mathcal{F}_{M_i}$
are isomorphic to the sheaves $\widetilde M_i$ by Lemma
\ref{lemma-compare-constructions} above. In particular we see that
$\mathcal{F}_{M_i}(D(f_if_j)) = (M_i)_{f_j}$, etc.
Also by Lemma \ref{lemma-compare-constructions} above we see
that $\psi_{ij}$ corresponds to a unique $R_{f_if_j}$-module isomorphism
$$
\psi_{ij} : (M_i)_{f_j} \longrightarrow (M_j)_{f_i}
$$
namely, the effect of $\psi_{ij}$ on sections over $D(f_if_j)$.
Moreover these then satisfy the cocycle condition that
$$
\xymatrix{
(M_i)_{f_jf_k}
\ar[rd]_{\psi_{ij}}
\ar[rr]^{\psi_{ik}}
& &
(M_k)_{f_if_j} \\
&
(M_j)_{f_if_k} \ar[ru]_{\psi_{jk}}
}
$$
commutes (for any triple $i, j, k$).

\medskip\noindent
Now Algebra, Lemma \ref{algebra-lemma-glue-modules}
shows that there exist an $R$-module $M$ such that
$M_i = M_{f_i}$ compatible with the morphisms $\psi_{ij}$.
Consider $\mathcal{F}_M = \widetilde M$. At this point it is
a formality to show that $\widetilde M$ is isomorphic to
the quasi-coherent sheaf $\mathcal{F}$ we started out with.
Namely, the sheaves $\mathcal{F}$ and $\widetilde M$ give
rise to isomorphic sets of glueing data of sheaves of $\mathcal{O}_X$-modules
with respect to the covering $X = \bigcup D(f_i)$, see
Sheaves, Section \ref{sheaves-section-glueing-sheaves}
and in particular Lemma \ref{sheaves-lemma-mapping-property-glue}.
Explicitly, in the current situation, this boils down to
the following argument: Let us construct an $R$-module map
$$
M \longrightarrow \Gamma(X, \mathcal{F}).
$$
Namely, given $m \in M$ we get $m_i = m/1 \in M_{f_i} = M_i$
by construction of $M$. By construction of $M_i$ this corresponds
to a section $s_i \in \mathcal{F}(U_i)$. (Namely, $\varphi^{-1}_i(m_i)$.)
We claim that $s_i|_{D(f_if_j)} = s_j|_{D(f_if_j)}$. This is
true because, by construction of $M$, we have $\psi_{ij}(m_i) = m_j$,
and by the construction of the $\psi_{ij}$. By the sheaf condition of
$\mathcal{F}$ this collection of sections gives rise to a unique
section $s$ of $\mathcal{F}$ over $X$. We leave it to the reader
to show that $m \mapsto s$ is a $R$-module map.
By Lemma \ref{lemma-compare-constructions} we obtain an associated
$\mathcal{O}_X$-module map
$$
\widetilde M \longrightarrow \mathcal{F}.
$$
By construction this map reduces to the isomorphisms
$\varphi_i^{-1}$ on each $D(f_i)$ and hence is an isomorphism.
\end{proof}

\begin{lemma}
\label{lemma-equivalence-quasi-coherent}
Let $(X, \mathcal{O}_X) = (\Spec(R), \mathcal{O}_{\Spec(R)})$
be an affine scheme.
The functors $M \mapsto \widetilde M$ and
$\mathcal{F} \mapsto \Gamma(X, \mathcal{F})$ define quasi-inverse
equivalences of categories
$$
\xymatrix{
\QCoh(\mathcal{O}_X)
\ar@<1ex>[r]
&
\text{Mod-}R
\ar@<1ex>[l]
}
$$
between the category of quasi-coherent $\mathcal{O}_X$-modules
and the category of $R$-modules.
\end{lemma}

\begin{proof}
See Lemmas \ref{lemma-compare-constructions}
and \ref{lemma-quasi-coherent-affine} above.
\end{proof}

\noindent
From now on we will not distinguish between quasi-coherent
sheaves on affine schemes and sheaves of the form $\widetilde M$.

\begin{lemma}
\label{lemma-kernel-cokernel-quasi-coherent}
Let $X = \Spec(R)$ be an affine scheme.
Kernels and cokernels of maps of quasi-coherent
$\mathcal{O}_X$-modules are quasi-coherent.
\end{lemma}

\begin{proof}
This follows from the exactness of the functor $\widetilde{\ }$
since by Lemma \ref{lemma-compare-constructions} we know that any map
$\psi : \widetilde{M} \to \widetilde{N}$ comes from
an $R$-module map $\varphi : M \to N$. (So we have
$\Ker(\psi) = \widetilde{\Ker(\varphi)}$ and
$\Coker(\psi) = \widetilde{\Coker(\varphi)}$.)
\end{proof}

\begin{lemma}
\label{lemma-colimit-quasi-coherent}
Let $X = \Spec(R)$ be an affine scheme.
The direct sum of an arbitrary collection of quasi-coherent sheaves
on $X$ is quasi-coherent. The same holds for colimits.
\end{lemma}

\begin{proof}
Suppose $\mathcal{F}_i$, $i \in I$ is a collection of quasi-coherent
sheaves on $X$. By Lemma \ref{lemma-equivalence-quasi-coherent}
above we can write $\mathcal{F}_i = \widetilde{M_i}$ for some $R$-module
$M_i$. Set $M = \bigoplus M_i$. Consider the sheaf $\widetilde{M}$.
For each standard open $D(f)$ we have
$$
\widetilde{M}(D(f)) = M_f =
\left(\bigoplus M_i\right)_f =
\bigoplus M_{i, f}.
$$
Hence we see that the quasi-coherent $\mathcal{O}_X$-module
$\widetilde{M}$ is the direct sum of the sheaves $\mathcal{F}_i$.
A similar argument works for general colimits.
\end{proof}

\begin{lemma}
\label{lemma-extension-quasi-coherent}
Let $(X, \mathcal{O}_X) = (\Spec(R), \mathcal{O}_{\Spec(R)})$
be an affine scheme. Suppose that
$$
0 \to
\mathcal{F}_1 \to
\mathcal{F}_2 \to
\mathcal{F}_3 \to
0
$$
is a short exact sequence of sheaves of $\mathcal{O}_X$-modules.
If two out of three are quasi-coherent then so is the third.
\end{lemma}

\begin{proof}
This is clear in case both $\mathcal{F}_1$ and $\mathcal{F}_2$ are
quasi-coherent because the functor $M \mapsto \widetilde M$
is exact, see Lemma \ref{lemma-spec-sheaves}.
Similarly in case both $\mathcal{F}_2$ and $\mathcal{F}_3$ are
quasi-coherent. Now, suppose that $\mathcal{F}_1 = \widetilde M_1$ and
$\mathcal{F}_3 = \widetilde M_3$ are quasi-coherent.
Set $M_2 = \Gamma(X, \mathcal{F}_2)$. We claim it suffices to show that
the sequence
$$
0 \to M_1 \to M_2 \to M_3 \to 0
$$
is exact. Namely, if this is the case, then (by using the mapping
property of Lemma \ref{lemma-compare-constructions}) we get a commutative
diagram
$$
\xymatrix{
0 \ar[r] &
\widetilde M_1 \ar[r] \ar[d] &
\widetilde M_2 \ar[r] \ar[d] &
\widetilde M_3 \ar[r] \ar[d] &
0 \\
0 \ar[r] &
\mathcal{F}_1 \ar[r] &
\mathcal{F}_2 \ar[r] &
\mathcal{F}_3 \ar[r] &
0
}
$$
and we win by the snake lemma.

\medskip\noindent
The ``correct'' argument here would be to show first
that $H^1(X, \mathcal{F}) = 0$ for any quasi-coherent sheaf $\mathcal{F}$.
This is actually not all that hard, but it is perhaps better to postpone
this till later. Instead we use a small trick.

\medskip\noindent
Pick $m \in M_3 = \Gamma(X, \mathcal{F}_3)$.
Consider the following set
$$
I = \{ f \in R \mid \text{the element }fm\text{ comes from }M_2\}.
$$
Clearly this is an ideal. It suffices to show $1 \in I$.
Hence it suffices to show that for any prime $\mathfrak p$
there exists an $f \in I$, $f \not\in \mathfrak p$.
Let $x \in X$ be the point corresponding to $\mathfrak p$.
Because surjectivity can be checked on stalks
there exists an open neighbourhood $U$ of $x$ such that
$m|_U$ comes from a local section $s \in \mathcal{F}_2(U)$.
In fact we may assume that $U = D(f)$ is a standard open,
i.e., $f \in R$, $f \not \in \mathfrak p$. We will show
that for some $N \gg 0$ we have $f^N \in I$, which
will finish the proof.

\medskip\noindent
Take any point $z \in V(f)$, say corresponding to the
prime $\mathfrak q \subset R$. We can also find a $g \in R$,
$g \not \in \mathfrak q$ such that $m|_{D(g)}$ lifts
to some $s' \in \mathcal{F}_2(D(g))$.
Consider the difference $s|_{D(fg)} - s'|_{D(fg)}$.
This is an element $m'$ of $\mathcal{F}_1(D(fg)) = (M_1)_{fg}$.
For some integer $n = n(z)$ the element $f^n m'$ comes
from some $m'_1 \in (M_1)_g$. We see that
$f^n s$ extends to a section $\sigma$ of $\mathcal{F}_2$ on $D(f) \cup D(g)$
because it agrees with the restriction of
$f^n s' + m'_1$ on $D(f) \cap D(g) = D(fg)$.
Moreover, $\sigma$ maps to the restriction of $f^n m$
to $D(f) \cup D(g)$.

\medskip\noindent
Since $V(f)$ is quasi-compact, there exists a finite list
of elements $g_1, \ldots, g_m \in R$ such that
$V(f) \subset \bigcup D(g_j)$, an integer $n > 0$ and sections
$\sigma_j \in \mathcal{F}_2(D(f) \cup D(g_j))$ such that
$\sigma_j|_{D(f)} = f^n s$ and $\sigma_j$ maps to the section
$f^nm|_{D(f) \cup D(g_j)}$ of $\mathcal{F}_3$.
Consider the differences
$$
\sigma_j|_{D(f) \cup D(g_jg_k)}
-
\sigma_k|_{D(f) \cup D(g_jg_k)}.
$$
These correspond to sections of $\mathcal{F}_1$
over $D(f) \cup D(g_jg_k)$ which are zero
on $D(f)$. In particular their images in
$\mathcal{F}_1(D(g_jg_k)) = (M_1)_{g_jg_k}$
are zero in $(M_1)_{g_jg_kf}$.
Thus some high power of $f$ kills each and every one of these.
In other words, the elements $f^N \sigma_j$, for some $N \gg 0$
satisfy the glueing condition of the sheaf property and
give rise to a section $\sigma $ of $\mathcal{F}_2$
over $\bigcup (D(f) \cup D(g_j)) = X$ as desired.
\end{proof}







\section{Closed subspaces of affine schemes}
\label{section-closed-in-affine}

\begin{example}
\label{example-closed-immersion-affines}
Let $R$ be a ring.
Let $I \subset R$ be an ideal.
Consider the morphism of affine schemes
$i : Z = \Spec(R/I) \to \Spec(R) = X$.
By Algebra, Lemma \ref{algebra-lemma-spec-closed}
this is a homeomorphism of $Z$ onto a closed
subset of $X$. Moreover, if $I \subset \mathfrak p \subset R$
is a prime corresponding to a point $x = i(z)$, $x \in X$,
$z \in Z$, then on stalks we get the map
$$
\mathcal{O}_{X, x} = R_{\mathfrak p}
\longrightarrow
R_{\mathfrak p}/IR_{\mathfrak p} = \mathcal{O}_{Z, z}
$$
Thus we see that $i$ is a closed immersion of locally ringed spaces,
see Definition \ref{definition-closed-immersion-locally-ringed-spaces}.
Clearly, this is (isomorphic) to the closed subspace associated
to the quasi-coherent sheaf of ideals $\widetilde I$, as in
Example \ref{example-closed-subspace}.
\end{example}

\begin{lemma}
\label{lemma-closed-immersion-affine-case}
\begin{slogan}
For affine schemes, closed immersions correspond to ideals.
\end{slogan}
Let $(X, \mathcal{O}_X) = (\Spec(R), \mathcal{O}_{\Spec(R)})$
be an affine scheme. Let $i : Z \to X$ be any closed immersion
of locally ringed spaces. Then there exists a unique ideal
$I \subset R$ such that the morphism $i : Z \to X$ can be identified
with the closed immersion $\Spec(R/I) \to \Spec(R)$
constructed in Example \ref{example-closed-immersion-affines} above.
\end{lemma}

\begin{proof}
This is kind of silly! Namely, by Lemma \ref{lemma-closed-immersion}
we can identify $Z \to X$ with the closed subspace associated to
a sheaf of ideals $\mathcal{I} \subset \mathcal{O}_X$ as in
Definition \ref{definition-closed-subspace} and
Example \ref{example-closed-subspace}.
By our conventions this sheaf of ideals is locally generated
by sections as a sheaf of $\mathcal{O}_X$-modules.
Hence the quotient sheaf $\mathcal{O}_X / \mathcal{I}$
is locally on $X$ the cokernel
of a map $\bigoplus_{j \in J} \mathcal{O}_U \to \mathcal{O}_U$.
Thus by definition, $\mathcal{O}_X / \mathcal{I}$ is quasi-coherent.
By our results in Section \ref{section-quasi-coherent-affine}
it is of the form $\widetilde S$ for some $R$-module $S$.
Moreover, since $\mathcal{O}_X = \widetilde R \to \widetilde S$
is surjective we see by Lemma \ref{lemma-extension-quasi-coherent}
that also $\mathcal{I}$ is quasi-coherent, say $\mathcal{I} = \widetilde I$.
Of course $I \subset R$ and $S = R/I$ and everything is clear.
\end{proof}













\section{Schemes}
\label{section-schemes}

\begin{definition}
\label{definition-scheme}
\begin{history}
In \cite{EGA1} what we call a scheme was called a ``pre-sch\'ema'' and the
name ``sch\'ema'' was reserved for what is a separated scheme in the
Stacks project. In the second edition \cite{EGA1-second} the terminology
was changed to the terminology that is now standard. However, one may
occasionally encounter the terminology ``prescheme'', for example in
\cite{Murre-lectures}.
\end{history}
A {\it scheme} is a locally ringed space with the property that
every point has an open neighbourhood which is an affine scheme.
A {\it morphism of schemes} is a morphism of locally
ringed spaces. The category of schemes will be denoted
$\Sch$.
\end{definition}

\noindent
Let $X$ be a scheme.
We will use the following (very slight) abuse of language.
We will say $U \subset X$ is an {\it affine open}, or an {\it open affine}
if the open subspace $U$ is an affine scheme. We will often
write $U = \Spec(R)$ to indicate that $U$ is isomorphic
to $\Spec(R)$ and moreover that we will identify (temporarily)
$U$ and $\Spec(R)$.

\begin{lemma}
\label{lemma-open-subspace-scheme}
Let $X$ be a scheme. Let $j : U \to X$ be an open immersion
of locally ringed spaces. Then $U$ is a scheme. In particular,
any open subspace of $X$ is a scheme.
\end{lemma}

\begin{proof}
Let $U \subset X$. Let $u \in U$.
Pick an affine open neighbourhood $u \in V \subset X$.
Because standard opens of $V$ form a basis of the topology
on $V$ we see that there exists a $f\in \mathcal{O}_V(V)$
such that $u \in D(f) \subset U$. And $D(f)$ is an affine scheme
by Lemma \ref{lemma-standard-open-affine}. This proves that every point
of $U$ has an open neighbourhood which is affine.
\end{proof}

\noindent
Clearly the lemma (or its proof) shows that any scheme
$X$ has a basis (see Topology, Section \ref{topology-section-bases})
for the topology consisting of affine opens.

\begin{example}
\label{example-not-affine}
Let $k$ be a field.
An example of a scheme which is not affine is
given by the open subspace
$U = \Spec(k[x, y]) \setminus \{ (x, y)\}$
of the affine scheme $X =\Spec(k[x, y])$.
It is covered by two affines, namely $D(x) = \Spec(k[x, y, 1/x])$
and $D(y) = \Spec(k[x, y, 1/y])$ whose intersection is
$D(xy) = \Spec(k[x, y, 1/xy])$. By the sheaf property
for $\mathcal{O}_U$ there is an exact sequence
$$
0 \to
\Gamma(U, \mathcal{O}_U) \to
k[x, y, 1/x] \times k[x, y, 1/y] \to
k[x, y, 1/xy]
$$
We conclude that the map $k[x, y] \to \Gamma(U, \mathcal{O}_U)$
(coming from the morphism $U \to X$) is an isomorphism.
Therefore $U$ cannot be affine since if it was then by
Lemma \ref{lemma-category-affine-schemes} we would have $U \cong X$.
\end{example}











\section{Immersions of schemes}
\label{section-immersions}

\noindent
In Lemma \ref{lemma-open-subspace-scheme} we saw that any open subspace
of a scheme is a scheme. Below we will prove that the same holds for
a closed subspace of a scheme.

\medskip\noindent
Note that the notion of a quasi-coherent sheaf of $\mathcal{O}_X$-modules
is defined for any ringed space $X$ in particular when $X$ is a scheme.
By our efforts in Section \ref{section-quasi-coherent-affine} we
know that such a sheaf is on any affine open $U \subset X$
of the form $\widetilde M$ for some $\mathcal{O}_X(U)$-module $M$.

\begin{lemma}
\label{lemma-closed-subspace-scheme}
Let $X$ be a scheme. Let $i : Z \to X$ be a closed immersion
of locally ringed spaces.
\begin{enumerate}
\item The locally ringed space $Z$ is a scheme,
\item the kernel $\mathcal{I}$ of the map
$\mathcal{O}_X \to i_*\mathcal{O}_Z$ is a quasi-coherent
sheaf of ideals,
\item for any affine open $U = \Spec(R)$ of $X$
the morphism $i^{-1}(U) \to U$ can be identified with
$\Spec(R/I) \to \Spec(R)$ for some ideal $I \subset R$, and
\item we have $\mathcal{I}|_U = \widetilde I$.
\end{enumerate}
In particular, any sheaf of ideals locally generated by sections
is a quasi-coherent sheaf of ideals (and vice versa),
and any closed subspace of $X$ is a scheme.
\end{lemma}

\begin{proof}
Let $i : Z \to X$ be a closed immersion.
Let $z \in Z$ be a point. Choose any affine open
neighbourhood $i(z) \in U \subset X$. Say $U = \Spec(R)$.
By Lemma \ref{lemma-closed-immersion-affine-case} we know
that $i^{-1}(U) \to U$ can be identified with the morphism
of affine schemes $\Spec(R/I) \to \Spec(R)$.
First of all this implies that $z \in i^{-1}(U) \subset Z$ is an
affine neighbourhood of $z$. Thus $Z$ is a scheme. Second
this implies that $\mathcal{I}|_U$ is $\widetilde I$.
In other words for every point $x \in i(Z)$ there exists an
open neighbourhood such that $\mathcal{I}$ is quasi-coherent in
that neighbourhood. Note that $\mathcal{I}|_{X \setminus i(Z)}
\cong \mathcal{O}_{X \setminus i(Z)}$. Thus the restriction
of the sheaf of ideals is quasi-coherent on $X \setminus i(Z)$
also. We conclude that $\mathcal{I}$ is quasi-coherent.
\end{proof}

\begin{definition}
\label{definition-immersion}
Let $X$ be a scheme.
\begin{enumerate}
\item A morphism of schemes is called an {\it open immersion}
if it is an open immersion of locally ringed spaces (see
Definition \ref{definition-immersion-locally-ringed-spaces}).
\item An {\it open subscheme} of $X$ is an open subspace of $X$
in the sense of Definition \ref{definition-open-subspace}; an open subscheme
of $X$ is a scheme by Lemma \ref{lemma-open-subspace-scheme}.
\item A morphism of schemes is called a {\it closed immersion}
if it is a closed immersion of locally ringed spaces (see
Definition \ref{definition-closed-immersion-locally-ringed-spaces}).
\item A {\it closed subscheme} of $X$ is a closed subspace of $X$
in the sense of Definition \ref{definition-closed-subspace}; a closed subscheme
is a scheme by Lemma \ref{lemma-closed-subspace-scheme}.
\item A morphism of schemes $f : X \to Y$ is called an {\it immersion},
or a {\it locally closed immersion} if it can be factored as
$j \circ i$ where $i$ is a closed immersion and $j$ is an open
immersion.
\end{enumerate}
\end{definition}

\noindent
It follows from the lemmas in Sections \ref{section-open-immersion} and
\ref{section-closed-immersion} that any open (resp.\ closed) immersion of
schemes is isomorphic to the inclusion of an open (resp.\ closed) subscheme
of the target.

\medskip\noindent
Our definition of a closed immersion is halfway between Hartshorne
and EGA. Hartshorne defines a closed immersion as a morphism $f : X \to Y$ of
schemes which induces a homeomorphism of $X$ onto a closed subset of
$Y$ such that $f^\# : \mathcal{O}_Y \to f_*\mathcal{O}_X$ is surjective, see
\cite[Page 85]{H}. We will show this is equivalent to our notion in
Lemma \ref{lemma-characterize-closed-immersions}. In \cite{EGA},
Grothendieck and Dieudonn\'e first define closed subschemes
via the construction of Example \ref{example-closed-subspace}
using quasi-coherent sheaves of ideals and then define a closed immersion
as a morphism $f : X \to Y$ which induces an isomorphism with a
closed subscheme. It follows from Lemma \ref{lemma-closed-subspace-scheme}
that this agrees with our notion.

\medskip\noindent
Pedagogically speaking the definition above is a disaster/nightmare.
In teaching this material to students, we have found it often
convenient to define a closed immersion as an affine morphism
$f : X \to Y$ of schemes such that $f^\# : \mathcal{O}_Y \to f_*\mathcal{O}_X$
is surjective. Namely, it turns out that the notion of an affine
morphism (Morphisms, Section \ref{morphisms-section-affine})
is quite natural and easy to understand.

\medskip\noindent
For more information on closed immersions we suggest the reader visit
Morphisms, Sections \ref{morphisms-section-closed-immersions} and
\ref{morphisms-section-closed-immersions-quasi-coherent}.

\medskip\noindent
We will discuss locally closed subschemes and immersions
at the end of this section.

\begin{remark}
\label{remark-not-reverse-open-closed}
If $f : X \to Y$ is an immersion of schemes, then it is in general
not possible to factor $f$ as an open immersion followed
by a closed immersion. See Morphisms, Example \ref{morphisms-example-thibaut}.
\end{remark}

\begin{lemma}
\label{lemma-immersion-when-closed}
Let $f : Y \to X$ be an immersion of schemes. Then $f$ is a closed
immersion if and only if $f(Y) \subset X$ is a closed subset.
\end{lemma}

\begin{proof}
If $f$ is a closed immersion then $f(Y)$ is closed by definition.
Conversely, suppose that $f(Y)$ is closed. By definition
there exists an open subscheme $U \subset X$ such that $f$ is the composition
of a closed immersion $i : Y \to U$ and the open immersion
$j : U \to X$. Let $\mathcal{I} \subset \mathcal{O}_U$ be the
quasi-coherent sheaf of ideals associated to the closed immersion
$i$. Note that
$\mathcal{I}|_{U \setminus i(Y)}
= \mathcal{O}_{U \setminus i(Y)}
= \mathcal{O}_{X \setminus i(Y)}|_{U \setminus i(Y)}$.
Thus we may glue (see Sheaves, Section \ref{sheaves-section-glueing-sheaves})
$\mathcal{I}$ and $\mathcal{O}_{X \setminus i(Y)}$ to a sheaf of
ideals $\mathcal{J} \subset \mathcal{O}_X$. Since every
point of $X$ has a neighbourhood where $\mathcal{J}$ is
quasi-coherent, we see that $\mathcal{J}$ is quasi-coherent
(in particular locally generated by sections).
By construction $\mathcal{O}_X/\mathcal{J}$ is supported
on $U$ and equal to $\mathcal{O}_U/\mathcal{I}$.
Thus we see that the closed subspaces associated to $\mathcal{I}$
and $\mathcal{J}$ are canonically isomorphic, see
Example \ref{example-closed-subspace}.
In particular the closed subspace of $U$ associated to $\mathcal{I}$
is isomorphic to a closed subspace of $X$.
Since $Y \to U$ is identified with the closed subspace
associated to $\mathcal{I}$, see Lemma \ref{lemma-closed-immersion},
we conclude that $Y \to U \to X$
is a closed immersion.
\end{proof}

\noindent
Let $f : Y \to X$ be an immersion.
Let $Z = \overline{f(Y)} \setminus f(Y)$ which is a closed subset of $X$.
Let $U = X \setminus Z$.
The lemma implies that $U$ is the biggest open subspace of $X$
such that $f : Y \to X$ factors through a closed immersion into $U$.
We define a {\it locally closed subscheme of $X$} as a pair
$(Z, U)$ consisting of a closed subscheme $Z$ of an open subscheme $U$
of $X$ such that in addition $\overline{Z} \cup U = X$. We usually
just say ``let $Z$ be a locally closed subscheme of $X$'' since we may
recover $U$ from the morphism $Z \to X$. The above
then shows that any immersion $f : Y \to X$ factors uniquely
as $Y \to Z \to X$ where $Z$ is a locally closed subspace of $X$
and $Y \to Z$ is an isomorphism.

\medskip\noindent
The interest of this is that the collection of locally closed subschemes
of $X$ forms a set. We may define a partial ordering on this set,
which we call inclusion for obvious reasons. To be explicit, if
$Z \to X$ and $Z' \to X$ are two locally closed subschemes of $X$, then
we say that {\it $Z$ is contained in $Z'$} simply if the morphism
$Z \to X$ factors through $Z'$. If it does, then of course $Z$ is identified
with a unique locally closed subscheme of $Z'$, and so on.

\medskip\noindent
For more information on immersions, we refer the reader to
Morphisms, Section \ref{morphisms-section-immersions}.










\section{Zariski topology of schemes}
\label{section-topology}

\noindent
See Topology, Section \ref{topology-section-introduction}
for some basic material in topology adapted to the Zariski
topology of schemes.

\begin{lemma}
\label{lemma-scheme-sober}
Let $X$ be a scheme.
Any irreducible closed subset of $X$ has a unique generic point.
In other words, $X$ is a sober topological space, see
Topology, Definition \ref{topology-definition-generic-point}.
\end{lemma}

\begin{proof}
Let $Z \subset X$ be an irreducible closed subset.
For every affine open $U \subset X$, $U = \Spec(R)$
we know that $Z \cap U = V(I)$ for a unique
radical ideal $I \subset R$. Note that $Z \cap U$ is either
empty or irreducible. In the second case (which occurs
for at least one $U$) we see that $I = \mathfrak p$
is a prime ideal, which is a generic point $\xi$ of $Z \cap U$.
It follows that $Z = \overline{\{\xi\}}$, in other words
$\xi$ is a generic point of $Z$. If $\xi'$ was a second
generic point, then $\xi' \in Z \cap U$ and it follows
immediately that $\xi' = \xi$.
\end{proof}

\begin{lemma}
\label{lemma-basis-affine-opens}
Let $X$ be a scheme. The collection of affine opens
of $X$ forms a basis for the topology on $X$.
\end{lemma}

\begin{proof}
This follows from the discussion on open subschemes
in Section \ref{section-schemes}.
\end{proof}

\begin{remark}
\label{remark-intersection-affine-opens}
In general the intersection of two affine opens in $X$
is not affine open. See Example \ref{example-affine-space-zero-doubled}.
\end{remark}

\begin{lemma}
\label{lemma-locally-quasi-compact}
The underlying topological space of any scheme is
locally quasi-compact, see
Topology, Definition \ref{topology-definition-locally-quasi-compact}.
\end{lemma}

\begin{proof}
This follows from Lemma \ref{lemma-basis-affine-opens} above
and the fact that the spectrum of ring is quasi-compact, see
Algebra, Lemma \ref{algebra-lemma-quasi-compact}.
\end{proof}

\begin{lemma}
\label{lemma-standard-open-two-affines}
Let $X$ be a scheme.
Let $U, V$ be affine opens of $X$, and let $x \in U \cap V$.
There exists an affine open neighbourhood $W$ of $x$
such that $W$ is a standard open of both $U$ and $V$.
\end{lemma}

\begin{proof}
Write $U = \Spec(A)$ and $V = \Spec(B)$.
Say $x$ corresponds to the prime $\mathfrak p \subset A$
and the prime $\mathfrak q \subset B$.
We may choose a $f \in A$, $f \not \in \mathfrak p$ such that
$D(f) \subset U \cap V$. Note that any standard open of $D(f)$
is a standard open of $\Spec(A) = U$. Hence we may assume
that $U \subset V$. In other words, now we may think of $U$
as an affine open of $V$. Next we choose a
$g \in B$, $g \not \in \mathfrak q$ such that
$D(g) \subset U$. In this case we see that $D(g) = D(g_A)$
where $g_A \in A$ denotes the image of $g$ by the map $B \to A$.
Thus the lemma is proved.
\end{proof}

\begin{lemma}
\label{lemma-good-subcover}
Let $X$ be a scheme.
Let $X = \bigcup_i U_i$ be an affine open covering.
Let $V \subset X$ be an affine open.
There exists a standard open covering
$V = \bigcup_{j = 1, \ldots, m} V_j$ (see
Definition \ref{definition-standard-covering})
such that each $V_j$ is a standard open in one of the $U_i$.
\end{lemma}

\begin{proof}
Pick $v \in V$. Then $v \in U_i$ for some $i$.
By Lemma \ref{lemma-standard-open-two-affines} above there exists an open
$v \in W_v \subset V \cap U_i$ such that
$W_v$ is a standard open in both $V$ and $U_i$.
Since $V$ is quasi-compact the lemma follows.
\end{proof}

\begin{lemma}
\label{lemma-sheaf-on-affines}
Let $X$ be a scheme. Let $\mathcal{B}$ be the set of affine opens of $X$.
Let $\mathcal{F}$ be a presheaf of sets on $\mathcal{B}$, see
Sheaves, Definition \ref{sheaves-definition-presheaf-basis}. The following
are equivalent
\begin{enumerate}
\item $\mathcal{F}$ is the restriction of a sheaf on $X$ to $\mathcal{B}$,
\item $\mathcal{F}$ is a sheaf on $\mathcal{B}$, and
\item $\mathcal{F}(\emptyset)$ is a singleton and
whenever $U = V \cup W$ with $U, V, W \in \mathcal{B}$ and
$V, W \subset U$ standard open
(Algebra, Definition \ref{algebra-definition-Zariski-topology})
the map
$$
\mathcal{F}(U) \longrightarrow \mathcal{F}(V) \times \mathcal{F}(W)
$$
is injective with image the set of pairs $(s, t)$
such that $s|_{V \cap W} = t|_{V \cap W}$.
\end{enumerate}
\end{lemma}

\begin{proof}
The equivalence of (1) and (2) is
Sheaves, Lemma \ref{sheaves-lemma-restrict-basis-equivalence}.
It is clear that (2) implies (3).
Hence it suffices to prove that (3) implies (2). By
Sheaves, Lemma \ref{sheaves-lemma-cofinal-systems-coverings-standard-case}
and Lemma \ref{lemma-standard-open} it suffices to prove the sheaf
condition holds for standard open coverings
(Definition \ref{definition-standard-covering})
of elements of $\mathcal{B}$.
Let $U = U_1 \cup \ldots \cup U_n$ be a standard open covering
with $U \subset X$ affine open.
We will prove the sheaf condition
for this covering by induction on $n$.
If $n = 0$, then $U$ is empty and we get the sheaf condition
by assumption. If $n = 1$, then there is nothing to prove.
If $n = 2$, then this is assumption (3).
If $n > 2$, then we write $U_i = D(f_i)$ for $f_i \in A = \mathcal{O}_X(U)$.
Suppose that $s_i \in \mathcal{F}(U_i)$ are sections such that
$s_i|_{U_i \cap U_j} = s_j|_{U_i \cap U_j}$ for all $1 \leq i < j \leq n$.
Since $U = U_1 \cup \ldots \cup U_n$ we have
$1 = \sum_{i = 1, \ldots, n} a_i f_i$ in $A$ for some $a_i \in A$, see
Algebra, Lemma \ref{algebra-lemma-Zariski-topology}.
Set $g = \sum_{i = 1, \ldots, n - 1} a_if_i$.
Then $U = D(g) \cup D(f_n)$.
Observe that $D(g) = D(gf_1) \cup \ldots \cup D(gf_{n - 1})$
is a standard open covering. By induction there is a unique section
$s' \in \mathcal{F}(D(g))$ which agrees with
$s_i|_{D(gfi)}$ for $i = 1, \ldots, n - 1$.
We claim that $s'$ and $s_n$ have the same restriction to
$D(gf_n)$. This is true by induction and the covering
$D(gf_n) = D(gf_nf_1) \cup \ldots \cup D(gf_nf_{n - 1})$.
Thus there is a unique section $s \in \mathcal{F}(U)$
whose restriction to $D(g)$ is $s'$ and whose restriction
to $D(f_n)$ is $s_n$. We omit the verification that $s$
restricts to $s_i$ on $D(f_i)$ for $i = 1, \ldots, n - 1$
and we omit the verification that $s$ is unique.
\end{proof}

\begin{lemma}
\label{lemma-scheme-finite-discrete-affine}
Let $X$ be a scheme whose underlying topological space
is a finite discrete set.
Then $X$ is affine.
\end{lemma}

\begin{proof}
Say $X = \{x_1, \ldots, x_n\}$. Then $U_i = \{x_i\}$ is an open neighbourhood
of $x_i$. By
Lemma \ref{lemma-basis-affine-opens}
it is affine. Hence $X$ is a finite disjoint union of affine schemes, and
hence is affine by
Lemma \ref{lemma-disjoint-union-affines}.
\end{proof}

\begin{example}
\label{example-scheme-without-closed-points}
There exists a scheme without closed points.
Namely, let $R$ be a local domain whose spectrum
looks like
$(0) = \mathfrak p_0 \subset \mathfrak p_1 \subset \mathfrak p_2
\subset \ldots \subset \mathfrak m$. Then the open subscheme
$\Spec(R) \setminus \{\mathfrak m\}$ does not have
a closed point. To see that such a ring $R$ exists, we use
that given any totally ordered group $(\Gamma, \geq)$
there exists a valuation ring $A$ with valuation group $(\Gamma, \geq)$,
see \cite{Krull}. See Algebra, Section \ref{algebra-section-valuation-rings}
for notation. We take
$\Gamma = \mathbf{Z}x_1 \oplus \mathbf{Z}x_2 \oplus \mathbf{Z}x_3 \oplus
\ldots$ and we define $\sum_i a_i x_i \geq 0$ if and only
if the first nonzero $a_i$ is $> 0$, or all $a_i = 0$.
So $x_1 \geq x_2 \geq x_3 \geq  \ldots \geq 0$.
The subsets $x_i + \Gamma_{\geq 0}$ are prime ideals
of $(\Gamma, \geq)$, see Algebra, notation above
Lemma \ref{algebra-lemma-ideals-valuation-ring}. These together
with $\emptyset$ and $\Gamma_{\geq 0}$ are the only prime ideals.
Hence $A$ is an example of a ring with the given structure of
its spectrum, by
Algebra, Lemma \ref{algebra-lemma-ideals-valuation-ring}.
\end{example}



















\section{Reduced schemes}
\label{section-reduced}

\begin{definition}
\label{definition-reduced}
Let $X$ be a scheme. We say $X$ is {\it reduced} if every local ring
$\mathcal{O}_{X, x}$ is reduced.
\end{definition}

\begin{lemma}
\label{lemma-reduced}
A scheme $X$ is reduced if and only if $\mathcal{O}_X(U)$
is a reduced ring for all $U \subset X$ open.
\end{lemma}

\begin{proof}
Assume that $X$ is reduced.
Let $f \in \mathcal{O}_X(U)$ be a section such that $f^n = 0$.
Then the image of $f$ in $\mathcal{O}_{U, u}$ is zero for
all $u \in U$. Hence $f$ is zero, see
Sheaves, Lemma \ref{sheaves-lemma-sheaf-subset-stalks}.
Conversely, assume that $\mathcal{O}_X(U)$ is reduced
for all opens $U$. Pick any nonzero element $f \in \mathcal{O}_{X, x}$.
Any representative $(U, f \in \mathcal{O}(U))$  of $f$ is nonzero and
hence not nilpotent. Hence $f$ is not nilpotent in $\mathcal{O}_{X, x}$.
\end{proof}

\begin{lemma}
\label{lemma-affine-reduced}
An affine scheme $\Spec(R)$ is reduced
if and only if $R$ is reduced.
\end{lemma}

\begin{proof}
The direct implication follows immediately from
Lemma \ref{lemma-reduced} above.
In the other direction it follows since any localization of
a reduced ring is reduced, and in particular the local rings
of a reduced ring are reduced.
\end{proof}

\begin{lemma}
\label{lemma-reduced-closed-subscheme}
Let $X$ be a scheme. Let $T \subset X$ be a closed subset.
There exists a unique closed subscheme $Z \subset X$ with
the following properties: (a) the underlying topological
space of $Z$ is equal to $T$, and (b) $Z$ is reduced.
\end{lemma}

\begin{proof}
Let $\mathcal{I} \subset \mathcal{O}_X$ be the sub presheaf
defined by the rule
$$
\mathcal{I}(U) = \{f \in \mathcal{O}_X(U) \mid
f(t) = 0\text{ for all }t \in T\cap U\}
$$
Here we use $f(t)$ to indicate the image of
$f$ in the residue field $\kappa(t)$ of $X$ at $t$.
Because of the local nature of the condition it is
clear that $\mathcal{I}$ is a sheaf of ideals. Moreover,
let $U = \Spec(R)$ be an affine open.
We may write $T \cap U = V(I)$ for a unique radical
ideal $I \subset R$. Given a prime $\mathfrak p \in V(I)$
corresponding to $t \in T \cap U$ and an element $f \in R$ we have
$f(t) = 0 \Leftrightarrow f \in \mathfrak p$.
Hence $\mathcal{I}(U) = \cap_{\mathfrak p \in V(I)} \mathfrak p
= I$ by Algebra, Lemma \ref{algebra-lemma-Zariski-topology}.
Moreover, for any standard open $D(g) \subset \Spec(R) = U$
we have $\mathcal{I}(D(g)) = I_g$ by the same reasoning.
Thus $\widetilde I$ and $\mathcal{I}|_U$ agree (as ideals)
on a basis of opens and hence are equal. Therefore
$\mathcal{I}$ is a quasi-coherent sheaf of ideals.

\medskip\noindent
At this point we may define $Z$ as the closed subspace
associated to the sheaf of ideals $\mathcal{I}$. For every
affine open $U = \Spec(R)$ of $X$ we see that
$Z \cap U = \Spec(R/I)$ where $I$ is a radical ideal and
hence $Z$ is reduced (by Lemma \ref{lemma-affine-reduced} above).
By construction the underlying closed subset of $Z$ is $T$.
Hence we have found a closed subscheme with properties (a) and (b).

\medskip\noindent
Let $Z' \subset X$ be a second closed subscheme with properties (a) and (b).
For every affine open $U = \Spec(R)$ of $X$ we see that
$Z' \cap U = \Spec(R/I')$ for some ideal $I' \subset R$.
By Lemma \ref{lemma-affine-reduced} the ring $R/I'$ is
reduced and hence $I'$ is radical. Since $V(I') = T \cap U = V(I)$
we deduced that $I = I'$ by
Algebra, Lemma \ref{algebra-lemma-Zariski-topology}.
Hence $Z'$ and $Z$ are defined by the same sheaf of ideals
and hence are equal.
\end{proof}

\begin{definition}
\label{definition-reduced-induced-scheme}
Let $X$ be a scheme. Let $Z \subset X$ be a closed subset.
A {\it scheme structure on $Z$} is given by a closed subscheme $Z'$ of
$X$ whose underlying set is equal to $Z$. We often say
``let $(Z, \mathcal{O}_Z)$ be a scheme structure on $Z$'' to
indicate this. The {\it reduced induced scheme structure}
on $Z$ is the one constructed in Lemma \ref{lemma-reduced-closed-subscheme}.
The {\it reduction $X_{red}$ of $X$} is the reduced induced scheme
structure on $X$ itself.
\end{definition}

\noindent
Often when we say ``let $Z \subset X$ be an irreducible component of $X$''
we think of $Z$ as a reduced closed subscheme of $X$ using the reduced induced
scheme structure.

\begin{remark}
\label{remark-reduced-induced-locally-closed}
Let $X$ be a scheme. Let $T \subset X$ be a locally closed subset.
In this situation we sometimes also use the phrase
``reduced induced scheme structure on $T$''. It refers
to the reduced induced scheme structure from
Definition \ref{definition-reduced-induced-scheme}
when we view $T$ as a closed subset of the open subscheme 
$X \setminus \partial T$ of $X$. Here
$\partial T = \overline{T} \setminus T$ is the ``boundary'' of $T$
in the topological space of $X$.
\end{remark}

\begin{lemma}
\label{lemma-map-into-reduction}
Let $X$ be a scheme.
Let $Z \subset X$ be a closed subscheme.
Let $Y$ be a reduced scheme.
A morphism $f : Y \to X$ factors through $Z$ if and only if
$f(Y) \subset Z$ (set theoretically). In particular, any
morphism $Y \to X$ factors as $Y \to X_{red} \to X$.
\end{lemma}

\begin{proof}
Assume $f(Y) \subset Z$ (set theoretically).
Let $\mathcal{I} \subset \mathcal{O}_X$ be the ideal sheaf of $Z$.
For any affine opens $U \subset X$, $\Spec(B) = V \subset Y$
with $f(V) \subset U$ and any $g \in \mathcal{I}(U)$
the pullback $b = f^\sharp(g) \in \Gamma(V, \mathcal{O}_Y) = B$
maps to zero in the residue field of any $y \in V$.
In other words $b \in \bigcap_{\mathfrak p \subset B} \mathfrak p$.
This implies $b = 0$ as $B$ is reduced (Lemma \ref{lemma-reduced}, and
Algebra, Lemma \ref{algebra-lemma-Zariski-topology}).
Hence $f$ factors through
$Z$ by Lemma \ref{lemma-characterize-closed-subspace}.
\end{proof}










\section{Points of schemes}
\label{section-points}

\noindent
Given a scheme $X$ we can define a functor
$$
h_X : \Sch^{opp}
\longrightarrow
\textit{Sets}, \quad
T \longmapsto \Mor(T, X).
$$
See Categories, Example \ref{categories-example-hom-functor}.
This is called the {\it functor of points of $X$}.
A fun part of scheme theory is to find descriptions of
the internal geometry of $X$ in terms of this functor $h_X$.
In this section we find a simple way to describe
points of $X$.

\medskip\noindent
Let $X$ be a scheme. Let $R$ be a local ring with maximal ideal
$\mathfrak m \subset R$. Suppose that $f : \Spec(R) \to X$
is a morphism of schemes. Let $x \in X$ be the image of the closed point
$\mathfrak m \in \Spec(R)$. Then we obtain a local homomorphism
of local rings
$$
f^\sharp :
\mathcal{O}_{X, x}
\longrightarrow
\mathcal{O}_{\Spec(R), \mathfrak m} = R.
$$

\begin{lemma}
\label{lemma-morphism-from-spec-local-ring}
Let $X$ be a scheme. Let $R$ be a local ring.
The construction above gives a bijective correspondence
between morphisms $\Spec(R) \to X$ and pairs
$(x, \varphi)$ consisting of a point $x \in X$ and
a local homomorphism of local rings $\varphi : \mathcal{O}_{X, x} \to R$.
\end{lemma}

\begin{proof}
Let $A$ be a ring. For any ring homomorphism $\psi : A \to R$
there exists a unique prime ideal $\mathfrak p \subset A$
and a factorization $A \to A_{\mathfrak p} \to R$ where the
last map is a local homomorphism of local rings. Namely,
$\mathfrak p = \psi^{-1}(\mathfrak m)$. Via
Lemma \ref{lemma-morphism-into-affine}
this proves that the lemma holds if $X$ is an affine scheme.

\medskip\noindent
Let $X$ be a general scheme. Any $x \in X$ is contained in
an open affine $U \subset X$. By the affine case we conclude that every pair
$(x, \varphi)$ occurs as the end product of the construction
above the lemma.

\medskip\noindent
To finish the proof it suffices to show that any morphism
$f : \Spec(R) \to X$ has image contained in any affine
open containing the image $x$ of the closed
point of $\Spec(R)$. In fact, let $x \in V \subset X$
be any open neighbourhood containing $x$. Then
$f^{-1}(V) \subset \Spec(R)$ is an open containing
the unique closed point and hence equal to $\Spec(R)$.
\end{proof}

\noindent
As a special case of the lemma above we obtain for every
point $x$ of a scheme $X$ a canonical morphism
\begin{equation}
\label{equation-canonical-morphism}
\Spec(\mathcal{O}_{X, x}) \longrightarrow X
\end{equation}
corresponding to the identity map on the local ring of $X$ at $x$.
We may reformulate the lemma above as saying that for any
morphism $f : \Spec(R) \to X$ there exists a unique point
$x \in X$ such that $f$ factors as
$\Spec(R) \to \Spec(\mathcal{O}_{X, x}) \to X$
where the first map comes from a local homomorphism
$\mathcal{O}_{X, x} \to R$.

\medskip\noindent
In case we have a morphism of schemes $f : X \to S$,
and a point $x$ mapping to a point $s \in S$ we obtain
a commutative diagram
$$
\xymatrix{
\Spec(\mathcal{O}_{X, x}) \ar[r] \ar[d] & X \ar[d] \\
\Spec(\mathcal{O}_{S, s}) \ar[r] & S
}
$$
where the left vertical map corresponds to the local ring map
$f^\sharp_x : \mathcal{O}_{S, s} \to \mathcal{O}_{X, x}$.

\begin{lemma}
\label{lemma-specialize-points}
Let $X$ be a scheme.
Let $x, x' \in X$ be points of $X$.
Then $x' \in X$ is a generalization of $x$ if and only if
$x'$ is in the image of the canonical morphism
$\Spec(\mathcal{O}_{X, x}) \to X$.
\end{lemma}

\begin{proof}
A continuous map preserves the relation of specialization/generalization.
Since every point of $\Spec(\mathcal{O}_{X, x})$ is a
generalization of the closed point we see every point in the image
of $\Spec(\mathcal{O}_{X, x}) \to X$ is a generalization of $x$.
Conversely, suppose that $x'$ is a generalization of $x$.
Choose an affine open neighbourhood $U = \Spec(R)$ of
$x$. Then $x' \in U$. Say $\mathfrak p \subset R$ and
$\mathfrak p' \subset R$ are the primes corresponding
to $x$ and $x'$. Since $x'$ is a generalization of $x$
we see that $\mathfrak p' \subset \mathfrak p$. This means
that $\mathfrak p'$ is in the image of the morphism
$\Spec(\mathcal{O}_{X, x}) = \Spec(R_{\mathfrak p})
\to \Spec(R) = U \subset X$ as desired.
\end{proof}

\noindent
Now, let us discuss morphisms from spectra of fields.
Let $(R, \mathfrak m, \kappa)$ be a local ring
with maximal ideal $\mathfrak m$ and residue field $\kappa$.
Let $K$ be a field. A local homomorphism $R \to K$ by definition
factors as $R \to \kappa \to K$, i.e., is the same thing as
a morphism $\kappa \to K$. Thus we see that morphisms
$$
\Spec(K) \longrightarrow X
$$
correspond to pairs $(x, \kappa(x) \to K)$. We may define
a preorder on morphisms of spectra of fields to $X$
by saying that $\Spec(K) \to X$ dominates
$\Spec(L) \to X$ if $\Spec(K) \to X$
factors through $\Spec(L) \to X$. This suggests
the following notion: Let us temporarily
say that two morphisms $p : \Spec(K) \to X$ and
$q : \Spec(L) \to X$ are {\it equivalent} if there exists
a third field $\Omega$ and a commutative diagram
$$
\xymatrix{
\Spec(\Omega) \ar[r] \ar[d] &
\Spec(L) \ar[d]^q \\
\Spec(K) \ar[r]^p &
X
}
$$
Of course this immediately implies that the unique points of
all three of the schemes $\Spec(K)$,
$\Spec(L)$, and $\Spec(\Omega)$
map to the same $x \in X$. Thus a diagram (by the remarks above)
corresponds to a point $x \in X$ and a commutative diagram
$$
\xymatrix{
\Omega &
L \ar[l] \\
K \ar[u] &
\kappa(x) \ar[l] \ar[u]
}
$$
of fields. This defines an equivalence relation, because given
any set of extensions $\kappa \subset K_i$ there exists
some field extension $\kappa \subset \Omega$ such that all
the field extensions $K_i$ are contained in the extension $\Omega$.

\begin{lemma}
\label{lemma-characterize-points}
Let $X$ be a scheme. Points of $X$ correspond bijectively
to equivalence classes of morphisms from spectra of
fields into $X$. Moreover, each equivalence class contains
a (unique up to unique isomorphism) smallest element
$\Spec(\kappa(x)) \to X$.
\end{lemma}

\begin{proof}
Follows from the discussion above.
\end{proof}

\noindent
Of course the morphisms $\Spec(\kappa(x)) \to X$
factor through the canonical morphisms
$\Spec(\mathcal{O}_{X, x}) \to X$.
And the content of Lemma \ref{lemma-specialize-points} is in
this setting that the morphism $\Spec(\kappa(x')) \to X$
factors as
$\Spec(\kappa(x')) \to \Spec(\mathcal{O}_{X, x}) \to X$
whenever $x'$ is a generalization of $x$.
In case we have a morphism of schemes $f : X \to S$,
and a point $x$ mapping to a point $s \in S$ we obtain
a commutative diagram
$$
\xymatrix{
\Spec(\kappa(x)) \ar[r] \ar[d] &
\Spec(\mathcal{O}_{X, x}) \ar[r] \ar[d] &
X \ar[d] \\
\Spec(\kappa(s)) \ar[r] &
\Spec(\mathcal{O}_{S, s}) \ar[r] &
S.
}
$$










\section{Glueing schemes}
\label{section-glueing-schemes}

\noindent
Let $I$ be a set.
For each $i \in I$ let $(X_i, \mathcal{O}_i)$ be
a locally ringed space. (Actually the construction that
follows works equally well for ringed spaces.)
For each pair $i, j \in I$ let $U_{ij} \subset X_i$
be an open subspace.
For each pair $i, j \in I$, let
$$
\varphi_{ij} : U_{ij} \to U_{ji}
$$
be an isomorphism of locally ringed spaces.
For convenience we assume that $U_{ii} = X_i$ and
$\varphi_{ii} = \text{id}_{X_i}$.
For each triple $i, j, k \in I$ assume that
\begin{enumerate}
\item we have
$\varphi_{ij}^{-1}(U_{ji} \cap U_{jk}) =  U_{ij} \cap U_{ik}$, and
\item the diagram
$$
\xymatrix{
U_{ij} \cap U_{ik} \ar[rr]_{\varphi_{ik}} \ar[rd]_{\varphi_{ij}} & &
U_{ki} \cap U_{kj} \\
& U_{ji} \cap U_{jk} \ar[ru]_{\varphi_{jk}}
}
$$
is commutative.
\end{enumerate}
Let us call a collection
$(I, (X_i)_{i\in I}, (U_{ij})_{i, j\in I}, (\varphi_{ij})_{i, j\in I})$
satisfying the conditions above a glueing data.

\begin{lemma}
\label{lemma-glue}
\begin{slogan}
If you have two locally ringed spaces, and a subspace of the first one
is isomorphic to a subspace of the other, then you can glue them together
into one big locally ringed space.
\end{slogan}
Given any glueing data of locally ringed spaces there
exists a locally ringed space $X$ and open subspaces
$U_i \subset X$ together with isomorphisms
$\varphi_i : X_i \to U_i$ of locally ringed spaces such that
\begin{enumerate}
\item $\varphi_i(U_{ij}) = U_i \cap U_j$, and
\item $\varphi_{ij} =
\varphi_j^{-1}|_{U_i \cap U_j} \circ \varphi_i|_{U_{ij}}$.
\end{enumerate}
The locally ringed space $X$ is characterized by the following
mapping properties: Given a locally ringed space $Y$ we have
\begin{eqnarray*}
\Mor(X, Y) & = & \{ (f_i)_{i\in I} \mid
f_i : X_i \to Y, \ f_j \circ \varphi_{ij} = f_i|_{U_{ij}}\} \\
f & \mapsto & (f|_{U_i} \circ \varphi_i)_{i \in I} \\
\Mor(Y, X) & = &
\left\{
\begin{matrix}
\text{open covering }Y = \bigcup\nolimits_{i \in I} V_i\text{ and }
(g_i : V_i \to X_i)_{i \in I}
\text{ such that}\\
g_i^{-1}(U_{ij}) = V_i \cap V_j
\text{ and }
g_j|_{V_i \cap V_j} = \varphi_{ij} \circ g_i|_{V_i \cap V_j}
\end{matrix}
\right\} \\
g & \mapsto &
V_i = g^{-1}(U_i), \ g_i = \varphi_i^{-1} \circ g|_{V_i}
\end{eqnarray*}
\end{lemma}

\begin{proof}
We construct $X$ in stages.
As a set we take
$$
X = (\coprod X_i) / \sim.
$$
Here given $x \in X_i$ and $x' \in X_j$ we say
$x \sim x'$ if and only if $x \in U_{ij}$, $x' \in U_{ji}$
and $\varphi_{ij}(x) = x'$. This is an equivalence relation
since if $x \in X_i$, $x' \in X_j$, $x'' \in X_k$, and $x \sim x'$ and
$x' \sim x''$, then $x' \in U_{ji} \cap U_{jk}$, hence by condition (1) of
a glueing data also $x \in U_{ij} \cap U_{ik}$ and
$x'' \in U_{ki} \cap U_{kj}$ and by condition (2)
we see that $\varphi_{ik}(x) = x''$. (Reflexivity and symmetry
follows from our assumptions that $U_{ii} = X_i$ and
$\varphi_{ii} = \text{id}_{X_i}$.)
Denote $\varphi_i : X_i \to X$
the natural maps. Denote $U_i = \varphi_i(X_i) \subset X$.
Note that $\varphi_i : X_i \to U_i$ is a bijection.

\medskip\noindent
The topology on $X$ is defined by the rule that
$U \subset X$ is open if and only if $\varphi_i^{-1}(U)$
is open for all $i$. We leave it to the reader to verify
that this does indeed define a topology.
Note that in particular $U_i$ is open since $\varphi_j^{-1}(U_i)
= U_{ji}$ which is open in $X_j$ for all $j$.
Moreover, for any open set $W \subset X_i$ the image
$\varphi_i(W) \subset U_i$ is open because
$\varphi_j^{-1}(\varphi_i(W)) = \varphi_{ji}^{-1}(W \cap U_{ij})$.
Therefore $\varphi_i : X_i \to U_i$ is a homeomorphism.

\medskip\noindent
To obtain a locally ringed space we have to construct the
sheaf of rings $\mathcal{O}_X$. We do this by glueing the
sheaves of rings $\mathcal{O}_{U_i} := \varphi_{i, *} \mathcal{O}_i$.
Namely, in the commutative diagram
$$
\xymatrix{
U_{ij} \ar[rr]_{\varphi_{ij}} \ar[rd]_{\varphi_i|_{U_{ij}}}
& &
U_{ji} \ar[ld]^{\varphi_j|_{U_{ji}}} \\
& U_i \cap U_j &
}
$$
the arrow on top is an isomorphism of ringed spaces,
and hence we get unique isomorphisms of sheaves of rings
$$
\mathcal{O}_{U_i}|_{U_i \cap U_j}
\longrightarrow
\mathcal{O}_{U_j}|_{U_i \cap U_j}.
$$
These satisfy a cocycle condition as in Sheaves,
Section \ref{sheaves-section-glueing-sheaves}.
By the results of that section we obtain a sheaf of rings
$\mathcal{O}_X$ on $X$ such that $\mathcal{O}_X|_{U_i}$
is isomorphic to $\mathcal{O}_{U_i}$ compatibly with
the glueing maps displayed above.
In particular $(X, \mathcal{O}_X)$ is a locally ringed
space since the stalks of $\mathcal{O}_X$ are equal
to the stalks of $\mathcal{O}_i$ at corresponding
points.

\medskip\noindent
The proof of the mapping properties is omitted.
\end{proof}

\begin{lemma}
\label{lemma-glue-schemes}
\begin{slogan}
Schemes can be glued to give new schemes.
\end{slogan}
In Lemma \ref{lemma-glue} above, assume that all
$X_i$ are schemes. Then the resulting locally ringed
space $X$ is a scheme.
\end{lemma}

\begin{proof}
This is clear since each of the $U_i$ is a scheme
and hence every $x \in X$ has an affine neighbourhood.
\end{proof}

\noindent
It is customary to think of $X_i$ as an open subspace of
$X$ via the isomorphisms $\varphi_i$. We will do this in
the next two examples.

\begin{example}[Affine space with zero doubled]
\label{example-affine-space-zero-doubled}
Let $k$ be a field. Let $n \geq 1$.
Let $X_1 = \Spec(k[x_1, \ldots, x_n])$,
let $X_2 = \Spec(k[y_1, \ldots, y_n])$.
Let $0_1 \in X_1$ be the point corresponding to the maximal ideal
$(x_1, \ldots, x_n) \subset k[x_1, \ldots, x_n]$.
Let $0_2 \in X_2$ be the point corresponding to the maximal ideal
$(y_1, \ldots, y_n) \subset k[y_1, \ldots, y_n]$.
Let $U_{12} = X_1 \setminus \{0_1\}$ and
let $U_{21} = X_2 \setminus \{0_2\}$. Let
$\varphi_{12} : U_{12} \to U_{21}$ be the isomorphism
coming from the isomorphism of $k$-algebras
$k[y_1, \ldots, y_n] \to k[x_1, \ldots, x_n]$
mapping $y_i$ to $x_i$ (which induces $X_1 \cong X_2$ mapping
$0_1$ to $0_2$).
Let $X$ be the scheme obtained from the glueing data
$(X_1, X_2, U_{12}, U_{21}, \varphi_{12},
\varphi_{21} = \varphi_{12}^{-1})$. Via the slight abuse
of notation introduced above the example we think of
$X_1, X_2 \subset X$ as open subschemes.
There is a morphism $f : X \to \Spec(k[t_1, \ldots, t_n])$
which on $X_1$ (resp.\ $X_2$) corresponds to $k$ algebra map
$k[t_1, \ldots, t_n] \to k[x_1, \ldots, x_n]$
(resp.\ $k[t_1, \ldots, t_n] \to k[y_1, \ldots, y_n]$)
mapping $t_i$ to $x_i$ (resp.\  $t_i$ to $y_i$).
It is easy to see that this morphism identifies
$k[t_1, \ldots, t_n]$ with $\Gamma(X, \mathcal{O}_X)$. Since
$f(0_1) = f(0_2)$ we see that $X$ is not affine.

\medskip\noindent
Note that $X_1$ and $X_2$ are affine opens of $X$.
But, if $n = 2$, then $X_1 \cap X_2$ is the scheme
described in Example \ref{example-not-affine} and hence not affine.
Thus in general the intersection of affine opens of a scheme
is not affine. (This fact holds more generally for any $n > 1$.)

\medskip\noindent
Another curious feature of this example is the following.
If $n > 1$ there are many irreducible closed subsets $T \subset X$
(take the closure of any non closed point in $X_1$ for example).
But unless $T = \{0_1\}$, or $T = \{0_2\}$ we have
$0_1 \in T \Leftrightarrow 0_2 \in T$. Proof omitted.
\end{example}

\begin{example}[Projective line]
\label{example-projective-line}
Let $k$ be a field.
Let $X_1 = \Spec(k[x])$,
let $X_2 = \Spec(k[y])$.
Let $0 \in X_1$ be the point corresponding to the maximal ideal
$(x) \subset k[x]$.
Let $\infty \in X_2$ be the point corresponding to the maximal ideal
$(y) \subset k[y]$.
Let $U_{12} = X_1 \setminus \{0\} = D(x) = \Spec(k[x, 1/x])$ and
let $U_{21} = X_2 \setminus \{\infty\} = D(y) = \Spec(k[y, 1/y])$.
Let $\varphi_{12} : U_{12} \to U_{21}$ be the isomorphism
coming from the isomorphism of $k$-algebras
$k[y, 1/y] \to k[x, 1/x]$ mapping $y$ to $1/x$.
Let $\mathbf{P}^1_k$ be the scheme obtained from the glueing data
$(X_1, X_2, U_{12}, U_{21}, \varphi_{12},
\varphi_{21} = \varphi_{12}^{-1})$. Via the slight abuse
of notation introduced above the example we think of
$X_i \subset \mathbf{P}^1_k$ as open subschemes. In this case
we see that $\Gamma(\mathbf{P}^1_k, \mathcal{O}) = k$ because the
only polynomials $g(x)$ in $x$ such that $g(1/y)$ is
also a polynomial in $y$ are constant polynomials.
Since $\mathbf{P}^1_k$ is infinite we see that $\mathbf{P}^1_k$ is not affine.

\medskip\noindent
We claim that there exists an affine open $U \subset \mathbf{P}^1_k$
which contains both $0$ and $\infty$. Namely, let
$U = \mathbf{P}^1_k \setminus \{1\}$, where $1$ is the point
of $X_1$ corresponding to the maximal ideal $(x - 1)$
and also the point of $X_2$ corresponding to the
maximal ideal $(y - 1)$. Then it is easy to see that
$s = 1/(x - 1) = y/(1 - y) \in \Gamma(U, \mathcal{O}_U)$.
In fact you can show that $\Gamma(U, \mathcal{O}_U)$
is equal to the polynomial ring $k[s]$ and that the
corresponding morphism $U \to \Spec(k[s])$ is
an isomorphism of schemes. Details omitted.
\end{example}




\section{A representability criterion}
\label{section-representable}

\noindent
In this section we reformulate the glueing lemma
of Section \ref{section-glueing-schemes} in terms
of functors. We recall some of the material from
Categories, Section \ref{categories-section-opposite}.
Recall that given a scheme $X$ we
can define a functor
$$
h_X : \Sch^{opp}
\longrightarrow
\textit{Sets}, \quad
T \longmapsto \Mor(T, X).
$$
This is called the {\it functor of points of $X$}.

\medskip\noindent
Let $F$ be a contravariant functor from the category
of schemes to the category of sets. In a formula
$$
F : \Sch^{opp}
\longrightarrow
\textit{Sets}.
$$
We will use the same terminology as in Sites, Section
\ref{sites-section-presheaves}. Namely, given a scheme
$T$, an element $\xi \in F(T)$, and a morphism $f : T' \to T$
we will denote $f^*\xi$ the element $F(f)(\xi)$, and sometimes
we will even use the notation $\xi|_{T'}$

\begin{definition}
\label{definition-representable-functor}
(See Categories, Definition \ref{categories-definition-representable-functor}.)
Let $F$ be a contravariant functor from the category
of schemes to the category of sets (as above).
We say that $F$ is {\it representable by a scheme}
or {\it representable} if there exists a scheme $X$
such that $h_X \cong F$.
\end{definition}

\noindent
Suppose that $F$ is representable by the scheme $X$ and that
$s : h_X \to F$ is an isomorphism.
By Categories, Yoneda Lemma \ref{categories-lemma-yoneda}
the pair $(X, s : h_X \to F)$ is unique up to unique
isomorphism if it exists.
Moreover, the Yoneda lemma says that
given any contravariant functor $F$ as above
and any scheme $Y$, we have a bijection
$$
\Mor_{\text{Fun}(\Sch^{opp}, \textit{Sets})} (h_Y, F)
\longrightarrow
F(Y), \quad
s \longmapsto s(\text{id}_Y).
$$
Here is the reverse construction. Given any $\xi \in F(Y)$
the transformation of functors $s_\xi : h_Y \to F$
associates to any morphism $f : T \to Y$ the
element $f^*\xi \in F(T)$.

\medskip\noindent
In particular, in the case that $F$ is representable, there exists
a scheme $X$ and an element $\xi \in F(X)$ such that the corresponding
morphism $h_X \to F$ is an isomorphism.
In this case we also say {\it the pair $(X, \xi)$ represents $F$}.
The element $\xi \in F(X)$
is often called the {\it ``universal family''} for reasons that will become
more clear when we talk about algebraic stacks (insert future reference here).
For the moment we simply observe that the fact that if the pair $(X, \xi)$
represents $F$, then every element $\xi' \in F(T)$ for any $T$ is of
the form $\xi' = f^*\xi$ for a unique morphism $f : T \to X$.

\begin{example}
\label{example-global-sections}
Consider the rule which associates to every scheme $T$ the set
$F(T) = \Gamma(T, \mathcal{O}_T)$. We can turn this into a contravariant
functor by using for a morphism $f : T' \to T$ the pullback map
$f^\sharp : \Gamma(T, \mathcal{O}_T) \to \Gamma(T', \mathcal{O}_{T'})$.
Given a ring $R$ and an element $t \in R$ there exists a unique
ring homomorphism $\mathbf{Z}[x] \to R$ which maps $x$ to $t$.
Thus, using Lemma \ref{lemma-morphism-into-affine}, we see that
$$
\Mor(T, \Spec(\mathbf{Z}[x])) =
\Hom(\mathbf{Z}[x], \Gamma(T, \mathcal{O}_T)) =
\Gamma(T, \mathcal{O}_T).
$$
This does indeed
give an isomorphism $h_{\Spec(\mathbf{Z}[x])} \to F$.
What is the ``universal family'' $\xi$? To get it we have to
apply the identifications above to $\text{id}_{\Spec(\mathbf{Z}[x])}$.
Clearly under the identifications above this gives
that
$\xi = x \in \Gamma(\Spec(\mathbf{Z}[x]),
\mathcal{O}_{\Spec(\mathbf{Z}[x])}) = \mathbf{Z}[x]$
as expected.
\end{example}

\begin{definition}
\label{definition-representable-by-open-immersions}
Let $F$ be a contravariant functor on the category
of schemes with values in sets.
\begin{enumerate}
\item We say that $F$ {\it satisfies the sheaf property for the
Zariski topology} if for every scheme $T$ and every open covering
$T = \bigcup_{i \in I} U_i$, and for any collection of elements
$\xi_i \in F(U_i)$ such that $\xi_i|_{U_i \cap U_j} =
\xi_j|_{U_i \cap U_j}$ there exists a unique element
$\xi \in F(T)$ such that $\xi_i = \xi|_{U_i}$ in $F(U_i)$.
\item A {\it subfunctor $H \subset F$} is a rule that associates
to every scheme $T$ a subset $H(T) \subset F(T)$ such that
the maps $F(f) : F(T) \to F(T')$ maps $H(T)$ into
$H(T')$ for all morphisms of schemes $f : T' \to T$.
\item Let $H \subset F$ be a subfunctor. We say that
$H \subset F$ is {\it representable by open immersions}
if for all pairs $(T, \xi)$, where $T$ is a scheme and $\xi \in F(T)$
there exists an open subscheme $U_\xi \subset T$ with the following
property:
\begin{itemize}
\item[(*)] A morphism $f : T' \to T$ factors through $U_\xi$ if and only
if $f^*\xi \in H(T')$.
\end{itemize}
\item Let $I$ be a set. For each $i \in I$ let $H_i \subset F$
be a subfunctor. We say that the collection $(H_i)_{i \in I}$
{\it covers $F$} if and only if for every $\xi \in F(T)$
there exists an open covering $T = \bigcup U_i$ such that
$\xi|_{U_i} \in H_i(U_i)$.
\end{enumerate}
\end{definition}

\noindent
In condition (4), if $H_i \subset F$ is representable by open immersions
for all $i$, then to check $(H_i)_{i \in I}$ covers $F$, it suffices to
check $F(T) = \bigcup H_i(T)$ whenever $T$ is the spectrum of a field.

\begin{lemma}
\label{lemma-glue-functors}
Let $F$ be a contravariant functor on the category of schemes
with values in the category of sets. Suppose that
\begin{enumerate}
\item $F$ satisfies the sheaf property for the Zariski topology,
\item there exists a set $I$ and a collection of subfunctors
$F_i \subset F$ such that
\begin{enumerate}
\item each $F_i$ is representable,
\item each $F_i \subset F$ is representable by open immersions, and
\item the collection $(F_i)_{i \in I}$ covers $F$.
\end{enumerate}
\end{enumerate}
Then $F$ is representable.
\end{lemma}

\begin{proof}
Let $X_i$ be a scheme representing $F_i$ and let
$\xi_i \in F_i(X_i) \subset F(X_i)$ be the ``universal family''.
Because $F_j \subset F$ is representable by open immersions,
there exists an open $U_{ij} \subset X_i$ such that
$T \to X_i$ factors through $U_{ij}$ if and only if
$\xi_i|_T \in F_j(T)$. In particular
$\xi_i|_{U_{ij}} \in F_j(U_{ij})$ and therefore we obtain a
canonical morphism $\varphi_{ij} : U_{ij} \to X_j$ such that
$\varphi_{ij}^*\xi_j = \xi_i|_{U_{ij}}$. By definition of $U_{ji}$
this implies that $\varphi_{ij}$ factors through $U_{ji}$.
Since $(\varphi_{ij} \circ \varphi_{ji})^*\xi_j
=\varphi_{ji}^*(\varphi_{ij}^*\xi_j) =
\varphi_{ji}^*\xi_i = \xi_j$ we conclude that
$\varphi_{ij} \circ \varphi_{ji} = \text{id}_{U_{ji}}$
because the pair $(X_j, \xi_j)$ represents $F_j$.
In particular the maps $\varphi_{ij} : U_{ij} \to U_{ji}$
are isomorphisms of schemes.
Next we have to show that
$\varphi_{ij}^{-1}(U_{ji} \cap U_{jk}) = U_{ij} \cap U_{ik}$.
This is true because (a) $U_{ji} \cap U_{jk}$ is the largest
open of $U_{ji}$ such that $\xi_j$ restricts to an element
of $F_k$, (b) $U_{ij} \cap U_{ik}$ is the largest
open of $U_{ij}$ such that $\xi_i$ restricts to an element
of $F_k$, and (c) $\varphi_{ij}^*\xi_j = \xi_i$. Moreover,
the cocycle condition in Section \ref{section-glueing-schemes}
follows because both
$\varphi_{jk}|_{U_{ji} \cap U_{jk}} \circ
\varphi_{ij}|_{U_{ij} \cap U_{ik}}$ and
$\varphi_{ik}|_{U_{ij} \cap U_{ik}}$ pullback $\xi_k$
to the element $\xi_i$.
Thus we may apply Lemma \ref{lemma-glue-schemes}
to obtain a scheme $X$ with an open
covering $X = \bigcup U_i$ and isomorphisms $\varphi_i : X_i \to U_i$
with properties as in Lemma \ref{lemma-glue}.
Let $\xi_i' = (\varphi_i^{-1})^* \xi_i$.
The conditions of Lemma \ref{lemma-glue} imply that
$\xi_i'|_{U_i \cap U_j} = \xi_j'|_{U_i \cap U_j}$.
Therefore, by the condition that $F$ satisfies the sheaf
condition in the Zariski topology we see that
there exists an element $\xi' \in F(X)$ such that
$\xi_i = \varphi_i^*\xi'|_{U_i}$ for all $i$.
Since $\varphi_i$ is an isomorphism we also get that
$(U_i, \xi'|_{U_i})$ represents the functor $F_i$.

\medskip\noindent
We claim that the pair $(X, \xi')$ represents the functor $F$.
To show this, let $T$ be a scheme and let $\xi \in F(T)$.
We will construct a unique morphism $g : T \to X$ such that
$g^*\xi' = \xi$. Namely, by the condition that the subfunctors
$F_i$ cover $F$ there exists an open covering $T = \bigcup V_i$
such that for each $i$ the restriction $\xi|_{V_i} \in F_i(V_i)$.
Moreover, since each of the inclusions $F_i \subset F$ are representable
by open immersions we may assume that each $V_i \subset T$ is maximal
open with this property.
Because, $(U_i, \xi'_{U_i})$ represents the functor $F_i$ we
get a unique morphism $g_i : V_i \to U_i$ such that
$g_i^*\xi'|_{U_i} = \xi|_{V_i}$. On the overlaps $V_i \cap V_j$
the morphisms $g_i$ and $g_j$ agree, for example because they both
pull back $\xi'|_{U_i \cap U_j} \in F_i(U_i \cap U_j)$
to the same element. Thus the morphisms $g_i$ glue to a unique morphism
from $T \to X$ as desired.
\end{proof}

\begin{remark}
\label{remark-representable-locally-ringed}
Suppose the functor $F$ is defined on all locally ringed spaces,
and if conditions of Lemma \ref{lemma-glue-functors} are replaced by
the following:
\begin{enumerate}
\item $F$ satisfies the sheaf property on the category of locally ringed
spaces,
\item there exists a set $I$ and a collection of subfunctors
$F_i \subset F$ such that
\begin{enumerate}
\item each $F_i$ is representable by a scheme,
\item each $F_i \subset F$ is representable by open immersions
on the category of locally ringed spaces, and
\item the collection $(F_i)_{i \in I}$ covers $F$
as a functor on the category of locally ringed spaces.
\end{enumerate}
\end{enumerate}
We leave it to the reader to spell this out further.
Then the end result is that the functor $F$ is
representable in the category of locally ringed spaces
and that the representing object is a scheme.
\end{remark}



\section{Existence of fibre products of schemes}
\label{section-existence-fibre-products}

\noindent
A very basic question is whether or not products and fibre products
exist on the category of schemes. We first prove abstractly that
products and fibre products exist, and in the next section
we show how we may think in a reasonable way about fibre
products of schemes.

\begin{lemma}
\label{lemma-fibre-products}
The category of schemes has a final object, products and fibre products.
In other words, the category of schemes has finite limits, see
Categories, Lemma \ref{categories-lemma-finite-limits-exist}.
\end{lemma}

\begin{proof}
Please skip this proof. It is more important to learn
how to work with the fibre product which is explained in
the next section.

\medskip\noindent
By Lemma \ref{lemma-morphism-into-affine}
the scheme $\Spec(\mathbf{Z})$ is a final
object in the category of locally ringed spaces. Thus it
suffices to prove that fibred products exist.

\medskip\noindent
Let $f : X \to S$ and $g : Y \to S$ be morphisms of schemes. We have to show
that the functor
\begin{eqnarray*}
F : \Sch^{opp} & \longrightarrow & \textit{Sets} \\
T & \longmapsto &
\Mor(T, X) \times_{\Mor(T, S)} \Mor(T, Y)
\end{eqnarray*}
is representable. We claim that Lemma \ref{lemma-glue-functors}
applies to the functor $F$. If we prove this then the lemma is proved.

\medskip\noindent
First we show that $F$ satisfies the sheaf property in the
Zariski topology. Namely, suppose that $T$ is a scheme,
$T = \bigcup_{i \in I} U_i$ is an open covering, and
$\xi_i \in F(U_i)$ such that
$\xi_i|_{U_i \cap U_j} =  \xi_j|_{U_i \cap U_j}$ for
all pairs $i, j$. By definition $\xi_i$ corresponds to
a pair $(a_i, b_i)$ where $a_i : U_i \to X$ and $b_i : U_i \to Y$
are morphisms of schemes such that $f \circ a_i = g \circ b_i$.
The glueing condition says that
$a_i|_{U_i \cap U_j} =  a_j|_{U_i \cap U_j}$
and
$b_i|_{U_i \cap U_j} =  b_j|_{U_i \cap U_j}$.
Thus by glueing the morphisms $a_i$ we obtain a morphism
of locally ringed spaces (i.e., a morphism of schemes)
$a : T \to X$ and similarly $b : T \to Y$ (see for example
the mapping property of Lemma \ref{lemma-glue}). Moreover,
on the members of an open covering the compositions
$f \circ a$ and $g \circ b$ agree. Therefore
$f \circ a = g \circ b$ and the pair $(a, b)$ defines
an element of $F(T)$ which restricts to the pairs
$(a_i, b_i)$ on each $U_i$. The sheaf condition is verified.

\medskip\noindent
Next, we construct the family of subfunctors.
Choose an open covering by open affines
$S = \bigcup\nolimits_{i \in I} U_i$.
For every $i \in I$ choose open coverings by open affines
$f^{-1}(U_i) = \bigcup\nolimits_{j \in J_i} V_j$ and
$g^{-1}(U_i) = \bigcup\nolimits_{k \in K_i} W_k$.
Note that $X = \bigcup_{i \in I} \bigcup_{j \in J_i} V_j$
is an open covering and similarly for $Y$.
For any $i \in I$ and each pair $(j, k) \in J_i \times K_i$
we have a commutative diagram
$$
\xymatrix{
    & W_k \ar[d] \ar[rd] &   \\
V_j \ar[rd] \ar[r] & U_i \ar[rd] & Y \ar[d] \\
    & X \ar[r]  & S
}
$$
where all the skew arrows are open immersions. For such a
triple we get a functor
\begin{eqnarray*}
F_{i, j, k} : \Sch^{opp} & \longrightarrow & \textit{Sets} \\
T & \longmapsto &
\Mor(T, V_j) \times_{\Mor(T, U_i)} \Mor(T, W_k).
\end{eqnarray*}
There is an obvious transformation of functors $F_{i, j, k} \to F$
(coming from the huge commutative diagram above)
which is injective, so we may think of $F_{i, j, k}$ as a subfunctor
of $F$.

\medskip\noindent
We check condition (2)(a) of Lemma \ref{lemma-glue-functors}.
This follows directly from Lemma \ref{lemma-fibre-product-affine-schemes}.
(Note that we use here that the fibre products in the category of
affine schemes are also fibre products in the whole category of locally
ringed spaces.)

\medskip\noindent
We check condition (2)(b) of Lemma \ref{lemma-glue-functors}.
Let $T$ be a scheme and let $\xi \in F(T)$. In other words,
$\xi = (a, b)$ where $a : T \to X$ and $b : T \to Y$ are
morphisms of schemes such that $f \circ a = g \circ b$.
Set $V_{i, j, k} = a^{-1}(V_j) \cap b^{-1}(W_k)$. For any
further morphism $h : T' \to T$ we have
$h^*\xi = (a \circ h, b \circ h)$. Hence we see that
$h^*\xi \in F_{i, j, k}(T')$ if and only if
$a(h(T')) \subset V_j$ and $b(h(T')) \subset W_k$.
In other words, if and only if $h(T') \subset V_{i, j, k}$.
This proves condition (2)(b).

\medskip\noindent
We check condition (2)(c) of Lemma \ref{lemma-glue-functors}.
Let $T$ be a scheme and let $\xi = (a, b) \in F(T)$ as above.
Set $V_{i, j, k} = a^{-1}(V_j) \cap b^{-1}(W_k)$ as above.
Condition (2)(c) just means that $T = \bigcup V_{i, j, k}$
which is evident. Thus the lemma is proved and fibre products
exist.
\end{proof}

\begin{remark}
\label{remark-fibre-product-schemes-locally-ringed}
Using Remark \ref{remark-representable-locally-ringed}
you can show that the fibre product of morphisms of schemes
exists in the category of locally ringed spaces and is a
scheme.
\end{remark}




\section{Fibre products of schemes}
\label{section-fibre-products}

\noindent
Here is a review of the general definition, even though
we have already shown that fibre products of schemes exist.

\begin{definition}
\label{definition-fibre-product}
Given morphisms of schemes $f : X \to S$ and $g : Y \to S$
the {\it fibre product} is a scheme $X \times_S Y$ together
with projection morphisms $p : X \times_S Y \to X$
and $q : X \times_S Y \to Y$ sitting into the following
commutative diagram
$$
\xymatrix{
X \times_S Y \ar[r]_q \ar[d]_p & Y \ar[d]^g \\
X \ar[r]^f & S
}
$$
which is universal among all diagrams of this sort,
see Categories, Definition \ref{categories-definition-fibre-products}.
\end{definition}

\noindent
In other words, given any solid commutative diagram of
morphisms of schemes
$$
\xymatrix{
T \ar[rrrd] \ar@{-->}[rrd] \ar[rrdd]
&
&
\\
&
&
X \times_S Y \ar[d] \ar[r]
&
Y \ar[d]
\\
&
&
X \ar[r]
&
S
}
$$
there exists a unique dotted arrow making the diagram commute.
We will prove some lemmas which will tell us how to think about
fibre products.

\begin{lemma}
\label{lemma-fibre-product-affines}
Let $f : X \to S$ and $g : Y \to S$ be morphisms of schemes
with the same target. If $X, Y, S$ are all affine then
$X \times_S Y$ is affine.
\end{lemma}

\begin{proof}
Suppose that $X = \Spec(A)$, $Y = \Spec(B)$
and $S = \Spec(R)$. By Lemma \ref{lemma-fibre-product-affine-schemes}
the affine scheme $\Spec(A \otimes_R B)$
is the fibre product $X \times_S Y$ in the category
of locally ringed spaces. Hence it is a fortiori the
fibre product in the category of schemes.
\end{proof}

\begin{lemma}
\label{lemma-open-fibre-product}
Let $f : X \to S$ and $g : Y \to S$ be morphisms of schemes
with the same target. Let $X \times_S Y$, $p$, $q$ be the fibre product.
Suppose that $U \subset S$,
$V \subset X$, $W \subset Y$ are open subschemes
such that $f(V) \subset U$ and $g(W) \subset U$.
Then the canonical morphism
$V \times_U W \to X \times_S Y$ is an open immersion
which identifies $V \times_U W$ with $p^{-1}(V) \cap q^{-1}(W)$.
\end{lemma}

\begin{proof}
Let $T$ be a scheme
Suppose $a : T \to V$ and $b : T \to W$ are morphisms
such that $f \circ a = g \circ b$ as morphisms into $U$.
Then they agree as morphisms into $S$.
By the universal property of the fibre product we get
a unique morphism $T \to X \times_S Y$. Of course this morphism
has image contained in the open $p^{-1}(V) \cap q^{-1}(W)$.
Thus $p^{-1}(V) \cap q^{-1}(W)$ is a fibre product of
$V$ and $W$ over $U$. The result follows from the uniqueness
of fibre products, see Categories, Section
\ref{categories-section-fibre-products}.
\end{proof}

\noindent
In particular this shows that $V \times_U W = V \times_S W$
in the situation of the lemma. Moreover, if $U, V, W$ are all
affine, then we know that $V \times_U W$ is affine. And of course
we may cover $X \times_S Y$ by such affine opens $V \times_U W$.
We formulate this as a lemma.

\begin{lemma}
\label{lemma-affine-covering-fibre-product}
\begin{slogan}
Bare-hands construction of fiber products: an affine open cover of a
fiber product of schemes can be assembled from compatible
affine open covers of the pieces.
\end{slogan}
Let $f : X \to S$ and $g : Y \to S$ be morphisms of schemes
with the same target. Let $S = \bigcup U_i$ be any affine open
covering of $S$. For each $i \in I$, let
$f^{-1}(U_i) = \bigcup_{j \in J_i} V_j$ be an affine open covering
of $f^{-1}(U_i)$ and let
$g^{-1}(U_i) = \bigcup_{k \in K_i} W_k$ be an affine open covering
of $g^{-1}(U_i)$. Then
$$
X \times_S Y =
\bigcup\nolimits_{i \in I}
\bigcup\nolimits_{j \in J_i, \ k \in K_i}
V_j \times_{U_i} W_k
$$
is an affine open covering of $X \times_S Y$.
\end{lemma}

\begin{proof}
See discussion above the lemma.
\end{proof}

\noindent
In other words, we might have used the previous lemma
as a way of construction the fibre product directly by
glueing the affine schemes. (Which is of course exactly
what we did in the proof of Lemma \ref{lemma-fibre-products} anyway.)
Here is a way to describe the set of points of a fibre product of schemes.

\begin{lemma}
\label{lemma-points-fibre-product}
Let $f : X \to S$ and $g : Y \to S$ be morphisms of schemes
with the same target. Points $z$ of $X \times_S Y$ are in bijective
correspondence to quadruples
$$
(x, y, s, \mathfrak p)
$$
where $x \in X$, $y \in Y$, $s \in S$ are points with
$f(x) = s$, $g(y) = s$ and $\mathfrak p$ is a prime ideal
of the ring $\kappa(x) \otimes_{\kappa(s)} \kappa(y)$.
The residue field of $z$ corresponds to
the residue field of the prime $\mathfrak p$.
\end{lemma}

\begin{proof}
Let $z$ be a point of $X \times_S Y$ and let us construct a
triple as above. Recall that we may think of $z$ as a morphism
$\Spec(\kappa(z)) \to X \times_S Y$, see
Lemma \ref{lemma-characterize-points}. This morphism corresponds
to morphisms $a : \Spec(\kappa(z)) \to X$
and $b : \Spec(\kappa(z)) \to Y$ such that
$f \circ a = g \circ b$. By the same lemma again
we get points $x \in X$, $y \in Y$ lying over the same point
$s \in S$ as well as field maps $\kappa(x) \to \kappa(z)$,
$\kappa(y) \to \kappa(z)$ such that the compositions
$\kappa(s) \to \kappa(x) \to \kappa(z)$
and
$\kappa(s) \to \kappa(y) \to \kappa(z)$
are the same. In other words we get a ring map
$\kappa(x) \otimes_{\kappa(s)} \kappa(y) \to \kappa(z)$.
We let $\mathfrak p$ be the kernel of this map.

\medskip\noindent
Conversely, given a quadruple $(x, y, s, \mathfrak p)$ we get a
commutative solid diagram
$$
\xymatrix{
X \times_S Y
\ar@/_/[dddr] \ar@/^/[rrrd]
& & & \\
&
\Spec(\kappa(x) \otimes_{\kappa(s)} \kappa(y)/\mathfrak p)
\ar[r] \ar[d] \ar@{-->}[lu]
&
\Spec(\kappa(y)) \ar[d] \ar[r] &
Y \ar[dd] \\
&
\Spec(\kappa(x)) \ar[r] \ar[d] &
\Spec(\kappa(s)) \ar[rd] &
\\
&
X \ar[rr] &
&
S
}
$$
see the discussion in Section \ref{section-points}. Thus we get the
dotted arrow. The corresponding point $z$ of $X \times_S Y$ is the
image of the generic point of
$\Spec(\kappa(x) \otimes_{\kappa(s)} \kappa(y)/\mathfrak p)$.
We omit the verification that the two constructions are inverse
to each other.
\end{proof}

\begin{lemma}
\label{lemma-fibre-product-immersion}
Let $f : X \to S$ and $g : Y \to S$ be morphisms of schemes
with the same target.
\begin{enumerate}
\item If $f : X \to S$ is a closed immersion,
then $X \times_S Y \to Y$ is a closed immersion.
Moreover, if $X \to S$ corresponds to the quasi-coherent
sheaf of ideals $\mathcal{I} \subset \mathcal{O}_S$, then
$X \times_S Y \to Y$ corresponds to the sheaf of ideals
$\Im(g^*\mathcal{I} \to \mathcal{O}_Y)$.
\item If $f : X \to S$ is an open immersion,
then $X \times_S Y \to Y$ is an open immersion.
\item If $f : X \to S$ is an immersion,
then $X \times_S Y \to Y$ is an immersion.
\end{enumerate}
\end{lemma}

\begin{proof}
Assume that $X \to S$ is a closed immersion corresponding
to the quasi-coherent sheaf of ideals $\mathcal{I} \subset \mathcal{O}_S$.
By Lemma \ref{lemma-restrict-map-to-closed} the closed subspace
$Z \subset Y$ defined by the sheaf of ideals
$\Im(g^*\mathcal{I} \to \mathcal{O}_Y)$ is the fibre product
in the category of locally ringed spaces.
By Lemma \ref{lemma-closed-subspace-scheme} $Z$ is a scheme.
Hence $Z = X \times_S Y$ and the first statement follows.
The second follows from Lemma \ref{lemma-open-fibre-product}
for example. The third is a combination of
the first two.
\end{proof}

\begin{definition}
\label{definition-inverse-image-closed-subscheme}
Let $f : X \to Y$ be a morphism of schemes. Let $Z \subset Y$ be a
closed subscheme of $Y$. The {\it inverse image $f^{-1}(Z)$ of the
closed subscheme $Z$} is the closed subscheme $Z \times_Y X$ of
$X$. See Lemma \ref{lemma-fibre-product-immersion} above.
\end{definition}

\noindent
We may occasionally also use this terminology with locally closed and
open subschemes.









\section{Base change in algebraic geometry}
\label{section-base-change}

\noindent
One motivation for the introduction of the language of schemes
is that it gives a very precise notion of what it means to
define a variety over a particular field. For example a variety
$X$ over $\mathbf{Q}$ is synonymous
(Varieties, Definition \ref{varieties-definition-variety})
with $X \to \Spec(\mathbf{Q})$ which is of finite type,
separated, irreducible and reduced\footnote{Of course algebraic
geometers still quibble over whether one should require $X$ to be
geometrically irreducible over $\mathbf{Q}$.}. In any case, the idea
is more generally to work with schemes over a given {\it base scheme},
often denoted $S$. We use the language: ``let $X$ be a scheme over $S$''
to mean simply that $X$ comes equipped with a morphism $X \to S$.
In diagrams we will try to picture the {\it structure morphism}
$X \to S$ as a downward arrow from $X$ to $S$.
We are often more interested in the properties of $X$ relative
to $S$ rather than the internal geometry of $X$.
For example, we would like to know things about the fibres of $X \to S$,
what happens to $X$ after base change, and so on.

\medskip\noindent
We introduce some of the language that is customarily used.
Of course this language is just a special case of thinking
about the category of objects over a given object in a category,
see Categories, Example \ref{categories-example-category-over-X}.

\begin{definition}
\label{definition-base-change}
Let $S$ be a scheme.
\begin{enumerate}
\item We say $X$ is a {\it scheme over $S$} to mean that $X$
comes equipped with a morphism of schemes $X \to S$.
The morphism $X \to S$ is sometimes called the
{\it structure morphism}.
\item If $R$ is a ring we say
$X$ is a {\it scheme over $R$} instead of
$X$ is a scheme over $\Spec(R)$.
\item A {\it morphism $f : X \to Y$ of schemes over $S$}
is a morphism of schemes such that the composition
$X \to Y \to S$ of $f$ with the structure morphism of $Y$ is
equal to the structure morphism of $X$.
\item We denote $\Mor_S(X, Y)$ the set of all morphisms
from $X$ to $Y$ over $S$.
\item Let $X$ be a scheme over $S$. Let $S' \to S$ be a
morphism of schemes. The {\it base change} of $X$
is the scheme $X_{S'} = S' \times_S X$ over $S'$.
\item Let $f : X \to Y$ be a morphism of schemes over $S$. Let $S' \to S$
be a morphism of schemes. The {\it base change} of $f$ is
the induced morphism $f' : X_{S'} \to Y_{S'}$ (namely the
morphism $\text{id}_{S'} \times_{\text{id}_S} f$).
\item Let $R$ be a ring. Let $X$ be a scheme over $R$.
Let $R \to R'$ be a ring map. The {\it base change} $X_{R'}$
is the scheme $\Spec(R') \times_{\Spec(R)} X$
over $R'$.
\end{enumerate}
\end{definition}

\noindent
Here is a typical result.

\begin{lemma}
\label{lemma-base-change-immersion}
Let $S$ be a scheme. Let $f : X \to Y$ be an
immersion (resp.\ closed immersion, resp. open immersion)
of schemes over $S$. Then any base change of $f$ is an
immersion (resp.\ closed immersion, resp. open immersion).
\end{lemma}

\begin{proof}
We can think of the base change of $f$ via the morphism
$S' \to S$ as the top left vertical arrow in the following
commutative diagram:
$$
\xymatrix{
X_{S'} \ar[r] \ar[d] & X \ar[d] \ar@/^4ex/[dd] \\
Y_{S'} \ar[r] \ar[d] & Y \ar[d] \\
S' \ar[r] & S
}
$$
The diagram implies $X_{S'} \cong Y_{S'} \times_Y X$,
and the lemma follows from Lemma \ref{lemma-fibre-product-immersion}.
\end{proof}

\noindent
In fact this type of result is so typical that there is a
piece of language to express it. Here it is.

\begin{definition}
\label{definition-preserved-by-base-change}
Properties and base change.
\begin{enumerate}
\item Let $\mathcal{P}$ be a property of schemes over a base.
We say that $\mathcal{P}$ is {\it preserved under arbitrary base change},
or simply that $\mathcal{P}$ is {\it preserved under base change}
if whenever $X/S$
has $\mathcal{P}$, any base change $X_{S'}/S'$ has $\mathcal{P}$.
\item Let $\mathcal{P}$ be a property of morphisms of schemes over a base.
We say that $\mathcal{P}$ is {\it preserved under arbitrary base change},
or simply that {\it preserved under base change} if whenever
$f : X \to Y$ over $S$ has $\mathcal{P}$, any base change
$f' : X_{S'} \to Y_{S'}$ over $S'$ has $\mathcal{P}$.
\end{enumerate}
\end{definition}

\noindent
At this point we can say that ``being a closed immersion'' is
preserved under arbitrary base change.

\begin{definition}
\label{definition-fibre}
Let $f : X \to S$ be a morphism of schemes.
Let $s \in S$ be a point.
The {\it scheme theoretic fibre $X_s$ of $f$ over $s$},
or simply the {\it fibre of $f$ over $s$},
is the scheme fitting in the following fibre product diagram
$$
\xymatrix{
X_s = \Spec(\kappa(s)) \times_S X \ar[r] \ar[d] &
X \ar[d] \\
\Spec(\kappa(s)) \ar[r] &
S
}
$$
We think of the fibre $X_s$ always as a scheme over $\kappa(s)$.
\end{definition}

\begin{lemma}
\label{lemma-fibre-topological}
Let $f : X \to S$ be a morphism of schemes.
Consider the diagrams
$$
\xymatrix{
X_s \ar[r] \ar[d] &
X \ar[d] &
\Spec(\mathcal{O}_{S, s}) \times_S X \ar[r] \ar[d] &
X \ar[d]
\\
\Spec(\kappa(s)) \ar[r] &
S &
\Spec(\mathcal{O}_{S, s}) \ar[r] &
S
}
$$
In both cases the top horizontal arrow is a homeomorphism
onto its image.
\end{lemma}

\begin{proof}
Choose an open affine $U \subset S$ that contains $s$.
The bottom horizontal morphisms factor through $U$, see
Lemma \ref{lemma-morphism-from-spec-local-ring} for example.
Thus we may assume that $S$ is affine. If $X$ is also affine, then
the result follows from
Algebra, Remark \ref{algebra-remark-fundamental-diagram}.
In the general case the result follows by covering $X$ by open affines.
\end{proof}







\section{Quasi-compact morphisms}
\label{section-quasi-compact}

\noindent
A scheme is {\it quasi-compact} if its underlying topological space is
quasi-compact. There is a relative notion which is defined as follows.

\begin{definition}
\label{definition-quasi-compact}
A morphism of schemes is called {\it quasi-compact}
if the underlying map of topological spaces is
quasi-compact, see
Topology, Definition \ref{topology-definition-quasi-compact}.
\end{definition}

\begin{lemma}
\label{lemma-quasi-compact-affine}
Let $f : X \to S$ be a morphism of schemes.
The following are equivalent
\begin{enumerate}
\item $f : X \to S$ is quasi-compact,
\item the inverse image of every affine open is quasi-compact, and
\item there exists some affine open covering $S = \bigcup_{i \in I} U_i$
such that $f^{-1}(U_i)$ is quasi-compact for all $i$.
\end{enumerate}
\end{lemma}

\begin{proof}
Suppose we are given a covering $S = \bigcup_{i \in I} U_i$ as in (3).
First, let $U \subset S$ be any affine open. For any $u \in U$
we can find an index $i(u) \in I$ such that $u \in U_{i(u)}$.
As standard opens form a basis for the topology on $U_{i(u)}$ we can find
$W_u \subset U \cap U_{i(u)}$ which is standard open in $U_{i(u)}$.
By compactness we can find finitely many points $u_1, \ldots, u_n \in U$
such that $U = \bigcup_{j = 1}^n W_{u_j}$. For each $j$ write
$f^{-1}U_{i(u_j)} = \bigcup_{k \in K_j} V_{jk}$ as a finite
union of affine opens. Since $W_{u_j} \subset U_{i(u_j)}$ is a standard
open we see that $f^{-1}(W_{u_j}) \cap V_{jk}$ is a standard
open of $V_{jk}$, see Algebra, Lemma \ref{algebra-lemma-spec-functorial}.
Hence $f^{-1}(W_{u_j}) \cap V_{jk}$ is affine, and so
$f^{-1}(W_{u_j})$ is a finite union of affines. This proves that the
inverse image of any affine open is a finite union of affine opens.

\medskip\noindent
Next, assume that the inverse image of every affine open is a finite
union of affine opens.
Let $K \subset S$ be any quasi-compact open. Since $S$ has a basis
of the topology consisting of affine opens we see that $K$ is a finite
union of affine opens. Hence the inverse image of $K$ is a finite
union of affine opens. Hence $f$ is quasi-compact.

\medskip\noindent
Finally, assume that $f$ is quasi-compact. In this case the argument
of the previous paragraph shows that the inverse image of any affine
is a finite union of affine opens.
\end{proof}

\begin{lemma}
\label{lemma-quasi-compact-preserved-base-change}
Being quasi-compact is a property of morphisms of schemes
over a base which is preserved under arbitrary base change.
\end{lemma}

\begin{proof}
Omitted.
\end{proof}

\begin{lemma}
\label{lemma-composition-quasi-compact}
The composition of quasi-compact morphisms is quasi-compact.
\end{lemma}

\begin{proof}
This follows from the definitions and
Topology, Lemma \ref{topology-lemma-composition-quasi-compact}.
\end{proof}

\begin{lemma}
\label{lemma-closed-immersion-quasi-compact}
A closed immersion is quasi-compact.
\end{lemma}

\begin{proof}
Follows from the definitions and
Topology, Lemma \ref{topology-lemma-closed-in-quasi-compact}.
\end{proof}

\begin{example}
\label{example-open-immersion-not-quasi-compact}
An open immersion is in general not quasi-compact.
The standard example of this is the open subspace
$U \subset X$, where $X = \Spec(k[x_1, x_2, x_3, \ldots])$,
where $U$ is $X \setminus \{0\}$, and where $0$ is the point
of $X$ corresponding to the maximal ideal
$(x_1, x_2, x_3, \ldots)$.
\end{example}

\begin{lemma}
\label{lemma-image-quasi-compact-closed}
Let $f : X \to S$ be a quasi-compact morphism of schemes.
The following are equivalent
\begin{enumerate}
\item $f(X) \subset S$ is closed, and
\item $f(X) \subset S$ is stable under specialization.
\end{enumerate}
\end{lemma}

\begin{proof}
We have (1) $\Rightarrow$ (2) by
Topology, Lemma \ref{topology-lemma-open-closed-specialization}.
Assume (2). Let $U \subset S$ be an affine open. It suffices to prove
that $f(X) \cap U$ is closed. Since $U \cap f(X)$ is stable under
specializations in $U$, we have reduced to the case where $S$ is affine.
Because $f$ is quasi-compact we deduce that $X = f^{-1}(S)$ is
quasi-compact as $S$ is affine. Thus we may write
$X = \bigcup_{i = 1}^n U_i$ with $U_i \subset X$ open affine.
Say $S = \Spec(R)$ and
$U_i = \Spec(A_i)$ for some $R$-algebra $A_i$.
Then $f(X) = \Im(\Spec(A_1 \times \ldots \times A_n)
\to \Spec(R))$. Thus the lemma follows from
Algebra, Lemma \ref{algebra-lemma-image-stable-specialization-closed}.
\end{proof}

\begin{lemma}
\label{lemma-quasi-compact-closed}
Let $f : X \to S$ be a quasi-compact morphism of schemes.
Then $f$ is closed if and only if specializations lift
along $f$, see
Topology, Definition \ref{topology-definition-lift-specializations}.
\end{lemma}

\begin{proof}
According to
Topology, Lemma \ref{topology-lemma-closed-open-map-specialization}
if $f$ is closed then specializations lift along $f$.
Conversely, suppose that specializations lift along $f$.
Let $Z \subset X$ be a closed subset. We may think of $Z$
as a scheme with the reduced induced scheme structure, see
Definition \ref{definition-reduced-induced-scheme}.
Since $Z \subset X$ is closed the restriction
of $f$ to $Z$ is still quasi-compact. Moreover specializations lift
along $Z \to S$ as well,
see Topology, Lemma \ref{topology-lemma-lift-specialization-composition}.
Hence it suffices to prove $f(X)$ is closed if specializations lift along $f$.
In particular $f(X)$ is stable under specializations, see
Topology, Lemma \ref{topology-lemma-lift-specializations-images}.
Thus $f(X)$ is closed by
Lemma \ref{lemma-image-quasi-compact-closed}.
\end{proof}










\section{Valuative criterion for universal closedness}
\label{section-valuative-criterion-universal-closedness}

\noindent
In Topology, Section \ref{topology-section-proper} there is a
discussion of proper maps as closed maps of topological spaces
all of whose fibres are quasi-compact, or as maps such that
all base changes are closed maps. Here is the corresponding notion
in algebraic geometry.

\begin{definition}
\label{definition-universally-closed}
A morphism of schemes $f : X \to S$ is said to be
{\it universally closed} if every base change
$f' : X_{S'} \to S'$ is closed.
\end{definition}

\noindent
In fact the adjective ``universally'' is often used in this way.
In other words, given a property $\mathcal{P}$ of morphisms
the we say that ``$X \to S$ is universally $\mathcal{P}$''
if and only if every base change $X_{S'} \to S'$ has $\mathcal{P}$.

\medskip\noindent
Please take a look at Morphisms, Section \ref{morphisms-section-proper} for a
more detailed discussion of the properties of universally closed morphisms.
In this section we restrict the discussion to the relationship between
universal closed morphisms and morphisms satisfying the
existence part of the valuative criterion.

\begin{lemma}
\label{lemma-specializations-lift}
Let $f : X \to S$ be a morphism of schemes.
\begin{enumerate}
\item If $f$ is universally closed then specializations lift
along any base change of $f$, see
Topology, Definition \ref{topology-definition-lift-specializations}.
\item If $f$ is quasi-compact and specializations lift
along any base change of $f$, then $f$ is universally closed.
\end{enumerate}
\end{lemma}

\begin{proof}
Part (1) is a direct consequence of
Topology, Lemma \ref{topology-lemma-closed-open-map-specialization}.
Part (2) follows from
Lemmas \ref{lemma-quasi-compact-closed} and
\ref{lemma-quasi-compact-preserved-base-change}.
\end{proof}

\begin{definition}
\label{definition-valuative-criterion}
Let $f : X \to S$ be a morphism of schemes. We say $f$
{\it satisfies the existence part of the valuative criterion}
if given any commutative solid diagram
$$
\xymatrix{
\Spec(K) \ar[r] \ar[d] & X \ar[d] \\
\Spec(A) \ar[r] \ar@{-->}[ru] & S
}
$$
where $A$ is a valuation ring with field of fractions $K$, the
dotted arrow exists. We say $f$ {\it satisfies the uniqueness
part of the valuative criterion} if there is at most one
dotted arrow given any diagram as above (without requiring
existence of course).
\end{definition}

\noindent
A {\it valuation ring} is a local domain maximal among the relation
of domination in its fraction field, see
Algebra, Definition \ref{algebra-definition-valuation-ring}.
Hence the spectrum of a valuation ring has a unique generic point
$\eta$ and a unique closed point $0$, and of course we have the
specialization $\eta \leadsto 0$.
The significance of valuation rings is that any specialization of
points in any scheme is the image of $\eta \leadsto 0$
under some morphism from the spectrum of some valuation ring.
Here is the precise result.

\begin{lemma}
\label{lemma-points-specialize}
Let $S$ be a scheme. Let $s' \leadsto s$ be a specialization of points of $S$.
Then
\begin{enumerate}
\item there exists a valuation ring $A$ and a morphism
$f : \Spec(A) \to S$ such that the generic point $\eta$ of
$\Spec(A)$ maps to $s'$ and the special point maps to $s$, and
\item given a field extension $\kappa(s') \subset K$
we may arrange it so that the extension
$\kappa(s') \subset \kappa(\eta)$ induced by $f$
is isomorphic to the given extension.
\end{enumerate}
\end{lemma}

\begin{proof}
Let $s' \leadsto s$ be a specialization in $S$, and let
$\kappa(s') \subset K$ be an extension of fields. By
Lemma \ref{lemma-specialize-points}
and the discussion following
Lemma \ref{lemma-characterize-points}
this leads to ring maps $\mathcal{O}_{S, s} \to \kappa(s') \to K$.
Let $A \subset K$ be any valuation ring whose field of fractions is
$K$ and which dominates the image of $\mathcal{O}_{S, s} \to K$, see
Algebra, Lemma \ref{algebra-lemma-dominate}.
The ring map $\mathcal{O}_{S, s} \to A$ induces the morphism
$f : \Spec(A) \to S$, see
Lemma \ref{lemma-morphism-from-spec-local-ring}.
This morphism has all the desired properties by construction.
\end{proof}

\begin{lemma}
\label{lemma-lift-specializations-valuative}
Let $f : X \to S$ be a morphism of schemes.
The following are equivalent
\begin{enumerate}
\item Specializations lift along any base change of $f$
\item The morphism $f$ satisfies the existence part of the
valuative criterion.
\end{enumerate}
\end{lemma}

\begin{proof}
Assume (1) holds. Let a solid diagram as in
Definition \ref{definition-valuative-criterion} be given.
In order to find the dotted arrow we may replace $X \to S$
by $X_{\Spec(A)} \to \Spec(A)$ since after all
the assumption is stable under base change.
Thus we may assume $S = \Spec(A)$.
Let $x' \in X$ be the image of $\Spec(K) \to X$, so
that we have $\kappa(x') \subset K$, see
Lemma \ref{lemma-characterize-points}.
By assumption there exists a specialization $x' \leadsto x$
in $X$ such that $x$ maps to the closed point of $S = \Spec(A)$.
We get a local ring map $A \to \mathcal{O}_{X, x}$ and a ring
map $\mathcal{O}_{X, x} \to \kappa(x')$, see
Lemma \ref{lemma-specialize-points} and the discussion following
Lemma \ref{lemma-characterize-points}. The composition
$A \to \mathcal{O}_{X, x} \to \kappa(x') \to K$ is the
given injection $A \to K$. Since $A \to \mathcal{O}_{X, x}$
is local, the image of $\mathcal{O}_{X, x} \to K$
dominates $A$ and hence is equal to $A$, by
Algebra, Definition \ref{algebra-definition-valuation-ring}.
Thus we obtain a ring map $\mathcal{O}_{X, x} \to A$ and
hence a morphism $\Spec(A) \to X$
(see Lemma \ref{lemma-morphism-from-spec-local-ring} and
discussion following it). This proves (2).

\medskip\noindent
Conversely, assume (2) holds. It is immediate that
the existence part of the valuative criterion holds for
any base change $X_{S'} \to S'$ of $f$ by considering
the following commutative diagram
$$
\xymatrix{
\Spec(K) \ar[r] \ar[d] & X_{S'} \ar[r] \ar[d] & X \ar[d] \\
\Spec(A) \ar[r] \ar@{-->}[ru] \ar@{-->}[rru] & S' \ar[r] & S
}
$$
Namely, the more horizontal dotted arrow will lead to the
other one by definition of the fibre product. OK, so it clearly
suffices to show that specializations lift along $f$.
Let $s' \leadsto s$ be a specialization in $S$, and let
$x' \in X$ be a point lying over $s'$. Apply
Lemma \ref{lemma-points-specialize}
to $s' \leadsto s$ and the extension of fields
$\kappa(s') \subset \kappa(x') = K$.
We get a commutative diagram
$$
\xymatrix{
\Spec(K) \ar[rr] \ar[d] & & X \ar[d] \\
\Spec(A) \ar[r] \ar@{-->}[rru] &
\Spec(\mathcal{O}_{S, s}) \ar[r] & S
}
$$
and by condition (2) we get the dotted arrow.
The image $x$ of the closed point of $\Spec(A)$
in $X$ will be a solution to our problem, i.e.,
$x$ is a specialization of $x'$ and maps to $s$.
\end{proof}

\begin{proposition}[Valuative criterion of universal closedness]
\label{proposition-characterize-universally-closed}
Let $f$ be a quasi-compact morphism of schemes.
Then $f$ is universally closed if and only if $f$
satisfies the existence part of the valuative criterion.
\end{proposition}

\begin{proof}
This is a formal consequence of
Lemmas \ref{lemma-specializations-lift} and
\ref{lemma-lift-specializations-valuative} above.
\end{proof}

\begin{example}
\label{example-projective-line-universally-closed}
Let $k$ be a field. Consider the structure morphism
$p : \mathbf{P}^1_k \to \Spec(k)$ of the projective
line over $k$, see Example \ref{example-projective-line}.
Let us use the valuative criterion above to prove that $p$
is universally closed.
By construction $\mathbf{P}^1_k$ is covered by two
affine opens and hence $p$ is quasi-compact.
Let a commutative diagram
$$
\xymatrix{
\Spec(K) \ar[r]_\xi \ar[d] & \mathbf{P}^1_k \ar[d] \\
\Spec(A) \ar[r]^\varphi & \Spec(k)
}
$$
be given, where $A$ is a valuation ring and $K$ is its field
of fractions. Recall that $\mathbf{P}^1_k$ is gotten by glueing
$\Spec(k[x])$ to $\Spec(k[y])$ by glueing
$D(x)$ to $D(y)$ via $x = y^{-1}$ (or more symmetrically $xy = 1$).
To show there is a morphism $\Spec(A) \to \mathbf{P}^1_k$
fitting diagonally into the diagram above we may assume that $\xi$
maps into the open $\Spec(k[x])$ (by symmetry). This gives the following
commutative diagram of rings
$$
\xymatrix{
K & k[x] \ar[l]^{\xi^\sharp} \\
A \ar[u] & k \ar[u] \ar[l]_{\varphi^\sharp}
}
$$
By Algebra, Lemma \ref{algebra-lemma-valuation-ring-x-or-x-inverse}
we see that either
$\xi^\sharp(x) \in A$ or $\xi^\sharp(x)^{-1} \in A$.
In the first case we get a ring map
$$
k[x] \to A,
\ \lambda \mapsto \varphi^\sharp(\lambda),
\  x \mapsto \xi^\sharp(x)
$$
fitting into the diagram of rings above, and we win.
In the second case we see that we get a ring map
$$
k[y] \to A,
\ \lambda \mapsto \varphi^\sharp(\lambda),
\ y \mapsto \xi^\sharp(x)^{-1}.
$$
This gives a morphism
$\Spec(A) \to \Spec(k[y]) \to \mathbf{P}^1_k$
which fits diagonally into the initial commutative diagram of this example
(check omitted).
\end{example}





























\section{Separation axioms}
\label{section-separation-axioms}

\noindent
A topological space $X$ is Hausdorff if and only if the
diagonal $\Delta \subset X \times X$ is a closed subset.
The analogue in algebraic geometry is, given a scheme $X$ over
a base scheme $S$, to consider the
diagonal morphism
$$
\Delta_{X/S} : X \longrightarrow X \times_S X.
$$
This is the unique morphism of schemes such that
$\text{pr}_1 \circ \Delta_{X/S} = \text{id}_X$ and
$\text{pr}_2 \circ \Delta_{X/S} = \text{id}_X$ (it exists in
any category with fibre products).

\begin{lemma}
\label{lemma-diagonal-affines-closed}
The diagonal morphism of a morphism between affines is closed.
\end{lemma}

\begin{proof}
The diagonal morphism associated to the morphism
$\Spec(S) \to \Spec(R)$ is the morphism on spectra
corresponding to the ring
map $S \otimes_R S \to S$, $a \otimes b \mapsto ab$.
This map is clearly surjective, so $S \cong S \otimes_R S/J$
for some ideal $J \subset S \otimes_R S$. Hence
$\Delta$ is a closed immersion according to
Example \ref{example-closed-immersion-affines}
\end{proof}

\begin{lemma}
\label{lemma-diagonal-immersion}
\begin{slogan}
The diagonal morphism for relative schemes is an immersion.
\end{slogan}
Let $X$ be a scheme over $S$.
The diagonal morphism $\Delta_{X/S}$ is an immersion.
\end{lemma}

\begin{proof}
Recall that if $V \subset X$ is affine open and maps into
$U \subset S$ affine open, then $V \times_U V$ is affine open
in $X \times_S X$, see Lemmas \ref{lemma-fibre-product-affines}
and \ref{lemma-open-fibre-product}.
Consider the open subscheme $W$ of $X \times_S X$ which
is the union of these affine opens $V \times_U V$.
By Lemma \ref{lemma-closed-local-target} it is enough
to show that each morphism
$\Delta_{X/S}^{-1}(V \times_U V) \to V \times_U V$ is
a closed immersion. Since $V = \Delta_{X/S}^{-1}(V \times_U V)$
we are just checking that $\Delta_{V/U}$ is a closed
immersion, which is Lemma \ref{lemma-diagonal-affines-closed}.
\end{proof}

\begin{definition}
\label{definition-separated}
Let $f : X \to S$ be a morphism of schemes.
\begin{enumerate}
\item We say $f$ is {\it separated} if the diagonal morphism $\Delta_{X/S}$
is a closed immersion.
\item We say $f$ is {\it quasi-separated} if the diagonal morphism
$\Delta_{X/S}$ is a quasi-compact morphism.
\item We say a scheme $Y$ is {\it separated} if the morphism
$Y \to \Spec(\mathbf{Z})$ is separated.
\item We say a scheme $Y$ is {\it quasi-separated} if the morphism
$Y \to \Spec(\mathbf{Z})$ is quasi-separated.
\end{enumerate}
\end{definition}

\noindent
By Lemmas \ref{lemma-diagonal-immersion} and \ref{lemma-immersion-when-closed}
we see that $\Delta_{X/S}$ is a closed immersion if an only if
$\Delta_{X/S}(X) \subset X \times_S X$ is a closed subset. Moreover,
by Lemma \ref{lemma-closed-immersion-quasi-compact} we see that a
separated morphism is quasi-separated. The reason for introducing
quasi-separated morphisms is that nonseparated morphisms come up naturally
in studying algebraic varieties (especially when doing moduli,
algebraic stacks, etc). But most often they are still quasi-separated.

\begin{example}
\label{example-not-quasi-separated}
Here is an example of a non-quasi-separated morphism.
Suppose $X = X_1 \cup X_2 \to S = \Spec(k)$ with
$X_1 = X_2 = \Spec(k[t_1, t_2, t_3, \ldots])$
glued along the complement of $\{0\} = \{(t_1, t_2, t_3, \ldots)\}$
(glued as in Example \ref{example-affine-space-zero-doubled}).
In this case the inverse image of the affine scheme
$X_1 \times_S X_2$ under $\Delta_{X/S}$ is the scheme
$\Spec(k[t_1, t_2, t_3, \ldots]) \setminus \{0\}$
which is not quasi-compact.
\end{example}

\begin{lemma}
\label{lemma-where-are-they-equal}
Let $X$, $Y$ be schemes over $S$.
Let $a, b : X \to Y$ be morphisms of schemes over $S$.
There exists a largest locally closed subscheme
$Z \subset X$ such that $a|_Z = b|_Z$. In fact $Z$ is
the equalizer of $(a, b)$. Moreover, if $Y$ is separated
over $S$, then $Z$ is a closed subscheme.
\end{lemma}

\begin{proof}
The equalizer of $(a, b)$ is for categorical reasons
the fibre product $Z$ in the following diagram
$$
\xymatrix{
Z = Y \times_{(Y \times_S Y)} X \ar[r] \ar[d] &
 X \ar[d]^{(a , b)} \\
Y \ar[r]^-{\Delta_{Y/S}} & Y \times_S Y
}
$$
Thus the lemma follows from Lemmas
\ref{lemma-base-change-immersion}, \ref{lemma-diagonal-immersion} and
Definition \ref{definition-separated}.
\end{proof}

\begin{lemma}
\label{lemma-characterize-quasi-separated}
Let $f : X \to S$ be a morphism of schemes.
The following are equivalent:
\begin{enumerate}
\item The morphism $f$ is quasi-separated.
\item For every pair of affine opens $U, V \subset X$
which map into a common affine open of $S$ the intersection
$U \cap V$ is a finite union of affine opens of $X$.
\item There exists an affine open covering $S = \bigcup_{i \in I} U_i$
and for each $i$ an affine open covering $f^{-1}U_i = \bigcup_{j \in I_i} V_j$
such that for each $i$ and each pair $j, j' \in I_i$ the
intersection $V_j \cap V_{j'}$ is a finite union of affine
opens of $X$.
\end{enumerate}
\end{lemma}

\begin{proof}
Let us prove that (3) implies (1).
By Lemma \ref{lemma-affine-covering-fibre-product}
the covering $X \times_S X = \bigcup_i \bigcup_{j, j'} V_j \times_{U_i} V_{j'}$
is an affine open covering of $X \times_S X$.
Moreover, $\Delta_{X/S}^{-1}(V_j \times_{U_i} V_{j'}) = V_j \cap V_{j'}$.
Hence the implication follows from Lemma \ref{lemma-quasi-compact-affine}.

\medskip\noindent
The implication (1) $\Rightarrow$ (2) follows from the fact
that under the hypotheses of (2) the fibre product
$U \times_S V$ is an affine open of $X \times_S X$.
The implication (2) $\Rightarrow$ (3) is trivial.
\end{proof}

\begin{lemma}
\label{lemma-characterize-separated}
Let $f : X \to S$ be a morphism of schemes.
\begin{enumerate}
\item If $f$ is separated then for every pair of affine
opens $(U, V)$ of $X$ which map into a
common affine open of $S$ we have
\begin{enumerate}
\item the intersection $U \cap V$ is affine.
\item the ring map
$\mathcal{O}_X(U) \otimes_{\mathbf{Z}} \mathcal{O}_X(V)
\to \mathcal{O}_X(U \cap V)$
is surjective.
\end{enumerate}
\item If any pair of points $x_1, x_2 \in X$ lying over a common
point $s \in S$ are contained in affine opens $x_1 \in U$,
$x_2 \in V$ which map into a common affine open of $S$ such
that (a), (b) hold, then $f$ is separated.
\end{enumerate}
\end{lemma}

\begin{proof}
Assume $f$ separated. Suppose $(U, V)$ is a pair as in (1).
Let $W = \Spec(R)$ be an affine open of $S$ containing
both $f(U)$ and $f(V)$. Write $U = \Spec(A)$ and
$V = \Spec(B)$ for $R$-algebras $A$ and $B$.
By Lemma \ref{lemma-open-fibre-product} we see that
$U \times_S V = U \times_W V = \Spec(A \otimes_R B)$
is an affine open of $X \times_S X$. Hence, by
Lemma \ref{lemma-closed-subspace-scheme} we see that
$\Delta^{-1}(U \times_S V) \to U \times_S V$
can be identified with $\Spec(A \otimes_R B/J)$
for some ideal $J \subset A \otimes_R B$.
Thus $U \cap V = \Delta^{-1}(U \times_S V)$ is affine.
Assertion (1)(b) holds because
$A \otimes_{\mathbf{Z}} B \to (A \otimes_R B)/J$ is surjective.

\medskip\noindent
Assume the hypothesis formulated in (2) holds.
Clearly the collection of affine opens $U \times_S V$
for pairs $(U, V)$ as in (2) form an affine open covering
of $X \times_S X$ (see e.g.\ Lemma \ref{lemma-affine-covering-fibre-product}).
Hence it suffices to show that each morphism
$U \cap V = \Delta_{X/S}^{-1}(U \times_S V) \to U \times_S V$
is a closed immersion, see Lemma \ref{lemma-closed-local-target}.
By assumption (a) we have $U \cap V = \Spec(C)$ for some ring $C$.
After choosing an affine open $W = \Spec(R)$ of $S$
into which both $U$ and $V$ map and writing $U = \Spec(A)$,
$V = \Spec(B)$ we see that the assumption (b) means
that the composition
$$
A \otimes_{\mathbf{Z}} B \to A \otimes_R B \to C
$$
is surjective. Hence $A \otimes_R B \to C$ is surjective and
we conclude that $\Spec(C) \to \Spec(A \otimes_R B)$
is a closed immersion.
\end{proof}

\begin{example}
\label{example-projective-line-separated}
Let $k$ be a field. Consider the structure morphism
$p : \mathbf{P}^1_k \to \Spec(k)$ of the projective
line over $k$, see Example \ref{example-projective-line}.
Let us use the lemma above to prove that $p$
is separated. By construction $\mathbf{P}^1_k$ is covered by two
affine opens $U = \Spec(k[x])$ and $V = \Spec(k[y])$
with intersection $U \cap V = \Spec(k[x, y]/(xy - 1))$
(using obvious notation). Thus it suffices to check that
conditions (2)(a) and (2)(b) of Lemma \ref{lemma-characterize-separated}
hold for the pairs of affine opens $(U, U)$, $(U, V)$, $(V, U)$
and $(V, V)$. For the pairs $(U, U)$ and $(V, V)$ this is trivial.
For the pair $(U, V)$ this amounts to proving
that $U \cap V$ is affine, which is true, and that the ring map
$$
k[x] \otimes_{\mathbf{Z}} k[y] \longrightarrow k[x, y]/(xy - 1)
$$
is surjective. This is clear because any element in the
right hand side can be written as a sum of a polynomial
in $x$ and a polynomial in $y$.
\end{example}

\begin{lemma}
\label{lemma-fibre-product-after-map}
Let $f : X \to T$ and $g : Y \to T$ be morphisms of schemes
with the same target. Let $h : T \to S$ be a morphism of schemes.
Then the induced morphism $i : X \times_T Y \to X \times_S Y$ is
an immersion. If $T \to S$ is separated, then $i$ is a closed
immersion. If $T \to S$ is quasi-separated, then $i$ is a
quasi-compact morphism.
\end{lemma}

\begin{proof}
By general category theory the following diagram
$$
\xymatrix{
X \times_T Y \ar[r] \ar[d] & X \times_S Y \ar[d] \\
T \ar[r]^{\Delta_{T/S}} \ar[r] & T \times_S T
}
$$
is a fibre product diagram. The lemma follows
from Lemmas \ref{lemma-diagonal-immersion},
\ref{lemma-fibre-product-immersion} and
\ref{lemma-quasi-compact-preserved-base-change}.
\end{proof}

\begin{lemma}
\label{lemma-semi-diagonal}
Let $g : X \to Y$ be a morphism of schemes over $S$.
The morphism $i : X \to X \times_S Y$ is an immersion.
If $Y$ is separated over $S$ it is a closed immersion.
If $Y$ is quasi-separated over $S$ it is quasi-compact.
\end{lemma}

\begin{proof}
This is a special case of Lemma \ref{lemma-fibre-product-after-map}
applied to the morphism $X = X \times_Y Y \to X \times_S Y$.
\end{proof}

\begin{lemma}
\label{lemma-section-immersion}
Let $f : X \to S$ be a morphism of schemes.
Let $s : S \to X$ be a section of $f$ (in a formula $f \circ s = \text{id}_S$).
Then $s$ is an immersion.
If $f$ is separated then $s$ is a closed immersion.
If $f$ is quasi-separated, then $s$ is quasi-compact.
\end{lemma}

\begin{proof}
This is a special case of Lemma \ref{lemma-semi-diagonal} applied to
$g =s$ so the morphism $i = s : S \to S \times_S X$.
\end{proof}

\begin{lemma}
\label{lemma-separated-permanence}
Permanence properties.
\begin{enumerate}
\item A composition of separated morphisms is separated.
\item A composition of quasi-separated morphisms is quasi-separated.
\item The base change of a separated morphism is separated.
\item The base change of a quasi-separated morphism is quasi-separated.
\item A (fibre) product of separated morphisms is separated.
\item A (fibre) product of quasi-separated morphisms is quasi-separated.
\end{enumerate}
\end{lemma}

\begin{proof}
Let $X \to Y \to Z$ be morphisms. Assume that $X \to Y$ and
$Y \to Z$ are separated. The composition
$$
X \to X \times_Y X \to X \times_Z X
$$
is closed because the first one is by assumption and the second
one by Lemma \ref{lemma-fibre-product-after-map}. The same argument
works for ``quasi-separated'' (with the same references).

\medskip\noindent
Let $f : X \to Y$ be a morphism of schemes over a base $S$.
Let $S' \to S$ be a morphism of schemes. Let $f' : X_{S'} \to Y_{S'}$
be the base change of $f$. Then the diagonal morphism
of $f'$ is a morphism
$$
\Delta_{f'} :
X_{S'} = S' \times_S X
\longrightarrow
X_{S'} \times_{Y_{S'}} X_{S'} = S' \times _S (X \times_Y X)
$$
which is easily seen to be the base change of $\Delta_f$.
Thus (3) and (4) follow from the fact that
closed immersions and quasi-compact morphisms are preserved
under arbitrary base change (Lemmas
\ref{lemma-fibre-product-immersion} and
\ref{lemma-quasi-compact-preserved-base-change}).

\medskip\noindent
If $f : X \to Y$ and $g : U \to V$ are morphisms of schemes over a base $S$,
then $f \times g$ is the composition of $X \times_S U \to X \times_S V$
(a base change of $g$) and $X \times_S V \to Y \times_S V$ (a base change
of $f$). Hence (5) and (6) follow from (1) -- (4).
\end{proof}

\begin{lemma}
\label{lemma-compose-after-separated}
\begin{slogan}
Separated and quasi-separated morphisms satisfy cancellation.
\end{slogan}
Let $f : X \to Y$ and $g : Y \to Z$ be morphisms of schemes.
If $g \circ f$ is separated then so is $f$.
If $g \circ f$ is quasi-separated then so is $f$.
\end{lemma}

\begin{proof}
Assume that $g \circ f$ is separated.
Consider the factorization $X \to X \times_Y X \to X \times_Z X$
of the diagonal morphism of $g \circ f$.
By Lemma \ref{lemma-fibre-product-after-map}
the last morphism is an immersion. By assumption the image
of $X$ in $X \times_Z X$ is closed. Hence it is also closed
in $X \times_Y X$. Thus we see that $X \to X \times_Y X$
is a closed immersion by Lemma \ref{lemma-immersion-when-closed}.

\medskip\noindent
Assume that $g \circ f$ is quasi-separated.
Let $V \subset Y$ be an affine open which maps into an affine
open of $Z$. Let $U_1, U_2 \subset X$ be affine opens which
map into $V$. Then $U_1 \cap U_2$ is a finite union of affine
opens because $U_1, U_2$ map into a common affine open
of $Z$. Since we may cover $Y$ by affine opens like $V$ we
deduce the lemma from Lemma \ref{lemma-characterize-quasi-separated}.
\end{proof}

\begin{lemma}
\label{lemma-quasi-compact-permanence}
Let $f : X \to Y$ and $g : Y \to Z$ be morphisms of schemes.
If $g \circ f$ is quasi-compact and $g$ is quasi-separated
then $f$ is quasi-compact.
\end{lemma}

\begin{proof}
This is true because $f$ equals the composition
$(1, f) : X \to X \times_Z Y \to Y$. The first map
is quasi-compact by Lemma \ref{lemma-section-immersion}
because it is a section of the quasi-separated morphism $X \times_Z Y \to X$
(a base change of $g$, see Lemma \ref{lemma-separated-permanence}).
The second map is quasi-compact as it
is the base change of $g \circ f$, see
Lemma \ref{lemma-quasi-compact-preserved-base-change}.
And compositions of quasi-compact
morphisms are quasi-compact, see Lemma \ref{lemma-composition-quasi-compact}.
\end{proof}

\begin{lemma}
\label{lemma-affine-separated}
An affine scheme is separated. A morphism from an affine scheme
to another scheme is separated.
\end{lemma}

\begin{proof}
Let $U = \Spec(A)$ be an affine scheme. Then $U \to \Spec(\mathbf{Z})$
has closed diagonal by Lemma \ref{lemma-diagonal-affines-closed}.
Thus $U$ is separated by Definition \ref{definition-separated}.
If $U \to X$ is a morphism of schemes, then we can apply
Lemma \ref{lemma-compose-after-separated}
to the morphisms $U \to X \to \Spec(\mathbf{Z})$ to conclude
that $U \to X$ is separated.
\end{proof}

\noindent
You may have been wondering whether the condition
of only considering pairs of affine opens whose image is contained
in an affine open is really necessary to be able to conclude
that their intersection is affine. Often it isn't!

\begin{lemma}
\label{lemma-curiosity}
Let $f : X \to S$ be a morphism.
Assume $f$ is separated and $S$ is a separated scheme.
Suppose $U \subset X$ and $V \subset X$ are affine.
Then $U \cap V$ is affine (and a closed subscheme of $U \times V$).
\end{lemma}

\begin{proof}
In this case $X$ is separated by Lemma \ref{lemma-separated-permanence}.
Hence $U \cap V$ is affine by
applying Lemma \ref{lemma-characterize-separated} to the
morphism $X \to \Spec(\mathbf{Z})$.
\end{proof}

\noindent
On the other hand, the following example shows that we cannot
expect the image of an affine to be contained in an affine.

\begin{example}
\label{example-image-affine-projective}
Consider the nonaffine scheme
$U = \Spec(k[x, y]) \setminus \{(x, y)\}$ of
Example \ref{example-not-affine}. On the other hand, consider the
scheme
$$
\mathbf{GL}_{2, k} = \Spec(k[a, b, c, d, 1/ad - bc]).
$$
There is a morphism $\mathbf{GL}_{2, k} \to U$ corresponding
to the ring map $x \mapsto a$, $y \mapsto b$. It is easy to see that
this is a surjective morphism, and hence the image is not contained
in any affine open of $U$. In fact, the affine scheme
$\mathbf{GL}_{2, k}$ also surjects onto $\mathbf{P}^1_k$, and
$\mathbf{P}^1_k$ does not even have an immersion into {\it any} affine scheme.
\end{example}

\begin{remark}
\label{remark-quasi-compact-and-quasi-separated}
The category of quasi-compact and quasi-separated schemes $\mathcal{C}$
has the following properties. If $X, Y \in \Ob(\mathcal{C})$, then any
morphism of schemes $f : X \to Y$ is quasi-compact and quasi-separated by
Lemmas \ref{lemma-quasi-compact-permanence} and
\ref{lemma-compose-after-separated}
with $Z = \Spec(\mathbf{Z})$. Moreover, if $X \to Y$ and $Z \to Y$
are morphisms $\mathcal{C}$, then $X \times_Y Z$ is an object of $\mathcal{C}$
too. Namely, the projection $X \times_Y Z \to Z$ is quasi-compact and
quasi-separated as a base change of the morphism $Z \to Y$, see
Lemmas \ref{lemma-separated-permanence} and
\ref{lemma-quasi-compact-preserved-base-change}.
Hence the composition $X \times_Y Z \to Z \to \Spec(\mathbf{Z})$
is quasi-compact and quasi-separated, see
Lemmas \ref{lemma-separated-permanence} and
\ref{lemma-composition-quasi-compact}.
\end{remark}





\section{Valuative criterion of separatedness}
\label{section-valuative-separatedness}

\begin{lemma}
\label{lemma-separated-implies-valuative}
Let $f : X \to S$ be a morphism of schemes.
If $f$ is separated, then $f$ satisfies the uniqueness
part of the valuative criterion.
\end{lemma}

\begin{proof}
Let a diagram as in Definition \ref{definition-valuative-criterion}
be given. Suppose there are two morphisms
$a, b : \Spec(A) \to X$ fitting into the diagram.
Let $Z \subset \Spec(A)$ be the equalizer of $a$ and $b$.
By Lemma \ref{lemma-where-are-they-equal} this is a closed
subscheme of $\Spec(A)$. By assumption it contains
the generic point of $\Spec(A)$. Since $A$ is a domain
this implies $Z = \Spec(A)$. Hence $a = b$ as desired.
\end{proof}

\begin{lemma}[Valuative criterion separatedness]
\label{lemma-valuative-criterion-separatedness}
\begin{reference}
\cite[II Proposition 7.2.3]{EGA}
\end{reference}
Let $f : X \to S$ be a morphism.
Assume
\begin{enumerate}
\item the morphism $f$ is quasi-separated, and
\item the morphism $f$ satisfies the uniqueness
part of the valuative criterion.
\end{enumerate}
Then $f$ is separated.
\end{lemma}

\begin{proof}
By assumption (1),
Proposition \ref{proposition-characterize-universally-closed}, and
Lemmas \ref{lemma-diagonal-immersion} and \ref{lemma-immersion-when-closed}
we see that it suffices to prove the morphism
$\Delta_{X/S} : X \to X \times_S X$ satisfies the existence
part of the valuative criterion.
Let a solid commutative diagram
$$
\xymatrix{
\Spec(K) \ar[r] \ar[d] & X \ar[d] \\
\Spec(A) \ar[r] \ar@{-->}[ru] & X \times_S X
}
$$
be given. The lower right arrow corresponds to a
pair of morphisms $a, b : \Spec(A) \to X$ over $S$.
By (2) we see that $a = b$. Hence using $a$ as the dotted
arrow works.
\end{proof}





\section{Monomorphisms}
\label{section-monomorphisms}

\begin{definition}
\label{definition-monomorphism}
A morphism of schemes is called a {\it monomorphism} if it is
a monomorphism in the category of schemes, see
Categories, Definition \ref{categories-definition-mono-epi}.
\end{definition}

\begin{lemma}
\label{lemma-monomorphism}
\begin{slogan}
A scheme morphism is a monomorphism iff its diagonal is an isomorphism.
\end{slogan}
Let $j : X \to Y$ be a morphism of schemes.
Then $j$ is a monomorphism if and only if the
diagonal morphism $\Delta_{X/Y} : X \to X \times_Y X$ is
an isomorphism.
\end{lemma}

\begin{proof}
This is true in any category with fibre products.
\end{proof}

\begin{lemma}
\label{lemma-monomorphism-separated}
A monomorphism of schemes is separated.
\end{lemma}

\begin{proof}
This is true because an isomorphism is a closed immersion,
and Lemma \ref{lemma-monomorphism} above.
\end{proof}

\begin{lemma}
\label{lemma-composition-monomorphism}
A composition of monomorphisms is a monomorphism.
\end{lemma}

\begin{proof}
True in any category.
\end{proof}

\begin{lemma}
\label{lemma-base-change-monomorphism}
The base change of a monomorphism is a monomorphism.
\end{lemma}

\begin{proof}
True in any category with fibre products.
\end{proof}

\begin{lemma}
\label{lemma-injective-points}
Let $j : X \to Y$ be a morphism of schemes.
If $j$ is injective on points, then $j$ is separated.
\end{lemma}

\begin{proof}
Let $z$ be a point of $X \times_Y X$. Then $x = \text{pr}_1(z)$ and
$\text{pr}_2(z)$ are the same because $j$ maps these points to
the same point $y$ of $Y$. Then we can choose an affine
open neighbourhood $V \subset Y$ of $y$ and an affine open
neighbourhood $U \subset X$ of $x$ with $j(U) \subset V$.
Then $z \in U \times_V U \subset X \times_Y X$.
Hence $X \times_Y X$ is the union of the affine
opens $U \times_V U$. Since $\Delta_{X/Y}^{-1}(U \times_V U) = U$
and since $U \to U \times_V U$ is a closed immersion, we conclude
that $\Delta_{X/Y}$ is a closed immersion (see argument in the
proof of Lemma \ref{lemma-diagonal-immersion}).
\end{proof}

\begin{lemma}
\label{lemma-injective-points-surjective-stalks}
Let $j : X \to Y$ be a morphism of schemes.
If
\begin{enumerate}
\item $j$ is injective on points, and
\item for any $x \in X$ the ring map
$j^\sharp_x : \mathcal{O}_{Y, j(x)} \to \mathcal{O}_{X, x}$
is surjective,
\end{enumerate}
then $j$ is a monomorphism.
\end{lemma}

\begin{proof}
Let $a, b : Z \to X$ be two morphisms of schemes such that
$j \circ a  = j \circ b$.
Then (1) implies $a = b$ as underlying
maps of topological spaces.
For any $z \in Z$ we have
$a^\sharp_z \circ j^\sharp_{a(z)} = b^\sharp_z \circ j^\sharp_{b(z)}$
as maps $\mathcal{O}_{Y, j(a(z))} \to \mathcal{O}_{Z, z}$.
The surjectivity of the maps
$j^\sharp_x$ forces $a^\sharp_z = b^\sharp_z$, $\forall z \in Z$.
This implies that $a^\sharp = b^\sharp$.
Hence we conclude $a = b$ as morphisms of schemes
as desired.
\end{proof}

\begin{lemma}
\label{lemma-immersions-monomorphisms}
An immersion of schemes is a monomorphism.
In particular, any immersion is separated.
\end{lemma}

\begin{proof}
We can see this by checking that the criterion of
Lemma \ref{lemma-injective-points-surjective-stalks} applies.
More elegantly perhaps, we can use that
Lemmas \ref{lemma-restrict-map-to-opens} and
\ref{lemma-characterize-closed-subspace} imply that
open and closed immersions are monomorphisms and hence
any immersion (which is a composition of such)
is a monomorphism.
\end{proof}

\begin{lemma}
\label{lemma-subscheme-of-separated-scheme}
Let $f : X \to S$ be a separated morphism.
Any locally closed subscheme $Z \subset X$ is separated over $S$.
\end{lemma}

\begin{proof}
Follows from Lemma \ref{lemma-immersions-monomorphisms}
and the fact that a composition of separated morphisms is separated
(Lemma \ref{lemma-separated-permanence}).
\end{proof}

\begin{example}
\label{example-Q-over-Z}
The morphism $\Spec(\mathbf{Q}) \to \Spec(\mathbf{Z})$
is a monomorphism. This is true because
$\mathbf{Q} \otimes_{\mathbf{Z}} \mathbf{Q} = \mathbf{Q}$.
More generally, for any scheme $S$ and any point $s \in S$ the
canonical morphism
$$
\Spec(\mathcal{O}_{S, s}) \longrightarrow S
$$
is a monomorphism.
\end{example}

\begin{lemma}
\label{lemma-mono-towards-spec-field}
Let $k_1, \ldots, k_n$ be fields.
For any monomorphism of schemes
$X \to \Spec(k_1 \times \ldots \times k_n)$
there exists a subset $I \subset \{1, \ldots, n\}$ such
that $X \cong \Spec(\prod_{i \in I} k_i)$ as
schemes over $\Spec(k_1 \times \ldots \times k_n)$.
More generally, if $X = \coprod_{i \in I} \Spec(k_i)$
is a disjoint union of spectra of fields and $Y \to X$ is a monomorphism,
then there exists a subset $J \subset I$ such that
$Y = \coprod_{i \in J} \Spec(k_i)$.
\end{lemma}

\begin{proof}
First reduce to the case $n = 1$ (or $\# I = 1$)
by taking the inverse images of the
open and closed subschemes $\Spec(k_i)$.
In this case $X$ has only one point hence is affine.
The corresponding algebra problem is this:
If $k \to R$ is an algebra map
with $R \otimes_k R \cong R$, then $R \cong k$ or $R = 0$.
This holds for dimension reasons.
See also
Algebra, Lemma \ref{algebra-lemma-epimorphism-over-field}
\end{proof}










\section{Functoriality for quasi-coherent modules}
\label{section-quasi-coherent}

\noindent
Let $X$ be a scheme. We denote $\QCoh(\mathcal{O}_X)$ the
category of quasi-coherent $\mathcal{O}_X$-modules as defined in
Modules, Definition \ref{modules-definition-quasi-coherent}.
We have seen in
Section \ref{section-quasi-coherent-affine}
that the category $\QCoh(\mathcal{O}_X)$ has a lot of good properties
when $X$ is affine. Since the property of being quasi-coherent is
local on $X$, these properties are inherited by the category
of quasi-coherent sheaves on any scheme $X$. We enumerate them here.
\begin{enumerate}
\item A sheaf of $\mathcal{O}_X$-modules $\mathcal{F}$ is
quasi-coherent if and only if the restriction of $\mathcal{F}$
to each affine open $U = \Spec(R)$ is of the form
$\widetilde M$ for some $R$-module $M$.
\item A sheaf of $\mathcal{O}_X$-modules $\mathcal{F}$ is
quasi-coherent if and only if the restriction of $\mathcal{F}$
to each of the members of an affine open covering is quasi-coherent.
\item Any direct sum of quasi-coherent sheaves is quasi-coherent.
\item Any colimit of quasi-coherent sheaves is quasi-coherent.
\item
\label{item-kernel}
The kernel and cokernel of a morphism of quasi-coherent sheaves
is quasi-coherent.
\item Given a short exact sequence of $\mathcal{O}_X$-modules
$0 \to \mathcal{F}_1 \to \mathcal{F}_2 \to \mathcal{F}_3 \to 0$
if two out of three are quasi-coherent so is the third.
\item Given a morphism of schemes $f : Y \to X$ the pullback
of a quasi-coherent $\mathcal{O}_X$-module is a quasi-coherent
$\mathcal{O}_Y$-module. See
Modules, Lemma \ref{modules-lemma-pullback-quasi-coherent}.
\item Given two quasi-coherent $\mathcal{O}_X$-modules
the tensor product is quasi-coherent, see
Modules, Lemma \ref{modules-lemma-tensor-product-permanence}.
\item Given a quasi-coherent $\mathcal{O}_X$-module $\mathcal{F}$
the tensor, symmetric and exterior algebras on $\mathcal{F}$
are quasi-coherent, see
Modules, Lemma \ref{modules-lemma-whole-tensor-algebra-permanence}.
\item Given two quasi-coherent $\mathcal{O}_X$-modules
$\mathcal{F}$, $\mathcal{G}$ such that $\mathcal{F}$
is of finite presentation, then the internal hom
$\SheafHom_{\mathcal{O}_X}(\mathcal{F}, \mathcal{G})$
is quasi-coherent, see
Modules, Lemma \ref{modules-lemma-internal-hom-locally-kernel-direct-sum}
and (\ref{item-kernel}) above.
\end{enumerate}
On the other hand, it is in general not the case that the
pushforward of a quasi-coherent module is quasi-coherent.
Here is a case where this does hold.

\begin{lemma}
\label{lemma-push-forward-quasi-coherent}
Let $f : X \to S$ be a morphism of schemes.
If $f$ is quasi-compact and quasi-separated then
$f_*$ transforms quasi-coherent $\mathcal{O}_X$-modules
into quasi-coherent $\mathcal{O}_S$-modules.
\end{lemma}

\begin{proof}
The question is local on $S$ and hence we may assume that
$S$ is affine. Because $X$ is quasi-compact we may write
$X = \bigcup_{i = 1}^n U_i$ with each $U_i$ open affine.
Because $f$ is quasi-separated we may write
$U_i \cap U_j = \bigcup_{k = 1}^{n_{ij}} U_{ijk}$ for some
affine open $U_{ijk}$, see Lemma \ref{lemma-characterize-quasi-separated}.
Denote $f_i : U_i \to S$ and $f_{ijk} : U_{ijk} \to S$ the
restrictions of $f$. For any open $V$ of $S$ and any sheaf
$\mathcal{F}$ on $X$ we have
\begin{eqnarray*}
f_*\mathcal{F}(V) & = & \mathcal{F}(f^{-1}V) \\
& = &
\Ker\left(
\bigoplus\nolimits_i \mathcal{F}(f^{-1}V \cap U_i)
\to
\bigoplus\nolimits_{i, j, k} \mathcal{F}(f^{-1}V \cap U_{ijk})\right) \\
& = &
\Ker\left(
\bigoplus\nolimits_i f_{i, *}(\mathcal{F}|_{U_i})(V)
\to
\bigoplus\nolimits_{i, j, k} f_{ijk, *}(\mathcal{F}|_{U_{ijk}})(V)\right) \\
& = &
\Ker\left(
\bigoplus\nolimits_i f_{i, *}(\mathcal{F}|_{U_i})
\to
\bigoplus\nolimits_{i, j, k} f_{ijk, *}(\mathcal{F}|_{U_{ijk}})\right)(V)
\end{eqnarray*}
In other words there is an exact sequence of sheaves
$$
0 \to f_*\mathcal{F}
\to \bigoplus f_{i, *}\mathcal{F}_i
\to \bigoplus f_{ijk, *}\mathcal{F}_{ijk}
$$
where $\mathcal{F}_i, \mathcal{F}_{ijk}$ denotes the
restriction of $\mathcal{F}$ to the corresponding open.
If $\mathcal{F}$ is a quasi-coherent $\mathcal{O}_X$-module
then $\mathcal{F}_i$ is a quasi-coherent $\mathcal{O}_{U_i}$-module
and $\mathcal{F}_{ijk}$ is a quasi-coherent $\mathcal{O}_{U_{ijk}}$-module.
Hence by Lemma \ref{lemma-widetilde-pullback} we see that the second and
third term of the exact sequence are quasi-coherent
$\mathcal{O}_S$-modules. Thus we conclude that
$f_*\mathcal{F}$ is a quasi-coherent $\mathcal{O}_S$-module.
\end{proof}

\noindent
Using this we can characterize (closed) immersions of schemes
as follows.

\begin{lemma}
\label{lemma-characterize-closed-immersions}
Let $f : X \to Y$ be a morphism of schemes.
Suppose that
\begin{enumerate}
\item $f$ induces a homeomorphism of $X$ with a
closed subset of $Y$, and
\item $f^\sharp : \mathcal{O}_Y \to f_*\mathcal{O}_X$
is surjective.
\end{enumerate}
Then $f$ is a closed immersion of schemes.
\end{lemma}

\begin{proof}
Assume (1) and (2). By (1) the morphism $f$ is quasi-compact
(see Topology, Lemma \ref{topology-lemma-closed-in-quasi-compact}).
Conditions (1) and (2) imply conditions (1) and (2) of
Lemma \ref{lemma-injective-points-surjective-stalks}.
Hence $f : X \to Y$ is a monomorphism. In particular, $f$
is separated, see Lemma \ref{lemma-monomorphism-separated}.
Hence Lemma \ref{lemma-push-forward-quasi-coherent}
above applies and we conclude that $f_*\mathcal{O}_X$ is a quasi-coherent
$\mathcal{O}_Y$-module. Therefore the kernel of
$\mathcal{O}_Y \to f_*\mathcal{O}_X$ is quasi-coherent
by Lemma \ref{lemma-extension-quasi-coherent}. Since a quasi-coherent
sheaf is locally generated by sections (see
Modules, Definition \ref{modules-definition-quasi-coherent})
this implies that $f$ is a closed immersion, see
Definition \ref{definition-closed-immersion-locally-ringed-spaces}.
\end{proof}

\noindent
We can use this lemma to prove the following lemma.

\begin{lemma}
\label{lemma-composition-immersion}
A composition of immersions of schemes is an immersion,
a composition of closed immersions of schemes is a closed immersion, and
a composition of open immersions of schemes is an open immersion.
\end{lemma}

\begin{proof}
This is clear for the case of open immersions since an open subspace of
an open subspace is also an open subspace.

\medskip\noindent
Suppose $a : Z \to Y$ and $b : Y \to X$ are closed immersions of schemes.
We will verify that $c = b \circ a$ is also a closed immersion.
The assumption implies that $a$ and $b$ are homeomorphisms onto closed subsets,
and hence also $c = b \circ a$ is a homeomorphism onto a closed subset.
Moreover, the map $\mathcal{O}_X \to c_*\mathcal{O}_Z$ is surjective
since it factors as the composition of the surjective maps
$\mathcal{O}_X \to b_*\mathcal{O}_Y$ and
$b_*\mathcal{O}_Y \to b_*a_*\mathcal{O}_Z$ (surjective as $b_*$ is exact,
see Modules, Lemma \ref{modules-lemma-i-star-exact}).
Hence by Lemma \ref{lemma-characterize-closed-immersions}
above $c$ is a closed immersion.

\medskip\noindent
Finally, we come to the case of immersions.
Suppose $a : Z \to Y$ and $b : Y \to X$ are immersions of schemes.
This means there exist open subschemes
$V \subset Y$ and $U \subset X$ such that
$a(Z) \subset V$, $b(Y) \subset U$ and $a : Z \to V$
and $b : Y \to U$ are closed immersions.
Since the topology on $Y$ is induced from the topology on
$U$ we can find an open $U' \subset U$ such that
$V = b^{-1}(U')$. Then we see that
$Z \to V = b^{-1}(U') \to U'$ is a composition of
closed immersions and hence a closed immersion.
This proves that $Z \to X$ is an immersion and we win.
\end{proof}











\begin{multicols}{2}[\section{Other chapters}]
\noindent
Preliminaries
\begin{enumerate}
\item \hyperref[introduction-section-phantom]{Introduction}
\item \hyperref[conventions-section-phantom]{Conventions}
\item \hyperref[sets-section-phantom]{Set Theory}
\item \hyperref[categories-section-phantom]{Categories}
\item \hyperref[topology-section-phantom]{Topology}
\item \hyperref[sheaves-section-phantom]{Sheaves on Spaces}
\item \hyperref[sites-section-phantom]{Sites and Sheaves}
\item \hyperref[stacks-section-phantom]{Stacks}
\item \hyperref[fields-section-phantom]{Fields}
\item \hyperref[algebra-section-phantom]{Commutative Algebra}
\item \hyperref[brauer-section-phantom]{Brauer Groups}
\item \hyperref[homology-section-phantom]{Homological Algebra}
\item \hyperref[derived-section-phantom]{Derived Categories}
\item \hyperref[simplicial-section-phantom]{Simplicial Methods}
\item \hyperref[more-algebra-section-phantom]{More on Algebra}
\item \hyperref[smoothing-section-phantom]{Smoothing Ring Maps}
\item \hyperref[modules-section-phantom]{Sheaves of Modules}
\item \hyperref[sites-modules-section-phantom]{Modules on Sites}
\item \hyperref[injectives-section-phantom]{Injectives}
\item \hyperref[cohomology-section-phantom]{Cohomology of Sheaves}
\item \hyperref[sites-cohomology-section-phantom]{Cohomology on Sites}
\item \hyperref[dga-section-phantom]{Differential Graded Algebra}
\item \hyperref[dpa-section-phantom]{Divided Power Algebra}
\item \hyperref[sdga-section-phantom]{Differential Graded Sheaves}
\item \hyperref[hypercovering-section-phantom]{Hypercoverings}
\end{enumerate}
Schemes
\begin{enumerate}
\setcounter{enumi}{25}
\item \hyperref[schemes-section-phantom]{Schemes}
\item \hyperref[constructions-section-phantom]{Constructions of Schemes}
\item \hyperref[properties-section-phantom]{Properties of Schemes}
\item \hyperref[morphisms-section-phantom]{Morphisms of Schemes}
\item \hyperref[coherent-section-phantom]{Cohomology of Schemes}
\item \hyperref[divisors-section-phantom]{Divisors}
\item \hyperref[limits-section-phantom]{Limits of Schemes}
\item \hyperref[varieties-section-phantom]{Varieties}
\item \hyperref[topologies-section-phantom]{Topologies on Schemes}
\item \hyperref[descent-section-phantom]{Descent}
\item \hyperref[perfect-section-phantom]{Derived Categories of Schemes}
\item \hyperref[more-morphisms-section-phantom]{More on Morphisms}
\item \hyperref[flat-section-phantom]{More on Flatness}
\item \hyperref[groupoids-section-phantom]{Groupoid Schemes}
\item \hyperref[more-groupoids-section-phantom]{More on Groupoid Schemes}
\item \hyperref[etale-section-phantom]{\'Etale Morphisms of Schemes}
\end{enumerate}
Topics in Scheme Theory
\begin{enumerate}
\setcounter{enumi}{41}
\item \hyperref[chow-section-phantom]{Chow Homology}
\item \hyperref[intersection-section-phantom]{Intersection Theory}
\item \hyperref[pic-section-phantom]{Picard Schemes of Curves}
\item \hyperref[weil-section-phantom]{Weil Cohomology Theories}
\item \hyperref[adequate-section-phantom]{Adequate Modules}
\item \hyperref[dualizing-section-phantom]{Dualizing Complexes}
\item \hyperref[duality-section-phantom]{Duality for Schemes}
\item \hyperref[discriminant-section-phantom]{Discriminants and Differents}
\item \hyperref[derham-section-phantom]{de Rham Cohomology}
\item \hyperref[local-cohomology-section-phantom]{Local Cohomology}
\item \hyperref[algebraization-section-phantom]{Algebraic and Formal Geometry}
\item \hyperref[curves-section-phantom]{Algebraic Curves}
\item \hyperref[resolve-section-phantom]{Resolution of Surfaces}
\item \hyperref[models-section-phantom]{Semistable Reduction}
\item \hyperref[equiv-section-phantom]{Derived Categories of Varieties}
\item \hyperref[pione-section-phantom]{Fundamental Groups of Schemes}
\item \hyperref[etale-cohomology-section-phantom]{\'Etale Cohomology}
\item \hyperref[crystalline-section-phantom]{Crystalline Cohomology}
\item \hyperref[proetale-section-phantom]{Pro-\'etale Cohomology}
\item \hyperref[more-etale-section-phantom]{More \'Etale Cohomology}
\item \hyperref[trace-section-phantom]{The Trace Formula}
\end{enumerate}
Algebraic Spaces
\begin{enumerate}
\setcounter{enumi}{62}
\item \hyperref[spaces-section-phantom]{Algebraic Spaces}
\item \hyperref[spaces-properties-section-phantom]{Properties of Algebraic Spaces}
\item \hyperref[spaces-morphisms-section-phantom]{Morphisms of Algebraic Spaces}
\item \hyperref[decent-spaces-section-phantom]{Decent Algebraic Spaces}
\item \hyperref[spaces-cohomology-section-phantom]{Cohomology of Algebraic Spaces}
\item \hyperref[spaces-limits-section-phantom]{Limits of Algebraic Spaces}
\item \hyperref[spaces-divisors-section-phantom]{Divisors on Algebraic Spaces}
\item \hyperref[spaces-over-fields-section-phantom]{Algebraic Spaces over Fields}
\item \hyperref[spaces-topologies-section-phantom]{Topologies on Algebraic Spaces}
\item \hyperref[spaces-descent-section-phantom]{Descent and Algebraic Spaces}
\item \hyperref[spaces-perfect-section-phantom]{Derived Categories of Spaces}
\item \hyperref[spaces-more-morphisms-section-phantom]{More on Morphisms of Spaces}
\item \hyperref[spaces-flat-section-phantom]{Flatness on Algebraic Spaces}
\item \hyperref[spaces-groupoids-section-phantom]{Groupoids in Algebraic Spaces}
\item \hyperref[spaces-more-groupoids-section-phantom]{More on Groupoids in Spaces}
\item \hyperref[bootstrap-section-phantom]{Bootstrap}
\item \hyperref[spaces-pushouts-section-phantom]{Pushouts of Algebraic Spaces}
\end{enumerate}
Topics in Geometry
\begin{enumerate}
\setcounter{enumi}{79}
\item \hyperref[spaces-chow-section-phantom]{Chow Groups of Spaces}
\item \hyperref[groupoids-quotients-section-phantom]{Quotients of Groupoids}
\item \hyperref[spaces-more-cohomology-section-phantom]{More on Cohomology of Spaces}
\item \hyperref[spaces-simplicial-section-phantom]{Simplicial Spaces}
\item \hyperref[spaces-duality-section-phantom]{Duality for Spaces}
\item \hyperref[formal-spaces-section-phantom]{Formal Algebraic Spaces}
\item \hyperref[restricted-section-phantom]{Algebraization of Formal Spaces}
\item \hyperref[spaces-resolve-section-phantom]{Resolution of Surfaces Revisited}
\end{enumerate}
Deformation Theory
\begin{enumerate}
\setcounter{enumi}{87}
\item \hyperref[formal-defos-section-phantom]{Formal Deformation Theory}
\item \hyperref[defos-section-phantom]{Deformation Theory}
\item \hyperref[cotangent-section-phantom]{The Cotangent Complex}
\item \hyperref[examples-defos-section-phantom]{Deformation Problems}
\end{enumerate}
Algebraic Stacks
\begin{enumerate}
\setcounter{enumi}{91}
\item \hyperref[algebraic-section-phantom]{Algebraic Stacks}
\item \hyperref[examples-stacks-section-phantom]{Examples of Stacks}
\item \hyperref[stacks-sheaves-section-phantom]{Sheaves on Algebraic Stacks}
\item \hyperref[criteria-section-phantom]{Criteria for Representability}
\item \hyperref[artin-section-phantom]{Artin's Axioms}
\item \hyperref[quot-section-phantom]{Quot and Hilbert Spaces}
\item \hyperref[stacks-properties-section-phantom]{Properties of Algebraic Stacks}
\item \hyperref[stacks-morphisms-section-phantom]{Morphisms of Algebraic Stacks}
\item \hyperref[stacks-limits-section-phantom]{Limits of Algebraic Stacks}
\item \hyperref[stacks-cohomology-section-phantom]{Cohomology of Algebraic Stacks}
\item \hyperref[stacks-perfect-section-phantom]{Derived Categories of Stacks}
\item \hyperref[stacks-introduction-section-phantom]{Introducing Algebraic Stacks}
\item \hyperref[stacks-more-morphisms-section-phantom]{More on Morphisms of Stacks}
\item \hyperref[stacks-geometry-section-phantom]{The Geometry of Stacks}
\end{enumerate}
Topics in Moduli Theory
\begin{enumerate}
\setcounter{enumi}{105}
\item \hyperref[moduli-section-phantom]{Moduli Stacks}
\item \hyperref[moduli-curves-section-phantom]{Moduli of Curves}
\end{enumerate}
Miscellany
\begin{enumerate}
\setcounter{enumi}{107}
\item \hyperref[examples-section-phantom]{Examples}
\item \hyperref[exercises-section-phantom]{Exercises}
\item \hyperref[guide-section-phantom]{Guide to Literature}
\item \hyperref[desirables-section-phantom]{Desirables}
\item \hyperref[coding-section-phantom]{Coding Style}
\item \hyperref[obsolete-section-phantom]{Obsolete}
\item \hyperref[fdl-section-phantom]{GNU Free Documentation License}
\item \hyperref[index-section-phantom]{Auto Generated Index}
\end{enumerate}
\end{multicols}


\bibliography{my}
\bibliographystyle{amsalpha}

\end{document}
