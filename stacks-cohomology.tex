\IfFileExists{stacks-project.cls}{%
\documentclass{stacks-project}
}{%
\documentclass{amsart}
}

% For dealing with references we use the comment environment
\usepackage{verbatim}
\newenvironment{reference}{\comment}{\endcomment}
%\newenvironment{reference}{}{}
\newenvironment{slogan}{\comment}{\endcomment}
\newenvironment{history}{\comment}{\endcomment}

% For commutative diagrams we use Xy-pic
\usepackage[all]{xy}

% We use 2cell for 2-commutative diagrams.
\xyoption{2cell}
\UseAllTwocells

% We use multicol for the list of chapters between chapters
\usepackage{multicol}

% This is generall recommended for better output
\usepackage{lmodern}
\usepackage[T1]{fontenc}

% For cross-file-references
\usepackage{xr-hyper}

% Package for hypertext links:
\usepackage{hyperref}

% For any local file, say "hello.tex" you want to link to please
% use \externaldocument[hello-]{hello}
\externaldocument[introduction-]{introduction}
\externaldocument[conventions-]{conventions}
\externaldocument[sets-]{sets}
\externaldocument[categories-]{categories}
\externaldocument[topology-]{topology}
\externaldocument[sheaves-]{sheaves}
\externaldocument[sites-]{sites}
\externaldocument[stacks-]{stacks}
\externaldocument[fields-]{fields}
\externaldocument[algebra-]{algebra}
\externaldocument[brauer-]{brauer}
\externaldocument[homology-]{homology}
\externaldocument[derived-]{derived}
\externaldocument[simplicial-]{simplicial}
\externaldocument[more-algebra-]{more-algebra}
\externaldocument[smoothing-]{smoothing}
\externaldocument[modules-]{modules}
\externaldocument[sites-modules-]{sites-modules}
\externaldocument[injectives-]{injectives}
\externaldocument[cohomology-]{cohomology}
\externaldocument[sites-cohomology-]{sites-cohomology}
\externaldocument[dga-]{dga}
\externaldocument[dpa-]{dpa}
\externaldocument[sdga-]{sdga}
\externaldocument[hypercovering-]{hypercovering}
\externaldocument[schemes-]{schemes}
\externaldocument[constructions-]{constructions}
\externaldocument[properties-]{properties}
\externaldocument[morphisms-]{morphisms}
\externaldocument[coherent-]{coherent}
\externaldocument[divisors-]{divisors}
\externaldocument[limits-]{limits}
\externaldocument[varieties-]{varieties}
\externaldocument[topologies-]{topologies}
\externaldocument[descent-]{descent}
\externaldocument[perfect-]{perfect}
\externaldocument[more-morphisms-]{more-morphisms}
\externaldocument[flat-]{flat}
\externaldocument[groupoids-]{groupoids}
\externaldocument[more-groupoids-]{more-groupoids}
\externaldocument[etale-]{etale}
\externaldocument[chow-]{chow}
\externaldocument[intersection-]{intersection}
\externaldocument[pic-]{pic}
\externaldocument[weil-]{weil}
\externaldocument[adequate-]{adequate}
\externaldocument[dualizing-]{dualizing}
\externaldocument[duality-]{duality}
\externaldocument[discriminant-]{discriminant}
\externaldocument[derham-]{derham}
\externaldocument[local-cohomology-]{local-cohomology}
\externaldocument[algebraization-]{algebraization}
\externaldocument[curves-]{curves}
\externaldocument[resolve-]{resolve}
\externaldocument[models-]{models}
\externaldocument[equiv-]{equiv}
\externaldocument[pione-]{pione}
\externaldocument[etale-cohomology-]{etale-cohomology}
\externaldocument[proetale-]{proetale}
\externaldocument[more-etale-]{more-etale}
\externaldocument[trace-]{trace}
\externaldocument[crystalline-]{crystalline}
\externaldocument[spaces-]{spaces}
\externaldocument[spaces-properties-]{spaces-properties}
\externaldocument[spaces-morphisms-]{spaces-morphisms}
\externaldocument[decent-spaces-]{decent-spaces}
\externaldocument[spaces-cohomology-]{spaces-cohomology}
\externaldocument[spaces-limits-]{spaces-limits}
\externaldocument[spaces-divisors-]{spaces-divisors}
\externaldocument[spaces-over-fields-]{spaces-over-fields}
\externaldocument[spaces-topologies-]{spaces-topologies}
\externaldocument[spaces-descent-]{spaces-descent}
\externaldocument[spaces-perfect-]{spaces-perfect}
\externaldocument[spaces-more-morphisms-]{spaces-more-morphisms}
\externaldocument[spaces-flat-]{spaces-flat}
\externaldocument[spaces-groupoids-]{spaces-groupoids}
\externaldocument[spaces-more-groupoids-]{spaces-more-groupoids}
\externaldocument[bootstrap-]{bootstrap}
\externaldocument[spaces-pushouts-]{spaces-pushouts}
\externaldocument[spaces-chow-]{spaces-chow}
\externaldocument[groupoids-quotients-]{groupoids-quotients}
\externaldocument[spaces-more-cohomology-]{spaces-more-cohomology}
\externaldocument[spaces-simplicial-]{spaces-simplicial}
\externaldocument[spaces-duality-]{spaces-duality}
\externaldocument[formal-spaces-]{formal-spaces}
\externaldocument[restricted-]{restricted}
\externaldocument[spaces-resolve-]{spaces-resolve}
\externaldocument[formal-defos-]{formal-defos}
\externaldocument[defos-]{defos}
\externaldocument[cotangent-]{cotangent}
\externaldocument[examples-defos-]{examples-defos}
\externaldocument[algebraic-]{algebraic}
\externaldocument[examples-stacks-]{examples-stacks}
\externaldocument[stacks-sheaves-]{stacks-sheaves}
\externaldocument[criteria-]{criteria}
\externaldocument[artin-]{artin}
\externaldocument[quot-]{quot}
\externaldocument[stacks-properties-]{stacks-properties}
\externaldocument[stacks-morphisms-]{stacks-morphisms}
\externaldocument[stacks-limits-]{stacks-limits}
\externaldocument[stacks-cohomology-]{stacks-cohomology}
\externaldocument[stacks-perfect-]{stacks-perfect}
\externaldocument[stacks-introduction-]{stacks-introduction}
\externaldocument[stacks-more-morphisms-]{stacks-more-morphisms}
\externaldocument[stacks-geometry-]{stacks-geometry}
\externaldocument[moduli-]{moduli}
\externaldocument[moduli-curves-]{moduli-curves}
\externaldocument[examples-]{examples}
\externaldocument[exercises-]{exercises}
\externaldocument[guide-]{guide}
\externaldocument[desirables-]{desirables}
\externaldocument[coding-]{coding}
\externaldocument[obsolete-]{obsolete}
\externaldocument[fdl-]{fdl}
\externaldocument[index-]{index}

% Theorem environments.
%
\theoremstyle{plain}
\newtheorem{theorem}[subsection]{Theorem}
\newtheorem{proposition}[subsection]{Proposition}
\newtheorem{lemma}[subsection]{Lemma}

\theoremstyle{definition}
\newtheorem{definition}[subsection]{Definition}
\newtheorem{example}[subsection]{Example}
\newtheorem{exercise}[subsection]{Exercise}
\newtheorem{situation}[subsection]{Situation}

\theoremstyle{remark}
\newtheorem{remark}[subsection]{Remark}
\newtheorem{remarks}[subsection]{Remarks}

\numberwithin{equation}{subsection}

% Macros
%
\def\lim{\mathop{\mathrm{lim}}\nolimits}
\def\colim{\mathop{\mathrm{colim}}\nolimits}
\def\Spec{\mathop{\mathrm{Spec}}}
\def\Hom{\mathop{\mathrm{Hom}}\nolimits}
\def\Ext{\mathop{\mathrm{Ext}}\nolimits}
\def\SheafHom{\mathop{\mathcal{H}\!\mathit{om}}\nolimits}
\def\SheafExt{\mathop{\mathcal{E}\!\mathit{xt}}\nolimits}
\def\Sch{\mathit{Sch}}
\def\Mor{\mathop{\mathrm{Mor}}\nolimits}
\def\Ob{\mathop{\mathrm{Ob}}\nolimits}
\def\Sh{\mathop{\mathit{Sh}}\nolimits}
\def\NL{\mathop{N\!L}\nolimits}
\def\CH{\mathop{\mathrm{CH}}\nolimits}
\def\proetale{{pro\text{-}\acute{e}tale}}
\def\etale{{\acute{e}tale}}
\def\QCoh{\mathit{QCoh}}
\def\Ker{\mathop{\mathrm{Ker}}}
\def\Im{\mathop{\mathrm{Im}}}
\def\Coker{\mathop{\mathrm{Coker}}}
\def\Coim{\mathop{\mathrm{Coim}}}

% Boxtimes
%
\DeclareMathSymbol{\boxtimes}{\mathbin}{AMSa}{"02}

%
% Macros for moduli stacks/spaces
%
\def\QCohstack{\mathcal{QC}\!\mathit{oh}}
\def\Cohstack{\mathcal{C}\!\mathit{oh}}
\def\Spacesstack{\mathcal{S}\!\mathit{paces}}
\def\Quotfunctor{\mathrm{Quot}}
\def\Hilbfunctor{\mathrm{Hilb}}
\def\Curvesstack{\mathcal{C}\!\mathit{urves}}
\def\Polarizedstack{\mathcal{P}\!\mathit{olarized}}
\def\Complexesstack{\mathcal{C}\!\mathit{omplexes}}
% \Pic is the operator that assigns to X its picard group, usage \Pic(X)
% \Picardstack_{X/B} denotes the Picard stack of X over B
% \Picardfunctor_{X/B} denotes the Picard functor of X over B
\def\Pic{\mathop{\mathrm{Pic}}\nolimits}
\def\Picardstack{\mathcal{P}\!\mathit{ic}}
\def\Picardfunctor{\mathrm{Pic}}
\def\Deformationcategory{\mathcal{D}\!\mathit{ef}}


% OK, start here.
%
\begin{document}

\title{Cohomology of Algebraic Stacks}

\maketitle

\phantomsection
\label{section-phantom}

\tableofcontents




\section{Introduction}
\label{section-introduction}

\noindent
In this chapter we write about cohomology of algebraic stacks.
This means in particular cohomology of quasi-coherent sheaves, i.e.,
we prove analogues of the results in the chapters entitled
``Cohomology of Schemes'' and ``Cohomology of Algebraic Spaces''.
The results in this chapter are different
from those in \cite{LM-B} mainly because we consistently use the
``big sites''. Before reading this chapter please take a quick look at
the chapter ``Sheaves on Algebraic Stacks'' in order to become
familiar with the terminology introduced there, see
Sheaves on Stacks, Section \ref{stacks-sheaves-section-introduction}.



\section{Conventions and abuse of language}
\label{section-conventions}

\noindent
We continue to use the conventions and the abuse of language
introduced in
Properties of Stacks, Section \ref{stacks-properties-section-conventions}.











\section{Notation}
\label{section-notation}

\noindent
Different topologies. If we indicate an algebraic stack by a calligraphic
letter, such as $\mathcal{X}, \mathcal{Y}, \mathcal{Z}$, then the notation
$\mathcal{X}_{Zar}, \mathcal{X}_\etale, \mathcal{X}_{smooth},
\mathcal{X}_{syntomic}, \mathcal{X}_{fppf}$ indicates the site introduced
in
Sheaves on Stacks, Definition
\ref{stacks-sheaves-definition-inherited-topologies}.
(Think ``big site''.) Correspondingly the structure sheaf of
$\mathcal{X}$ is a sheaf on $\mathcal{X}_{fppf}$.
On the other hand, algebraic spaces and schemes
are usually indicated by roman capitals, such as $X, Y, Z$, and in this case
$X_\etale$ indicates the small \'etale site of $X$ (as
defined in
Topologies, Definition
\ref{topologies-definition-big-small-etale}
or
Properties of Spaces, Definition
\ref{spaces-properties-definition-etale-site}).
It seems that the distinction should be clear enough.

\medskip\noindent
The default topology is the fppf topology. Hence we will sometimes
say ``sheaf on $\mathcal{X}$'' or ``sheaf of $\mathcal{O}_\mathcal{X}$-modules''
when we mean sheaf on $\mathcal{X}_{fppf}$ or object of
$\textit{Mod}(\mathcal{X}_{fppf}, \mathcal{O}_\mathcal{X})$.

\medskip\noindent
If $f : \mathcal{X} \to \mathcal{Y}$ is a morphism of algebraic
stacks, then the functors $f_*$ and $f^{-1}$ defined on presheaves
preserves sheaves for any of the topologies mentioned above. In particular
when we discuss the pushforward or pullback of a sheaf we don't have to
mention which topology we are working with. The same isn't true
when we compute cohomology groups and/or higher direct images. In this
case we will always mention which topology we are working with.

\medskip\noindent
Suppose that $f : X \to \mathcal{Y}$ is a morphism from an algebraic
space $X$ to an algebraic stack $\mathcal{Y}$. Let $\mathcal{G}$ be
a sheaf on $\mathcal{Y}_\tau$ for some topology $\tau$. In this case
$f^{-1}\mathcal{G}$ is a sheaf for the $\tau$ topology on $\mathcal{S}_X$
(the algebraic stack associated to $X$) because (by our conventions) $f$
really is a $1$-morphism $f : \mathcal{S}_X \to \mathcal{Y}$.
If $\tau = \etale$ or stronger, then we write
$f^{-1}\mathcal{G}|_{X_\etale}$
to denote the restriction to the \'etale site of $X$, see
Sheaves on Stacks, Section \ref{stacks-sheaves-section-compare}.
If $\mathcal{G}$ is an $\mathcal{O}_\mathcal{X}$-module we sometimes
write $f^*\mathcal{G}$ and $f^*\mathcal{G}|_{X_\etale}$
instead.




\section{Pullback of quasi-coherent modules}
\label{section-pullback}

\noindent
Let $f : \mathcal{X} \to \mathcal{Y}$ be a morphism of algebraic stacks.
It is a very general fact that quasi-coherent modules on ringed topoi
are compatible with pullbacks. In particular the pullback $f^*$ preserves
quasi-coherent modules and we obtain a functor
$$
f^* :
\QCoh(\mathcal{O}_\mathcal{Y})
\longrightarrow
\QCoh(\mathcal{O}_\mathcal{X}),
$$
see Sheaves on Stacks, Lemma
\ref{stacks-sheaves-lemma-pullback-quasi-coherent}.
In general this functor isn't exact, but if $f$ is flat then it is.

\begin{lemma}
\label{lemma-flat-pullback-quasi-coherent}
If $f : \mathcal{X} \to \mathcal{Y}$ is a flat morphism of algebraic stacks
then $f^* : \QCoh(\mathcal{O}_\mathcal{Y}) \to
\QCoh(\mathcal{O}_\mathcal{X})$ is an exact functor.
\end{lemma}

\begin{proof}
Choose a scheme $V$ and a surjective smooth morphism $V \to \mathcal{Y}$.
Choose a scheme $U$ and a surjective smooth morphism
$U \to V \times_\mathcal{Y} \mathcal{X}$. Then $U \to \mathcal{X}$ is
still smooth and surjective as a composition of two such morphisms.
From the commutative diagram
$$
\xymatrix{
U \ar[d] \ar[r]_{f'} & V \ar[d] \\
\mathcal{X} \ar[r]^f & \mathcal{Y}
}
$$
we obtain a commutative diagram
$$
\xymatrix{
\QCoh(\mathcal{O}_U) & \QCoh(\mathcal{O}_V) \ar[l] \\
\QCoh(\mathcal{O}_\mathcal{X}) \ar[u] &
\QCoh(\mathcal{O}_\mathcal{Y}) \ar[l] \ar[u]
}
$$
of abelian categories. Our proof that the bottom two categories in this
diagram are abelian showed that the vertical functors are faithful
exact functors (see proof of
Sheaves on Stacks, Lemma
\ref{stacks-sheaves-lemma-quasi-coherent-algebraic-stack}).
Since $f'$ is a flat morphism of schemes (by our definition of
flat morphisms of algebraic stacks) we see that $(f')^*$ is an
exact functor on quasi-coherent sheaves on $V$. Thus we win.
\end{proof}






\section{The key lemma}
\label{section-key}

\noindent
The following lemma is the basis for our understanding of
higher direct images of certain types of sheaves of modules.
There are two versions: one for the \'etale topology and
one for the fppf topology.

\begin{lemma}
\label{lemma-general-pushforward}
Let $\mathcal{M}$ be a rule which associates to every algebraic stack
$\mathcal{X}$ a subcategory $\mathcal{M}_\mathcal{X}$ of
$\textit{Mod}(\mathcal{X}_\etale, \mathcal{O}_\mathcal{X})$
such that
\begin{enumerate}
\item $\mathcal{M}_\mathcal{X}$ is a weak Serre subcategory
of $\textit{Mod}(\mathcal{X}_\etale, \mathcal{O}_\mathcal{X})$
(see Homology, Definition \ref{homology-definition-serre-subcategory})
for all algebraic stacks $\mathcal{X}$,
\item for a smooth morphism of algebraic stacks
$f : \mathcal{Y} \to \mathcal{X}$ the functor $f^*$ maps
$\mathcal{M}_\mathcal{X}$ into $\mathcal{M}_\mathcal{Y}$,
\item if $f_i : \mathcal{X}_i \to \mathcal{X}$ is a family of smooth
morphisms of algebraic stacks with
$|\mathcal{X}| = \bigcup |f_i|(|\mathcal{X}_i|)$, then an object
$\mathcal{F}$ of
$\textit{Mod}(\mathcal{X}_\etale, \mathcal{O}_\mathcal{X})$
is in $\mathcal{M}_\mathcal{X}$ if and only if
$f_i^*\mathcal{F}$ is in $\mathcal{M}_{\mathcal{X}_i}$ for all $i$, and
\item if $f : \mathcal{Y} \to \mathcal{X}$ is a morphism of algebraic
stacks such that $\mathcal{X}$ and $\mathcal{Y}$ are representable
by affine schemes, then $R^if_*$ maps $\mathcal{M}_\mathcal{Y}$
into $\mathcal{M}_\mathcal{X}$.
\end{enumerate}
Then for any quasi-compact and quasi-separated morphism 
$f : \mathcal{Y} \to \mathcal{X}$ of algebraic stacks
$R^if_*$ maps $\mathcal{M}_\mathcal{Y}$
into $\mathcal{M}_\mathcal{X}$. (Higher direct images computed in \'etale
topology.)
\end{lemma}

\begin{proof}
Let $f : \mathcal{Y} \to \mathcal{X}$ be a quasi-compact and quasi-separated
morphism of algebraic stacks and let $\mathcal{F}$ be an object of
$\mathcal{M}_\mathcal{Y}$. Choose a surjective smooth morphism
$\mathcal{U} \to \mathcal{X}$ where $\mathcal{U}$ is representable by
a scheme. By
Sheaves on Stacks, Lemma
\ref{stacks-sheaves-lemma-base-change-higher-direct-images}
taking higher direct images commutes with base change.
Assumption (2) shows that the pullback of $\mathcal{F}$
to $\mathcal{U} \times_\mathcal{X} \mathcal{Y}$ is in
$\mathcal{M}_{\mathcal{U} \times_\mathcal{X} \mathcal{Y}}$
because the projection
$\mathcal{U} \times_\mathcal{X} \mathcal{Y} \to \mathcal{Y}$
is smooth as a base change of a smooth morphism. Hence (3) shows we may
replace $\mathcal{Y} \to \mathcal{X}$ by the projection
$\mathcal{U} \times_\mathcal{X} \mathcal{Y} \to \mathcal{U}$.
In other words, we may assume that $\mathcal{X}$
is representable by a scheme.
Using (3) once more, we see that the question is Zariski local on
$\mathcal{X}$, hence we may assume that $\mathcal{X}$ is representable by
an affine scheme. Since $f$ is quasi-compact this implies that also
$\mathcal{Y}$ is quasi-compact. Thus we may choose a surjective smooth
morphism $g : \mathcal{V} \to \mathcal{Y}$ where $\mathcal{V}$ is representable
by an affine scheme.

\medskip\noindent
In this situation we have the spectral sequence
$$
E_2^{p, q} = R^q(f \circ g_p)_*g_p^*\mathcal{F}
\Rightarrow
R^{p + q}f_*\mathcal{F}
$$
of
Sheaves on Stacks, Proposition
\ref{stacks-sheaves-proposition-smooth-covering-compute-direct-image}.
Recall that this is a first quadrant spectral sequence hence we may
use the last part of Homology, Lemma \ref{homology-lemma-first-quadrant-ss}.
Note that the morphisms
$$
g_p : \mathcal{V}_p =
\mathcal{V} \times_\mathcal{Y} \ldots \times_\mathcal{Y} \mathcal{V}
\longrightarrow
\mathcal{Y}
$$
are smooth as compositions of base changes of the smooth morphism $g$.
Thus the sheaves $g_p^*\mathcal{F}$ are in
$\mathcal{M}_{\mathcal{V}_p}$ by (2). Hence it suffices to prove that the
higher direct images of objects of $\mathcal{M}_{\mathcal{V}_p}$ under
the morphisms
$$
\mathcal{V}_p =
\mathcal{V} \times_\mathcal{Y} \ldots \times_\mathcal{Y} \mathcal{V}
\longrightarrow
\mathcal{X}
$$
are in $\mathcal{M}_\mathcal{X}$. The algebraic stacks $\mathcal{V}_p$
are quasi-compact and quasi-separated by
Morphisms of Stacks, Lemma
\ref{stacks-morphisms-lemma-quasi-compact-quasi-separated-permanence}.
Of course each $\mathcal{V}_p$ is representable by an algebraic space
(the diagonal of the algebraic stack $\mathcal{Y}$ is representable
by algebraic spaces). This reduces us to the case where
$\mathcal{Y}$ is representable by an algebraic space and $\mathcal{X}$
is representable by an affine scheme.

\medskip\noindent
In the situation where $\mathcal{Y}$ is representable by an algebraic
space and $\mathcal{X}$ is representable by an affine scheme, we choose
anew a surjective smooth morphism $\mathcal{V} \to \mathcal{Y}$ where
$\mathcal{V}$ is representable by an affine scheme. Going through the
argument above once again we once again reduce to the morphisms
$\mathcal{V}_p \to \mathcal{X}$. But in the current situation the algebraic
stacks $\mathcal{V}_p$ are representable by quasi-compact and quasi-separated
schemes (because the diagonal of an algebraic space is representable by
schemes).

\medskip\noindent
Thus we may assume $\mathcal{Y}$ is representable by a scheme and
$\mathcal{X}$ is representable by an affine scheme. Choose (again)
a surjective smooth morphism $\mathcal{V} \to \mathcal{Y}$ where
$\mathcal{V}$ is representable by an affine scheme. In this case all
the algebraic stacks $\mathcal{V}_p$ are representable by separated
schemes (because the diagonal of a scheme is separated).

\medskip\noindent
Thus we may assume $\mathcal{Y}$ is representable by a separated scheme and
$\mathcal{X}$ is representable by an affine scheme. Choose (yet again)
a surjective smooth morphism $\mathcal{V} \to \mathcal{Y}$ where
$\mathcal{V}$ is representable by an affine scheme. In this case all
the algebraic stacks $\mathcal{V}_p$ are representable by affine schemes
(because the diagonal of a separated scheme is a closed immersion hence affine)
and this case is handled by assumption (4).
This finishes the proof.
\end{proof}

\noindent
Here is the version for the fppf topology.

\begin{lemma}
\label{lemma-general-pushforward-fppf}
Let $\mathcal{M}$ be a rule which associates to every algebraic stack
$\mathcal{X}$ a subcategory $\mathcal{M}_\mathcal{X}$ of
$\textit{Mod}(\mathcal{O}_\mathcal{X})$
such that
\begin{enumerate}
\item $\mathcal{M}_\mathcal{X}$ is a weak Serre subcategory
of $\textit{Mod}(\mathcal{O}_\mathcal{X})$
for all algebraic stacks $\mathcal{X}$,
\item for a smooth morphism of algebraic stacks
$f : \mathcal{Y} \to \mathcal{X}$ the functor $f^*$ maps
$\mathcal{M}_\mathcal{X}$ into $\mathcal{M}_\mathcal{Y}$,
\item if $f_i : \mathcal{X}_i \to \mathcal{X}$ is a family of smooth
morphisms of algebraic stacks with
$|\mathcal{X}| = \bigcup |f_i|(|\mathcal{X}_i|)$, then an object
$\mathcal{F}$ of $\textit{Mod}(\mathcal{O}_\mathcal{X})$
is in $\mathcal{M}_\mathcal{X}$ if and only if
$f_i^*\mathcal{F}$ is in $\mathcal{M}_{\mathcal{X}_i}$ for all $i$, and
\item if $f : \mathcal{Y} \to \mathcal{X}$ is a morphism of algebraic
stacks and $\mathcal{X}$ and $\mathcal{Y}$ are representable
by affine schemes, then $R^if_*$ maps $\mathcal{M}_\mathcal{Y}$
into $\mathcal{M}_\mathcal{X}$.
\end{enumerate}
Then for any quasi-compact and quasi-separated morphism 
$f : \mathcal{Y} \to \mathcal{X}$ of algebraic stacks
$R^if_*$ maps $\mathcal{M}_\mathcal{Y}$
into $\mathcal{M}_\mathcal{X}$. (Higher direct images computed in fppf
topology.)
\end{lemma}

\begin{proof}
Identical to the proof of Lemma \ref{lemma-general-pushforward}.
\end{proof}


\section{Locally quasi-coherent modules}
\label{section-locally-quasi-coherent}

\noindent
Let $\mathcal{X}$ be an algebraic stack. Let $\mathcal{F}$ be a presheaf
of $\mathcal{O}_\mathcal{X}$-modules. We can ask whether $\mathcal{F}$
is {\it locally quasi-coherent}, see
Sheaves on Stacks, Definition
\ref{stacks-sheaves-definition-locally-quasi-coherent}.
Briefly, this means $\mathcal{F}$ is an $\mathcal{O}_\mathcal{X}$-module
for the \'etale topology such that for any morphism $f : U \to \mathcal{X}$
the restriction $f^*\mathcal{F}|_{U_\etale}$ is quasi-coherent
on $U_\etale$. (The actual definition is slightly different, but
equivalent.) A useful fact is that
$$
\textit{LQCoh}(\mathcal{O}_\mathcal{X}) \subset
\textit{Mod}(\mathcal{X}_\etale, \mathcal{O}_\mathcal{X})
$$
is a weak Serre subcategory, see
Sheaves on Stacks, Lemma \ref{stacks-sheaves-lemma-lqc-colimits}.

\begin{lemma}
\label{lemma-check-lqc-on-etale-covering}
Let $\mathcal{X}$ be an algebraic stack. Let
$f_j : \mathcal{X}_j \to \mathcal{X}$ be a family of smooth
morphisms of algebraic stacks with
$|\mathcal{X}| =\bigcup |f_j|(|\mathcal{X}_j|)$.
Let $\mathcal{F}$ be a sheaf of $\mathcal{O}_\mathcal{X}$-modules
on $\mathcal{X}_\etale$. If each $f_j^{-1}\mathcal{F}$
is locally quasi-coherent, then so is $\mathcal{F}$.
\end{lemma}

\begin{proof}
We may replace each of the algebraic stacks $\mathcal{X}_j$ by
a scheme $U_j$ (using that any algebraic stack has a smooth covering by
a scheme and that compositions of smooth morphisms are smooth, see
Morphisms of Stacks, Lemma \ref{stacks-morphisms-lemma-composition-smooth}).
The pullback of $\mathcal{F}$ to $(\Sch/U_j)_\etale$ is still
locally quasi-coherent, see
Sheaves on Stacks, Lemma \ref{stacks-sheaves-lemma-pullback-lqc}.
Then $f = \coprod f_j : U = \coprod U_j \to \mathcal{X}$ is a surjective
smooth morphism. Let $x$ be an object of $\mathcal{X}$. By
Sheaves on Stacks, Lemma
\ref{stacks-sheaves-lemma-surjective-flat-locally-finite-presentation}
there exists an \'etale covering $\{x_i \to x\}_{i \in I}$
such that each $x_i$ lifts to an object $u_i$ of $(\Sch/U)_\etale$.
This just means that $x$, $x_i$ live over schemes $V$, $V_i$, that
$\{V_i \to V\}$ is an \'etale covering, and that $x_i$ comes from
a morphism $u_i : V_i \to U$. The restriction
$x_i^*\mathcal{F}|_{V_{i, \etale}}$ is equal to the restriction
of $f^*\mathcal{F}$ to $V_{i, \etale}$, see
Sheaves on Stacks, Lemma \ref{stacks-sheaves-lemma-comparison}.
Hence $x^*\mathcal{F}|_{V_\etale}$
is a sheaf on the small \'etale site of $V$ which is quasi-coherent
when restricted to $V_{i, \etale}$ for each $i$.
This implies that it is quasi-coherent (as desired), for example by
Properties of Spaces, Lemma
\ref{spaces-properties-lemma-characterize-quasi-coherent}.
\end{proof}

\begin{lemma}
\label{lemma-pushforward-locally-quasi-coherent}
Let $f : \mathcal{X} \to \mathcal{Y}$ be a quasi-compact and
quasi-separated morphism of algebraic stacks. Let 
$\mathcal{F}$ be a locally quasi-coherent
$\mathcal{O}_\mathcal{X}$-module on $\mathcal{X}_\etale$.
Then $R^if_*\mathcal{F}$ (computed in the \'etale topology) is
locally quasi-coherent on $\mathcal{Y}_\etale$.
\end{lemma}

\begin{proof}
We will use
Lemma \ref{lemma-general-pushforward}
to prove this. We will check its assumptions (1) -- (4).
Parts (1) and (2) follows from
Sheaves on Stacks, Lemma \ref{stacks-sheaves-lemma-lqc-colimits}.
Part (3) follows from
Lemma \ref{lemma-check-lqc-on-etale-covering}.
Thus it suffices to show (4).

\medskip\noindent
Suppose $f : \mathcal{X} \to \mathcal{Y}$ is a morphism of algebraic stacks
such that $\mathcal{X}$ and $\mathcal{Y}$ are representable by affine
schemes $X$ and $Y$. Choose any object $y$ of $\mathcal{Y}$ lying over a
scheme $V$. For clarity, denote $\mathcal{V} = (\Sch/V)_{fppf}$ the
algebraic stack corresponding to $V$. Consider the cartesian diagram
$$
\xymatrix{
\mathcal{Z} \ar[d] \ar[r]_g \ar[d]_{f'} & \mathcal{X} \ar[d]^f \\
\mathcal{V} \ar[r]^y & \mathcal{Y}
}
$$
Thus $\mathcal{Z}$ is representable by the scheme $Z = V \times_Y X$
and $f'$ is quasi-compact and separated (even affine). By
Sheaves on Stacks, Lemma
\ref{stacks-sheaves-lemma-compare-representable-morphism-cohomology}
we have
$$
R^if_*\mathcal{F}|_{V_\etale} =
R^if'_{small, *}\big(g^*\mathcal{F}|_{Z_\etale}\big)
$$
The right hand side is a quasi-coherent sheaf on $V_\etale$ by
Cohomology of Spaces, Lemma
\ref{spaces-cohomology-lemma-higher-direct-image}.
This implies the left hand side is quasi-coherent which is what
we had to prove.
\end{proof}

\begin{lemma}
\label{lemma-check-lqc-on-flat-covering}
Let $\mathcal{X}$ be an algebraic stack. Let
$f_j : \mathcal{X}_j \to \mathcal{X}$ be a family of flat
and locally finitely presented morphisms of algebraic stacks with
$|\mathcal{X}| =\bigcup |f_j|(|\mathcal{X}_j|)$.
Let $\mathcal{F}$ be a sheaf of $\mathcal{O}_\mathcal{X}$-modules
on $\mathcal{X}_{fppf}$. If each $f_j^{-1}\mathcal{F}$
is locally quasi-coherent, then so is $\mathcal{F}$.
\end{lemma}

\begin{proof}
First, suppose there is a morphism $a : \mathcal{U} \to \mathcal{X}$
which is surjective, flat, locally of finite presentation, quasi-compact,
and quasi-separated such that $a^*\mathcal{F}$ is locally quasi-coherent.
Then there is an exact sequence
$$
0 \to \mathcal{F} \to a_*a^*\mathcal{F} \to b_*b^*\mathcal{F}
$$
where $b$ is the morphism
$b : \mathcal{U} \times_\mathcal{X} \mathcal{U} \to \mathcal{X}$, see
Sheaves on Stacks, Proposition
\ref{stacks-sheaves-proposition-exactness-cech-complex} and
Lemma \ref{stacks-sheaves-lemma-surjective-flat-locally-finite-presentation}.
Moreover, the pullback $b^*\mathcal{F}$ is the pullback of $a^*\mathcal{F}$
via one of the projection morphisms, hence is locally quasi-coherent
(Sheaves on Stacks, Lemma \ref{stacks-sheaves-lemma-pullback-lqc}).
The modules $a_*a^*\mathcal{F}$ and $b_*b^*\mathcal{F}$ are locally
quasi-coherent by Lemma \ref{lemma-pushforward-locally-quasi-coherent}.
(Note that $a_*$ and $b_*$ don't care about which topology is
used to calculate them.)
We conclude that $\mathcal{F}$ is locally quasi-coherent, see
Sheaves on Stacks, Lemma \ref{stacks-sheaves-lemma-lqc-colimits}.

\medskip\noindent
We are going to reduce the proof of the general case the
situation in the first paragraph. Let $x$ be an object of $\mathcal{X}$
lying over the scheme $U$. We have to show that
$\mathcal{F}|_{U_\etale}$ is a quasi-coherent $\mathcal{O}_U$-module.
It suffices to do this (Zariski) locally on $U$, hence we may
assume that $U$ is affine. By
Morphisms of Stacks, Lemma
\ref{stacks-morphisms-lemma-surjective-family-flat-locally-finite-presentation}
there exists an fppf covering $\{a_i : U_i \to U\}$ such that
each $x \circ a_i$ factors through some $f_j$. Hence $a_i^*\mathcal{F}$
is locally quasi-coherent on $(\Sch/U_i)_{fppf}$. After refining
the covering we may assume $\{U_i \to U\}_{i = 1, \ldots, n}$
is a standard fppf covering. Then $x^*\mathcal{F}$ is an fppf
module on $(\Sch/U)_{fppf}$ whose pullback by the morphism
$a : U_1 \amalg \ldots \amalg U_n \to U$ is locally quasi-coherent.
Hence by the first paragraph we see that $x^*\mathcal{F}$ is locally
quasi-coherent, which certainly implies that $\mathcal{F}|_{U_\etale}$
is quasi-coherent.
\end{proof}






\section{Flat comparison maps}
\label{section-flat-comparison}

\noindent
Let $\mathcal{X}$ be an algebraic stack and let $\mathcal{F}$ be an object
of $\textit{Mod}(\mathcal{X}_\etale, \mathcal{O}_\mathcal{X})$.
Given an object $x$ of $\mathcal{X}$ lying over the scheme $U$ the
restriction $\mathcal{F}|_{U_\etale}$ is the restriction of
$x^{-1}\mathcal{F}$ to the small \'etale site of $U$, see
Sheaves on Stacks, Definition \ref{stacks-sheaves-definition-pullback}.
Next, let $\varphi : x \to x'$ be a morphism of $\mathcal{X}$ lying
over a morphism of schemes $f : U \to U'$. Thus a $2$-commutative diagram
$$
\xymatrix{
U \ar[rd]_x \ar[rr]_f & & U' \ar[ld]^{x'} \\
& \mathcal{X}
}
$$
Associated to $\varphi$ we obtain a comparison map between restrictions
\begin{equation}
\label{equation-comparison-modules}
c_\varphi :
f_{small}^*(\mathcal{F}|_{U'_\etale})
\longrightarrow
\mathcal{F}|_{U_\etale}
\end{equation}
see Sheaves on Stacks, Equation
(\ref{stacks-sheaves-equation-comparison-modules}).
In this situation we can consider the following property
of $\mathcal{F}$.

\begin{definition}
\label{definition-flat-base-change}
Let $\mathcal{X}$ be an algebraic stack and let $\mathcal{F}$ in
$\textit{Mod}(\mathcal{X}_\etale, \mathcal{O}_\mathcal{X})$.
We say $\mathcal{F}$ has the {\it flat base change property}\footnote{This
may be nonstandard notation.}
if and only if $c_\varphi$ is an isomorphism whenever $f$ is flat.
\end{definition}

\noindent
Here is a lemma with some properties of this notion.

\begin{lemma}
\label{lemma-check-flat-comparison-on-etale-covering}
Let $\mathcal{X}$ be an algebraic stack. Let $\mathcal{F}$
be an $\mathcal{O}_\mathcal{X}$-module on $\mathcal{X}_\etale$.
\begin{enumerate}
\item If $\mathcal{F}$ has the flat base change property then for any morphism
$g : \mathcal{Y} \to \mathcal{X}$ of algebraic stacks, the
pullback $g^*\mathcal{F}$ does too.
\item The full subcategory of
$\textit{Mod}(\mathcal{X}_\etale, \mathcal{O}_\mathcal{X})$
consisting of modules with the flat base change property
is a weak Serre subcategory.
\item  Let $f_i : \mathcal{X}_i \to \mathcal{X}$ be a family of
smooth morphisms of algebraic stacks such that
$|\mathcal{X}| = \bigcup_i |f_i|(|\mathcal{X}_i|)$. If each
$f_i^*\mathcal{F}$ has the flat base change property then so does
$\mathcal{F}$.
\item The category of $\mathcal{O}_\mathcal{X}$-modules
on $\mathcal{X}_\etale$ with the flat base change property
has colimits and they agree with colimits in
$\textit{Mod}(\mathcal{X}_\etale, \mathcal{O}_\mathcal{X})$.
\end{enumerate}
\end{lemma}

\begin{proof}
Let $g : \mathcal{Y} \to \mathcal{X}$ be as in (1).
Let $y$ be an object of $\mathcal{Y}$ lying over a scheme $V$. By
Sheaves on Stacks, Lemma \ref{stacks-sheaves-lemma-comparison}
we have
$(g^*\mathcal{F})|_{V_\etale} = \mathcal{F}|_{V_\etale}$.
Moreover a comparison mapping for the sheaf $g^*\mathcal{F}$ on $\mathcal{Y}$
is a special case of a comparison map for the sheaf $\mathcal{F}$ on
$\mathcal{X}$, see
Sheaves on Stacks, Lemma \ref{stacks-sheaves-lemma-comparison}.
In this way (1) is clear.

\medskip\noindent
Proof of (2). We use the characterization of weak Serre subcategories of
Homology, Lemma \ref{homology-lemma-characterize-weak-serre-subcategory}.
Kernels and cokernels of
maps between sheaves having the flat base change property
also have the flat base change property. This is clear because
$f_{small}^*$ is exact for a flat morphism of schemes and since the
restriction functors $(-)|_{U_\etale}$ are exact (because we
are working in the \'etale topology). Finally, if
$0 \to \mathcal{F}_1 \to \mathcal{F}_2 \to \mathcal{F}_3 \to 0$
is a short exact sequence of
$\textit{Mod}(\mathcal{X}_\etale, \mathcal{O}_\mathcal{X})$
and the outer two sheaves have the flat base change property then
the middle one does as well, again because of the exactness of
$f_{small}^*$ and the restriction functors (and the 5 lemma).

\medskip\noindent
Proof of (3).
Let $f_i : \mathcal{X}_i \to \mathcal{X}$ be a jointly surjective family of
smooth morphisms of algebraic stacks and assume each $f_i^*\mathcal{F}$
has the flat base change property. By part (1), the definition of
an algebraic stack, and the fact that compositions of smooth morphisms
are smooth (see
Morphisms of Stacks, Lemma \ref{stacks-morphisms-lemma-composition-smooth})
we may assume that each $\mathcal{X}_i$ is representable by a scheme.
Let $\varphi : x \to x'$ be a morphism of $\mathcal{X}$ lying over
a flat morphism $a : U \to U'$ of schemes. By
Sheaves on Stacks, Lemma
\ref{stacks-sheaves-lemma-surjective-flat-locally-finite-presentation}
there exists a jointly surjective family of \'etale morphisms
$U'_i \to U'$ such that $U'_i \to U' \to \mathcal{X}$ factors through
$\mathcal{X}_i$. Thus we obtain commutative diagrams
$$
\xymatrix{
U_i = U \times_{U'} U_i' \ar[r]_-{a_i} \ar[d] &
U_i' \ar[r]_{x_i'} \ar[d] & \mathcal{X}_i \ar[d]^{f_i} \\
U \ar[r]^a & U' \ar[r]^{x'} & \mathcal{X}
}
$$
Note that each $a_i$ is a flat morphism of schemes as a base change of $a$.
Denote $\psi_i : x_i \to x'_i$ the morphism of $\mathcal{X}_i$ lying over
$a_i$ with target $x_i'$. By assumption the comparison maps
$c_{\psi_i} :
(a_i)_{small}^*\big(f_i^*\mathcal{F}|_{(U'_i)_\etale}\big)
\to f_i^*\mathcal{F}|_{(U_i)_\etale}$ is an isomorphism.
Because the vertical arrows $U_i' \to U'$ and $U_i \to U$ are \'etale,
the sheaves $f_i^*\mathcal{F}|_{(U_i')_\etale}$ and
$f_i^*\mathcal{F}|_{(U_i)_\etale}$ are the restrictions of
$\mathcal{F}|_{U'_\etale}$ and $\mathcal{F}|_{U_\etale}$
and the map $c_{\psi_i}$ is the restriction of $c_\varphi$ to
$(U_i)_\etale$, see
Sheaves on Stacks, Lemma \ref{stacks-sheaves-lemma-comparison}.
Since $\{U_i \to U\}$ is an \'etale covering, this implies
that the comparison map $c_\varphi$ is an isomorphism which is what
we wanted to prove.

\medskip\noindent
Proof of (4). Let
$\mathcal{I} \to
\textit{Mod}(\mathcal{X}_\etale, \mathcal{O}_\mathcal{X})$,
$i \mapsto \mathcal{F}_i$ be a diagram and assume each $\mathcal{F}_i$
has the flat base change property. Recall that $\colim_i \mathcal{F}_i$
is the sheafification of the presheaf colimit. As we are using the
\'etale topology, it is clear that
$$
(\colim_i \mathcal{F}_i)|_{U_\etale} =
\colim_i {\mathcal{F}_i}|_{U_\etale}
$$
As $f_{small}^*$ commutes with colimits (as a left adjoint)
we see that (4) holds.
\end{proof}

\begin{lemma}
\label{lemma-flat-comparison}
Let $f : \mathcal{X} \to \mathcal{Y}$ be a quasi-compact and
quasi-separated morphism of algebraic stacks. Let 
$\mathcal{F}$ be an object of
$\textit{Mod}(\mathcal{X}_\etale, \mathcal{O}_\mathcal{X})$
which is locally quasi-coherent and has the flat base change property.
Then each $R^if_*\mathcal{F}$ (computed in the \'etale topology)
has the flat base change property.
\end{lemma}

\begin{proof}
We will use
Lemma \ref{lemma-general-pushforward}
to prove this. For every algebraic stack $\mathcal{X}$ let
$\mathcal{M}_\mathcal{X}$ denote the full subcategory of
$\textit{Mod}(\mathcal{X}_\etale, \mathcal{O}_\mathcal{X})$
consisting of locally quasi-coherent sheaves with the flat base
change property. Once we verify conditions (1) -- (4) of
Lemma \ref{lemma-general-pushforward}
the lemma will follow. Properties (1), (2), and (3) follow from
Sheaves on Stacks, Lemmas \ref{stacks-sheaves-lemma-pullback-lqc} and
\ref{stacks-sheaves-lemma-lqc-colimits}
and
Lemmas \ref{lemma-check-lqc-on-etale-covering} and
\ref{lemma-check-flat-comparison-on-etale-covering}.
Thus it suffices to show part (4).

\medskip\noindent
Suppose $f : \mathcal{X} \to \mathcal{Y}$ is a morphism of algebraic stacks
such that $\mathcal{X}$ and $\mathcal{Y}$ are representable by affine
schemes $X$ and $Y$. In this case, suppose that
$\psi : y \to y'$ is a morphism of $\mathcal{Y}$ lying over
a flat morphism $b : V \to V'$ of schemes. For clarity denote
$\mathcal{V} = (\Sch/V)_{fppf}$ and $\mathcal{V}' = (\Sch/V')_{fppf}$
the corresponding algebraic stacks. Consider the diagram
of algebraic stacks
$$
\xymatrix{
\mathcal{Z} \ar[d]_{f''} \ar[r]_a &
\mathcal{Z}' \ar[r]_{x'} \ar[d]_{f'} & \mathcal{X} \ar[d]^f \\
\mathcal{V} \ar[r]^b & \mathcal{V}' \ar[r]^{y'} & \mathcal{Y}
}
$$
with both squares cartesian. As $f$ is representable by schemes
(and quasi-compact and separated -- even affine) we see that $\mathcal{Z}$ and
$\mathcal{Z}'$ are representable by schemes $Z$ and $Z'$ and in
fact $Z = V \times_{V'} Z'$. Since $\mathcal{F}$ has the flat
base change property we see that
$$
a_{small}^*\big(\mathcal{F}|_{Z'_\etale}\big)
\longrightarrow
\mathcal{F}|_{Z_\etale}
$$
is an isomorphism. Moreover,
$$
R^if_*\mathcal{F}|_{V'_\etale} =
R^i(f')_{small, *}\big(\mathcal{F}|_{Z'_\etale}\big)
$$
and
$$
R^if_*\mathcal{F}|_{V_\etale} =
R^i(f'')_{small, *}\big(\mathcal{F}|_{Z_\etale}\big)
$$
by
Sheaves on Stacks, Lemma
\ref{stacks-sheaves-lemma-compare-representable-morphism-cohomology}.
Hence we see that the comparison map
$$
c_\psi :
b_{small}^*(R^if_*\mathcal{F}|_{V'_\etale})
\longrightarrow
R^if_*\mathcal{F}|_{V_\etale}
$$
is an isomorphism by
Cohomology of Spaces, Lemma
\ref{spaces-cohomology-lemma-flat-base-change-cohomology}.
Thus $R^if_*\mathcal{F}$ has the flat base change property.
Since $R^if_*\mathcal{F}$ is locally quasi-coherent by
Lemma \ref{lemma-pushforward-locally-quasi-coherent}
we win.
\end{proof}

\begin{proposition}
\label{proposition-lcq-flat-base-change}
Summary of results on locally quasi-coherent modules having the flat
base change property.
\begin{enumerate}
\item Let $\mathcal{X}$ be an algebraic stack.
If $\mathcal{F}$ is an object of
$\textit{Mod}(\mathcal{X}_\etale, \mathcal{O}_\mathcal{X})$
which is locally quasi-coherent and has the flat base change property,
then $\mathcal{F}$ is a sheaf for the fppf topology, i.e., it is
an object of $\textit{Mod}(\mathcal{O}_\mathcal{X})$.
\item The category of modules which are locally quasi-coherent
and have the flat base change property is a weak Serre subcategory
$\mathcal{M}_\mathcal{X}$ of both $\textit{Mod}(\mathcal{O}_\mathcal{X})$
and $\textit{Mod}(\mathcal{X}_\etale, \mathcal{O}_\mathcal{X})$.
\item Pullback $f^*$ along any morphism of algebraic stacks
$f : \mathcal{X} \to \mathcal{Y}$ induces a functor
$f^* : \mathcal{M}_\mathcal{Y} \to \mathcal{M}_\mathcal{X}$.
\item If $f : \mathcal{X} \to \mathcal{Y}$ is a
quasi-compact and quasi-separated morphism of algebraic stacks
and $\mathcal{F}$ is an object of $\mathcal{M}_\mathcal{X}$, then
\begin{enumerate}
\item the derived direct image $Rf_*\mathcal{F}$ and the higher direct
images $R^if_*\mathcal{F}$ can be computed in either the \'etale or the
fppf topology with the same result, and
\item each $R^if_*\mathcal{F}$ is an object of $\mathcal{M}_\mathcal{Y}$.
\end{enumerate}
\item The category $\mathcal{M}_\mathcal{X}$ has colimits and they agree
with colimits in
$\textit{Mod}(\mathcal{X}_\etale, \mathcal{O}_\mathcal{X})$
as well as in $\textit{Mod}(\mathcal{O}_\mathcal{X})$.
\end{enumerate}
\end{proposition}

\begin{proof}
Part (1) is
Sheaves on Stacks, Lemma
\ref{stacks-sheaves-lemma-lqc-flat-base-change-fppf-sheaf}.

\medskip\noindent
Part (2) for the embedding $\mathcal{M}_\mathcal{X} \subset
\textit{Mod}(\mathcal{X}_\etale, \mathcal{O}_\mathcal{X})$
we have seen in the proof of
Lemma \ref{lemma-flat-comparison}.
Let us prove (2) for the embedding
$\mathcal{M}_\mathcal{X} \subset \textit{Mod}(\mathcal{O}_\mathcal{X})$.
Let $\varphi : \mathcal{F} \to \mathcal{G}$ be a morphism between
objects of $\mathcal{M}_\mathcal{X}$. Since $\Ker(\varphi)$
is the same whether computed in the \'etale or the fppf
topology, we see that $\Ker(\varphi)$ is in
$\mathcal{M}_\mathcal{X}$ by the \'etale case. On the other hand,
the cokernel computed in the fppf topology is the fppf sheafification
of the cokernel computed in the \'etale topology. However, this
\'etale cokernel is in $\mathcal{M}_\mathcal{X}$ hence an fppf sheaf
by (1) and we see that the cokernel is in $\mathcal{M}_\mathcal{X}$.
Finally, suppose that
$$
0 \to \mathcal{F}_1 \to \mathcal{F}_2 \to \mathcal{F}_3 \to 0
$$
is an exact sequence in $\textit{Mod}(\mathcal{O}_\mathcal{X})$
(i.e., using the fppf topology) with $\mathcal{F}_1$, $\mathcal{F}_2$
in $\mathcal{M}_\mathcal{X}$. In order to show that $\mathcal{F}_2$
is an object of $\mathcal{M}_\mathcal{X}$ it suffices to show that
the sequence is also exact in the \'etale topology. To do this it
suffices to show that any element of $H^1_{fppf}(x, \mathcal{F}_1)$
becomes zero on the members of an \'etale covering of $x$ (for any
object $x$ of $\mathcal{X}$). This is true because
$H^1_{fppf}(x, \mathcal{F}_1) = H^1_\etale(x, \mathcal{F}_1)$ by
Sheaves on Stacks, Lemma \ref{stacks-sheaves-lemma-compare-fppf-etale}
and because of locality of cohomology, see
Cohomology on Sites, Lemma
\ref{sites-cohomology-lemma-kill-cohomology-class-on-covering}.
This proves (2).

\medskip\noindent
Part (3) follows from
Lemma \ref{lemma-check-flat-comparison-on-etale-covering}
and
Sheaves on Stacks, Lemma \ref{stacks-sheaves-lemma-pullback-lqc}.

\medskip\noindent
Part (4)(b) for $R^if_*\mathcal{F}$ computed in the \'etale cohomology
follows from Lemma \ref{lemma-flat-comparison}.
Whereupon part (4)(a) follows from
Sheaves on Stacks, Lemma \ref{stacks-sheaves-lemma-compare-fppf-etale}
combined with (1) above.

\medskip\noindent
Part (5) for the \'etale topology follows from
Sheaves on Stacks, Lemma \ref{stacks-sheaves-lemma-lqc-colimits} and
Lemma \ref{lemma-check-flat-comparison-on-etale-covering}.
The fppf version then follows as the colimit in the \'etale
topology is already an fppf sheaf by part (1).
\end{proof}

\begin{lemma}
\label{lemma-check-lqc-fbc-on-covering}
Let $\mathcal{X}$ be an algebraic stack. With $\mathcal{M}_\mathcal{X}$
the category of locally quasi-coherent modules with the flat base change
property.
\begin{enumerate}
\item Let $f_j : \mathcal{X}_j \to \mathcal{X}$ be a family of smooth
morphisms of algebraic stacks with
$|\mathcal{X}| =\bigcup |f_j|(|\mathcal{X}_j|)$.
Let $\mathcal{F}$ be a sheaf of $\mathcal{O}_\mathcal{X}$-modules
on $\mathcal{X}_\etale$. If each $f_j^{-1}\mathcal{F}$
is in $\mathcal{M}_{\mathcal{X}_i}$, then $\mathcal{F}$ is in
$\mathcal{M}_\mathcal{X}$.
\item  Let $f_j : \mathcal{X}_j \to \mathcal{X}$ be a family of flat
and locally finitely presented morphisms of algebraic stacks with
$|\mathcal{X}| =\bigcup |f_j|(|\mathcal{X}_j|)$.
Let $\mathcal{F}$ be a sheaf of $\mathcal{O}_\mathcal{X}$-modules
on $\mathcal{X}_{fppf}$. If each $f_j^{-1}\mathcal{F}$
is in $\mathcal{M}_{\mathcal{X}_i}$, then $\mathcal{F}$ is in
$\mathcal{M}_\mathcal{X}$.
\end{enumerate}
\end{lemma}

\begin{proof}
Part (1) follows from a combination of
Lemmas \ref{lemma-check-lqc-on-etale-covering} and
\ref{lemma-check-flat-comparison-on-etale-covering}.
The proof of (2) is analogous to the proof of
Lemma \ref{lemma-check-lqc-on-flat-covering}.
Let $\mathcal{F}$ of a sheaf of $\mathcal{O}_\mathcal{X}$-modules
on $\mathcal{X}_{fppf}$.

\medskip\noindent
First, suppose there is a morphism $a : \mathcal{U} \to \mathcal{X}$
which is surjective, flat, locally of finite presentation, quasi-compact,
and quasi-separated such that $a^*\mathcal{F}$ is locally quasi-coherent
and has the flat base change property.
Then there is an exact sequence
$$
0 \to \mathcal{F} \to a_*a^*\mathcal{F} \to b_*b^*\mathcal{F}
$$
where $b$ is the morphism
$b : \mathcal{U} \times_\mathcal{X} \mathcal{U} \to \mathcal{X}$, see
Sheaves on Stacks, Proposition
\ref{stacks-sheaves-proposition-exactness-cech-complex} and
Lemma \ref{stacks-sheaves-lemma-surjective-flat-locally-finite-presentation}.
Moreover, the pullback $b^*\mathcal{F}$ is the pullback of $a^*\mathcal{F}$
via one of the projection morphisms, hence is locally quasi-coherent
and has the flat base change property, see
Proposition \ref{proposition-lcq-flat-base-change}.
The modules $a_*a^*\mathcal{F}$ and $b_*b^*\mathcal{F}$ are locally
quasi-coherent and have the flat base change property by
Proposition \ref{proposition-lcq-flat-base-change}.
We conclude that $\mathcal{F}$ is locally quasi-coherent and
has the flat base change property by
Proposition \ref{proposition-lcq-flat-base-change}.

\medskip\noindent
Choose a scheme $U$ and a surjective smooth morphism $x : U \to \mathcal{X}$.
By part (1) it suffices to show that $x^*\mathcal{F}$ is locally
quasi-coherent and has the flat base change property.
Again by part (1) it suffices to do this (Zariski) locally on $U$,
hence we may assume that $U$ is affine. By
Morphisms of Stacks, Lemma
\ref{stacks-morphisms-lemma-surjective-family-flat-locally-finite-presentation}
there exists an fppf covering $\{a_i : U_i \to U\}$ such that
each $x \circ a_i$ factors through some $f_j$. Hence the module
$a_i^*\mathcal{F}$ on $(\Sch/U_i)_{fppf}$
is locally quasi-coherent and has the flat base change property.
After refining the covering we may assume $\{U_i \to U\}_{i = 1, \ldots, n}$
is a standard fppf covering. Then $x^*\mathcal{F}$ is an fppf
module on $(\Sch/U)_{fppf}$ whose pullback by the morphism
$a : U_1 \amalg \ldots \amalg U_n \to U$ is locally quasi-coherent
and has the flat base change property.
Hence by the previous paragraph we see that $x^*\mathcal{F}$ is locally
quasi-coherent and has the flat base change property as desired.
\end{proof}






\section{Parasitic modules}
\label{section-parasitic}

\noindent
The following definition is compatible with
Descent, Definition \ref{descent-definition-parasitic}.

\begin{definition}
\label{definition-parasitic}
Let $\mathcal{X}$ be an algebraic stack.
A presheaf of $\mathcal{O}_\mathcal{X}$-modules $\mathcal{F}$ is
{\it parasitic} if we have $\mathcal{F}(x) = 0$ for any object $x$
of $\mathcal{X}$ which lies over a scheme $U$ such that the corresponding
morphism $x : U \to \mathcal{X}$ is flat.
\end{definition}

\noindent
Here is a lemma with some properties of this notion.

\begin{lemma}
\label{lemma-parasitic}
Let $\mathcal{X}$ be an algebraic stack. Let $\mathcal{F}$
be a presheaf of $\mathcal{O}_\mathcal{X}$-modules.
\begin{enumerate}
\item If $\mathcal{F}$ is parasitic and
$g : \mathcal{Y} \to \mathcal{X}$ is a flat morphism of algebraic stacks,
then $g^*\mathcal{F}$ is parasitic.
\item For $\tau \in \{Zariski, \etale, smooth, syntomic, fppf\}$
we have
\begin{enumerate}
\item the $\tau$ sheafification of a parasitic presheaf of modules is
parasitic, and
\item the full subcategory of
$\textit{Mod}(\mathcal{X}_\tau, \mathcal{O}_\mathcal{X})$
consisting of parasitic modules is a Serre subcategory.
\end{enumerate}
\item Suppose $\mathcal{F}$ is a sheaf for the \'etale topology.
Let $f_i : \mathcal{X}_i \to \mathcal{X}$ be a family of
smooth morphisms of algebraic stacks such that
$|\mathcal{X}| = \bigcup_i |f_i|(|\mathcal{X}_i|)$. If each
$f_i^*\mathcal{F}$ is parasitic then so is $\mathcal{F}$.
\item Suppose $\mathcal{F}$ is a sheaf for the fppf topology.
Let $f_i : \mathcal{X}_i \to \mathcal{X}$ be a family of
flat and locally finitely presented morphisms of algebraic stacks such that
$|\mathcal{X}| = \bigcup_i |f_i|(|\mathcal{X}_i|)$. If each
$f_i^*\mathcal{F}$ is parasitic then so is $\mathcal{F}$.
\end{enumerate}
\end{lemma}

\begin{proof}
To see part (1) let $y$ be an object of $\mathcal{Y}$ which lies
over a scheme $V$ such that the corresponding morphism $y : V \to \mathcal{Y}$
is flat. Then $g(y) : V \to \mathcal{Y} \to \mathcal{X}$ is flat
as a composition of flat morphisms (see
Morphisms of Stacks, Lemma \ref{stacks-morphisms-lemma-composition-flat})
hence $\mathcal{F}(g(y))$ is zero by assumption. Since
$g^*\mathcal{F} = g^{-1}\mathcal{F}(y) = \mathcal{F}(g(y))$ we conclude
$g^*\mathcal{F}$ is parasitic.

\medskip\noindent
To see part (2)(a) note that if $\{x_i \to x\}$ is a $\tau$-covering
of $\mathcal{X}$, then each of the morphisms $x_i \to x$ lies
over a flat morphism of schemes. Hence if $x$ lies over a scheme
$U$ such that $x : U \to \mathcal{X}$ is flat, so do all of the
objects $x_i$. Hence the presheaf $\mathcal{F}^+$ (see
Sites, Section \ref{sites-section-sheafification})
is parasitic if the presheaf $\mathcal{F}$ is
parasitic. This proves (2)(a) as the sheafification of $\mathcal{F}$
is $(\mathcal{F}^+)^+$.

\medskip\noindent
Let $\mathcal{F}$ be a parasitic $\tau$-module. It is immediate from the
definitions that any submodule of $\mathcal{F}$ is parasitic. On the other
hand, if $\mathcal{F}' \subset \mathcal{F}$ is a submodule, then it is
equally clear that the presheaf
$x \mapsto \mathcal{F}(x)/\mathcal{F}'(x)$
is parasitic. Hence the quotient $\mathcal{F}/\mathcal{F}'$ is a parasitic
module by (2)(a). Finally, we have to show that given a short exact sequence
$0 \to \mathcal{F}_1 \to \mathcal{F}_2 \to \mathcal{F}_3 \to 0$
with $\mathcal{F}_1$ and $\mathcal{F}_3$ parasitic, then $\mathcal{F}_2$
is parasitic. This follows immediately on evaluating on $x$ lying
over a scheme flat over $\mathcal{X}$. This proves (2)(b), see
Homology, Lemma \ref{homology-lemma-characterize-serre-subcategory}.

\medskip\noindent
Let $f_i : \mathcal{X}_i \to \mathcal{X}$ be a jointly surjective family of
smooth morphisms of algebraic stacks and assume each $f_i^*\mathcal{F}$
is parasitic. Let $x$ be an object of $\mathcal{X}$ which lies over a
scheme $U$ such that $x : U \to \mathcal{X}$ is flat. Consider a surjective
smooth covering $W_i \to U \times_{x, \mathcal{X}} \mathcal{X}_i$.
Denote $y_i : W_i \to \mathcal{X}_i$ the projection. It follows
that $\{f_i(y_i) \to x\}$ is a covering for the smooth topology
on $\mathcal{X}$. Since a composition of flat morphisms is flat we see that
$f_i^*\mathcal{F}(y_i) = 0$. On the other hand, as we saw in the proof
of (1), we have $f_i^*\mathcal{F}(y_i) = \mathcal{F}(f_i(y_i))$.
Hence we see that for some smooth covering $\{x_i \to x\}_{i \in I}$
in $\mathcal{X}$ we have $\mathcal{F}(x_i) = 0$. This implies
$\mathcal{F}(x) = 0$ because the smooth topology is the same
as the \'etale topology, see
More on Morphisms, Lemma \ref{more-morphisms-lemma-etale-dominates-smooth}.
Namely, $\{x_i \to x\}_{i \in I}$ lies over a smooth covering
$\{U_i \to U\}_{i \in I}$ of schemes. By the lemma just referenced
there exists an \'etale covering $\{V_j \to U\}_{j \in J}$ which
refines $\{U_i \to U\}_{i \in I}$. Denote $x'_j = x|_{V_j}$.
Then $\{x'_j \to x\}$ is an \'etale covering in $\mathcal{X}$
refining $\{x_i \to x\}_{i \in I}$. This means the map
$\mathcal{F}(x) \to \prod_{j \in J} \mathcal{F}(x'_j)$, which is
injective as $\mathcal{F}$ is a sheaf in the \'etale topology,
factors through $\mathcal{F}(x) \to \prod_{i \in I} \mathcal{F}(x_i)$
which is zero. Hence $\mathcal{F}(x) = 0$ as desired.

\medskip\noindent
Proof of (4): omitted. Hint: similar, but simpler, than the proof of (3).
\end{proof}

\noindent
Parasitic modules are preserved under absolutely any pushforward.

\begin{lemma}
\label{lemma-pushforward-parasitic}
Let $\tau \in \{\etale, fppf\}$.
Let $\mathcal{X}$ be an algebraic stack.
Let $\mathcal{F}$ be a parasitic object of
$\textit{Mod}(\mathcal{X}_\tau, \mathcal{O}_\mathcal{X})$.
\begin{enumerate}
\item $H^i_\tau(\mathcal{X}, \mathcal{F}) = 0$ for all $i$.
\item Let $f : \mathcal{X} \to \mathcal{Y}$ be a morphism of algebraic stacks.
Then $R^if_*\mathcal{F}$ (computed in $\tau$-topology) is a
parasitic object of $\textit{Mod}(\mathcal{Y}_\tau, \mathcal{O}_\mathcal{Y})$.
\end{enumerate}
\end{lemma}

\begin{proof}
We first reduce (2) to (1).
By Sheaves on Stacks, Lemma \ref{stacks-sheaves-lemma-pushforward-restriction}
we see that $R^if_*\mathcal{F}$ is the sheaf associated to the presheaf
$$
y \longmapsto
H^i_\tau\Big(V \times_{y, \mathcal{Y}} \mathcal{X},
\ \text{pr}^{-1}\mathcal{F}\Big)
$$
Here $y$ is a typical object of $\mathcal{Y}$ lying over the scheme $V$.
By Lemma \ref{lemma-parasitic} it suffices to show that
these cohomology groups are zero when $y : V \to \mathcal{Y}$ is flat.
Note that $\text{pr} : V \times_{y, \mathcal{Y}} \mathcal{X} \to \mathcal{X}$
is flat as a base change of $y$. Hence by
Lemma \ref{lemma-parasitic} we see that $\text{pr}^{-1}\mathcal{F}$
is parasitic. Thus it suffices to prove (1).

\medskip\noindent
To see (1) we can use the spectral sequence of
Sheaves on Stacks, Proposition
\ref{stacks-sheaves-proposition-smooth-covering-compute-cohomology}
to reduce this to the case where $\mathcal{X}$
is an algebraic stack representable by an algebraic space.
Note that in the spectral sequence each
$f_p^{-1}\mathcal{F} = f_p^*\mathcal{F}$ is a parasitic module by
Lemma \ref{lemma-parasitic} because the morphisms
$f_p : \mathcal{U}_p =
\mathcal{U} \times_\mathcal{X} \ldots
\times_\mathcal{X} \mathcal{U} \to \mathcal{X}$ are flat.
Reusing this spectral sequence one more time (as in the
proof of the key Lemma \ref{lemma-general-pushforward})
we reduce to the case where the
algebraic stack $\mathcal{X}$ is representable by a scheme $X$.
Then $H^i_\tau(\mathcal{X}, \mathcal{F}) = H^i((\Sch/X)_\tau, \mathcal{F})$.
In this case the vanishing follows easily from an argument
with {\v C}ech coverings, see
Descent, Lemma \ref{descent-lemma-cohomology-parasitic}.
\end{proof}

\noindent
The following lemma is one of the major reasons we care about
parasitic modules. To understand the statement, recall that
the functors
$\QCoh(\mathcal{O}_\mathcal{X}) \to
\textit{Mod}(\mathcal{X}_\etale, \mathcal{O}_\mathcal{X})$
and
$\QCoh(\mathcal{O}_\mathcal{X}) \to
\textit{Mod}(\mathcal{O}_\mathcal{X})$
aren't exact in general.

\begin{lemma}
\label{lemma-exact-sequence-quasi-coherent-parasitic-cohomology}
Let $\mathcal{X}$ be an algebraic stack. Let
$\mathcal{F}^\bullet$ be an
exact complex in $\QCoh(\mathcal{O}_\mathcal{X})$.
Then the cohomology sheaves of $\mathcal{F}^\bullet$
in either the \'etale or the fppf topology
are parasitic $\mathcal{O}_\mathcal{X}$-modules.
\end{lemma}

\begin{proof}
Let $x : U \to \mathcal{X}$ be a flat morphism where $U$ is a scheme.
Then $x^*\mathcal{F}^\bullet$ is exact by
Lemma \ref{lemma-flat-pullback-quasi-coherent}.
Hence the restriction $x^*\mathcal{F}^\bullet|_{U_\etale}$
is exact which is what we had to prove.
\end{proof}





\section{Quasi-coherent modules, I}
\label{section-quasi-coherent}

\noindent
We have seen that the category of quasi-coherent modules on an algebraic
stack is equivalent to the category of quasi-coherent modules on a
presentation, see
Sheaves on Stacks, Section
\ref{stacks-sheaves-section-quasi-coherent-algebraic-stacks}.
This fact is the basis for the following.

\begin{lemma}
\label{lemma-adjoint}
Let $\mathcal{X}$ be an algebraic stack. Let $\mathcal{M}_\mathcal{X}$
be the category of locally quasi-coherent modules with the
flat base change property, see
Proposition \ref{proposition-lcq-flat-base-change}.
The inclusion functor
$i : \QCoh(\mathcal{O}_\mathcal{X}) \to \mathcal{M}_\mathcal{X}$
has a right adjoint
$$
Q : \mathcal{M}_\mathcal{X} \to \QCoh(\mathcal{O}_\mathcal{X})
$$
such that $Q \circ i$ is the identity functor.
\end{lemma}

\begin{proof}
Choose a scheme $U$ and a surjective smooth morphism $f : U \to \mathcal{X}$.
Set $R = U \times_\mathcal{X} U$ so that we obtain a smooth groupoid
$(U, R, s, t, c)$ in algebraic spaces with the property that
$\mathcal{X} = [U/R]$, see
Algebraic Stacks, Lemma \ref{algebraic-lemma-stack-presentation}.
We may and do replace $\mathcal{X}$ by $[U/R]$.
In the proof of
Sheaves on Stacks, Proposition \ref{stacks-sheaves-proposition-quasi-coherent}
we constructed a functor
$$
q_1 :
\QCoh(U, R, s, t, c)
\longrightarrow
\QCoh(\mathcal{O}_\mathcal{X}).
$$
The construction of the inverse functor in the proof of
Sheaves on Stacks, Proposition \ref{stacks-sheaves-proposition-quasi-coherent}
works for objects of $\mathcal{M}_\mathcal{X}$ and induces a functor
$$
q_2 :
\mathcal{M}_\mathcal{X}
\longrightarrow
\QCoh(U, R, s, t, c).
$$
Namely, if $\mathcal{F}$ is an object of $\mathcal{M}_\mathcal{X}$
the we set
$$
q_2(\mathcal{F}) = (f^*\mathcal{F}|_{U_\etale}, \alpha)
$$
where $\alpha$ is the isomorphism
$$
t_{small}^*(f^*\mathcal{F}|_{U_\etale})
\to
t^*f^*\mathcal{F}|_{R_\etale} \to
s^*f^*\mathcal{F}|_{R_\etale} \to
s_{small}^*(f^*\mathcal{F}|_{U_\etale})
$$
where the outer two morphisms are the comparison maps. Note that
$q_2(\mathcal{F})$ is quasi-coherent precisely because $\mathcal{F}$ is
locally quasi-coherent (and we used the flat base change property
in the construction of the descent datum $\alpha$). We omit the
verification that the cocycle condition (see
Groupoids in Spaces, Definition
\ref{spaces-groupoids-definition-groupoid-module})
holds. We define $Q = q_1 \circ q_2$.
Let $\mathcal{F}$ be an object of $\mathcal{M}_\mathcal{X}$ and
let $\mathcal{G}$ be an object of $\QCoh(\mathcal{O}_\mathcal{X})$.
We have
\begin{align*}
\Mor_{\mathcal{M}_\mathcal{X}}(i(\mathcal{G}), \mathcal{F})
& =
\Mor_{\QCoh(U, R, s, t, c)}(q_2(\mathcal{G}), q_2(\mathcal{F})) \\
& =
\Mor_{\QCoh(\mathcal{O}_\mathcal{X})}(\mathcal{G}, Q(\mathcal{F}))
\end{align*}
where the first equality is
Sheaves on Stacks, Lemma \ref{stacks-sheaves-lemma-map-from-quasi-coherent}
and the second equality holds because $q_1$ and $q_2$ are inverse
equivalences of categories. The assertion $Q \circ i \cong \text{id}$
is a formal consequence of the fact that $i$ is fully faithful.
\end{proof}

\begin{lemma}
\label{lemma-adjoint-kernel-parasitic}
Let $\mathcal{X}$ be an algebraic stack.
Let $Q : \mathcal{M}_\mathcal{X} \to \QCoh(\mathcal{O}_\mathcal{X})$
be the functor constructed in Lemma \ref{lemma-adjoint}.
\begin{enumerate}
\item The kernel of $Q$ is exactly the collection of parasitic objects
of $\mathcal{M}_\mathcal{X}$.
\item For any object $\mathcal{F}$
of $\mathcal{M}_\mathcal{X}$ both the kernel and the cokernel of the
adjunction map $Q(\mathcal{F}) \to \mathcal{F}$ are parasitic.
\item The functor $Q$ is exact.
\end{enumerate}
\end{lemma}

\begin{proof}
Write $\mathcal{X} = [U/R]$ as in the proof of Lemma \ref{lemma-adjoint}.
Let $\mathcal{F}$ be an object of $\mathcal{M}_\mathcal{X}$.
It is clear from the proof of Lemma \ref{lemma-adjoint}
that $\mathcal{F}$ is in the kernel of $Q$ if and only if
$\mathcal{F}|_{U_\etale} = 0$.
In particular, if $\mathcal{F}$ is parasitic then $\mathcal{F}$ is in
the kernel. Next, let $x : V \to \mathcal{X}$ be a flat morphism, where
$V$ is a scheme. Set $W = V \times_\mathcal{X} U$ and consider the diagram
$$
\xymatrix{
W \ar[d]_p \ar[r]_q & V \ar[d] \\
U \ar[r] & \mathcal{X}
}
$$
Note that the projection $p : W \to U$ is flat and the projection
$q : W \to V$ is smooth and surjective. This implies that $q_{small}^*$
is a faithful functor on quasi-coherent modules. By assumption $\mathcal{F}$
has the flat base change property so that we obtain
$p_{small}^*\mathcal{F}|_{U_\etale} \cong
q_{small}^*\mathcal{F}|_{V_\etale}$. Thus if $\mathcal{F}$
is in the kernel of $Q$, then $\mathcal{F}|_{V_\etale} = 0$
which completes the proof of (1).

\medskip\noindent
Part (2) follows from the discussion above and the fact
that the map $Q(\mathcal{F}) \to \mathcal{F}$ becomes an isomorphism after
restricting to $U_\etale$.

\medskip\noindent
To see part (3) note that $Q$
is left exact as a right adjoint. Suppose that
$0 \to \mathcal{F} \to \mathcal{G} \to \mathcal{H} \to 0$
is a short exact sequence in $\mathcal{M}_\mathcal{X}$. Let
$\mathcal{E} = \Coker(Q(\mathcal{G}) \to Q(\mathcal{H}))$ in
$\QCoh(\mathcal{O}_\mathcal{X})$. Since
$\QCoh(\mathcal{O}_\mathcal{X}) \to \mathcal{M}_\mathcal{X}$
is a left adjoint it is right exact. Hence we see that
$Q(\mathcal{G}) \to Q(\mathcal{H}) \to \mathcal{E} \to 0$
is exact in $\mathcal{M}_\mathcal{X}$. Using
Lemma \ref{lemma-exact-sequence-quasi-coherent-parasitic-cohomology}
we find that the top row of the following commutative diagram
has parasitic cohomology sheaves at $Q(\mathcal{F})$ and $Q(\mathcal{G})$:
$$
\xymatrix{
0 \ar[r] &
Q(\mathcal{F}) \ar[r] \ar[d]_a &
Q(\mathcal{G}) \ar[r] \ar[d]_b &
Q(\mathcal{H}) \ar[r] \ar[d]_c &
\mathcal{E} \ar[r] \ar[d] & 0 \\
0 \ar[r] &
\mathcal{F} \ar[r] &
\mathcal{G} \ar[r] &
\mathcal{H} \ar[r] & 0
}
$$
The bottom row is exact and the vertical arrows $a, b, c$
have parasitic kernel and cokernels by part (2). It follows that
$\mathcal{E}$ is parasitic: in the quotient category of
$\textit{Mod}(\mathcal{O}_\mathcal{X})/\text{Parasitic}$
(see Homology, Lemma \ref{homology-lemma-serre-subcategory-is-kernel} and 
Lemma \ref{lemma-parasitic})
we see that $a, b, c$ are isomorphisms and that the top row becomes
exact. As it is also quasi-coherent, we conclude
that $\mathcal{E}$ is zero because $\mathcal{E} = Q(\mathcal{E}) = 0$ by
part (1).
\end{proof}












\section{Pushforward of quasi-coherent modules}
\label{section-pushforward-quasi-coherent}

\noindent
Let $f : \mathcal{X} \to \mathcal{Y}$ be a morphism of algebraic stacks.
Consider the pushforward
$$
f_* :
\textit{Mod}(\mathcal{O}_\mathcal{X})
\longrightarrow
\textit{Mod}(\mathcal{O}_\mathcal{Y})
$$
It turns out that this functor almost never preserves the subcategories
of quasi-coherent sheaves. For example, consider the morphism of schemes
$$
j : X = \mathbf{A}^2_k \setminus \{0\} \longrightarrow \mathbf{A}^2_k = Y.
$$
Associated to this we have the corresponding morphism of algebraic stacks
$$
f = j_{big} : \mathcal{X} = (\Sch/X)_{fppf} \to
(\Sch/Y)_{fppf} = \mathcal{Y}
$$
The pushforward $f_*\mathcal{O}_\mathcal{X}$ of the structure sheaf has
global sections $k[x, y]$. Hence if $f_*\mathcal{O}_\mathcal{X}$ is
quasi-coherent on $\mathcal{Y}$ then we would have
$f_*\mathcal{O}_\mathcal{X} = \mathcal{O}_\mathcal{Y}$. However,
consider $T = \Spec(k) \to \mathbf{A}^2_k = Y$ mapping to $0$.
Then $\Gamma(T, f_*\mathcal{O}_\mathcal{X}) = 0$ because
$X \times_Y T = \emptyset$ whereas $\Gamma(T, \mathcal{O}_\mathcal{Y}) = k$.
On the positive side, we know from
Cohomology of Schemes, Lemma \ref{coherent-lemma-flat-base-change-cohomology}
that for any flat morphism $T \to Y$ we have the equality
$\Gamma(T, f_*\mathcal{O}_\mathcal{X}) = \Gamma(T, \mathcal{O}_\mathcal{Y})$
(this uses that $j$ is quasi-compact and quasi-separated).

\medskip\noindent
Let $f : \mathcal{X} \to \mathcal{Y}$ be a quasi-compact and
quasi-separated morphism of algebraic stacks. We work around the problem
mentioned above using the following three observations:
\begin{enumerate}
\item $f_*$ does preserve locally quasi-coherent
modules (Lemma \ref{lemma-pushforward-locally-quasi-coherent}),
\item $f_*$ transforms a quasi-coherent sheaf into a locally quasi-coherent
sheaf whose flat comparison maps are isomorphisms
(Lemma \ref{lemma-flat-comparison}), and
\item locally quasi-coherent $\mathcal{O}_\mathcal{Y}$-modules
with the flat base change property give rise to quasi-coherent
modules on a presentation of $\mathcal{Y}$ and hence quasi-coherent
modules on $\mathcal{Y}$, see
Sheaves on Stacks, Section
\ref{stacks-sheaves-section-quasi-coherent-algebraic-stacks}.
\end{enumerate}
Thus we obtain a functor
$$
f_{\QCoh, *} :
\QCoh(\mathcal{O}_\mathcal{X})
\longrightarrow
\QCoh(\mathcal{O}_\mathcal{Y})
$$
which is a right adjoint to
$f^* : \QCoh(\mathcal{O}_\mathcal{Y}) \to
\QCoh(\mathcal{O}_\mathcal{X})$
such that moreover
$$
\Gamma(y, f_*\mathcal{F}) = \Gamma(y, f_{\QCoh, *}\mathcal{F})
$$
for any $y \in \Ob(\mathcal{Y})$ such that the associated
$1$-morphism $y : V \to \mathcal{Y}$ is flat, see (insert future
reference here).
Moreover, a similar construction will produce functors
$R^if_{\QCoh, *}$.
However, these results will not be sufficient to produce a
total direct image functor (of complexes with quasi-coherent
cohomology sheaves).

\begin{proposition}
\label{proposition-direct-image-quasi-coherent}
Let $f : \mathcal{X} \to \mathcal{Y}$ be a quasi-compact and quasi-separated
morphism of algebraic stacks. The functor
$f^* : \QCoh(\mathcal{O}_\mathcal{Y}) \to
\QCoh(\mathcal{O}_\mathcal{X})$
has a right adjoint
$$
f_{\QCoh, *} :
\QCoh(\mathcal{O}_\mathcal{X})
\longrightarrow
\QCoh(\mathcal{O}_\mathcal{Y})
$$
which can be defined as the composition
$$
\QCoh(\mathcal{O}_\mathcal{X}) \to \mathcal{M}_\mathcal{X}
\xrightarrow{f_*} \mathcal{M}_\mathcal{Y}
\xrightarrow{Q} \QCoh(\mathcal{O}_\mathcal{Y})
$$
where the functors $f_*$ and $Q$ are as in
Proposition \ref{proposition-lcq-flat-base-change}
and
Lemma \ref{lemma-adjoint}.
Moreover, if we define $R^if_{\QCoh, *}$ as the composition
$$
\QCoh(\mathcal{O}_\mathcal{X}) \to \mathcal{M}_\mathcal{X}
\xrightarrow{R^if_*} \mathcal{M}_\mathcal{Y}
\xrightarrow{Q} \QCoh(\mathcal{O}_\mathcal{Y})
$$
then the sequence of functors $\{R^if_{\QCoh, *}\}_{i \geq 0}$
forms a cohomological $\delta$-functor.
\end{proposition}

\begin{proof}
This is a combination of the results mentioned in the statement.
The adjointness can be shown as follows: Let $\mathcal{F}$
be a quasi-coherent $\mathcal{O}_\mathcal{X}$-module and let
$\mathcal{G}$ be a quasi-coherent $\mathcal{O}_\mathcal{Y}$-module.
Then we have
\begin{align*}
\Mor_{\QCoh(\mathcal{O}_\mathcal{X})}(f^*\mathcal{G}, \mathcal{F})
& =
\Mor_{\mathcal{M}_\mathcal{Y}}(\mathcal{G}, f_*\mathcal{F}) \\
& =
\Mor_{\QCoh(\mathcal{O}_\mathcal{Y})}(\mathcal{G}, Q(f_*\mathcal{F}))
\\
& =
\Mor_{\QCoh(\mathcal{O}_\mathcal{Y})}(\mathcal{G},
f_{\QCoh, *}\mathcal{F})
\end{align*}
the first equality by adjointness of $f_*$ and $f^*$ (for arbitrary sheaves
of modules). By
Proposition \ref{proposition-lcq-flat-base-change}
we see that $f_*\mathcal{F}$ is an object of $\mathcal{M}_\mathcal{Y}$
(and can be computed in either the fppf or \'etale topology) and we
obtain the second equality by Lemma \ref{lemma-adjoint}. The third
equality is the definition of $f_{\QCoh, *}$.

\medskip\noindent
To see that $\{R^if_{\QCoh, *}\}_{i \geq 0}$ is a cohomological
$\delta$-functor as defined in
Homology, Definition \ref{homology-definition-cohomological-delta-functor}
let
$$
0 \to \mathcal{F}_1 \to \mathcal{F}_2 \to \mathcal{F}_3 \to 0
$$
be a short exact sequence of $\QCoh(\mathcal{O}_\mathcal{X})$.
This sequence may not be an exact sequence in
$\textit{Mod}(\mathcal{O}_\mathcal{X})$ but we know that it is
up to parasitic modules, see
Lemma \ref{lemma-exact-sequence-quasi-coherent-parasitic-cohomology}.
Thus we may break up the sequence into short exact sequences
$$
\begin{matrix}
0 \to \mathcal{P}_1 \to \mathcal{F}_1 \to \mathcal{I}_2 \to 0 \\
0 \to \mathcal{I}_2 \to \mathcal{F}_2 \to \mathcal{Q}_2 \to 0 \\
0 \to \mathcal{P}_2 \to \mathcal{Q}_2 \to \mathcal{I}_3 \to 0 \\
0 \to \mathcal{I}_3 \to \mathcal{F}_3 \to \mathcal{P}_3 \to 0
\end{matrix}
$$
of $\textit{Mod}(\mathcal{O}_\mathcal{X})$ with $\mathcal{P}_i$ parasitic.
Note that each of the sheaves
$\mathcal{P}_j$, $\mathcal{I}_j$, $\mathcal{Q}_j$ is an object of
$\mathcal{M}_\mathcal{X}$, see
Proposition \ref{proposition-lcq-flat-base-change}.
Applying $R^if_*$ we obtain long exact sequences 
$$
\begin{matrix}
0 \to f_*\mathcal{P}_1 \to f_*\mathcal{F}_1 \to f_*\mathcal{I}_2 \to
R^1f_*\mathcal{P}_1 \to \ldots \\
0 \to f_*\mathcal{I}_2 \to f_*\mathcal{F}_2 \to f_*\mathcal{Q}_2 \to
R^1f_*\mathcal{I}_2 \to \ldots \\
0 \to f_*\mathcal{P}_2 \to f_*\mathcal{Q}_2 \to f_*\mathcal{I}_3 \to
R^1f_*\mathcal{P}_2 \to \ldots \\
0 \to f_*\mathcal{I}_3 \to f_*\mathcal{F}_3 \to f_*\mathcal{P}_3 \to
R^1f_*\mathcal{I}_3 \to \ldots
\end{matrix}
$$
where are the terms are objects of $\mathcal{M}_\mathcal{Y}$ by
Proposition \ref{proposition-lcq-flat-base-change}.
By
Lemma \ref{lemma-pushforward-parasitic}
the sheaves $R^if_*\mathcal{P}_j$ are parasitic, hence vanish on applying
the functor $Q$, see
Lemma \ref{lemma-adjoint-kernel-parasitic}.
Since $Q$ is exact the maps
$$
Q(R^if_*\mathcal{F}_3)
\cong
Q(R^if_*\mathcal{I}_3)
\cong
Q(R^if_*\mathcal{Q}_2)
\rightarrow
Q(R^{i + 1}f_*\mathcal{I}_2)
\cong
Q(R^{i + 1}f_*\mathcal{F}_1)
$$
can serve as the connecting map which turns the family of functors
$\{R^if_{\QCoh, *}\}_{i \geq 0}$
into a cohomological $\delta$-functor.
\end{proof}

\begin{lemma}
\label{lemma-leray}
Let $f : \mathcal{X} \to \mathcal{Y}$
be a quasi-compact and quasi-separated morphism of algebraic stacks.
Let $\mathcal{F}$ be a quasi-coherent sheaf on $\mathcal{X}$. Then
there exists a spectral sequence with $E_2$-page
$$
E_2^{p, q} = H^p(\mathcal{Y}, R^qf_{\QCoh, *}\mathcal{F})
$$
converging to $H^{p + q}(\mathcal{X}, \mathcal{F})$.
\end{lemma}

\begin{proof}
By Cohomology on Sites, Lemma \ref{sites-cohomology-lemma-Leray}
the Leray spectral sequence with
$$
E_2^{p, q} = H^p(\mathcal{Y}, R^qf_*\mathcal{F})
$$
converges to $H^{p + q}(\mathcal{X}, \mathcal{F})$. The kernel and cokernel
of the adjunction map
$$
R^qf_{\QCoh, *}\mathcal{F} \longrightarrow R^qf_*\mathcal{F}
$$
are parasitic modules on $\mathcal{Y}$
(Lemma \ref{lemma-adjoint-kernel-parasitic})
hence have vanishing cohomology
(Lemma \ref{lemma-pushforward-parasitic}).
It follows formally that
$H^p(\mathcal{Y}, R^qf_{\QCoh, *}\mathcal{F}) =
H^p(\mathcal{Y}, R^qf_*\mathcal{F})$ and we win.
\end{proof}

\begin{lemma}
\label{lemma-relative-leray}
Let $f : \mathcal{X} \to \mathcal{Y}$ and $g : \mathcal{Y} \to \mathcal{Z}$
be quasi-compact and quasi-separated morphisms of algebraic stacks.
Let $\mathcal{F}$ be a quasi-coherent sheaf on $\mathcal{X}$. Then
there exists a spectral sequence with $E_2$-page
$$
E_2^{p, q} = R^pg_{\QCoh, *}(R^qf_{\QCoh, *}\mathcal{F})
$$
converging to $R^{p + q}(g \circ f)_{\QCoh, *}\mathcal{F}$.
\end{lemma}

\begin{proof}
By Cohomology on Sites, Lemma \ref{sites-cohomology-lemma-relative-Leray}
the Leray spectral sequence with
$$
E_2^{p, q} = R^pg_*(R^qf_*\mathcal{F})
$$
converges to $R^{p + q}(g \circ f)_*\mathcal{F}$. By the results of
Proposition \ref{proposition-lcq-flat-base-change}
all the terms of this spectral sequence are objects of
$\mathcal{M}_\mathcal{Z}$. Applying the exact functor
$Q_\mathcal{Z} : \mathcal{M}_\mathcal{Z} \to
\QCoh(\mathcal{O}_\mathcal{Z})$ we obtain a spectral sequence in
$\QCoh(\mathcal{O}_\mathcal{Z})$ covering to
$R^{p + q}(g \circ f)_{\QCoh, *}\mathcal{F}$. Hence
the result follows if we can show that
$$
Q_\mathcal{Z}(R^pg_*(R^qf_*\mathcal{F})) =
Q_\mathcal{Z}(R^pg_*(Q_\mathcal{X}(R^qf_*\mathcal{F}))
$$
This follows from the fact that the kernel and cokernel of the map
$$
Q_\mathcal{X}(R^qf_*\mathcal{F}) \longrightarrow R^qf_*\mathcal{F}
$$
are parasitic (Lemma \ref{lemma-adjoint-kernel-parasitic}) and that
$R^pg_*$ transforms parasitic modules into parasitic modules
(Lemma \ref{lemma-pushforward-parasitic}).
\end{proof}

\noindent
To end this section we make explicit the spectral sequences
associated to a smooth covering by a scheme. Please compare with
Sheaves on Stacks, Sections \ref{stacks-sheaves-section-cohomology} and
\ref{stacks-sheaves-section-higher-direct-images}.

\begin{proposition}
\label{proposition-smooth-covering-compute-cohomology}
Let $f : \mathcal{U} \to \mathcal{X}$ be a morphism of algebraic stacks.
Assume $f$ is representable by algebraic spaces, surjective, flat, and
locally of finite presentation. Let $\mathcal{F}$ be a quasi-coherent
$\mathcal{O}_\mathcal{X}$-module. Then there is a spectral sequence
$$
E_2^{p, q} = H^q(\mathcal{U}_p, f_p^*\mathcal{F})
\Rightarrow
H^{p + q}(\mathcal{X}, \mathcal{F})
$$
where $f_p$ is the morphism
$\mathcal{U} \times_\mathcal{X} \ldots \times_\mathcal{X} \mathcal{U} \to
\mathcal{X}$ ($p + 1$ factors).
\end{proposition}

\begin{proof}
This is a special case of
Sheaves on Stacks, Proposition
\ref{stacks-sheaves-proposition-smooth-covering-compute-cohomology}.
\end{proof}

\begin{proposition}
\label{proposition-smooth-covering-compute-direct-image}
Let $f : \mathcal{U} \to \mathcal{X}$ and $g : \mathcal{X} \to \mathcal{Y}$
be composable morphisms of algebraic stacks.
Assume that
\begin{enumerate}
\item $f$ is representable by algebraic spaces, surjective,
flat, locally of finite presentation, quasi-compact, and quasi-separated, and
\item $g$ is quasi-compact and quasi-separated.
\end{enumerate}
If $\mathcal{F}$ is in $\QCoh(\mathcal{O}_\mathcal{X})$ then
there is a spectral sequence
$$
E_2^{p, q} = R^q(g \circ f_p)_{\QCoh, *}f_p^*\mathcal{F}
\Rightarrow
R^{p + q}g_{\QCoh, *}\mathcal{F}
$$
in $\QCoh(\mathcal{O}_\mathcal{Y})$.
\end{proposition}

\begin{proof}
Note that each of the morphisms
$f_p : \mathcal{U} \times_\mathcal{X} \ldots \times_\mathcal{X} \mathcal{U} \to
\mathcal{X}$ is quasi-compact and quasi-separated, hence $g \circ f_p$
is quasi-compact and quasi-separated, hence the assertion makes sense
(i.e., the functors $R^q(g \circ f_p)_{\QCoh, *}$ are defined).
There is a spectral sequence
$$
E_2^{p, q} = R^q(g \circ f_p)_*f_p^{-1}\mathcal{F}
\Rightarrow
R^{p + q}g_*\mathcal{F}
$$
by Sheaves on Stacks, Proposition
\ref{stacks-sheaves-proposition-smooth-covering-compute-direct-image}.
Applying the exact functor
$Q_\mathcal{Y} : \mathcal{M}_\mathcal{Y} \to
\QCoh(\mathcal{O}_\mathcal{Y})$ gives the desired spectral sequence in
$\QCoh(\mathcal{O}_\mathcal{Y})$.
\end{proof}









\section{The lisse-\'etale and the flat-fppf sites}
\label{section-lisse-etale}

\noindent
In the book \cite{LM-B} many of the results above are proved using the
lisse-\'etale site of an algebraic stack. We define this site here.
In Examples, Section \ref{examples-section-lisse-etale-not-functorial}
we show that the lisse-\'etale site isn't functorial.
We also define its analogue, the flat-fppf site, which is better suited
to the development of algebraic stacks as given in the Stacks project
(because we use the fppf topology as our base topology). Of course the
flat-fppf site isn't functorial either.

\begin{definition}
\label{definition-lisse-etale}
Let $\mathcal{X}$ be an algebraic stack.
\begin{enumerate}
\item The {\it lisse-\'etale site} of $\mathcal{X}$ is the full subcategory
$\mathcal{X}_{lisse,\etale}$\footnote{In the literature the
site is denoted $\text{Lis-\'et}(\mathcal{X})$ or
$\text{Lis-Et}(\mathcal{X})$ and the associated topos is denoted
$\mathcal{X}_{\text{lis-\'e}t}$ or $\mathcal{X}_{\text{lis-et}}$.
In the Stacks project our convention is to name the site and
denote the corresponding topos by $\Sh(\mathcal{C})$.} of $\mathcal{X}$
whose objects are those $x \in \Ob(\mathcal{X})$ lying over a scheme $U$
such that $x : U \to \mathcal{X}$ is smooth. A covering of
$\mathcal{X}_{lisse,\etale}$ is a family of morphisms
$\{x_i \to x\}_{i \in I}$ of $\mathcal{X}_{lisse,\etale}$
which forms a covering of $\mathcal{X}_\etale$.
\item The {\it flat-fppf site} of $\mathcal{X}$ is the full subcategory
$\mathcal{X}_{flat,fppf}$ of $\mathcal{X}$
whose objects are those $x \in \Ob(\mathcal{X})$ lying over a scheme $U$
such that $x : U \to \mathcal{X}$ is flat. A covering of
$\mathcal{X}_{flat,fppf}$ is a family of morphisms
$\{x_i \to x\}_{i \in I}$ of $\mathcal{X}_{flat,fppf}$
which forms a covering of $\mathcal{X}_{fppf}$.
\end{enumerate}
\end{definition}

\noindent
We denote $\mathcal{O}_{\mathcal{X}_{lisse,\etale}}$
the restriction of $\mathcal{O}_\mathcal{X}$ to the lisse-\'etale site
and similarly for $\mathcal{O}_{\mathcal{X}_{flat,fppf}}$.
The relationship between the lisse-\'etale site and the \'etale site is
as follows (we mainly stick to ``topological'' properties in this lemma).

\begin{lemma}
\label{lemma-lisse-etale}
Let $\mathcal{X}$ be an algebraic stack.
\begin{enumerate}
\item The inclusion functor
$\mathcal{X}_{lisse,\etale} \to \mathcal{X}_\etale$
is fully faithful, continuous and cocontinuous. It follows that
\begin{enumerate}
\item there is a morphism of topoi
$$
g :
\Sh(\mathcal{X}_{lisse,\etale})
\longrightarrow
\Sh(\mathcal{X}_\etale)
$$
with $g^{-1}$ given by restriction,
\item the functor $g^{-1}$ has a left adjoint $g_!^{Sh}$ on sheaves of sets,
\item the adjunction maps $g^{-1}g_* \to \text{id}$ and
$\text{id} \to g^{-1}g_!^{Sh}$ are isomorphisms,
\item the functor $g^{-1}$ has a left adjoint $g_!$ on abelian sheaves,
\item the adjunction map $\text{id} \to g^{-1}g_!$ is an isomorphism, and
\item we have $g^{-1}\mathcal{O}_\mathcal{X} =
\mathcal{O}_{\mathcal{X}_{lisse,\etale}}$ hence $g$ induces a flat
morphism of ringed topoi such that $g^{-1} = g^*$.
\end{enumerate}
\item The inclusion functor
$\mathcal{X}_{flat,fppf} \to \mathcal{X}_{fppf}$
is fully faithful, continuous and cocontinuous. It follows that
\begin{enumerate}
\item there is a morphism of topoi
$$
g :
\Sh(\mathcal{X}_{flat,fppf})
\longrightarrow
\Sh(\mathcal{X}_{fppf})
$$
with $g^{-1}$ given by restriction,
\item the functor $g^{-1}$ has a left adjoint $g_!^{Sh}$ on sheaves of sets,
\item the adjunction maps $g^{-1}g_* \to \text{id}$ and
$\text{id} \to g^{-1}g_!^{Sh}$ are isomorphisms,
\item the functor $g^{-1}$ has a left adjoint $g_!$ on abelian sheaves,
\item the adjunction map $\text{id} \to g^{-1}g_!$ is an isomorphism, and
\item we have $g^{-1}\mathcal{O}_\mathcal{X} =
\mathcal{O}_{\mathcal{X}_{flat,fppf}}$ hence $g$ induces a flat
morphism of ringed topoi such that $g^{-1} = g^*$.
\end{enumerate}
\end{enumerate}
\end{lemma}

\begin{proof}
In both cases it is immediate that the functor is fully faithful,
continuous, and cocontinuous (see
Sites, Definitions \ref{sites-definition-continuous} and
\ref{sites-definition-cocontinuous}).
Hence properties (a), (b), (c) follow from
Sites, Lemmas \ref{sites-lemma-when-shriek} and
\ref{sites-lemma-back-and-forth}.
Parts (d), (e) follow from
Modules on Sites, Lemmas \ref{sites-modules-lemma-g-shriek-adjoint} and
\ref{sites-modules-lemma-back-and-forth}.
Part (f) is immediate.
\end{proof}

\begin{lemma}
\label{lemma-lisse-etale-modules}
Let $\mathcal{X}$ be an algebraic stack. Notation as in
Lemma \ref{lemma-lisse-etale}.
\begin{enumerate}
\item There exists a functor
$$
g_! :
\textit{Mod}(\mathcal{X}_{lisse,\etale},
\mathcal{O}_{\mathcal{X}_{lisse,\etale}})
\longrightarrow
\textit{Mod}(\mathcal{X}_\etale, \mathcal{O}_{\mathcal{X}})
$$
which is left adjoint to $g^*$. Moreover it agrees with the functor $g_!$
on abelian sheaves and $g^*g_! = \text{id}$.
\item There exists a functor
$$
g_! :
\textit{Mod}(\mathcal{X}_{flat,fppf},
\mathcal{O}_{\mathcal{X}_{flat,fppf}})
\longrightarrow
\textit{Mod}(\mathcal{X}_{fppf}, \mathcal{O}_{\mathcal{X}})
$$
which is left adjoint to $g^*$. Moreover it agrees with the functor $g_!$
on abelian sheaves and $g^*g_! = \text{id}$.
\end{enumerate}
\end{lemma}

\begin{proof}
In both cases, the existence of the functor $g_!$ follows from
Modules on Sites, Lemma \ref{sites-modules-lemma-lower-shriek-modules}.
To see that $g_!$ agrees with the functor on abelian sheaves we will
show the maps Modules on Sites, Equation
(\ref{sites-modules-equation-compare-on-localizations})
are isomorphisms.

\medskip\noindent
Lisse-\'etale case. Let $x \in \Ob(\mathcal{X}_{lisse,\etale})$
lying over a scheme $U$ with $x : U \to \mathcal{X}$ smooth.
Consider the induced fully faithful functor
$$
g' :
\mathcal{X}_{lisse,\etale}/x
\longrightarrow
\mathcal{X}_\etale/x
$$
The right hand side is identified with $(\Sch/U)_\etale$ and the
left hand side with the full subcategory of schemes $U'/U$ such that
the composition $U' \to U \to \mathcal{X}$ is smooth. Thus
\'Etale Cohomology, Lemma
\ref{etale-cohomology-lemma-compare-structure-sheaves}
applies.

\medskip\noindent
Flat-fppf case. Let $x \in \Ob(\mathcal{X}_{flat,fppf})$
lying over a scheme $U$ with $x : U \to \mathcal{X}$ flat.
Consider the induced fully faithful functor
$$
g' :
\mathcal{X}_{flat,fppf}/x
\longrightarrow
\mathcal{X}_{fppf}/x
$$
The right hand side is identified with $(\Sch/U)_{fppf}$ and the
left hand side with the full subcategory of schemes $U'/U$ such that
the composition $U' \to U \to \mathcal{X}$ is flat. Thus
\'Etale Cohomology, Lemma
\ref{etale-cohomology-lemma-compare-structure-sheaves}
applies.

\medskip\noindent
In both cases the equality $g^*g_! = \text{id}$ follows from
$g^* = g^{-1}$ and the
equality for abelian sheaves in Lemma \ref{lemma-lisse-etale}.
\end{proof}

\begin{lemma}
\label{lemma-lisse-etale-structure-sheaf}
Let $\mathcal{X}$ be an algebraic stack. Notation as in
Lemmas \ref{lemma-lisse-etale} and \ref{lemma-lisse-etale-modules}.
\begin{enumerate}
\item We have $g_!\mathcal{O}_{\mathcal{X}_{lisse,\etale}} =
\mathcal{O}_\mathcal{X}$.
\item We have $g_!\mathcal{O}_{\mathcal{X}_{flat, fppf}} =
\mathcal{O}_\mathcal{X}$.
\end{enumerate}
\end{lemma}

\begin{proof}
In this proof we write
$\mathcal{C} = \mathcal{X}_\etale$
(resp.\ $\mathcal{C} = \mathcal{X}_{fppf}$)
and we denote
$\mathcal{C}' = \mathcal{X}_{lisse,\etale}$
(resp.\ $\mathcal{C}' = \mathcal{X}_{flat, fppf}$).
Then $\mathcal{C}'$ is a full subcategory of $\mathcal{C}$.
In this proof we will think of objects $V$ of $\mathcal{C}$
as schemes over $\mathcal{X}$ and objects $U$ of $\mathcal{C}'$
as schemes smooth (resp.\ flat) over $\mathcal{X}$.
Finally, we write $\mathcal{O} = \mathcal{O}_\mathcal{X}$
and $\mathcal{O}' = \mathcal{O}_{\mathcal{X}_{lisse,\etale}}$
(resp.\ $\mathcal{O}' = \mathcal{O}_{\mathcal{X}_{flat,fppf}}$).
In the notation above we have $\mathcal{O}(V) = \Gamma(V, \mathcal{O}_V)$
and $\mathcal{O}'(U) = \Gamma(U, \mathcal{O}_U)$.
Consider the $\mathcal{O}$-module homomorphism
$g_!\mathcal{O}' \to \mathcal{O}$
adjoint to the identification $\mathcal{O}' = g^{-1}\mathcal{O}$.

\medskip\noindent
Recall that $g_!\mathcal{O}'$ is the sheaf associated to the presheaf
$g_{p!}\mathcal{O}'$ given by the rule
$$
V \longmapsto \colim_{V \to U} \mathcal{O}'(U)
$$
where the colimit is taken in the category of abelian groups
(Modules on Sites, Definition \ref{sites-modules-definition-g-shriek}).
Below we will use frequently that if
$$
V \to U \to U'
$$
are morphisms and if $f' \in \mathcal{O}'(U')$ restricts to
$f \in \mathcal{O}'(U)$, then $(V \to U, f)$ and $(V \to U', f')$
define the same element of the colimit. Also,
$g_!\mathcal{O}' \to \mathcal{O}$ maps the element
$(V \to U, f)$ simply to the pullback of $f$ to $V$.

\medskip\noindent
To see that $g_!\mathcal{O}' \to \mathcal{O}$ is surjective it
suffices to show that $1 \in \Gamma(\mathcal{C}, \mathcal{O})$ is
locally in the image. Choose an object $U$ of $\mathcal{C}'$
corresponding to a surjective smooth morphism $U \to \mathcal{X}$.
Then viewing $U$ both as an object of $\mathcal{C}'$ and $\mathcal{C}$
we see that $(U \to U, 1)$ is an element of the colimit above which
maps to $1 \in \mathcal{O}(U)$. Since $U$ surjects onto the final
object of $\Sh(\mathcal{C})$ we conclude $g_!\mathcal{O}' \to \mathcal{O}$
is surjective.

\medskip\noindent
Suppose that $s \in g_!\mathcal{O}'(V)$ is a section
mapping to zero in $\mathcal{O}(V)$. To finish the proof we have to show
that $s$ is zero. After replacing $V$ by the members
of a covering we may assume $s$ is an element of the colimit
$$
\colim_{V \to U} \mathcal{O}'(U)
$$
Say $s = \sum (\varphi_i, s_i)$ is a finite sum with
$\varphi_i : V \to U_i$, $U_i$ smooth (resp.\ flat) over $\mathcal{X}$, and
$s_i \in \Gamma(U_i, \mathcal{O}_{U_i})$. Choose a scheme $W$ surjective
\'etale over the algebraic space
$U = U_1 \times_\mathcal{X} \ldots \times_\mathcal{X} U_n$.
Note that $W$ is still smooth (resp.\ flat) over $\mathcal{X}$, i.e.,
defines an object of $\mathcal{C}'$. The fibre product
$$
V' = V \times_{(\varphi_1, \ldots, \varphi_n), U} W
$$
is surjective \'etale over $V$, hence it suffices to show that $s$ maps
to zero in $g_!\mathcal{O}'(V')$. Note that the restriction
$\sum (\varphi_i, s_i)|_{V'}$ corresponds to the sum of the pullbacks
of the functions $s_i$ to $W$. In other words, we have reduced to the case
of $(\varphi, s)$ where $\varphi : V \to U$ is a morphism with $U$ in
$\mathcal{C}'$ and $s \in \mathcal{O}'(U)$ restricts to zero in
$\mathcal{O}(V)$. By the commutative diagram
$$
\xymatrix{
V \ar[rr]_-{(\varphi, 0)} \ar[rrd]_\varphi & & U \times \mathbf{A}^1 \\
& & U \ar[u]_{(\text{id}, 0)}
}
$$
we see that
$((\varphi, 0) : V \to U \times \mathbf{A}^1, \text{pr}_2^*x)$
represents zero in the colimit above. Hence we may
replace $U$ by $U \times \mathbf{A}^1$, $\varphi$ by $(\varphi, 0)$
and $s$ by $\text{pr}_1^*s + \text{pr}_2^*x$. Thus we may assume that
the vanishing locus $Z : s = 0$ in $U$ of $s$ is smooth (resp.\ flat)
over $\mathcal{X}$. Then we see that $(V \to Z, 0)$ and $(\varphi, s)$
have the same value in the colimit, i.e., we see that the element $s$
is zero as desired.
\end{proof}

\noindent
The lisse-\'etale and the flat-fppf sites can be used to characterize
parasitic modules as follows.

\begin{lemma}
\label{lemma-parasitic-in-terms-flat-fppf}
Let $\mathcal{X}$ be an algebraic stack.
\begin{enumerate}
\item Let $\mathcal{F}$ be an $\mathcal{O}_\mathcal{X}$-module
with the flat base change property on $\mathcal{X}_\etale$.
The following are equivalent
\begin{enumerate}
\item $\mathcal{F}$ is parasitic, and
\item $g^*\mathcal{F} = 0$ where
$g : \Sh(\mathcal{X}_{lisse,\etale}) \to
\Sh(\mathcal{X}_\etale)$ is as in Lemma \ref{lemma-lisse-etale}.
\end{enumerate}
\item Let $\mathcal{F}$ be an $\mathcal{O}_\mathcal{X}$-module on
$\mathcal{X}_{fppf}$. The following are equivalent
\begin{enumerate}
\item $\mathcal{F}$ is parasitic, and
\item $g^*\mathcal{F} = 0$ where
$g :  \Sh(\mathcal{X}_{flat,fppf}) \to \Sh(\mathcal{X}_{fppf})$
is as in Lemma \ref{lemma-lisse-etale}.
\end{enumerate}
\end{enumerate}
\end{lemma}

\begin{proof}
Part (2) is immediate from the definitions (this is one of the advantages
of the flat-fppf site over the lisse-\'etale site). The implication
(1)(a) $\Rightarrow$ (1)(b) is immediate as well. To see (1)(b)
$\Rightarrow$ (1)(a) let $U$ be a scheme and let $x : U \to \mathcal{X}$
be a surjective smooth morphism. Then $x$ is an object of the
lisse-\'etale site of $\mathcal{X}$. Hence we see that (1)(b)
implies that $\mathcal{F}|_{U_\etale} = 0$. Let $V \to \mathcal{X}$
be an flat morphism where $V$ is a scheme. Set $W = U \times_\mathcal{X} V$
and consider the diagram
$$
\xymatrix{
W \ar[d]_p \ar[r]_q & V \ar[d] \\
U \ar[r] & \mathcal{X}
}
$$
Note that the projection $p : W \to U$ is flat and the projection
$q : W \to V$ is smooth and surjective. This implies that $q_{small}^*$
is a faithful functor on quasi-coherent modules. By assumption $\mathcal{F}$
has the flat base change property so that we obtain
$p_{small}^*\mathcal{F}|_{U_\etale} \cong
q_{small}^*\mathcal{F}|_{V_\etale}$. Thus if $\mathcal{F}$
is in the kernel of $g^*$, then $\mathcal{F}|_{V_\etale} = 0$
as desired.
\end{proof}

\noindent
The lisse-\'etale site is functorial for smooth morphisms of algebraic stacks
and the flat-fppf site is functorial for flat morphisms of algebraic stacks.

\begin{lemma}
\label{lemma-lisse-etale-functorial}
Let $f : \mathcal{X} \to \mathcal{Y}$ be a morphism of algebraic stacks.
\begin{enumerate}
\item If $f$ is smooth, then $f$ restricts to a continuous and cocontinuous
functor
$\mathcal{X}_{lisse,\etale} \to \mathcal{Y}_{lisse,\etale}$
which gives a morphism of ringed topoi fitting into the following
commutative diagram
$$
\xymatrix{
\Sh(\mathcal{X}_{lisse,\etale}) \ar[r]_{g'} \ar[d]_{f'} &
\Sh(\mathcal{X}_\etale) \ar[d]^f \\
\Sh(\mathcal{Y}_{lisse,\etale}) \ar[r]^g &
\Sh(\mathcal{Y}_\etale)
}
$$
We have $f'_*(g')^{-1} = g^{-1}f_*$ and $g'_!(f')^{-1} = f^{-1}g_!$.
\item If $f$ is flat, then $f$ restricts to a continuous and cocontinuous
functor
$\mathcal{X}_{flat,fppf} \to \mathcal{Y}_{flat,fppf}$
which gives a morphism of ringed topoi fitting into the following
commutative diagram
$$
\xymatrix{
\Sh(\mathcal{X}_{flat,fppf}) \ar[r]_{g'} \ar[d]_{f'} &
\Sh(\mathcal{X}_{fppf}) \ar[d]^f \\
\Sh(\mathcal{Y}_{flat,fppf}) \ar[r]^g &
\Sh(\mathcal{Y}_{fppf})
}
$$
We have $f'_*(g')^{-1} = g^{-1}f_*$ and $g'_!(f')^{-1} = f^{-1}g_!$.
\end{enumerate}
\end{lemma}

\begin{proof}
The initial statement comes from the fact that if $x \in \Ob(\mathcal{X})$
lies over a scheme $U$ such that $x : U \to \mathcal{X}$ is smooth
(resp.\ flat) and if $f$ is smooth (resp.\ flat) then
$f(x) : U \to \mathcal{Y}$ is smooth (resp.\ flat), see
Morphisms of Stacks, Lemmas \ref{stacks-morphisms-lemma-composition-smooth} and
\ref{stacks-morphisms-lemma-composition-flat}. The induced functor
$\mathcal{X}_{lisse,\etale} \to \mathcal{Y}_{lisse,\etale}$
(resp.\ $\mathcal{X}_{flat,fppf} \to \mathcal{Y}_{flat,fppf}$) is
continuous and cocontinuous by our definition of coverings in these
categories. Finally, the commutativity of the diagram is a consequence of the
fact that the horizontal morphisms are given by the inclusion functors (see
Lemma \ref{lemma-lisse-etale}) and
Sites, Lemma \ref{sites-lemma-composition-cocontinuous}.

\medskip\noindent
To show that $f'_*(g')^{-1} = g^{-1}f_*$ let $\mathcal{F}$ be a sheaf
on $\mathcal{X}_\etale$ (resp.\ $\mathcal{X}_{fppf}$).
There is a canonical pullback map
$$
g^{-1}f_*\mathcal{F} \longrightarrow f'_*(g')^{-1}\mathcal{F}
$$
see Sites, Section \ref{sites-section-pullback}.
We claim this map is an isomorphism.
To prove this pick an object $y$ of $\mathcal{Y}_{lisse,\etale}$
(resp.\ $\mathcal{Y}_{flat,fppf}$). Say $y$ lies over the scheme $V$
such that $y : V \to \mathcal{Y}$ is smooth (resp.\ flat). Since
$g^{-1}$ is the restriction we find that
$$
\left(g^{-1}f_*\mathcal{F}\right)(y) =
\Gamma(V \times_{y, \mathcal{Y}} \mathcal{X},\ \text{pr}^{-1}\mathcal{F})
$$
by Sheaves on Stacks, Equation (\ref{stacks-sheaves-equation-pushforward}).
Let
$(V \times_{y, \mathcal{Y}} \mathcal{X})' \subset
V \times_{y, \mathcal{Y}} \mathcal{X}$
be the full subcategory consisting of objects
$z : W \to V \times_{y, \mathcal{Y}} \mathcal{X}$ such that the induced
morphism $W \to \mathcal{X}$ is smooth (resp.\ flat). Denote
$$
\text{pr}' :
(V \times_{y, \mathcal{Y}} \mathcal{X})'
\longrightarrow
\mathcal{X}_{lisse,\etale}
\ (\text{resp. }\mathcal{X}_{flat,fppf})
$$
the restriction of the functor $\text{pr}$ used in the formula above.
Exactly the same argument that proves
Sheaves on Stacks, Equation (\ref{stacks-sheaves-equation-pushforward})
shows that for any sheaf $\mathcal{H}$ on $\mathcal{X}_{lisse,\etale}$
(resp.\ $\mathcal{X}_{flat,fppf}$) we have
\begin{equation}
\label{equation-pushforward-lisse-etale}
f'_*\mathcal{H}(y) =
\Gamma((V \times_{y, \mathcal{Y}} \mathcal{X})',
\ (\text{pr}')^{-1}\mathcal{H})
\end{equation}
Since $(g')^{-1}$ is restriction we see that
$$
\left(f'_*(g')^{-1}\mathcal{F}\right)(y) =
\Gamma((V \times_{y, \mathcal{Y}} \mathcal{X})',
\ \text{pr}^{-1}\mathcal{F}|_{(V \times_{y, \mathcal{Y}} \mathcal{X})'})
$$
By
Sheaves on Stacks, Lemma \ref{stacks-sheaves-lemma-cohomology-on-subcategory}
we see that
$$
\Gamma((V \times_{y, \mathcal{Y}} \mathcal{X})',
\ \text{pr}^{-1}\mathcal{F}|_{(V \times_{y, \mathcal{Y}} \mathcal{X})'})
=
\Gamma(V \times_{y, \mathcal{Y}} \mathcal{X},\ \text{pr}^{-1}\mathcal{F})
$$
are equal as desired; although we omit the verification of the assumptions
of the lemma we note that the fact that $V \to \mathcal{Y}$ is smooth
(resp.\ flat) is used to verify the second condition.

\medskip\noindent
Finally, the equality $g'_!(f')^{-1} = f^{-1}g_!$ follows formally from
the equality $f'_*(g')^{-1} = g^{-1}f_*$ by the adjointness of
$f^{-1}$ and $f_*$, the adjointness of $g_!$ and $g^{-1}$, and their
``primed'' versions.
\end{proof}






\section{Quasi-coherent modules, II}
\label{section-quasi-coherent-modules-II}

\noindent
In this section we explain how to think of quasi-coherent modules
on an algebraic stack in terms of its lisse-\'etale or flat-fppf site.

\begin{lemma}
\label{lemma-check-qc-on-etale-covering}
Let $\mathcal{X}$ be an algebraic stack.
\begin{enumerate}
\item Let $f_j : \mathcal{X}_j \to \mathcal{X}$ be a family of smooth
morphisms of algebraic stacks with
$|\mathcal{X}| =\bigcup |f_j|(|\mathcal{X}_j|)$.
Let $\mathcal{F}$ be a sheaf of $\mathcal{O}_\mathcal{X}$-modules
on $\mathcal{X}_\etale$. If each $f_j^{-1}\mathcal{F}$
is quasi-coherent, then so is $\mathcal{F}$.
\item Let $f_j : \mathcal{X}_j \to \mathcal{X}$ be a family of flat and
locally finitely presented morphisms of algebraic stacks with
$|\mathcal{X}| =\bigcup |f_j|(|\mathcal{X}_j|)$.
Let $\mathcal{F}$ be a sheaf of $\mathcal{O}_\mathcal{X}$-modules
on $\mathcal{X}_{fppf}$. If each $f_j^{-1}\mathcal{F}$
is quasi-coherent, then so is $\mathcal{F}$.
\end{enumerate}
\end{lemma}

\begin{proof}
Proof of (1). We may replace each of the algebraic stacks $\mathcal{X}_j$
by a scheme $U_j$ (using that any algebraic stack has a smooth covering by
a scheme and that compositions of smooth morphisms are smooth, see
Morphisms of Stacks, Lemma \ref{stacks-morphisms-lemma-composition-smooth}).
The pullback of $\mathcal{F}$ to $(\Sch/U_j)_\etale$ is still
quasi-coherent, see
Modules on Sites, Lemma \ref{sites-modules-lemma-local-pullback}.
Then $f = \coprod f_j : U = \coprod U_j \to \mathcal{X}$ is a smooth surjective
morphism. Let $x : V \to \mathcal{X}$ be an object of $\mathcal{X}$. By
Sheaves on Stacks, Lemma
\ref{stacks-sheaves-lemma-surjective-flat-locally-finite-presentation}
there exists an \'etale covering $\{x_i \to x\}_{i \in I}$
such that each $x_i$ lifts to an object $u_i$ of $(\Sch/U)_\etale$.
This just means that $x_i$ lives over a scheme $V_i$, that
$\{V_i \to V\}$ is an \'etale covering, and that $x_i$ comes from
a morphism $u_i : V_i \to U$. Then
$x_i^*\mathcal{F} = u_i^*f^*\mathcal{F}$ is quasi-coherent.
This implies that $x^*\mathcal{F}$ on $(\Sch/V)_\etale$
is quasi-coherent, for example by
Modules on Sites, Lemma \ref{sites-modules-lemma-local-final-object}.
By Sheaves on Stacks, Lemma
\ref{stacks-sheaves-lemma-characterize-quasi-coherent-bis}
we see that $x^*\mathcal{F}$ is an fppf sheaf and since $x$
was arbitrary we see that $\mathcal{F}$ is a sheaf in the
fppf topology. Applying Sheaves on Stacks, Lemma
\ref{stacks-sheaves-lemma-characterize-quasi-coherent}
we see that $\mathcal{F}$ is quasi-coherent.

\medskip\noindent
Proof of (2). This is proved using exactly the same argument, which we fully
write out here. We may replace each of the algebraic stacks $\mathcal{X}_j$
by a scheme $U_j$ (using that any algebraic stack has a smooth covering by
a scheme and that flat and locally finite presented morphisms are preserved
by composition, see Morphisms of Stacks, Lemmas
\ref{stacks-morphisms-lemma-composition-flat} and
\ref{stacks-morphisms-lemma-composition-finite-presentation}).
The pullback of $\mathcal{F}$ to $(\Sch/U_j)_\etale$ is still
locally quasi-coherent, see
Sheaves on Stacks, Lemma \ref{stacks-sheaves-lemma-pullback-quasi-coherent}.
Then $f = \coprod f_j : U = \coprod U_j \to \mathcal{X}$ is a surjective,
flat, and locally finitely presented morphism. Let
$x : V \to \mathcal{X}$ be an object of $\mathcal{X}$. By
Sheaves on Stacks, Lemma
\ref{stacks-sheaves-lemma-surjective-flat-locally-finite-presentation}
there exists an fppf covering $\{x_i \to x\}_{i \in I}$
such that each $x_i$ lifts to an object $u_i$ of $(\Sch/U)_\etale$.
This just means that $x_i$ lives over a scheme $V_i$, that
$\{V_i \to V\}$ is an fppf covering, and that $x_i$ comes from
a morphism $u_i : V_i \to U$. Then
$x_i^*\mathcal{F} = u_i^*f^*\mathcal{F}$ is quasi-coherent.
This implies that $x^*\mathcal{F}$ on $(\Sch/V)_\etale$
is quasi-coherent, for example by
Modules on Sites, Lemma \ref{sites-modules-lemma-local-final-object}.
By Sheaves on Stacks, Lemma
\ref{stacks-sheaves-lemma-characterize-quasi-coherent}
we see that $\mathcal{F}$ is quasi-coherent.
\end{proof}

\noindent
We recall that we have defined the notion of a quasi-coherent module on
any ringed topos in
Modules on Sites, Section \ref{sites-modules-section-local}.

\begin{lemma}
\label{lemma-shriek-quasi-coherent}
Let $\mathcal{X}$ be an algebraic stack. Notation as in
Lemma \ref{lemma-lisse-etale}.
\begin{enumerate}
\item Let $\mathcal{H}$ be a quasi-coherent
$\mathcal{O}_{\mathcal{X}_{lisse,\etale}}$-module 
on the lisse-\'etale site of $\mathcal{X}$. Then $g_!\mathcal{H}$ is a
quasi-coherent module on $\mathcal{X}$.
\item Let $\mathcal{H}$ be a quasi-coherent
$\mathcal{O}_{\mathcal{X}_{flat,fppf}}$-module 
on the flat-fppf site of $\mathcal{X}$. Then $g_!\mathcal{H}$ is a
quasi-coherent module on $\mathcal{X}$.
\end{enumerate}
\end{lemma}

\begin{proof}
Pick a scheme $U$ and a surjective smooth morphism $x : U \to \mathcal{X}$.
By
Modules on Sites, Definition \ref{sites-modules-definition-site-local}
there exists an \'etale (resp.\ fppf) covering
$\{U_i \to U\}_{i \in I}$ such that each pullback $f_i^{-1}\mathcal{H}$
has a global presentation (see
Modules on Sites, Definition \ref{sites-modules-definition-global}).
Here $f_i : U_i \to \mathcal{X}$ is the composition
$U_i \to U \to \mathcal{X}$ which is a morphism of algebraic stacks.
(Recall that the pullback ``is'' the restriction to $\mathcal{X}/f_i$, see
Sheaves on Stacks, Definition \ref{stacks-sheaves-definition-pullback}
and the discussion following.) Since each $f_i$ is smooth (resp.\ flat) by
Lemma \ref{lemma-lisse-etale-functorial}
we see that $f_i^{-1}g_!\mathcal{H} = g_{i, !}(f'_i)^{-1}\mathcal{H}$.
Using Lemma \ref{lemma-check-qc-on-etale-covering}
we reduce the statement of the lemma to the case where $\mathcal{H}$
has a global presentation. Say we have
$$
\bigoplus\nolimits_{j \in J} \mathcal{O} \longrightarrow
\bigoplus\nolimits_{i \in I} \mathcal{O} \longrightarrow
\mathcal{H} \longrightarrow 0
$$
of $\mathcal{O}$-modules where
$\mathcal{O} = \mathcal{O}_{\mathcal{X}_{lisse,\etale}}$
(resp.\ $\mathcal{O} = \mathcal{O}_{\mathcal{X}_{flat,fppf}}$).
Since $g_!$ commutes with arbitrary colimits (as a left adjoint functor, see
Lemma \ref{lemma-lisse-etale-modules} and
Categories, Lemma \ref{categories-lemma-adjoint-exact})
we conclude that there exists an exact sequence
$$
\bigoplus\nolimits_{j \in J} g_!\mathcal{O} \longrightarrow
\bigoplus\nolimits_{i \in I} g_!\mathcal{O} \longrightarrow
g_!\mathcal{H} \longrightarrow 0
$$
Lemma \ref{lemma-lisse-etale-structure-sheaf}
shows that $g_!\mathcal{O} = \mathcal{O}_\mathcal{X}$.
In case (2) we are done. In case (1) we apply
Sheaves on Stacks, Lemma
\ref{stacks-sheaves-lemma-characterize-quasi-coherent-bis}
to conclude.
\end{proof}

\begin{lemma}
\label{lemma-quasi-coherent}
Let $\mathcal{X}$ be an algebraic stack. Let $\mathcal{M}_\mathcal{X}$
be the category of locally quasi-coherent $\mathcal{O}_\mathcal{X}$-modules
with the flat base change property.
\begin{enumerate}
\item With $g$ as in Lemma \ref{lemma-lisse-etale}
for the lisse-\'etale site we have
\begin{enumerate}
\item the functors $g^{-1}$ and $g_!$ define mutually inverse functors
$$
\xymatrix{
\QCoh(\mathcal{O}_\mathcal{X}) \ar@<1ex>[r]^-{g^{-1}} &
\QCoh(\mathcal{X}_{lisse,\etale},
\mathcal{O}_{\mathcal{X}_{lisse,\etale}}) \ar@<1ex>[l]^-{g_!}
}
$$
\item if $\mathcal{F}$ is in $\mathcal{M}_\mathcal{X}$
then $g^{-1}\mathcal{F}$ is in
$\QCoh(\mathcal{X}_{lisse,\etale},
\mathcal{O}_{\mathcal{X}_{lisse,\etale}})$ and
\item $Q(\mathcal{F}) = g_!g^{-1}\mathcal{F}$ where $Q$ is as in
Lemma \ref{lemma-adjoint}.
\end{enumerate}
\item With $g$ as in Lemma \ref{lemma-lisse-etale}
for the flat-fppf site we have
\begin{enumerate}
\item the functors $g^{-1}$ and $g_!$ define mutually inverse functors
$$
\xymatrix{
\QCoh(\mathcal{O}_\mathcal{X}) \ar@<1ex>[r]^-{g^{-1}} &
\QCoh(\mathcal{X}_{flat,fppf},
\mathcal{O}_{\mathcal{X}_{flat,fppf}}) \ar@<1ex>[l]^-{g_!}
}
$$
\item if $\mathcal{F}$ is in $\mathcal{M}_\mathcal{X}$
then $g^{-1}\mathcal{F}$ is in
$\QCoh(\mathcal{X}_{flat,fppf}, \mathcal{O}_{\mathcal{X}_{flat,fppf}})$
and
\item $Q(\mathcal{F}) = g_!g^{-1}\mathcal{F}$ where $Q$ is as in
Lemma \ref{lemma-adjoint}.
\end{enumerate}
\end{enumerate}
\end{lemma}

\begin{proof}
Pullback by any morphism of ringed topoi preserves categories of quasi-coherent
modules, see
Modules on Sites, Lemma \ref{sites-modules-lemma-local-pullback}.
Hence $g^{-1}$ preserves the categories of quasi-coherent modules;
here we use that
$\QCoh(\mathcal{O}_\mathcal{X}) =
\QCoh(\mathcal{X}_\etale, \mathcal{O}_\mathcal{X})$
by Sheaves on Stacks, Lemma
\ref{stacks-sheaves-lemma-characterize-quasi-coherent-bis}.
The same is true for $g_!$ by
Lemma \ref{lemma-shriek-quasi-coherent}.
We know that $\mathcal{H} \to g^{-1}g_!\mathcal{H}$ is an isomorphism by
Lemma \ref{lemma-lisse-etale}.
Conversely, if $\mathcal{F}$ is in $\QCoh(\mathcal{O}_\mathcal{X})$
then the map $g_!g^{-1}\mathcal{F} \to \mathcal{F}$ is a map of quasi-coherent
modules on $\mathcal{X}$ whose restriction to any scheme smooth over
$\mathcal{X}$ is an isomorphism. Then the discussion in
Sheaves on Stacks, Sections
\ref{stacks-sheaves-section-quasi-coherent-presentation} and
\ref{stacks-sheaves-section-quasi-coherent-algebraic-stacks}
(comparing with quasi-coherent modules on presentations)
shows it is an isomorphism. This proves (1)(a) and (2)(a).

\medskip\noindent
Let $\mathcal{F}$ be an object of $\mathcal{M}_\mathcal{X}$. By
Lemma \ref{lemma-adjoint-kernel-parasitic}
the kernel and cokernel of the map
$Q(\mathcal{F}) \to \mathcal{F}$ are parasitic. Hence by
Lemma \ref{lemma-parasitic-in-terms-flat-fppf}
and since $g^* = g^{-1}$ is exact, we conclude
$g^*Q(\mathcal{F}) \to g^*\mathcal{F}$ is an isomorphism. Thus
$g^*\mathcal{F}$ is quasi-coherent. This proves (1)(b) and (2)(b).
Finally, (1)(c) and (2)(c) follow because
$g_!g^*Q(\mathcal{F}) \to Q(\mathcal{F})$ is an isomorphism by
our arguments above.
\end{proof}

\begin{remark}
\label{remark-bousfield-colocalization}
Let $\mathcal{X}$ be an algebraic stack. The results of
Lemmas \ref{lemma-adjoint} and \ref{lemma-adjoint-kernel-parasitic}
imply that
$$
\QCoh(\mathcal{O}_\mathcal{X}) =
\mathcal{M}_\mathcal{X} / \text{Parasitic} \cap \mathcal{M}_\mathcal{X}
$$
in words: the category of quasi-coherent modules is the category
of locally quasi-coherent modules with the flat base change property
divided out by the Serre subcategory consisting of parasitic objects.
See Homology, Lemma \ref{homology-lemma-serre-subcategory-is-kernel}.
The existence of the inclusion functor
$i : \QCoh(\mathcal{O}_\mathcal{X}) \to \mathcal{M}_\mathcal{X}$
which is left adjoint to the quotient functor means that
$\mathcal{M}_\mathcal{X} \to \QCoh(\mathcal{O}_\mathcal{X})$
is a {\it Bousfield colocalization} or a {\it right Bousfield localization}
(insert future reference here). Our next goal is to show a similar result
holds on the level of derived categories.
\end{remark}

\begin{lemma}
\label{lemma-quasi-coherent-weak-serre}
Let $\mathcal{X}$ be an algebraic stack.
\begin{enumerate}
\item $\QCoh(\mathcal{O}_{\mathcal{X}_{lisse,\etale}})$
is a weak Serre subcategory of
$\textit{Mod}(\mathcal{O}_{\mathcal{X}_{lisse,\etale}})$.
\item $\QCoh(\mathcal{O}_{\mathcal{X}_{flat,fppf}})$
is a weak Serre subcategory of
$\textit{Mod}(\mathcal{O}_{\mathcal{X}_{flat,fppf}})$.
\end{enumerate}
\end{lemma}

\begin{proof}
We will verify conditions (1), (2), (3), (4) of
Homology, Lemma \ref{homology-lemma-characterize-weak-serre-subcategory}.
Since $0$ is a quasi-coherent module on any ringed site we see that (1)
holds. By definition $\QCoh(\mathcal{O})$
is a strictly full subcategory $\textit{Mod}(\mathcal{O})$, so (2) holds.
Let $\varphi : \mathcal{G} \to \mathcal{F}$ be a morphism of quasi-coherent
modules on $\mathcal{X}_{lisse,\etale}$ or $\mathcal{X}_{flat,fppf}$.
We have $g^*g_!\mathcal{F} = \mathcal{F}$ and similarly for
$\mathcal{G}$ and $\varphi$, see Lemma \ref{lemma-lisse-etale-modules}.
By Lemma \ref{lemma-shriek-quasi-coherent}
we see that $g_!\mathcal{F}$ and $g_!\mathcal{G}$ are quasi-coherent
$\mathcal{O}_\mathcal{X}$-modules. Hence we see that
$\Ker(g_!\varphi)$ and $\Coker(g_!\varphi)$ are quasi-coherent
modules on $\mathcal{X}$. Since $g^*$ is exact (see
Lemma \ref{lemma-lisse-etale}) we see that
$g^*\Ker(g_!\varphi) = \Ker(g^*g_!\varphi) = \Ker(\varphi)$
and
$g^*\Coker(g_!\varphi) = \Coker(g^*g_!\varphi) =
\Coker(\varphi)$
are quasi-coherent too (see Lemma \ref{lemma-quasi-coherent}).
This proves (3).

\medskip\noindent
Finally, suppose that
$$
0 \to \mathcal{F} \to \mathcal{E} \to \mathcal{G} \to 0
$$
is an extension of $\mathcal{O}_{\mathcal{X}_{lisse,\etale}}$-modules
(resp.\ $\mathcal{O}_{\mathcal{X}_{flat,fppf}}$-modules) with $\mathcal{F}$
and $\mathcal{G}$ quasi-coherent. We have to show that $\mathcal{E}$
is quasi-coherent on $\mathcal{X}_{lisse,\etale}$
(resp.\ $\mathcal{X}_{flat,fppf}$). We strongly urge the reader to write
out what this means on a napkin and prove it him/herself rather than
reading the somewhat labyrinthine proof that follows.
By Lemma \ref{lemma-quasi-coherent}
this is true if and only if $g_!\mathcal{E}$ is quasi-coherent.
By Lemmas \ref{lemma-check-qc-on-etale-covering} and
Lemma \ref{lemma-lisse-etale-functorial}
we may check this after replacing $\mathcal{X}$ by a smooth
(resp.\ fppf) cover. Choose a scheme $U$ and a smooth surjective
morphism $U \to \mathcal{X}$. By definition there exists an \'etale
(resp.\ fppf) covering $\{U_i \to U\}_i$ such that $\mathcal{G}$
has a global presentation over each $U_i$. Replacing $\mathcal{X}$
by $U_i$ (which is permissible by the discussion above)
we may assume that the site $\mathcal{X}_{lisse,\etale}$
(resp.\ $\mathcal{X}_{flat,fppf}$) has a final object $U$
(in other words $\mathcal{X}$ is representable by the scheme $U$)
and that $\mathcal{G}$ has a global presentation
$$
\bigoplus\nolimits_{j \in J} \mathcal{O} \longrightarrow
\bigoplus\nolimits_{i \in I} \mathcal{O} \longrightarrow
\mathcal{G} \longrightarrow 0
$$
of $\mathcal{O}$-modules where
$\mathcal{O} = \mathcal{O}_{\mathcal{X}_{lisse,\etale}}$
(resp.\ $\mathcal{O} = \mathcal{O}_{\mathcal{X}_{flat,fppf}}$).
Let $\mathcal{E}'$ be the pullback of $\mathcal{E}$ via the map
$\bigoplus\nolimits_{i \in I} \mathcal{O} \to \mathcal{G}$.
Then we see that $\mathcal{E}$ is the cokernel of a map
$\bigoplus\nolimits_{j \in J} \mathcal{O} \to \mathcal{E}'$
hence by property (3) which we proved above, it suffices to prove
that $\mathcal{E}'$ is quasi-coherent. Consider the exact sequence
$$
L_1g_!\left(\bigoplus\nolimits_{i \in I}\mathcal{O}\right) \to
g_!\mathcal{F} \to
g_!\mathcal{E}' \to
g_!\left(\bigoplus\nolimits_{i \in I}\mathcal{O}\right) \to 0
$$
where $L_1g_!$ is the first left derived functor of
$g_! : \textit{Mod}(\mathcal{O}_{\mathcal{X}_{lisse,\etale}}) \to
\textit{Mod}(\mathcal{X}_\etale, \mathcal{O}_\mathcal{X})$
(resp.\ $g_! :
\textit{Mod}(\mathcal{X}_{flat,fppf}, \mathcal{O}_{\mathcal{X}_{flat,fppf}})
\to \textit{Mod}(\mathcal{X}_{fppf}, \mathcal{O}_{\mathcal{X}})$).
This derived functor exists by Cohomology on Sites,
Lemma \ref{sites-cohomology-lemma-existence-derived-lower-shriek}.
Moreover, since $\mathcal{O} = j_{U!}\mathcal{O}_U$ we have
$Lg_!\mathcal{O} = g_!\mathcal{O} = \mathcal{O}_\mathcal{X}$
also by Cohomology on Sites,
Lemma \ref{sites-cohomology-lemma-existence-derived-lower-shriek}.
Thus the left hand term vanishes and we obtain a short exact sequence
$$
0 \to
g_!\mathcal{F} \to
g_!\mathcal{E}' \to
\bigoplus\nolimits_{i \in I}\mathcal{O}_\mathcal{X} \to 0
$$
By Proposition \ref{proposition-lcq-flat-base-change}
it follows that $g_!\mathcal{E}'$ is locally quasi-coherent with the
flat base change property. Finally, Lemma \ref{lemma-quasi-coherent}
implies that $\mathcal{E}' = g^{-1}g_!\mathcal{E}'$ is quasi-coherent
as desired.
\end{proof}












\begin{multicols}{2}[\section{Other chapters}]
\noindent
Preliminaries
\begin{enumerate}
\item \hyperref[introduction-section-phantom]{Introduction}
\item \hyperref[conventions-section-phantom]{Conventions}
\item \hyperref[sets-section-phantom]{Set Theory}
\item \hyperref[categories-section-phantom]{Categories}
\item \hyperref[topology-section-phantom]{Topology}
\item \hyperref[sheaves-section-phantom]{Sheaves on Spaces}
\item \hyperref[sites-section-phantom]{Sites and Sheaves}
\item \hyperref[stacks-section-phantom]{Stacks}
\item \hyperref[fields-section-phantom]{Fields}
\item \hyperref[algebra-section-phantom]{Commutative Algebra}
\item \hyperref[brauer-section-phantom]{Brauer Groups}
\item \hyperref[homology-section-phantom]{Homological Algebra}
\item \hyperref[derived-section-phantom]{Derived Categories}
\item \hyperref[simplicial-section-phantom]{Simplicial Methods}
\item \hyperref[more-algebra-section-phantom]{More on Algebra}
\item \hyperref[smoothing-section-phantom]{Smoothing Ring Maps}
\item \hyperref[modules-section-phantom]{Sheaves of Modules}
\item \hyperref[sites-modules-section-phantom]{Modules on Sites}
\item \hyperref[injectives-section-phantom]{Injectives}
\item \hyperref[cohomology-section-phantom]{Cohomology of Sheaves}
\item \hyperref[sites-cohomology-section-phantom]{Cohomology on Sites}
\item \hyperref[dga-section-phantom]{Differential Graded Algebra}
\item \hyperref[dpa-section-phantom]{Divided Power Algebra}
\item \hyperref[sdga-section-phantom]{Differential Graded Sheaves}
\item \hyperref[hypercovering-section-phantom]{Hypercoverings}
\end{enumerate}
Schemes
\begin{enumerate}
\setcounter{enumi}{25}
\item \hyperref[schemes-section-phantom]{Schemes}
\item \hyperref[constructions-section-phantom]{Constructions of Schemes}
\item \hyperref[properties-section-phantom]{Properties of Schemes}
\item \hyperref[morphisms-section-phantom]{Morphisms of Schemes}
\item \hyperref[coherent-section-phantom]{Cohomology of Schemes}
\item \hyperref[divisors-section-phantom]{Divisors}
\item \hyperref[limits-section-phantom]{Limits of Schemes}
\item \hyperref[varieties-section-phantom]{Varieties}
\item \hyperref[topologies-section-phantom]{Topologies on Schemes}
\item \hyperref[descent-section-phantom]{Descent}
\item \hyperref[perfect-section-phantom]{Derived Categories of Schemes}
\item \hyperref[more-morphisms-section-phantom]{More on Morphisms}
\item \hyperref[flat-section-phantom]{More on Flatness}
\item \hyperref[groupoids-section-phantom]{Groupoid Schemes}
\item \hyperref[more-groupoids-section-phantom]{More on Groupoid Schemes}
\item \hyperref[etale-section-phantom]{\'Etale Morphisms of Schemes}
\end{enumerate}
Topics in Scheme Theory
\begin{enumerate}
\setcounter{enumi}{41}
\item \hyperref[chow-section-phantom]{Chow Homology}
\item \hyperref[intersection-section-phantom]{Intersection Theory}
\item \hyperref[pic-section-phantom]{Picard Schemes of Curves}
\item \hyperref[weil-section-phantom]{Weil Cohomology Theories}
\item \hyperref[adequate-section-phantom]{Adequate Modules}
\item \hyperref[dualizing-section-phantom]{Dualizing Complexes}
\item \hyperref[duality-section-phantom]{Duality for Schemes}
\item \hyperref[discriminant-section-phantom]{Discriminants and Differents}
\item \hyperref[derham-section-phantom]{de Rham Cohomology}
\item \hyperref[local-cohomology-section-phantom]{Local Cohomology}
\item \hyperref[algebraization-section-phantom]{Algebraic and Formal Geometry}
\item \hyperref[curves-section-phantom]{Algebraic Curves}
\item \hyperref[resolve-section-phantom]{Resolution of Surfaces}
\item \hyperref[models-section-phantom]{Semistable Reduction}
\item \hyperref[equiv-section-phantom]{Derived Categories of Varieties}
\item \hyperref[pione-section-phantom]{Fundamental Groups of Schemes}
\item \hyperref[etale-cohomology-section-phantom]{\'Etale Cohomology}
\item \hyperref[crystalline-section-phantom]{Crystalline Cohomology}
\item \hyperref[proetale-section-phantom]{Pro-\'etale Cohomology}
\item \hyperref[more-etale-section-phantom]{More \'Etale Cohomology}
\item \hyperref[trace-section-phantom]{The Trace Formula}
\end{enumerate}
Algebraic Spaces
\begin{enumerate}
\setcounter{enumi}{62}
\item \hyperref[spaces-section-phantom]{Algebraic Spaces}
\item \hyperref[spaces-properties-section-phantom]{Properties of Algebraic Spaces}
\item \hyperref[spaces-morphisms-section-phantom]{Morphisms of Algebraic Spaces}
\item \hyperref[decent-spaces-section-phantom]{Decent Algebraic Spaces}
\item \hyperref[spaces-cohomology-section-phantom]{Cohomology of Algebraic Spaces}
\item \hyperref[spaces-limits-section-phantom]{Limits of Algebraic Spaces}
\item \hyperref[spaces-divisors-section-phantom]{Divisors on Algebraic Spaces}
\item \hyperref[spaces-over-fields-section-phantom]{Algebraic Spaces over Fields}
\item \hyperref[spaces-topologies-section-phantom]{Topologies on Algebraic Spaces}
\item \hyperref[spaces-descent-section-phantom]{Descent and Algebraic Spaces}
\item \hyperref[spaces-perfect-section-phantom]{Derived Categories of Spaces}
\item \hyperref[spaces-more-morphisms-section-phantom]{More on Morphisms of Spaces}
\item \hyperref[spaces-flat-section-phantom]{Flatness on Algebraic Spaces}
\item \hyperref[spaces-groupoids-section-phantom]{Groupoids in Algebraic Spaces}
\item \hyperref[spaces-more-groupoids-section-phantom]{More on Groupoids in Spaces}
\item \hyperref[bootstrap-section-phantom]{Bootstrap}
\item \hyperref[spaces-pushouts-section-phantom]{Pushouts of Algebraic Spaces}
\end{enumerate}
Topics in Geometry
\begin{enumerate}
\setcounter{enumi}{79}
\item \hyperref[spaces-chow-section-phantom]{Chow Groups of Spaces}
\item \hyperref[groupoids-quotients-section-phantom]{Quotients of Groupoids}
\item \hyperref[spaces-more-cohomology-section-phantom]{More on Cohomology of Spaces}
\item \hyperref[spaces-simplicial-section-phantom]{Simplicial Spaces}
\item \hyperref[spaces-duality-section-phantom]{Duality for Spaces}
\item \hyperref[formal-spaces-section-phantom]{Formal Algebraic Spaces}
\item \hyperref[restricted-section-phantom]{Algebraization of Formal Spaces}
\item \hyperref[spaces-resolve-section-phantom]{Resolution of Surfaces Revisited}
\end{enumerate}
Deformation Theory
\begin{enumerate}
\setcounter{enumi}{87}
\item \hyperref[formal-defos-section-phantom]{Formal Deformation Theory}
\item \hyperref[defos-section-phantom]{Deformation Theory}
\item \hyperref[cotangent-section-phantom]{The Cotangent Complex}
\item \hyperref[examples-defos-section-phantom]{Deformation Problems}
\end{enumerate}
Algebraic Stacks
\begin{enumerate}
\setcounter{enumi}{91}
\item \hyperref[algebraic-section-phantom]{Algebraic Stacks}
\item \hyperref[examples-stacks-section-phantom]{Examples of Stacks}
\item \hyperref[stacks-sheaves-section-phantom]{Sheaves on Algebraic Stacks}
\item \hyperref[criteria-section-phantom]{Criteria for Representability}
\item \hyperref[artin-section-phantom]{Artin's Axioms}
\item \hyperref[quot-section-phantom]{Quot and Hilbert Spaces}
\item \hyperref[stacks-properties-section-phantom]{Properties of Algebraic Stacks}
\item \hyperref[stacks-morphisms-section-phantom]{Morphisms of Algebraic Stacks}
\item \hyperref[stacks-limits-section-phantom]{Limits of Algebraic Stacks}
\item \hyperref[stacks-cohomology-section-phantom]{Cohomology of Algebraic Stacks}
\item \hyperref[stacks-perfect-section-phantom]{Derived Categories of Stacks}
\item \hyperref[stacks-introduction-section-phantom]{Introducing Algebraic Stacks}
\item \hyperref[stacks-more-morphisms-section-phantom]{More on Morphisms of Stacks}
\item \hyperref[stacks-geometry-section-phantom]{The Geometry of Stacks}
\end{enumerate}
Topics in Moduli Theory
\begin{enumerate}
\setcounter{enumi}{105}
\item \hyperref[moduli-section-phantom]{Moduli Stacks}
\item \hyperref[moduli-curves-section-phantom]{Moduli of Curves}
\end{enumerate}
Miscellany
\begin{enumerate}
\setcounter{enumi}{107}
\item \hyperref[examples-section-phantom]{Examples}
\item \hyperref[exercises-section-phantom]{Exercises}
\item \hyperref[guide-section-phantom]{Guide to Literature}
\item \hyperref[desirables-section-phantom]{Desirables}
\item \hyperref[coding-section-phantom]{Coding Style}
\item \hyperref[obsolete-section-phantom]{Obsolete}
\item \hyperref[fdl-section-phantom]{GNU Free Documentation License}
\item \hyperref[index-section-phantom]{Auto Generated Index}
\end{enumerate}
\end{multicols}


\bibliography{my}
\bibliographystyle{amsalpha}

\end{document}
